Within the context of higher education, learning programming is hard for both beginners and students with past experience. 

It is suggested by some [...] that learning programming is no different than most other complex skills: it takes roughly ten years (ten thousand hours) to become truly proficient.

The reason why it takes so long is disarmingly simple. Programming requires both the ability to \textbf{understand} and to \textbf{design} code. 

\subsection{Understanding code}
Understanding code is a passive skill, but not any simpler because of it. The true meaning of code is the sequence of steps that the machine will actually perform: every single bit that will be read and written as a result of an instruction is part of the meaning of that instruction, just like every cache hit-or-miss, the activation of the CPU ALU(s), network channels, operating systems, interpreters, just-in-time compilers, and ultimately interactions with users. Being able to figure precisely what a program does, and how it does it (also in terms of performance) requires the ability to formulate an abstract idea of the program behaviour, and the mapping of this abstract idea to the concrete components when more specific reasoning is needed.

The sheer size of the machinery involved in the execution of even the simplest program is simply very large, and the ability to think hierarchically and zoom in and out of the details as needed takes a lot of experience.
