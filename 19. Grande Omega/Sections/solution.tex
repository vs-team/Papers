The didactic model we propose (just like the implementing software) is called GrandeOmega.

Whereas many “revolutionary” didactic models completely remove emphasis from the role of the teacher, thereby forsaking the richest source of knowledge and explanation available in the vicinity of students, GrandeOmega sets out to empower both teacher and student. 

The teacher sets up his course, in a way similar to traditional lectures. Each lecture is made up of slides, and these slides follow some visual formalism to explain code. One  of the most used such formalism is (recursive) tables of names and values. Traditional slides are \textit{passive}.

Some of the slides, ideally one every quarter to half an hour, are active slides. The active slides use the \textit{very same formalism of the passive slides, but with a major difference: students can experiment with them}. This means that the theory discussed by (and with) the teacher is not separated from the practice, but comes to life and strengthens the practice just as much as the practice gives context and use to the theory.

Experimentation within active slides takes two forms: \textit{forward} and \textit{backward} assignments.

\paragraph{Forward assignments}
Within forward assignments students practice their understanding of code. They get to see whole (small) programs, and they are asked by the system to provide the values of variables or constants (\texttt{REASON\textunderscore MODEL} and \texttt{REASON\textunderscore DESIGN} issues from Table \ref{tab:issues}). The values that need to be predicted by students can range from the integer value of a global variable (\texttt{x=1}), to the value of a reference in the virtual heap (\texttt{l=ref(10)}).

\paragraph{Backward assignments}
Within backward assignments students practice with design strategies to build code. They get to see whole (small) programs where some bits have been hidden, and they also get to see all the steps that the complete program would take. Students must guess back the hidden bits of code (\texttt{EXTENDED\textunderscore DESIGN} and \texttt{EXTENDED\textunderscore BUILDUP} issues from Table \ref{tab:issues}). The hidden bits of code range from a constant (\texttt{5}) or a variable name (\texttt{x}) to whole expressions (\texttt{i*2}), to whole instructions, functions, methods, classes, queries, potentially up to the whole program.

\paragraph{Immediate feedback}
Each and every mistake automatically produces immediate, visual feedback in the form of colors and icons. The input from students \textit{is evaluated, not just compared as text}.

\paragraph{Gamification}
Progress of a student, and every success, is stored and clearly shown in context by means of colors and icons. A student can see, at a glance, how far he is: whether lagging behind, or following nicely.

\paragraph{Insight to teachers}
All the \textit{data gathered} by the logging systems is \textit{analysed and presented to teachers}. This makes it possible to identify difficult topics (assignments where the majority of students is getting stuck), identify struggling students (who are getting stuck in most of the assignments), and to identify negative behaviours (too few hours of study, lack of regularity, etc.). As stated above, we strongly believe that modern didactics should not try to provide a revolutionary removal of the teacher: \textit{the teacher remains the director of the educational orchestra, even though lots of active work is performed by the students}.

\paragraph{Extra ingredients}
Understanding and designing code on a smaller scale, with a gradual buildup in size and complexity, makes it possible for students to gain a deep understanding of the underlying logical mechanisms.

To further increase efficiency of a curriculum, the task of learning should be cornered from multiple sides. Students should also be presented with the challenge of freely experimenting with open designs and projects to build. Thus, the didactic method as a whole can be summarised as:

\begin{itemize}[noitemsep]
	\item forward assignments to learn \textit{understanding};
	\item backward assignments to learn \textit{design};
	\item projects to \textit{sum it up}.
\end{itemize}

