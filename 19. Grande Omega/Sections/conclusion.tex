In this work we addressed the problem of teaching and learning how to program and we proposed the use of a didactic model called GrandeOmega. The web implementation of GrandeOmega was tested on a subset of students from Hogeschool Rotterdam with promising results: the use of the didactic model managed, in some cases, to improve the pass rate of the students, while it always improved their overall performance (measured as a percentage of the maximum score). Furthermore, the classes that did not use GrandeOmega performed generally more poorly to the point that, in one of them, no students managed to successfully complete any of the assignments. The tool is also able to predict with a reliability of 77\% if a student will pass the course on the base of the amount of completed assignments. We can thus conclude that using GrandeOmega boosts the general performance of the students, measured as the percentage of correctly given answers, and, in some cases, improves also the pass rate of the course. On the other hand, students who did not use GrandeOmega at all performed, on average, worse and had a lower passing percentage.