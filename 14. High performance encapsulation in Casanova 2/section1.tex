Video games industry is an ever growing sector with sales surpassing 100 million \$ in 2014 \cite{GAME_SALES}. Video games are not only built for entertainment purposes but they are also used by schools, government agencies, military, and research. These ``serious'' games are not allocated the development budgets available for the entertainment industry, thus the developers are interested in tools capable of overcoming the difficulties deriving from the complexity of games, and the long development time required to build their product.

Video games are complex and composed of several inter-operating components, which accomplish different and coordinated tasks, such as drawing a game object, running the physics simulation of bodies, and the artificial intelligence moving non-playable characters. These components are periodically activated in turn to update the game state and draw the scene. The rate at which these components update the game state can seriously affect the quality of the game, as the player becomes less efficient in performing precision tasks (such as jumping across a chasm, clicking a small interface button, etc.) and also perceives a less overall quality in the play experience \cite{FPS_RATE2, FPS_RATE1}. Computer games are still mostly developed using general purpose programming languages, such as C++ and C\#. Given their general purpose nature, these languages are not ideal to define common patterns used in games, thus forcing the programmers to implement them by hand. Manual implementations seek high-performance but lose readability of the code, since they alter the program structure by adding auxiliary data structures. These data structures introduce additional dependencies among the game elements to run the optimization, which affect the understandability and complicate the program modification \cite{ENCAPSULATION_AND_INHERITANCE_IN_OOP}. A solution is using the encapsulation design pattern to increase readability but is low-performance \cite{ENCAPSULATION_DISADVANTAGES}, thus developers usually discard this design pattern for a high-performance solution.

\vspace{0.5cm}
\noindent
In this paper we present a solution to the loss of performance in encapsulated programs by using a domain specific language, Casanova, that allows developers to write readable and maintainable code and, at the same time, relives them from writing optimizations by hand to gain high-performance.

\vspace{0.5cm}
\noindent
In Section \ref{sec:encapsulation} we introduce a sample of a game implemented both with encapsulated code and a faster implementation which breaks encapsulation. We will then discuss the complexity of both solutions. In Section \ref{sec:optimization_overview} we explain the idea behind our transformation system. In Section \ref{sec:implementation} we build the transformation system inside Casanova compiler. In Section \ref{sec:evaluation} we show a comparison between our optimized encapsulated code and a not encapsulated implementation.