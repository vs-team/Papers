In this section we discuss first common issue arising from traditional facilities for game development. We then introduce a short example to explain the problem of encapsulation in games. Eventually, we discuss the advantages and disadvantages of using encapsulation when designing a game.

\subsection{Common issues}
\label{Common issues}
In the panorama of game development the two main approaches to game development are either the use of \textbf{tools}, or the use of \textbf{languages} \cite{maggiore2013casanova}.

\textbf{Tools} are environments where developers are assisted in the creation of games through visual instruments and built-in features (such as physics). Tools are generally focused on specific genres. A typical aspect of these tools is that they offer developers predefined functionalities (such as path finding, collision detection, and rendering), which would take a lot of time to develop and debug. These functionalities are often available in the shape of menu objects in the development environment. The goal of tools, in general, is to allow developers to quickly prototype and deploy games, while relieving them from common tasks in game development. Typical tools, such as GameMaker, Corona, Unity3D, and RPGmaker, provide an easy-to-use interface and shortcuts for dealing with entity behavior. As long as the developer limits himself to using the components provided, the tools produce high-performance game code, because of their specific application in the domain of games. For behaviors that are not expressed natively by the tool components, tools often offer scripting languages that allow developers to define custom behaviors. Unfortunately, the expressiveness of such scripting languages is affected by the mechanics of the tools to which they are adapted. Thus, in order to effectively use these languages developers are supposed to make a considerable effort with understanding the mechanics of the chosen tool and to adapt their solutions accordingly. Moreover, the scripting languages used by tools are usually interpreted (like LUA and JavaScript \cite{anderson2011classification}), which considerably affects performance.

General-purpose \textbf{languages} (GPLs) are languages that provide powerful, composable, abstractions for expressing general data structures and algorithms. These properties allow such GPLs to express solutions for different kinds of applications including the one of game development. C\#, Python, and Objective-C are typical examples of languages used for game development. A typical limitation in using GPLs is in expressing performance patterns \cite{cheung2015bridging}. Performance in games is very important, since it is strongly related to the perceived quality of it. Due to this lack of support by GPLs, a game featuring complex data structures and algorithms will require developers to implement optimization by hand, thus increasing the costs of implementing them. Unless the developers have access to large financial resources, the use of a GPL for game development is not a good choice. However, developers can use domain-specific languages (DSLs) for implementing their games, since they are capable not only of expressing domain-specific abstractions easily, but also to perform domain-specific optimization. For instance: the Conceptual Domain Modeling Language (CDM) \cite{best2009searching} supports an efficient parallel implementation of a producer/consumer pattern; Inform \cite{reed2010creating}, and Zillions of Games \cite{mallett1998zillions} provide domain-specific abstractions focused on the definition of specific genres, such as storytelling or board games; monadic frameworks have been developed to tackle common problems of game development such as combining and defining behaviors that depend on the flow of time and their efficient implementation, to integrate within game engines \cite{maggiore2011monadic} and reduce code complexity \cite{tabareau2013typed}. The only issues of such languages is that in many cases they are prototypes and supported by small communities, which translates into the fact that code that is supposed to work actually does not always because of compiler issues. Moreover, every DSL has its own learning curve, since similarities between DSLs are few (unlike for GPLs): every DSL comes with its own philosophy, constructs, primitives, etc. The advantage of Casanova 2 over other DSLs is that the latter are usually limited to one or few game genres sharing similar characteristics, while the former can be used to develop any kind of game genre. This will require, of course, to learn Casanova 2 but not other languages because of language limitations.


\subsection{Running example}

%recall encapsulation again, and why games would benefit from it?

To illustrate the discussions hereafter, we now present a game that contains typical elements that are often encountered in game development. The game consists of a set of planets linked together by routes. A player can move fleets from his planets to attack and conquer enemy planets. Fleets reach other planets by using the provided routes. Whenever a fleet gets close enough to an enemy planet, it starts fighting the defending fleets orbiting the planet. The game can be considered the basis for a typical \emph{Planet Wars} strategy game (such as Galcon \cite{wiki:galcon}). We define a \textit{frame} to be a single update cycle of all the game's data structures.

In our running example, we assume that a \texttt{Route} is represented by a data structure containing (\textit{i}) the start and end points as references to \texttt{Planets}, and (\textit{ii}) a list of \texttt{Fleets} travelling via such route. \texttt{Planet} is a data structure containing (\textit{i}) a list of defending \texttt{Fleets}, (\textit{ii}) a list of attacking \texttt{Fleets}, and (\textit{iii}) an \texttt{Owner}. Each fleet has an owner as well. Each data structure contains a method called \texttt{Update}, which updates the state of its associated object at every frame. Furthermore, we assume that all the game objects have direct access to the global game state, which contains the list of all routes in the game scenario.

Following the definition of encapsulation given in Section \ref{sec:introduction}, each entity is responsible for updating its own fields in such a way that it maintains its own invariant.

\subsection{Design techniques and operations}

In our running example the modules are the \texttt{Planet} and \texttt{Route} classes defined above; \textit{data} are their fields. To support \emph{encapsulation}, in the following implementation each entity is responsible for updating its fields with respect to the world dynamics. The \textit{operations} for each entity are the following:
\begin{itemize}
    \item[] \textbf{Planet}: Takes the enemy fleets travelling along its incoming routes, which are close to the planet, and moves them into the attacking fleets list. We assume that \texttt{Planet} contains a method to check whether a fleet is contained in the list of attacking fleets, called \texttt{IsAttackedBy};
    \item[] \textbf{Route}: Removes the travelling fleets, which have been placed in the attacking fleets of the destination planet from the list of travelling fleets.
\end{itemize}

\begin{lstlisting}
class Route
  Planet Start, Planet End,
  List<Fleet> TravellingFleets,
  Player Owner
  void Update()
    foreach fleet in TravellingFleets
      if End.IsAttackedBy(fleet)
        this.TravellingFleets.Remove(fleet)
class Planet
  List<Fleet> DefendingFleets,
  List<Fleet> AttackingFleets
  void Update()
    foreach route in GetState().Routes
      if route.End = this then
        foreach fleet in route.TravellingFleets
          if distance(fleet.Position, this.Position) < min_dist && 
             fleet.Owner != this.Owner then
            this.AttackingFleets.Add(fleet)
\end{lstlisting}


An alternative design, which does not use encapsulation, allows the route to move the fleets close to the destination planet directly into the attacking fleets by writing into the planet fields. In this scenario the route is modifying data related to the planet and the route is writing into a reference to a planet.
\begin{lstlisting}
class Route
  Planet Start, Planet End,
  List<Fleet> TravellingFleets
  void Update()
    foreach fleet in this.TravellingFleets
      if distance(fleet.Position, this.Position) < min_dist && 
         fleet.Owner != End.Owner then
          this.TravellingFleets.Remove(fleet)
          End.AttackingFleets.Add(fleet)
\end{lstlisting}
\subsection{Discussion}
In our running example a programmer is left with the choice of (\textit{i}) either using the paradigm of encapsulation, which improves the understandability of programs and eases their modification \cite{ENCAPSULATION_AND_INHERITANCE_IN_OOP}, or (\textit{ii}) breaking encapsulation by writing directly into the planet fields from an external class, which, as we will show below, is more efficient but potentially dangerous \cite{eder1994coupling}.

As far as \emph{performance} is concerned, in the encapsulated version, the planet queries the game state to obtain all routes where endpoints are the planet itself, and for every route selects the enemy travelling fleets that are close enough to the planet. At the same time, a \texttt{Route} checks the list of attacking fleets of its endpoints and removes the fleets that are contained in both lists from the travelling fleets. If we consider a scenario containing $m$ planets, $n$ routes, and at most $k$ travelling fleets per route, each planet should check the distance condition for $O(nk)$ ships; thus the overall complexity is $O(mnk)$. The non-encapsulated version checks, for each route, the distance for a maximum of $k$ ships and then directly moves those close to the planet, for which the overall complexity is $O(nk)$. Therefore, the performance on the non-encapsulated version is better. One could argue that adding a spatial index, such as a KD-Tree for fast entities lookup \cite{white2007scaling}, in the planet containing the incoming routes could lead to higher performance; however this would break the \textit{SOLID} (single responsibility, open-closed, Liskov substitution, interface segregation and dependency inversion) principle of Design Patterns, as a planet would contain information on the topology of part of the map. In particular the \textit{Single Responsibility} is violated, as the task of the planet is less deducible.

As far as \emph{maintainability} is concerned, in a game containing planets, many entities might need to interact with each planet (such as fleets, upgrades, and special weapons). Assume that a special action freezes all the activities of a planet. We have to propagate this behavior into the code of all the entities in the game that may interact with a planet, disabling such interactions when the planet is frozen. In the encapsulated version of the code, such behavior needs only be implemented in one place, namely in the planet. In the non-encapsulated version, it must be implemented in each and every entity that may interact with a planet. Moreover, if the developer forgets to make this change even in just one of the entities, the game no longer functions correctly; i.e., bugs associated with planets might actually find their cause in other entities. It is clear that the maintainability of the encapsulated version of the code is much better than the maintainability of the non-encapsulated version.

The main advantage of using encapsulation is related to the maintainability of code, because encapsulated operations that alter the state of an entity are strictly defined within the entity definition. This helps to reduce the amount of code to maintain in case the entity changes the \textit{normal} behavior of an entity. In our scenario all the activities that alter the planet are inside the planet, so if we remove (or disable) a planet then all its operations are suspended.

What we desire to achieve is the maintainability of encapsulated game code, combined with the performance of non-encapsulated code. In the following sections, we show how this can be achieved with Casanova. 