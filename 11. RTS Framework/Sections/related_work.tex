Currently there are many commercial and open source solutions for developing RTS games which result to be often too specific, thus inflexible or not scalable. When users would like to extend these frameworks, this often turns out to be difficult, if not impossible, unless they change the entire structure of the project by changing the structures of entities and the connections among them.
There are not many specific RTS engines but some of the most common used are listed below.
\subsubsection*{Game maker:}
Game maker is a tool which joins the visual development with a limited scripting language. The scripting language allows only the use of strings and real numbers, possibly indexed as arrays. However, it is neither possible to pass an array as a script argument nor accessing it with a pointer except by passing a string holding the name of the array itself. R.E.A. instead is an extension of a well defined and structured programming language like Casanova with no such limitations and workarounds.
\subsubsection*{ORTS – Open real-time strategy engine:}
ORTS is a domain specific language for making RTS games based on scripts. The language of the scripts is limited; what is not supported by its primitives must be written in C. However, this lack of expressiveness is compensated by being domain specific. ORTS is not designed to be very general because it is specific only for the RTS genre, and in particular for RTS's without an articulated logic. Finally, native optimization, which is provided by our solution, is not possible in ORTS, as explained above, unless the developer codes it by himself.
\subsubsection*{Spring engine:}
Spring engine is a framework for creating RTS games. The engine specifies predefined boundaries on game dynamics, which cannot be extended. The developer has to learn a long series of keywords. Moreover, getting out of the predefined context, requires to code in a different semantic level using scripting languages such as LUA. However, Spring engine is a good RTS framework which implements a wide variety of options and, in some cases, native optimization (such as spatial optimization for collision detection). Spring engine also presents the same problems, as for the scripting language, of the other engines listed above.