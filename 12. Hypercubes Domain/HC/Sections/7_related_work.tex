Various numerical domains have been studied in the literature, and they can be classified with respect to a number of different dimensions: finite (e.g., \textit{Sign}) versus infinite (e.g., \textit{Intervals}) height, relational (e.g., Octagons \cite{MIN06}) versus non-relational (e.g., \textit{Intervals}), convex (e.g., Polyhedra \cite{CH78}) versus possibly non-convex (e.g., donut-like domains \cite{G12}). Hypercubes track disjunctive information relying on Intervals. Similarly, the powerset operator \cite{FR99} allows one to track disjunctive information, but the complexity of the analysis grows up exponentially. Instead, we designed a specific disjunctive domain that reduces the practical complexity of the analysis by adopting indexes and offsets. 

%A number of different numerical abstract domains have been studied in the literature, and they can be classified with respect to a number of different dimensions: finite versus infinite height, relational versus non-relational, convex versus possibly non-convex, and so on. 
%
%The computational cost increases when lifting from finite non–relational domains like \textit{Sign} or \textit{Parity}, to 
%infinite non–relational domains like \textit{Intervals}, to sophisticated infinite relational domains like \textit{Octagons} \cite{MIN06}, \textit{Polyhedra} \cite{CH78}, \textit{Pentagons} \cite{LF08}, and \textit{Stripes} \cite{F08}, or to donut-like non-convex domains \cite{G12}. Moreover, when considering possibly non-convex disjunctive domains, as obtained through the powerset operator \cite{FR99}, the complexity of the analysis is growing (as well as its accuracy) in a full orthogonal (exponential) way. 

Noticeable efforts have been put both to reduce the loss of precision due to the upper bound operation, and to accelerate the convergence of the Kleene iterative algorithm \cite{HMG06,SB13,SISG06,BHZ07}, but they do not track disjunctive information.

%Noticeable efforts have been put both to reduce the loss of precision due to the upper bound operation, and to accelerate the convergence of the Kleene iterative algorithm. Some ways to reduce the space dimension in polyhedra computations relying on variable elimination and Cartesian factoring are introduced in \cite{HMG06}.  
%Seladji and Bouissou \cite{SB13} designed refinement tools based on convex analysis to express the convergence of convex sets using support functions, and on numerical analysis to accelerate this convergence applying sequence transformations. 
%On the other hand, Sankaranarayanan et al. \cite{SISG06} faced the issue of reducing the computational cost of the analysis using a powerset domain, by adopting restrictions based on \textquotedblleft on the fly elaborations\textquotedblright\ of the program's control flow graph. Efficiency issues about convergence acceleration by widening in the case of a powerset domain have been studied by Bagnara et al. in \cite{BHZ07}. All these domains do not track disjunctive information.

The trace partitioning technique designed by Mauborgne and Rival \cite{MR05} provides automatic procedures to build suitable partitions of the traces yielding to a refinement that has great impact both on the accuracy and on the efficiency of the analysis. This approach tracks disjunctive information, and it works quite well when the single partitions are carefully designed by an expert user. Unluckily, given the high number of hypercubes tracked by our analysis, this approach is definitely too slow for the scenario we are targeting.

Our spatial representation and width adjustment resembles the hierarchical data-structure of quadtrees in \cite{HKL10}. However, this paper contains only a preliminary discussion of the quadtree domain, and as far as we know it has not been further developed nor applied. %the authors are ``neutral" to the applicability of quadtrees for use in practical analyzers. 
Moreover, their domain is targeted to analyse only machine integers %(while we deal with real values) 
and the width is the same in each spatial axis. %(while we use a different width for each variable and offsets). %also, no offsets!

Our self-adaptive parametrization of the width shares some common concepts with  the CounterExample Guided Abstraction Refinement (CEGAR) \cite{CGJ00}. CEGAR begins checking with a coarse (imprecise) abstraction of the system and progressively refines it, based on spurious counterexamples seen in prior model checking runs. The process continues until either an abstraction proves the correctness of the system or a valid counterexample is generated. 
%This tool targets model checking systems, and it refines the level of abstraction by looking to a counter-example.

\cite{GC10} introduced the Boxes domain, a refinement of the Interval domain with finite disjunctions: an element of Boxes is a finite union of boxes. Each value of Boxes is a propositional formula over interval constraints and it is represented by the Linear Decision Diagrams data structure (LDDs). Note that the size of an LDD is exponential in the number of variables. We use a fixed width and a fixed partitioning on each hypercube dimension, while they do not employ constraints of this kind. In addition, Boxes uses a specific abstract transformer for each possible operation (for example, distinguishing $x = x + v$, $x = a \times x$, $x = a \times y$ and also making assumptions on the sign of constants) while our definitions are more generic. Finally, Boxes' implementation is based on the specific data structure of LDDs and cannot be extended to other base domains. %, while our approach can. 

%The Parametric Hypercubes proposal presented in this paper can be seen as a selection and a combination of most of these techniques, tailored to get a solution that properly suits the features of Computer Games Software applications. In this scenario, we had to track precisely a lot of disjunctive information. Therefore, we needed to introduce a targeted domain \adomain\ for physics simulations. On the other hand, some of the domain's features (like width parameter tuning, and interval offsets) may also be applied in other domains. 

If on the one hand Parametric Hypercubes have been tailored to Computer Games Software applications, on the other hand some of their features may also be applied to other contexts. In particular, our definition of  Computer Games Software applications (i.e., an infinite reactive loop, a complex state space with many real-valued variables, and strong dependencies among variables) exactly matches that of real-time synchronous control-command software (found in many industries such as aerospace and automotive industries). Hybrid systems and hybrid automata have been widely applied to verify this software. The formal analysis of large scale hybrid systems is known to be a very difficult process \cite{AHLP00}. In general, existing approaches suffer from performance issues or limitations on the property to prove, on the shape of the program, etc. For instance, \cite{BMC12} deals a simpler example than ours (a bouncing ball with only vertical motion) and in their benchmarks the variable space is quite limited: the velocity is a fixed constant, and the starting position varies only between 10 and 10.1. Instead, our Hypercubes can deal with velocities and positions bound inside any intervals of values. Also in \cite{B09} the variable space is more restricted than in our approach. 
%happens the same: the benchmark \emph{Heater} works on the variable space $[0..5]$, the \emph{Navigation} one on $[3.5..3.6] \times [3.5..3.6]$ and the \emph{Two-tanks} on $[5..5] \times [6..6]$; 
In addition, this analysis returns an abstraction of the final state of the program, while we also give information about which starting values are responsible for the property verification and which not.
%In \cite{HHW97} a lot of benchmarks are cited but no table with results and performances is presented. 
%\cite{ACH95} presents a general framework for the formal specification and algorithmic analysis of hybrid systems, while we proposed a specific analysis. 
\cite{HRP94} presents an application of the abstract interpretation by means of convex polyhedra to hybrid systems. This work is focused on a particular class of hybrid systems (\emph{linear} ones), and it is able to represent only convex regions of the space, since it employs the convex hull approximation of a set of values. \cite{ADI02} presents algorithms and tools for reachability analysis of hybrid systems by relying on predicate abstraction and polyhedra. However, this solution suffers from the exponential growth of abstract states and relies on expensive abstract domains. Finally, \cite{RS05} concerns safety verification of non-linear hybrid systems, starting from a classical method that uses interval arithmetic to check whether trajectories can move over the boundaries in a rectangular grid. This approach is similar to ours in the data representation (boxes). However, they do not employ any concept of offset, their space partitioning is not fixed and the examples they experimented with cover a very limited variable space.

