A number of different numerical abstract domains have been studied in the literature, and they can be classified with respect to a number of different dimensions: finite versus infinite height, relational versus non-relational, convex versus possibly non-convex, and so on. 

The computational cost raises when lifting from finite non–relational domains like \textit{Sign} or \textit{Parity}, to 
infinite non–relational domains like \textit{Intervals}, to sophisticated infinite relational domains like \textit{Octagons} \cite{MIN06}, \textit{Polyhedra} \cite{CH78}, \textit{Pentagons} \cite{LF08}, and \textit{Stripes} \cite{F08}, or to donut-like non-convex domains \cite{G12}. Moreover, when considering possibly non-convex disjunctive domains, as obtained through the powerset operator \cite{FR99}, the complexity of the analysis is growing (as well as its accuracy) in a full orthogonal (exponential) way.

Noticeable efforts have been put both to reduce the loss of precision due to the upper bound operation, and to accelerate the convergence of the Kleene iterative algorithm. Some ways to reduce the space dimension in polyhedra computations relying on variable elimination and Cartesian factoring are introduced in \cite{HMG06}.  
Seladji and Bouissou \cite{SB13} designed refinement tools based on convex analysis to express the convergence of convex sets using support functions, and on numerical analysis to accelerate this convergence applying sequence transformations. 
On the other hand, Sankaranarayanan et al. \cite{SISG06} faced the issue of reducing the computational cost of the analysis using a powerset domain, by adopting restrictions based on \textquotedblleft on the fly elaborations\textquotedblright\ of the program's control flow graph. Efficiency issues about convergence acceleration by widening in the case of a powerset domain have been studied by Bagnara et al. in \cite{BHZ07}. All these domains do not track disjunctive information.

Finally, the trace partitioning technique designed by Mauborgne and Rival \cite{MR05} provides automatic procedures to build suitable partitions of the traces yielding to a refinement that has great impact both on the accuracy and on the efficiency of the analysis. This approach tracks disjunctive information, and it works quite well when the single partitions are carefully designed by an expert user. Unluckily, given the high number of hypercubes tracked by our analysis, this approach is definitely too slow for the scenario we are targeting.

The Parametric Hypercubes proposal presented in this paper can be seen as a selection and a combination of most of these techniques, tailored to get a solution that properly suits the features of Computer Games Software applications. In this scenario, we had to track precisely a lot of disjunctive information. Therefore, we needed to introduce a targeted domain \adomain\ for physics simulations. On the other hand, some of the domain's features (like width parameter tuning, and interval offsets) may also be applied in other domains. 

\subsection{Future work}

We observe that our approach offers plenty of venues in order to improve its results, thanks to its flexible and parametric nature. In particular, we could: (i) increase the precision by intersecting our hypercubes with arbitrary bounding volumes which restrict the relationships between variables (the offsets presented in Section \ref{sec:tuning} are the simplest version of this extension); (ii) increase the performance of Algorithm \ref{alg:widthAdjusting} by halving the widths only on some axes, chosen through an analysis of the distribution of hypercubes in the \emph{yes,no,maybe} sets; and (iii) study the derivative with respect to time of the iterations of the main loop in order to define temporal trends to refine the widening operator.
In addition, our domain is modular w.r.t. the non-relational abstract domain adopted to represent the hypercube dimensions. By using other abstract domains it is possible to track relationships between variables which do not necessarily represent physical quantities.

