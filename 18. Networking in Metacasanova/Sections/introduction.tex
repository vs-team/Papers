Adding multiplayer support to games is a desirable feature to increase the game popularity by attracting millions of players \cite{ducheneaut2006alone}. The benefit for the popularity is given by the fact that, after completing the singleplayer experience, the user may keep playing the game online in cooperation or against other players, in different modes that are more interesting and challenging than their singleplayer counterparts \cite{pantel2002impact}. Multiplayer games are very sensitive to network delay (latency), packet loss, and bandwidth limits, thus requiring heavy optimizations that are different and depend on the specific instance of the game. For instance, a Turn-based strategy game will require less optimization with respect to bandwidth usage or delays, because the data transmission happens once, while in a First-person shooter, which gameplay is based on fast-paced action, low latency is paramount. In addition to games developed for the entertainment sector, networking is a crucial feature also in the field of serious games. However, serious games developers do not have access to the same amount of resources and manpower available at big companies such as EA or Ubisoft, while having to deal with the same degree of complexity in their products.

Notable examples of s

For this reason, manually implementing the networking module is not a feasible solution in the field of serious games, as serious games developers have to rely on game engines, such as Unity 3D or Unreal Engine, which offer a built-in API to deal with networking. Although being a better approach than developing a hand-made networking protocol for a specific game, game engines require extensive knowledge to be used effectively and require, to some extent, to still deal directly with low-level issues related to data transmission and remote procedure calls. This problem can be solved in part by developing a Domain-specific language including abstractions to define the behaviour of the game in a networking scenario. Choosing this approach requires to implement a compiler for the language, which can be argued to be as time-consuming as developing a hand-made networking implementation. In this paper we propose an alternative approach to the development of a hard-coded compiler for a Domain-specific based on meta-compilation, which has been proven to be an effective technique in different works \cite{kaagedal1998generating,DiGiacomo2017}, in order to have the benefits of a domain-specific language for networking and, at the same time, a lower development effort.
 
 \textbf{Complete with summary of the paper}