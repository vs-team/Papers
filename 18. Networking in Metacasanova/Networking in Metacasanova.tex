% This is LLNCS.DEM the demonstration file of
% the LaTeX macro package from Springer-Verlag
% for Lecture Notes in Computer Science,
\documentclass[runningheads,a4paper]{llncs}
\usepackage{makeidx}  % allows for indexgeneration
\usepackage{amsmath}
\usepackage{amsfonts}
\usepackage{amssymb}
\usepackage{listings}
\usepackage{mathpartir}
\usepackage{graphicx}
\usepackage{comment}
\usepackage{pifont}
\usepackage{enumitem}
\usepackage{url}
\lstset
{
	basicstyle = \ttfamily\tiny,
	breaklines = true,
	frame = single
}

\begin{document}
\mainmatter

\pagestyle{headings} 

\title{Automated networking in Meta-Casanova}

\titlerunning{Automated networking in Meta-Casanova}

\author{Francesco Di Giacomo\inst{1} \and Mohamed Abbadi\inst{1} \and Agostino Cortesi\inst{1}
	\and Pieter Spronck\inst{2}
	\and Giuseppe Maggiore\inst{3}}

\institute{Universita' Ca' Foscari, Venezia,\\
	\email{francesco.digiacomo@unive.it, cortesi@unive.it, mohamed.abbadi@unive.it}
	\and
	Tiburg University,
	\email{p.spronck@uvt.nl}
	\and
	Hogeschool Rotterdam,
	\email{maggg@hr.nl}}

\authorrunning{Francesco Di Giacomo et al.}
	
\maketitle


\abstract{Multiplayer support is a desirable feature in commercial games since it allows to extend the longevity of a title. The same importance can be found in several examples of serious games, as they are often used in training where several individuals must cooperate to reach a common goal. This paper has a dual purpose: (\textit{i}) showing how to extend an existing Domain-Specific language for game development with the semantics of networking primitives that allows the programmer to abstract away from the low-level details of the networking layer for a multiplayer game, and (\textit{ii}) proposing a novel approach to implement such semantics by using a Meta-compiler, which eases and speeds up the development process of the language. Our results show that the use of language abstractions significantly reduces the line of code necessary to develop a multiplayer game, and that the use of a meta-compiler significantly reduces the effort necessary to implement a compiler for the language with respect to a hard-coded implementation.}
	
\section{Introduction}
The number of programming languages available on the market has dramatically increased during the last years. The tiobe index \cite{tiobe2018}, a ranking of programming languages based on their popularity, lists 50 programming languages for 2018. This number is only a small glimpse of the real amount, since it does not take into account several languages dedicated to specific applications. This growth has brought a further need for new compilers that are able to translate programs written in those languages into executable code. The goal of this work is to investigate how the development speed of a compiler can be boosted by employing meta-compilers, programs that generalize the task performed by a normal compiler. In particular the goal of this research is creating a meta-compiler that significantly reduces the amount of code needed to define a language and its compilation steps, while maintaining acceptable performance.

This chapter introduces the issue of expressing the solution of problems in terms of algorithms in Section \ref{sec:ch1_algorithms}. Then we proceed by defining how the semi-formal definition of an algorithm must be translated into code executable by a processor (Section \ref{sec:ch1_programming_languages}). In this section we discuss the advantages and disadvantages of using different kinds of programming languages with respect to their affinity with the specific hardware architecture and the scope of the domain they target. In Section \ref{sec:ch1_compilers} we explain the reason behind compilers and we explain why building a compiler is a time-consuming task. In Section \ref{sec:ch1_metacompilers} we introduce the idea of meta-compilers as a further step into generalizing the task of compilers. In this section we also explain the requirements, benefits, and the relevance as a scientific topic. Finally in Section \ref{sec:ch1_problem_statement} we formulate the problem statement and the research questions that this work will answer.

\section{Algorithms and problems}
\label{sec:ch1_algorithms}
Since the ancient age, there has always been the need of describing the sequence of activities needed to perform a specific task \cite{barbin2012history}, to which we refer with the name of \textit{Algorithm}. The allegedly most ancient known example of this dates back to the Ancient Greek, when Hero invented an algorithm to perform the factorization and the approximation of the square root, discovered also by other civilizations \cite{ bailey2012ancient, smith1923history} . Regardless of the specific details of each algorithm, one needs to use some kind of language  to define the sequence of steps to perform. In the past people used natural language to describe such steps but, with the advent of the computer era, the choice of the language has been strictly connected with the possibility of its implementation. Natural languages are not suitable for the implementation, as they are known to be verbose and ambiguous \cite{church1982coping, resnik1999semantic}. For this reason, several kind of formal solutions have been employed, which are described below.

\subsubsection*{Flow charts}
A flow chart is a diagram where the steps of an algorithm are defined by using boxes of different kinds, connected by arrows to define their ordering in the sequence. The boxes are rectangular-shaped if they define an \textit{activity} (or processing step), while they are diamond-shaped if they define a \textit{decision}. A rectangle with rounded corners denotes the initial step. An example of a flow chart describing how to sum the numbers in a sequence is described in Figure \ref{fig:ch1_flow_chart}.

\begin{figure}
	\centering
	\includegraphics[width = \textwidth]{Figures/flow_chart}
	\caption{Flow chart for the sum of a sequence of numbers}
	\label{fig:ch1_flow_chart}
\end{figure}

\subsubsection*{Pseudocode}
Pseudocode is a semi-formal language that might contain also statements expressed in natural language and omits system specific code like opening file writers, printing messages on the standard output, or even some data structure declaration and initialization. It is intended mainly for human reading rather than machine reading. The pseudocode to sum a sequence of numbers is shown in Algorithm \ref{alg:ch1_pseudocode}.

\begin{algorithm}
	\caption{Pseudocode to perform the sum of a sequence of integer numbers}
	\label{alg:ch1_pseudocode}
	\begin{algorithmic}
		\Function{SumIntegers}{$l \text{ list of integers}$}
			\State $sum \gets 0$
			\ForAll {$x \text{ in } l$}
				\State $sum \gets sum + x$
			\EndFor
			\State \Return $sum$
		\EndFunction
	\end{algorithmic}
\end{algorithm}

\subsubsection*{Advantages and disadvantages}
Using flow charts or pseudo-code has the advantage of being able to define an algorithm in a way which is very close to the abstractions employed when using natural language: a flow chart combines both the use of natural language and a visual interface to describe an algorithm, pseudo-code allows to employ several abstractions and even define some steps in terms of natural language. The drawback of these two formal representations is that, when it comes to the implementation, the definition of the algorithm must be translated by hand into code that the hardware is able to execute. This could be done by implementing the algorithm in a low-level or high-level programming language. This process affects at different levels how the logic of the algorithm is presented, as explained further.

\section{Programming languages}
\label{sec:ch1_programming_languages}
A programming language is a formal language that is used to define instructions that a machine, usually a computer, must perform in order to produce a result through computation \cite{mordechai1996, narasimhan1967programming, oxford2008}. There is a wide taxonomy used to classify programming languages depending on their use \cite{kelleher2005lowering, myers1986visual, myers1990taxonomies}, but all can be grouped according to two main characteristics: the level of abstraction, or how close to the specific targeted hardware they are, and the domain, which defines the range of applicability of a programming language. In the following sections we give an exhaustive explanation of the aforementioned characteristics.

\subsection{Low-level programming languages}
\label{subsec:ch1_ll_languages}
A low-level programming language is a programming language that provides little to no abstraction from the hardware architecture of a processor. This means that it is strongly connected with the instruction set of the targeted machine, the set of instructions a processor is able to execute. These languages are divided into two sub-categories: \textit{first-generation} and \textit{second-generation} languages:

\subsubsection*{First-generation languages}
\textit{Machine code} falls into the category of first-generation languages. In this category we find all those languages that do not require code transformations to be executed by the processor. These languages were used mainly during the dawn of computer age and are rarely employed by programmers nowadays. Machine code is made of stream of binary data, that represents the instruction codes and their arguments \cite{guide2011intel, seal2001arm}. Usually this stream of data is treated by programmers in hexadecimal format, which is then remapped into binary code. The programs written in machine code were once loaded into the processor through a front panel, a controller that allowed the display and alteration of the registers and memory (see Figure \ref{fig:ch1_front_panel}). An example of machine code for a program that computes the sum of a sequence of integer numbers can be seen in Listing \ref{lst:ch1_machine_code}.

\begin{figure}
	\centering
	\includegraphics[width = \textwidth]{Figures/ch1_front_panel}
	\caption{Front panel of IBM 1620}
	\label{fig:ch1_front_panel}
\end{figure}

\begin{minipage}{\linewidth}
\begin{lstlisting}[numbers = left, caption = Machine code to compute the sum of a sequence of numbers, label = lst:ch1_machine_code]
 00075	c7 45 b8 00 00
 00 00
 0007c	eb 09	
 0007e	8b 45 b8
 00081	83 c0 01
 00084	89 45 b8
 00087	83 7d b8 0a
 0008b	7d 0f
 0008d	8b 45 b8
 00090	8b 4d c4
 00093	03 4c 85 d0
 00097	89 4d c4
 0009a	eb e2
\end{lstlisting}
\end{minipage}

\subsubsection*{Second-generation languages}
The languages in this category provides an abstraction layer over the machine code by expressing processor instructions with mnemonic names both for the instruction code and the arguments. For example, the arithmetic sum instruction \texttt{add} is the mnemonic name for the instruction code \texttt{0x00} in \texttt{x86} processors. Among these languages we find \textit{Assembly}, that is mapped with an \textit{Assembler} to machine code. The Assembler can load directly the code or link different \textit{object files} to generate a single executable by using a \textit{linker}. An example of assembly \texttt{x86} code corresponding to the machine code in Listing \ref{lst:ch1_machine_code} can be found in Listing \ref{lst:ch1_assembly_code}. You can see that the code in the machine code \texttt{00081	83 c0 01} at line 5 has been replaced by its mnemonic representation in Assembly as \texttt{add	eax, 1}.

\begin{minipage}{\linewidth}
\begin{lstlisting}[numbers = left, caption = Assembly x86 code to compute the sum of a sequence of numbers, label = lst:ch1_assembly_code]
mov	DWORD PTR _i$1[ebp], 0
jmp	SHORT $LN4@main
$LN2@main:
mov	eax, DWORD PTR _i$1[ebp]
add	eax, 1
mov	DWORD PTR _i$1[ebp], eax
$LN4@main:
cmp	DWORD PTR _i$1[ebp], 10			; 0000000aH
jge	SHORT $LN3@main
mov	eax, DWORD PTR _i$1[ebp]
mov	ecx, DWORD PTR _sum$[ebp]
add	ecx, DWORD PTR _numbers$[ebp+eax*4]
mov	DWORD PTR _sum$[ebp], ecx
jmp	SHORT $LN2@main
\end{lstlisting}
\end{minipage}

\subsubsection*{Advantages and disadvantages}
Writing a program in low-level programming languages might produce programs that are generally more efficient than their high-level counterparts, as ad-hoc optimizations are possible. However, the high-performance comes at great costs: (\textit{i}) the programmer must be an expert of the underlying architecture and of the specific instruction set of the processor, (\textit{ii}) the program loses portability because the low-level code is tightly bound to the specific hardware architecture it targets, (\textit{iii}) the logic and readability of the program is hidden among the details of the instruction set itself, and (\textit{iv}) developing a program in assembly requires a considerable effort in terms of time and debugging \cite{frampton2009demystifying}: assembly lacks any abstraction from the concrete hardware architecture, such as a type system, that partially ensures the correctness of the program or high-level constructs that allow to manipulate the execution of the program.

\subsection{High-level programming languages}
\label{subsec:ch1_hl_languages}
A high-level programming language is a programming language that offers a high level of abstraction from the specific hardware architecture of the machine. Unlike machine code (and in some way also assembly), high-level languages are not directly executable by the processor and they require some kind of translation process into machine code. The level of abstraction offered by the language defines how high level the language is. Several categories of high-level programming language exist, but the main one are described below.

\subsubsection*{Imperative programming languages}
\textit{Imperative programming languages} model the computation as a sequence of statements that alter the state of the program (usually the memory state). A program in such languages consists then of a sequence of \textit{commands}. Notable examples are FORTRAN, C, and PASCAL. An example of the program used in Listing \ref{lst:ch1_machine_code} and \ref{lst:ch1_assembly_code} written in C can be seen in Listing \ref{lst:ch1_c_code}. Line 5 to 9 corresponds to the Assembly code in Listing \ref{lst:ch1_assembly_code}.

\begin{lstlisting}[numbers = left, caption = C code to compute the sum of a sequence of numbers, label = lst:ch1_c_code]
int main()
{
  int numbers[10] = { 1, 6, 8, -2, 4, 3, 0, 1, 10, -5 };
  int sum = 0;
  for (int i = 0; i < 10; i++)
  {
    sum += numbers[i];
  }
  printf("%d\n", sum);
}
\end{lstlisting}

\subsection*{Declarative programming languages}
\textit{Declarative programming languages} are antithetical to those based on imperative programming, as they model computation as an evaluation of expressions and not as a sequence of commands to execute. Declarative programming languages are called as such when they are side-effects free or referentially transparent. The definition of referential transparency varies \cite{quine2013word}, but it is usually explained with the substitution principle, which states that a language is referentially transparent if any expression can be replaced by its value without altering the behaviour of the program \cite{mitchell2003concepts}. For instance, the following sentences in natural language are both true

\begin{lstlisting}
Cicero = Tullius

''Cicero`` contains six letters
\end{lstlisting} 

\noindent
but they are not referentially transparent, since replacing the last name with the middle name falsifies the second sentence.

A similar situation in programming languages is met when considering variable assignments: the statement

\begin{lstlisting}
x = x + 5
\end{lstlisting}

\noindent
is not referentially transparent. Let us assume this statement appears twice in a program and that at the beginning x = 0. Clearly the expression \texttt{x + 5} results in the value 5 the first time, but the second time the same statement is executed the expression has value 10. Thus replacing all the occurrences of \texttt{x + 5} with 5 is wrong, which is why imperative languages are not referentially transparent. A more rigorous definition of referential transparency can be found in \cite{sondergaard1990referential}.

Declarative programming languages are often compared to imperative programming languages by stating that declarative programming defines \textit{what} to compute and not \textit{how} to compute it. This family of languages include \textit{functional programming}, \textit{logic programming}, and \textit{database query languages}. Notable examples are F\#, Haskell, Prolog, SQL, and Linq (which is a query language embedded in C\#). Listing \ref{lst:ch1_fsharp_code_rec} shows the code to perform the sum of a sequence of integer numbers in F\# with a recursive function. Higher-order functions, such as \texttt{fold}, allow even to capture the same recursive pattern into a single function as shown in Listing \ref{lst:ch1_fsharp_code_fold}. Both implementations are referentially transparent.

\begin{lstlisting}[caption = Recursive F\# code to compute the sum of a sequence of numbers, label = lst:ch1_fsharp_code_rec]
let rec sumList l =
  match l with
  | [] -> 0
  | x :: xs -> x + (sumList xs)
\end{lstlisting}

\begin{lstlisting}[caption = F\# code to compute the sum of a sequence of numbers using higher-order functions, label = lst:ch1_fsharp_code_fold]
let sumList l = l |> List.fold (+) 0
\end{lstlisting}

\subsection{General-purpose vs Domain-specific languages}
\label{sec:ch1_dsl}
\textit{General-purpose languages} are defined as languages that can be used across different application domains and lack abstractions that specifically target elements of a single domain. Example of these are languages such as C, C++, C\#, and Java. Although several applications are still being developed by using general-purpose programming languages, in several contexts it is more convenient to rely on \textit{domain-specific languages}, because they offer abstractions relative to the problem domain that are unavailable in general-purpose languages \cite{van2000domain, voelter2013dsl}. Notable examples of the use of domain-specific languages are listed below.

\subsubsection*{Graphics programming}
Rendering a scene in a 3D space is often performed by relying on dedicated hardware. Modern graphics processors rely on shaders to create various effects that are rendered in the 3D scene. Shaders are written in domain-specific languages, such as GLSL or HLSL \cite{glhl2014, hlsl2018, hlslref2018}, that offer abstractions to compute operations at GPU level that are often used in computer graphics, such as vertices and pixel transformations, matrix multiplications, and interpolation of textures. Listing \ref{lst:ch1_hlsl_code} shows the code to implement light reflections in HLSL. At line 4 you can, for example, see the use of matrix multiplication provided as a language abstraction in HLSL.

\begin{lstlisting}[numbers = left, caption = HLSL code to compute the light reflection, label = lst:ch1_hlsl_code]
VertexShaderOutput VertexShaderSpecularFunction(VertexShaderInput input, float3 Normal : NORMAL)
{
  VertexShaderOutput output;
  float4 worldPosition = mul(input.Position, World);
  float4 viewPosition = mul(worldPosition, View);
  output.Position = mul(viewPosition, Projection);
  float3 normal = normalize(mul(Normal, World));
  output.Normal = normal;
  output.View = normalize(float4(EyePosition,1.0f) - worldPosition);
  return output;
}
\end{lstlisting}

\subsubsection*{Game programming}
Computer games are a field where domain-specific languages are widely employed, as they contain complex behaviours that often require special constructs to model timing event-based primitives, or to execute tasks in parallel. These behaviours cannot be modelled, for performance reasons, by using threads. Therefore, in the past, domain-specific languages which provide these abstractions have been implemented \cite{nwnlexicon2018, jass2011, unrealscript2018, sqf2018}. In Listing \ref{lst:ch1_sqf_code} an example of the SQF domain-specific language for the game ArmA2 is shown. This language offers abstractions to wait for a specific amount of time, to wait for a condition, and to spawn scripts that run in parallel to the callee, that you can respectively see at lines 18, 12, and 10.

\begin{lstlisting}[numbers = left, caption = ArmA 2 scripting language, label = lst:ch1_sqf_code]
"colorCorrections" ppEffectAdjust [1, pi, 0, [0.0, 0.0, 0.0, 0.0], [0.05, 0.18, 0.45, 0.5], [0.5, 0.5, 0.5, 0.0]];  
"colorCorrections" ppEffectCommit 0;  
"colorCorrections" ppEffectEnable true;

thanatos switchMove "AmovPpneMstpSrasWrflDnon";
[[],(position tower) nearestObject 6540,[["USMC_Soldier",west]],4,true,[]] execVM "patrolBuilding.sqf";
playMusic "Intro";

titleCut ["", "BLACK FADED", 999];
[] Spawn 
{
	waitUntil{!(isNil "BIS_fnc_init")};
	[
	  localize "STR_TITLE_LOCATION" ,
	  localize "STR_TITLE_PERSON",
	  str(date select 1) + "." + str(date select 2) + "." + str(date select 0)
	] spawn BIS_fnc_infoText;
	sleep 3;
	"dynamicBlur" ppEffectEnable true;   
	"dynamicBlur" ppEffectAdjust [6];   
	"dynamicBlur" ppEffectCommit 0;     
	"dynamicBlur" ppEffectAdjust [0.0];  
	"dynamicBlur" ppEffectCommit 7;
	titleCut ["", "BLACK IN", 5];
};
\end{lstlisting}

\subsubsection*{Shell scripting languages}
Shell scripting languages, such as the \textit{Unix Shell script}, are used to manipulate files or user input in different ways. They generally offer abstractions to the operating system interface in the form of dedicated commands. Listing \ref{lst:ch1_shell_code} shows an example of a program written in Unix shell script to convert an image from JPG to PNG format. At line 3 you can see the use of the statement \texttt{echo} to display a message in the standard output.

\begin{lstlisting}[numbers = left, caption = Unix shell code, label = lst:ch1_shell_code]
for jpg; do                                  
  png="${jpg%.jpg}.png"                    
  echo converting "$jpg" ...               
  if convert "$jpg" jpg.to.png ; then      
    mv jpg.to.png "$png"                 
  else                                     
    echo 'jpg2png: error: failed output saved in "jpg.to.png".' >&2
    exit 1
  fi                                       
done                                         
echo all conversions successful              
exit 0
\end{lstlisting}

\subsubsection*{Advantages and disadvantages}
High-level programming languages offer a variety of abstractions over the specific hardware the program targets. The obvious advantage of this is that the programmer does not need to be an expert of the underlying hardware architecture or instruction set. A further advantage is that the available abstractions are closer to the semi-formal description of the underlying algorithm as pseudo-code. This produces two desirable effects: (\textit{i}) the readability of the program is increased as the available abstractions are closer to the natural language than the equivalent machine code, and (\textit{ii}) that being able to mimic the semi-formal version of an algorithm, which is generally how the algorithm is presented and on which its correctness is proven, grants a higher degree of correctness in the specific implementation.

The use of a high-level programming language might, in general, not achieve the same high-performance as writing the same program with a low-level programming language  \cite{chatzigeorgiou2002evaluating}, but modern code-generation optimization techniques can generally mitigate this gap \cite{amarasinghe1993communication, wang2007code}. A further major issue in using high-level programming languages is that the machine cannot directly execute the code, thus the use of a compiler that translates the high-level program into machine code is necessary.

The portability of a high-level programming language depends on the architecture of the underlying compiler, thus some languages are portable and the same code can be run on different machines (for example Java), while others might require to be compiled to target a specific architecture (for example C++).

\section{Compilers}
\label{sec:ch1_compilers}
A compiler is a program that transforms source code defined in a programming language into another computer language, which usually is object code but can also be code in a high-level programming language \cite{aho2007compilers, appel2002javacompiler}. Writing a compiler is a necessary step to implementing a high-level programming language. Indeed, a high-level programming language, unlike low-level ones, are not executable directly by the processor and need to be translated into machine code, as stated in Section \ref{subsec:ch1_ll_languages} and \ref{subsec:ch1_hl_languages}.

The first complete compiler was developed by IBM for the FORTRAN language and required 18 person-years for its development \cite{backus1957fortran}. This clearly shows that writing a compiler is a hard and time-consuming task.

A compiler is a complex piece of software made of several components that implement a step in the translation process. The translation process performed by a compiler involves the following steps:

\begin{enumerate}
	\item \textit{syntactical analysis:} In this phase the compiler checks that the program is written according to the grammar rules of the language. In this phase the compiler must be able to recognize the \textit{syntagms} of the language (the ``words'') and also check if the program conforms to the syntax rules of the language through a grammar specification.
	\item \textit{type checking:} In this phase the compiler checks that a \textit{syntactically correct program} performs operations conform to a defined \textit{type system}. A type system is a set of rules that assign properties called types to the constructs of a computer program \cite{pierce2002types}. The use of a type system drastically reduces the chance of having bugs in a computer program \cite{cardelli1996type} . This phase can be performed at compile time (\textit{static typing}) or the generated code could contain the code to perform the type checking at runtime (\textit{dynamic typing}). 
	\item \textit{code generation:} In this phase the compiler takes the \textit{syntactically and type-correct program} and performs the translation step. At this point an equivalent program in a target language will be generated. The target language can be object code, another high-level programming language, or even a bytecode that can be interpreted by a virtual machine.
\end{enumerate}

All the previous steps are always the same regardless of the language the compiler translates from and they are not part of the creative aspect of the language design \cite{book1970cwic}. Approaches to automating the construction of the syntactical analyser are well known in literature \cite{mcpeak2004elkhound, nivre2006maltparser, parr1995antlr}, to the point that several lexer/parser generators are available for programmers, for example all those belonging to the \texttt{yacc} family such as \texttt{yacc} for C/C++, \texttt{fsyacc} for F\#, \texttt{cup} for Java, and \texttt{Happy} for Haskell. On the other hand, developers lack a set of tools to automate the implementation of the last two steps, namely the type checking and the code generation.

For this reason, when implementing a compiler, the formal type system definition and the operational semantics, which is tightly connected to the code generation and defines how the constructs of the language behave, must be translated into the abstractions provided by the host language in which the compiler will be implemented. Other than being a time-consuming activity itself, this causes that (\textit{i}) the logic of the type system and operational semantics is lost inside the abstraction of the host-language, and (\textit{ii}) it is difficult to extend the language with new features.

\section{Meta-compilers}
\label{sec:ch1_metacompilers}
In Section \ref{sec:ch1_compilers} we described how the steps involved in designing and implementing a compiler do not require creativity and are always the same, regardless of the language the compiler is built for. The first step, namely the syntactical analysis, can be automated by using one of the several lexer/parser generators available, but the implementation of a type checker and a code generator still relies on a manual implementation. This is where meta-compilers come into play: a meta-compiler is a program that takes the source code of another program written in a specific language and the language definition itself, and generates executable code. The language definition is written in a programming language, referred to as \textit{meta-language}, which should provide the abstractions necessary to define the syntax, type system, and operational semantics of the language, in order to implement all the steps above.

\subsection{Requirements}
As stated in Section \ref{sec:ch1_metacompilers}, a meta-compiler should provide a meta-language that is able to define the syntax, type system, and operational semantics of a programming language. In Section \ref{sec:ch1_compilers} we discussed how methods to automate the implementation of syntactical analyser are already known in scientific literature. For this reason, in this work, we will focus exclusively on automating the implementation of the type system and of the operational semantics. Given this focus, we formulate the following requirements:

\begin{itemize}
	\item The meta-language should provide abstractions to define the constructs of the language. This includes the possibility of defining control structures, operators with any form of prefix or infix notation, and the priority of the constructs that is used when evaluating their behaviour. Furthermore, it must be possible to define the equivalence of language constructs. For instance, an integer constant might be considered both a value and a basic arithmetic expression.
	
	\item The meta-language must be able to mimic as close as possible the formal definition of a programming language. This will bring the following benefits: (\textit{i}) Implementing the language in the meta-compiler will just involve re-writing almost one-to-one the type system or the semantics of the language with little or no change; (\textit{ii}) the correctness and soundness \cite{cardelli1996type, milner1972proving} of the language formal definition will be directly reflected in the implementation of the language; indeed if a meta-program allows to mimic directly the type system and semantics of the language their correctness is transferred also in the implementation, while this might not be trivial when translating them in the abstractions of a high-level programming language; (\textit{iii}) any extension of the language definition can be just added as an additional rule in the type system or the semantics.
	
	\item The meta-compiler must be able to embed libraries from external languages, so that they can be used to implement specific behaviours such as networking transmission or specific data structure usage.
\end{itemize}

\subsection{Benefits}
\label{sec:ch1_benefits}
%list the benefits first and then explain. Add a paragraph also about correctness
Programming languages usually are released with a minimal (but sufficient to be Turing-complete) set of features, and later extended in functionality in successive versions. This process tends to be slow and often significant improvements or additions are only seen years after the first release. For example, Java was released in 1996 and lacked an important feature such as Generics until 2004, when J2SE 5.0 was released. Furthermore, Java and C++ lacked constructs from functional programming, which is becoming more and more popular with the years \cite{thompson1995miranda}, such as lambda abstractions until 2016, while a similar language like C\# 3.0 was released with such capability in 2008. The slow rate of change of programming languages is due to the fact that every abstraction added to the language must be reflected in all the modules of its compiler: the grammar must be extended to support new syntactical rules, the type checking of the new constructs must be added, and the appropriate code generation must be implemented. Given the complexity of compilers, this process requires a huge amount of work, and it is often obstructed by the low flexiblity of the compiler as piece of software. Using a meta-compiler would speed up the extension of an existing language because it would require only to change on paper the type system and the operational semantics, and then add the new definitions to their counterpart written in the meta-language. This process is easier because the meta-language should mimic as close as possible their behaviour. Moreover, backward compatibility is automatically granted because an older program will simply use the extended language version to be compiled by the meta-compiler.

To this we add the fact that, in general, for the same reasons, the development of a new programming language is generally faster when using a meta-compiler. This could be beneficial to the development of a high variety of domain-specific languages. Indeed, such languages are often employed in situations where the developers have little or no resources to develop a fully-fledged hard-coded compiler by hand. For instance, it is desirable for game developers to focus on aspects that are strictly tied to the game itself, for example the development of an efficient graphics engine or to improve the game logic. At the same time they would need a domain-specific language to express some behaviours typical of games, things that could be achieved by using a meta-compiler rather than on a hand-made implementation.

\subsection{Scientific relevance} %change the tile
\label{sec:ch1_scientific_relevance}
Meta-compilers have been researched since the 1960's \cite{schorre1964meta} and several implementations have been proposed \cite{ braborovansky1998overview, venboer2008stratego, klint2009rascal, pettersson1996compiler, verdejo2006executable}. In general meta-compilers perform poorly compared to hard-coded compilers because they add the additional layer of abstraction of the meta-language. Moreover, a specific implementation of a compiler opens up the possibility of implementing language-specific optimizations during the code generation phase. Meta-compilers have been used in a wide range of applications, such as source code analysis and manipulation and physical simulations \cite{kaagedal1998generating}, but no use up to our knowledge was made in the field of domain-specific languages for games. Since games are pieces of software that are very demanding in terms of performance, we think that it could be of interest to investigate the applicability of meta-compilers in the scope of domain-specific languages for games and the development speed up introduced by the use of such a tool. In this work we present Metacasanova, a meta-compiler based on natural semantics that was born from the intent of easing the development of the domain-specific language for game development Casanova, and we analyse the benefit of using it for a re-implementation and extension of Casanova.

\section{Problem statement}
\label{sec:ch1_problem_statement}
In Section \ref{sec:ch1_programming_languages} we showed the advantages of using high-level programming languages when implementing an algorithm. Among such languages, it is sometimes desirable to employ domain-specific languages that offer abstractions relative to a specific application domain (Section \ref{sec:ch1_dsl}). In Section \ref{sec:ch1_compilers} we described the need of a compiler for such languages, and that developing one is a time-consuming activity despite the process being, in great part, non-creative. In Section \ref{sec:ch1_metacompilers} we introduced the role of meta-compilers to speed up the process of developing a compiler and we listed the requirements and the benefits that one should have. In Section \ref{sec:ch1_scientific_relevance} we explained why we believe that meta-compilers are a relevant scientific topic if coupled with the problem of of developing domain-specific languages in response to the their increasing need. We can now formulate our problem statement:

\vspace{0.5cm}
\noindent
\textbf{Problem statement: } \textit{\psContent}

\vspace{0.5cm}
\noindent
The first parameter we need to evaluate in order to answer this question is the size of the code reduction needed to implement the domain-specific language. At this purpose, the following research question arises:

\vspace{0.5cm}
\noindent
\textbf{Research question 1: } \textit{\rqContentOne}

\vspace{0.5cm}
\noindent
The second parameter we need to evaluate is the eventual performance loss caused by introducing the abstraction layer provided by the meta-compiler. This leads to the following research question:

\vspace{0.5cm}
\noindent
\textbf{Research question 2: } \textit{\rqContentTwo}

\vspace{0.5cm}
\noindent
In case of a performance loss, we need to identify the cause of this performance loss and if an improvement is possible. This leads to the following research question:

\vspace{0.5cm}
\noindent
\textbf{Research question 3: } \textit{\rqContentThree}

\vspace{0.5cm}
\noindent

\section{Thesis structure}
This thesis describes the architecture of Metacasanova, a meta-compiler whose meta-language is based on operational semantics, and a possible optimization for such meta-compiler. It also shows its the capabilities by implementing a small imperative language and re-implementing the existing domain-specific language for games \textit{Casanova 2}, extending it with abstractions to express network operations for multiplayer games.

In Chapter \ref{ch:background} we provide background information in order to understand the choices made for this work. The chapter presents the state of the art in designing and implementing compilers and existing research on meta-compilers.

In Chapter \ref{ch:metacasanova} we present the architecture of Metacasanova by extensively describing the implementation of all its modules.

In Chapter \ref{ch:languages} we show how to use Metacasanova to implement two languages: a small imperative language, and \textit{Casanova 2}, a language for game development. At the end of the chapter we provide an evaluation of the performance of the two languages and their implementation length with respect to existing compilers, thus answering to Research Question 1 and 2.

In Chapter \ref{ch:functors} we discuss the performance loss of the implementation of the presented languages and we propose an extension of Metacasanova that aims to improve the performance of the generated code, thus answering Research Question 3.

In Chapter \ref{ch:functor_languages} we show how to use functors to improve the performance of Casanova implemented in Metacasanova, comparing this approach and the one presented in Chapter \ref{ch:languages} with respect to the execution time of a sample in Casanova. 

In Chapter \ref{ch:networking} we propose an extension of Casanova 2 for multiplayer game development. We first provide its hard-coded compiler solution and then we show how to extend the implementation in the meta-compiler to include the same extension. In this chapter we evaluate the performance of a multiplayer game implemented in Casanova with this extension with respect to the same game implemented in C\#, and we measure the effort of realising such extension in the hard-coded compiler of Casanova versus the implementation with Metacasanova.

In Chapter \ref{ch:discussion} we discuss the result and answer the research questions.

\section{Challenges of multiplayer games}
Within the context of higher education, learning programming is hard for both beginners and students with past experience. 

It is suggested by some \cite{tan2009learning} that learning programming is no different than most other complex skills: it takes roughly ten years (ten thousand hours) to become truly proficient.

The reason why it takes so long is disarmingly simple. Programming requires both the ability to \textit{understand} and to \textit{design} code. 

\subsection{Understanding code}
Understanding code is a passive skill, but not any simpler because of it. The true meaning of code is the sequence of steps that the machine will actually perform: every single bit that will be read and written as a result of an instruction is part of the meaning of that instruction, just like every cache hit-or-miss, the activation of the CPU ALU(s), network channels, operating systems, interpreters, just-in-time compilers, and ultimately interactions with users. Being able to figure precisely what a program does, and how it does it (also in terms of performance) requires the ability to formulate an abstract idea of the program behaviour, and the mapping of this \textit{abstract idea} to the concrete components when more specific reasoning is needed.

The sheer size of the machinery involved in the execution of even the simplest program is simply \textit{very large}, and \textit{the ability to think hierarchically and zoom in and out of the details as needed takes a lot of experience}.

\subsection{Designing code}
Designing code is an active skill, and as such intrinsically complex. Designing code requires the formulation (and therefore the choice) of a design strategy, usually in a top-down fashion, which is then recursively turned into a more and more concrete definition of the program. Being able to design a program effectively requires the ability to choose a specific design among a series of possible designs, which taken together form \textit{the abstract meta-strategies} that characterise a programmer’s knowledge, style, and experience.

The sheer size of the design space of even the simples program is \textit{so large as to be essentially infinite} (we are talking about finite machines after all!), and the ability to \textit{formulate meta-strategies and employ them recursively takes a lot of experience}.

\subsection{Size matters (and so does structure)}
We believe the size of the domain to be the core of the issue. Even though some students might already know a few tricks to produce working programs in some very narrow domain, the fundamental ability to abstractly reason about code (both for understanding and designing programs) is usually severely lacking in first year students.

Moreover, we cannot just solve the problem by throwing unstructured assignments such as “read this code” or “try and write this program”, as we must train the specific mental activities that we wish to stimulate in students. Specific training must be structured in order to gently guide the activation of the proper thought structures, in a slow buildup of complexity and freedom to express one’s own creativity.

\subsection{The issues}
We close this session by identifying a series of practical, concrete issues that we believe sum up the discussion so far:

\begin{table}[!h]
	\begin{tabular}{|c|p{8cm}|}
		\hline
		\textbf{ID} & \textbf{Issue} \\
		\hline
		REASON\textunderscore MODEL & Students need extensive practice with reasoning about models of programs \\
		\hline
		REASON\textunderscore DESIGN & Students need extensive practice with reasoning about existing design strategies \\
		\hline
		EXTEND\textunderscore DESIGN & Students need extensive practice with extending existing programs (which should follow a formative design). \\
		\hline
		EXTEND\textunderscore BUILDUP & Students need a buildup in complexity with their extension activities. \\
		\hline
	\end{tabular}
	\caption{Issues about learning programming}
	\label{tab:issues}
\end{table}

By keeping these issues in mind, we will now setup a proposed didactic model which we believe can mitigate them or outright resolve them.





\section{Networking architecture and semantics}
\begin{comment}
\begin{mathpar}
\tiny
\inferrule
{E \vdash t \; : \; \mathtt{float}}
{E \vdash \mathtt{wait} \; t \; : \; \mathtt{void}}

\inferrule
{E \vdash c \; : \; \mathtt{bool}}
{E \vdash \mathtt{when} \; c : \; \mathtt{void}}
\end{mathpar}
\end{comment}

In this section we give an overview of Casanova 2, a declarative domain-specific language oriented towards game development. In this section we just give a brief informal overview of the language; for further technical details we point the reader to \cite{AbbadiThesis2017,Abbadi2015}. We then present the language extension to include networking abstractions, which prototype was presented in \cite{DiGiacomo201725} and hard-coded in its compiler, and we finally present its formal semantics.

\subsection{Overview of Casanova 2}
Casanova 2 is a declarative language oriented to game development. A program in Casanova 2 consists of a set of entities that define the dynamic elements of the game. These range from elements which interact or are interacted with by the player, such as characters, weapons, or interactive scene elements, to elements not directly operable such as bullets or non-playable characters. The entities are organized into a tree structure, at which root you find a special entity called \textit{World}. Each entity defines a set of fields, in the fashion of a class, and a set of rules that define its temporal evolution. Unlike object-oriented programming, entity fields are not directly modifiable through variable assignments (although they can always be read), but only through rules: each rule takes as input a domain, which is a subset of fields that it can modify, and they can be updated only by calling a dedicated statement called \texttt{yield}. If a rule tries to update a field outside its domain, this is captured at type-system level. Moreover, each rule works in the fashion of threads (although the implementation in the back-end of the compiler is completely different, as it makes use of state machines rather than a scheduler), meaning that they can be interrupted by using built-in abstractions in the language, such as \texttt{wait} statements, or the \texttt{yield} itself, which stops the rule execution by one game frame. Finally, each rule takes as input a reference to the current entity (\texttt{this}) and a parameter \texttt{dt} which contains the elapsed time between the current game frame and the previous update.

In the next section we present a language extension for Casanova 2 in order to include abstractions to express networking synchronization.

\subsection{Networking architecture}
The networking architecture that we present is peer-to-peer based. This means that we do not have a centralized authoritative server that constantly validates the local versions of the game state on the clients, but rather each client is tasked with maintaining and validating a portion of the game state.

In this scenario, we identified the following abstractions that must be provided by the language:

\begin{itemize}
	\item \textit{Sending and receiving data:} the developer should be able to specify how and what data to send and receive to and from other clients.
	\item \textit{Connection:} the developer should be able to specify what happens when a new player connects to the game and how the existing players react to this event.
	\item \textit{Managing the local and remote game state:} the developer should be able to define how local and remote entities evolve on each client.
\end{itemize}

Below we present the extensions implementing such abstractions.

\subsection{Data transmission}

The language extension should include primitives to send data. These primitives should allow the programmer to choose whether to send data in a reliable (meaning that the receiver will be sure to receive the data at some point) or an unreliable way (meaning that the data could be lost because of packets losses). To this end, we define the primitives (\texttt{send} and \texttt{send\textunderscore reliable}. The programmer should be provided with the means of receiving data, which are given by the primitives \texttt{receive} and \texttt{receive\textunderscore many}: the former receive a single entity or atomic value, the second is able to receive a list of entities or values. The \texttt{receive} operations can be used in combination with the \texttt{yield} statement to simultaneously receive the data and update an entity field. Furthermore, we will be needing of a language primitive that waits until a received value is available and then binds it to a variable, called \texttt{let!} to distinguish it from a normal \texttt{let} binding. The following snippet illustrates an example using some of these primitives (the \texttt{@} operator denotes lists concatenation):

\begin{lstlisting}
world Shooter = {
...
rule master Ships =
  let! receivedShips = receive_many
  yield Ships @ receivedShips
}
\end{lstlisting}

\subsubsection{Handling connections}
When a new client connects to the game, it instantiates a local copy of the game state. This game state is incomplete, meaning that it does not contain an information about the game state views of the other clients, nor the other clients know anything about the local view of the new client. At this purpose, we extend Casanova 2 rules with two clauses: \texttt{connecting} and \texttt{connected}. A rule defined as \texttt{connecting} is executed only once when a new client connects to the game. On the other hand, a rule marked as \texttt{connected} is executed by all clients when a new client connects. An example of this is shown below: a client connects to the game and instantiates a ship locally, which is sent to the other clients. At the same time, the rule defined as \texttt{connected} is executed by the existing clients which sends their ships. The data transmission primitives will be discussed further below.

\begin{lstlisting}
world Shooter = {
  ...
  rule connecting Ships =
    yield send_reliable Ships
    
  rule connected Ships =
    yield send_reliable Ships
}
\end{lstlisting}

\noindent
Note that, even if the two rules seem to be identical, the semantics is completely different: the first is run by a connecting client and sends the only local ship instantiated when connecting to the game to all the existing clients, the second is executed by all the other clients, each one sending a list with the existing Ships.

\subsection{Local and remote entities}
To handle local and remote portions of the game state we must divide accordingly the entities in two sets: those instantiated by the current client, which are under its direct control, and those instantiated by other clients and under their control. To further illustrate this situation, consider a simple shooter game where every client controls a ship that can shoot the other clients' ships. The client can use input devices (such as mouse and keyboard or a gamepad) to perform an action with its ship, but it should be unable to control the other clients' ships. At this purpose the client should send the updates performed locally on the ship it controls, and receive the updates performed at the same time on the other clients for all the other ships. Entities instantiated locally by a client are defined as \texttt{master} entities, while entities that were instantiated on remote clients and which copy is replicated in the current client are defined as \texttt{slave}. We extend the rules with this two additional identifiers \texttt{master} and \texttt{slave}. Rules defined as \texttt{master} are run only for instances of entities that were locally instantiated by a client, while rules defined as \texttt{slave} are run only for instances of entities that were instantiated remotely. Note that rules that are not marked either as \texttt{master} or \texttt{slave} are always run independently on the entity being instantiated locally or remotely. This could be helpful to perform dynamics that do not need a synchronization with other clients, such as reacting to an interaction with a GUI element. An example of rules defining the dynamics for local entities is shown below:

\begin{lstlisting}
entity Ship = {
   ...
   rule master Position =
     wait world.Input.KeyDown(Keys.W)
     let vp = new Vector2(Math.Cos(Rotation),Math.Sin(Rotation)) * 300.0f
     let p = Position + vp * dt
     yield send p     
}
\end{lstlisting}

\subsection{Formal semantics}
In what follows we present the operational semantics of the networking language extension that was introduced above. In order to keep consistency with how Casanova semantics was defined \cite{Abbadi2015}, we use a rewrite-based definition \cite{klop1992term}, where the world is transformed into a new one containing an updated version of its fields and rules.

The semantics of Casanova 2 must be extended to reflect the fact that, in a networking scenario, there is a \textit{Network State} (see Section \ref{sec:preliminaries}) consisting of local approximated views of the global state. At this purpose we define this state as a set $N_{s} = \lbrace S_{c_{i}} | c_{i} \in C \rbrace$ where $C$ is the set of clients connected to the game and $S_{c_{i}}$ is the local game state for client $c_{i}$. Each local state is a pair $S_{c_{i}} = (E_{L},E_{R})$ of local (instantiated on the client $c_{i}$) and remote (instantiated on a different client $c_{j}, \; j \neq i$) entities. Each entity $e \; | \; e \in E_{L} \vee e \in E_{R}$ is a set of fields and rules:

\begin{lstlisting}[mathescape = true]
E = {$Field_{1} = f_{1}, ..., Field_{n} = f_{n}$
     $Rule_{1} = r_{1}, ..., Rule_{m} = r_{m}$} 
\end{lstlisting}

In order to include the definition of \texttt{master}, \texttt{slave}, \texttt{connecting}, and \texttt{connected} rules, we consider their respective sets $R_{m}$, $R_{s}$, $R_{c}$, and $R_{c}'$.

\subsubsection{Entity update in a networking scenario}

The update of the game state in Casanova 2 is realised by using a recursive function \texttt{tick} that recursively updates all the fields of an entity and, at the same time, executes all its rules through another function \texttt{step}.

\begin{lstlisting}[mathescape]
tick(e:E, dt) =
{ Field$_1$=tick(f$_1^m$, dt); $\dots$; Field$_n$=tick(f$_n^m$, dt);
  Rule$_1$=r$_1'$; $\dots$; Rule$_m$=r$_m'$ }
  where
   f$_1^m$, $\dots$, f$_n^m$, r$_m'$ = step(f$_1^{m-1}$, $\dots$, f$_n^{m-1}$, r$_m$)
   .
   .
   f$_1^1$, $\dots$, f$_n^1$, r$_1'$ = step(f$_1$, $\dots$, f$_n$, r$_1$)}
\end{lstlisting}

In what follows we will refer to \texttt{tick} and \texttt{step} as defined above, where  the \texttt{tick} function is extended with an additional argument \texttt{c} denoting that it is being performed by client \texttt{c}. For further details see \cite{Abbadi2015}.

\paragraph{Connecting rule semantics}
As explained above, a connecting rule is executed once on a client connecting to the game for the first time. Thus, if the client does not yet exist among the clients participating to the game session the rule is executed

\begin{lstlisting}[mathescape]
tick(e:E, dt, c) =
{ Field$_1$=tick(f$_1^m$, dt); $\dots$; Field$_n$=tick(f$_n^m$, dt);
  Rule$_1$=r$_1'$; $\dots$; Rule$_j$ = r$_j'$; $\dots$; Rule$_m$=r$_m'$ }
  where
    r$_j$ $\in$ $R_{c}$, c $\notin C$
    f$_1^m$, $\dots$, f$_n^m$, r$_m'$ = step(f$_1^{m-1}$, $\dots$, f$_n^{m-1}$, r$_m$)
    .
    .
    f$_1^j$, $\dots$, f$_n^j$, r$_j'$ = step(f$_1^{j-1}$, $\dots$, f$_n^{j-1}$, r$_j$)
    .
    .
    f$_1^1$, $\dots$, f$_n^1$, r$_1'$ = step(f$_1$, $\dots$, f$_n$, r$_1$)
\end{lstlisting}

\noindent
On the other hand, if the client already exists among the current clients, the rule has no effect on the entity.

\begin{lstlisting}[mathescape]
tick(e:E, dt, c) =
{ Field$_1$=tick(f$_1^m$, dt); $\dots$; Field$_n$=tick(f$_n^m$, dt);
  Rule$_1$=r$_1'$; $\dots$; Rule$_j$ = r$_j$; $\dots$; Rule$_m$=r$_m'$ }
  where
    r$_j$ $\in$ $R_{c}$, c $\in C$
    f$_1^m$, $\dots$, f$_n^m$, r$_m'$ = step(f$_1^{m-1}$, $\dots$, f$_n^{m-1}$, r$_m$)
    .
    .
    f$_1^{j-1}$, $\dots$, f$_n^{j-1}$,r$_j$ = step(f$_1^{j-1}$, $\dots$, f$_n^{j-1}$, r$_j$)
    .
    .
    f$_1^1$, $\dots$, f$_n^1$, r$_1'$ = step(f$_1$, $\dots$, f$_n$, r$_1$)
\end{lstlisting}

\paragraph{Connected rule semantics}
A connected rule is executed when a new client connects to the game, so when there exists a client that does not belong to the set of connected clients.

\begin{lstlisting}[mathescape]
tick(e:E, dt, c) =
{ Field$_1$=tick(f$_1^m$, dt); $\dots$; Field$_n$=tick(f$_n^m$, dt);
  Rule$_1$=r$_1'$; $\dots$; Rule$_j$ = r$_j'$; $\dots$; Rule$_m$=r$_m'$ }
  where
    r$_j$ $\in$ $R_{c}'$, $\exists c' | c' \notin C$
    f$_1^m$, $\dots$, f$_n^m$, r$_m'$ = step(f$_1^{m-1}$, $\dots$, f$_n^{m-1}$, r$_m$)
    .
    .
    f$_1^j$, $\dots$, f$_n^j$,r$_j'$ = step(f$_1^{j-1}$, $\dots$, f$_n^{j-1}$, r$_j$)
    .
    .
    f$_1^1$, $\dots$, f$_n^1$, r$_1'$ = step(f$_1$, $\dots$, f$_n$, r$_1$)
\end{lstlisting}

\noindent
Again, if no new client has connected to the game, then the rule has simply no effect.

\begin{lstlisting}[mathescape]
tick(e:E, dt, c) =
{ Field$_1$=tick(f$_1^m$, dt); $\dots$; Field$_n$=tick(f$_n^m$, dt);
  Rule$_1$=r$_1'$; $\dots$; Rule$_j$ = r$_j$; $\dots$; Rule$_m$=r$_m'$ }
  where
    r$_j$ $\in$ $R_{c}$, $\forall c' | c' \in C$
   f$_1^m$, $\dots$, f$_n^m$, r$_m'$ = step(f$_1^{m-1}$, $\dots$, f$_n^{m-1}$, r$_m$)
   .
   .
   f$_1^{j-1}$, $\dots$, f$_n^{j-1}$,r$_j$ = step(f$_1^{j-1}$, $\dots$, f$_n^{j-1}$, r$_j$)
   .
   .
   f$_1^1$, $\dots$, f$_n^1$, r$_1'$ = step(f$_1$, $\dots$, f$_n$, r$_1$)
\end{lstlisting}

\paragraph{Master rule semantics}
Master rules are executed on instances of entities that were instantiated on the current client, that is, when the entity belongs to the set of local entities of the current client state.

\begin{lstlisting}[mathescape]
tick(e:E, dt, c) =
{ Field$_1$=tick(f$_1^m$, dt); $\dots$; Field$_n$=tick(f$_n^m$, dt);
  Rule$_1$=r$_1'$; $\dots$; Rule$_j$ = r$_j'$; $\dots$; Rule$_m$=r$_m'$ }
  where
    r$_j$ $\in$ $R_{m}$, e $\in E_{L}$
    f$_1^m$, $\dots$, f$_n^m$, r$_m'$ = step(f$_1^{m-1}$, $\dots$, f$_n^{m-1}$, r$_m$)
    .
    .
    f$_1^j$, $\dots$, f$_n^j$,r$_j'$ = step(f$_1^{j-1}$, $\dots$, f$_n^{j-1}$, r$_j$)
    .
    .
    f$_1^1$, $\dots$, f$_n^1$, r$_1'$ = step(f$_1$, $\dots$, f$_n$, r$_1$)
\end{lstlisting}

\paragraph{Slave rule semantics}
Slave rules are executed on instances of entities that were instantiated on a remote client, that is, when the entity belongs to the set of remote entities of the current client state.

\begin{lstlisting}[mathescape]
tick(e:E, dt, c) =
{ Field$_1$=tick(f$_1^m$, dt); $\dots$; Field$_n$=tick(f$_n^m$, dt);
  Rule$_1$=r$_1'$; $\dots$; Rule$_j$ = r$_j'$; $\dots$; Rule$_m$=r$_m'$ }
  where
    r$_j$ $\in$ $R_{s}$, e $\in E_{R}$
    f$_1^m$, $\dots$, f$_n^m$, r$_m'$ = step(f$_1^{m-1}$, $\dots$, f$_n^{m-1}$, r$_m$)
    .
    .
    f$_1^j$, $\dots$, f$_n^j$,r$_j'$ = step(f$_1^{j-1}$, $\dots$, f$_n^{j-1}$, r$_j$)
    .
    .
    f$_1^1$, $\dots$, f$_n^1$, r$_1'$ = step(f$_1$, $\dots$, f$_n$, r$_1$)
\end{lstlisting}

\subsubsection{Semantics of data transmission} The networking transmission semantics must include the possibility that some messages might be lost during the transmission. At this purpose we introduce a parameter $p_{r} \in \; ]0,1[$ for the \textit{network reliability} which represents the probability that a message is lost before reaching its destination. We also use a parameter $p_{m} \in \; ]0,1[$ which is a random number generated for each message to simulate the message losses with respect to $p_{r}$. In what follows we change the definition of the \texttt{step} function to include the fact that there are multiple clients running the rules in the network state, and that a receive on a client usually corresponds to a send on another client.

\paragraph{Sending and receiving unreliable messages} The language provides two primitives to send messages, one which reliably sends the message, meaning that it ensures that the receiver receives the message, and one unreliable primitive, with which the message could be lost with no chance of recovering it. In the case of an unreliable send, the client send the message and if received, that is if $p_{m} \geq p_{r}$ then the fields of the receiving client are updated with the message value, otherwise the receiver does not receive the message and simply skips to the next statement.

\begin{lstlisting}[mathescape = true]
step(f$_1^{c_{i}}$},...,f$_n^{c_{i}}$,{send x; k$^{c_i}$}),f$_1^{c_{j}}$},...,f$_n^{c_{j}}$,{yield receive x; k$_{c_j}$}) =
(step(f$_1^{c_{i}}$},...,f$_n^{c_{i}}$,{k$_{c_i}$}),(x, {k$_{c_j}$})
when $p_{m} \geq p_{r}$
\end{lstlisting}

\begin{lstlisting}[mathescape = true]
step(f$_1^{c_{i}}$},...,f$_n^{c_{i}}$,{send x; k$^{c_i}$}),f$_1^{c_{j}}$},...,f$_n^{c_{j}}$,{yield receive x; k$_{c_j}$}) = (step(f$_1^{c_{i}}$},...,f$_n^{c_{i}}$,{k$_{c_i}$}),(f$_1^{c_{j}}$,...,f$_n^{c_{j}}$,{ k$_{c_j}$})
when $p_{m} < p_{r}$
\end{lstlisting}



\section{Implementation in Metacasanova}
\begin{itemize}
	\item Introduction to Metacasanova with explanation on rule evaluation.
	\item Overview of the implementation of Casanova 2 with Metacasanova.
	\item Implementation of networking in Metacasanova.
\end{itemize}

Implementation in Metacasanova.


\section{Evaluation}
An extensive evaluation of Casanova implemented in Metacasanova, which we omit for brevity, can be found in \cite{DiGiacomo2017}. The implementation of Casanova operational semantics in Metacasanova is almost 5 times shorter than the corresponding F\# implementation in the hard-coded compiler. In addition to Casanova, we have implemented a subset of the C language called C-{}-. This language supports \texttt{if-then-else}, \texttt{while-loop}, and \texttt{for} statements, as well as local scoping of variables. The total length of the language definition in Metacasanova is 353 lines of code. The corresponding C\# code to implement the operational semantics of the language is 3123 lines, thus the code reduction with Metacasanova is roughly 8.84 times. For comparison, in Table \ref{tab:cmm} it is possible to see the code length to implement three different statements, both in Metacasanova and C\#. We tested C-{}- against Python by computing the average running time to compute the factorial of a number. C-{}- results to be 50 times slower than Python. This result is worse than what we obtained with Casanova, because in order to emulate the interruptible rule mechanism of Casanova in Python you must rely on coroutines that are slower than a program containing simple statements. Moreover, we tested the performance improvement of the optimization using Functors to represent records against the standard one using dynamic symbol tables. The test was run using records with a number of fields ranging from 1 to 10 and updating from 10000 to 1000000 instances of such records. In Table \ref{tab:functors}, we can see that the optimization using Functors leads to a performance increase on average of about 11 times, with peaks of 30 times. The gain increases with the number of fields, thus Functors are particularly effective for records with high number of fields. Figure \ref{fig:chart} shows a chart of the overall performance of the two techniques (the data points are taken from Table \ref{tab:functors}). The horizontal axis contains the amount of fields per record, while the vertical axis contains the number of records that are being updated. We can see that the performance of the dynamic table degrades considerably when increasing the number of fields, and that the higher the amount of records is, the steeper the curve is. On the other hand, the performance of the implementation with Functors is almost constant, regardless of the amount of fields or records that are being updated. Moreover, note that the performance of the dynamic table is improved by the fact that we are using a dictionary implemented in .NET, which can access the entries in $O(\log n)$. If the symbol table were represented as a meta-data structure in the language the performance would be even worse, since it would have to be encoded as a list of pairs with the field name and its value, and its manipulation would be affected by the evaluation rules that should implement this behaviour.

\begin{table}	
	\caption{Running time with the functor optimization and the dynamic table with 10000, 100000, and 1000000 records.}
	\begin{tabular}{|c|c|c|c|}
		\hline
		\textbf{FIELDS}& \textbf{Functors (ms)}&\textbf{Dynamic Table (ms)} & \textbf{Gain}\\ \hline
		1&	1.00E-05&	5.00E-06&	0.50\\ \hline
		2&	9.00E-06&	1.30E-05&	1.44\\ \hline
		3&	9.00E-06&	2.70E-05&	3.00\\ \hline
		4&	9.00E-06&	4.50E-05&	5.00\\ \hline
		5&	9.00E-06&	7.00E-05&	7.78\\ \hline
		6&	9.00E-06&	9.90E-05&	11.00\\ \hline
		7&	9.00E-06&	1.33E-04&	14.78\\ \hline
		8&	9.00E-06&	1.75E-04&	19.44\\ \hline
		9&	9.00E-06&	2.20E-04&	24.44\\ \hline
		10&	9.00E-06&	2.70E-04&	30.00\\ \hline
		\multicolumn{2}{c|}{} & \textbf{Average gain} & 11.74\\ \cline{3-4}			
	\end{tabular}

	\vspace{0.15cm}
	\begin{tabular}{|c|c|c|c|}
		\hline
		\textbf{FIELDS}& \textbf{Functors (ms)}&\textbf{Dynamic Table (ms)} & \textbf{Gain}\\ \hline
		1&	9.60E-05&	6.30E-05&	0.66\\ \hline
		2&	9.40E-05&	1.59E-04&	1.69\\ \hline
		3&	9.50E-05&	3.04E-04&	3.20\\ \hline
		4&	9.60E-05&	5.03E-04&	5.24\\ \hline
		5&	9.60E-05&	7.52E-04&	7.83\\ \hline
		6&	9.60E-05&	1.05E-03&	10.95\\ \hline
		7&	9.70E-05&	1.41E-03&	14.57\\ \hline
		8&	9.80E-05&	1.82E-03&	18.59\\ \hline
		9&	9.90E-05&	2.29E-03&	23.17\\ \hline
		10&	1.00E-04&	2.81E-03&	28.05\\ \hline
		\multicolumn{2}{c|}{} & \textbf{Average gain} & 11.39\\ \cline{3-4}						
	\end{tabular}

	\vspace{0.15cm}
	\begin{tabular}{|c|c|c|c|}
		\hline
		\textbf{FIELDS}& \textbf{Functors (ms)}&\textbf{Dynamic Table (ms)} & \textbf{Gain}\\ \hline
		1&	9.47E-04&	7.29E-04&	0.77\\ \hline
		2&	9.51E-04&	1.78E-03&	1.87\\ \hline
		3&	9.50E-04&	3.33E-03&	3.51\\ \hline
		4&	9.60E-04&	5.43E-03&	5.66\\ \hline
		5&	9.65E-04&	8.03E-03&	8.32\\ \hline
		6&	9.71E-04&	1.11E-02&	11.44\\ \hline
		7&	9.75E-04&	1.47E-02&	15.12\\ \hline
		8&	9.82E-04&	1.89E-02&	19.28\\ \hline
		9&	9.92E-04&	2.37E-02&	23.86\\ \hline
		10&	1.00E-03&	2.87E-02&	28.62\\ \hline
		\multicolumn{2}{c|}{} & \textbf{Average gain} & 11.84\\ \cline{3-4}						
	\end{tabular}
	\label{tab:functors}
\end{table}

\begin{table}
	\centering
	\caption{Code length implementation of C-{}- and run-time performance}
	\begin{tabular}{|c|c|c|}
		\hline
		\textbf{Statement} & \textbf{Metacasanova} & \textbf{C\#}\\
		\hline
		\texttt{if-then-else} & 4 & 103 \\
		\hline
		\texttt{while} & 7 & 73 \\
		\hline
		\texttt{For} & 11 & 81\\
		\hline
	\end{tabular}
	
	\vspace{0.15cm}
	\begin{tabular}{|c|c|}
		\hline
		\textbf{C-{}-} & \textbf{Python} \\
		\hline
		1.26ms & $2.36 \cdot 10^{-2}$ms \\
		\hline
	\end{tabular}
	\label{tab:cmm}
\end{table}

\begin{figure}
	\includegraphics[width = \columnwidth]{Figures/chart.jpg}
	\caption{Execution time of the different memory models}
	\label{fig:chart}
\end{figure}

\section{Conclusion}
This chapter provides an answer to the problem statement and research questions presented in Section \ref{sec:ch1_problem_statement}. The goal of the first research question is measuring the benefits of using a Metacompiler in terms of development speed when used to implement a domain-specific language for game development with respect to the implementation measured in code length. The goal of the second research question is aimed to determine the trade-off between a manual implementation of the language and an implementation with Metacasanova. The goal of the third research question is to identify reasons for this trade-off and propose an optimization to reduce it. The last part of this chapter answers the problem statement, provides an overview of future work and adds final remarks for this thesis.

\section{Answer to research questions}
\label{sec:ch_conclusion_answer_research_questions}
The three research questions stated in Section \ref{sec:ch1_problem_statement} are now answered  in Sections \ref{subsec:ch_conclusion_rq1}, \ref{subsec:ch_conclusion_rq2}, and \ref{subsec:ch_conclusion_rq3} respectively.

\subsection{Ease of development}
\label{subsec:ch_conclusion_rq1}

The first research question reads:\\

\researchQuestion{To what extent can a meta-compiler reduce the amount of code required to create a compiler for a given programming language?}\\

The answer to this research question is derived from the results shown in Chapter \ref{ch:languages}: in Section \ref{sec:ch_mcnv_languages_evaluation} we showed how the use of Metacasanova reduces the effort in term of code writing for the compiler of Casanova as the code required for the definition of the language semantics is roughly 5 times shorter in Metacasanova than the hard-coded version of the compiler written in F\#. This improvement is due to the fact that, in Metacasanova, it is possible to express the semantics of the language by mimicking almost directly the definition of Casanova written in natural semantics. 

\subsection{Performance trade-off}
\label{subsec:ch_conclusion_rq2}

\researchQuestion{How much is the performance loss introduced by the meta-compiler with respect to an implementation written in a language compiled with a traditional compiler and is this loss acceptable when considering game development?}

\subsection{Optimization}
\label{subsec:ch_conclusion_rq3}

\researchQuestion{What is the cause of the performance degradation when employing a meta-compiler and how can this be improved?}


\bibliography{bibliography}
\bibliographystyle{plain}

\end{document}