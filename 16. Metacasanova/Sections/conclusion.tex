In this work we proposed an alternative technique to implement a DSL for games by using a metacompiler called Metacasanova. As a case study we re-implemented the Casanova language, a DSL for game development, in Metacasanova. Our results show that the code required to re-implement Casanova in Metacasanova is (\textit{i}) shorter, and (\textit{ii}) more readable with respect to the existing hard-coded compiler for the same language. Moreover we showed that the language behaviour can be expressed in a way that directly mimics the formal semantics definition of the language. Adding the layer of the meta-compiler to the language affects the performance of the generated code so that we cannot achieve the same performance as with the manual implementation. Despite this, we managed to achieve performance similar to Python, a language typically used as a scripting language to define the game logic in several commercial games. We believe that this is a promising result for a meta-compiler which is at its first iteration.

Currently Metacasanova meta-compiler misses some features, such as a much stronger type system of modules, functions, higher kinded modules. This will allow to get rid of the memory map in Casanova 2.5 overhead by inlining fields directly into the rule updates. Furthermore we want to integrate networking primitives in Casanova 2.5 to ease the development of multiplayer games.

Despite this, adding additional features to Casanova in the implementation in Metacasanova requires little or no effort compared to what should be implemented in the hard-coded compiler that was developed for Casanova 2, at the same time generating code with performance good enough that could be used in a real scenario.