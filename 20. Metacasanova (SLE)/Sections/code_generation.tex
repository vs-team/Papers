In Section \ref{sec:semantics} we defined the syntax and semantics of Metacasanova. In this section we explain how the abstractions of the language are compiled into the generated code. We chose C\# as target language because the development of Metacasanova started with the idea of expanding the DSL for game development Casanova with further functionalities. Casanova hard-coded compiler generates C\# code as well because it is compatible with game engines such as Unity3D and Monogame. At the same time, C\# grants decent performance without having to manually manage the memory such as for lower-level languages like C/C++. Code generation in different target languages is possible but still an ongoing project (see Section \ref{sec:conclusion}).

\subsection{Data Structures Code Generation}
The type of each data structure is generated as an interface in C\#. Each data structure defined in Metacasanova is mapped to a \texttt{class} in C\# that implements such interface. The class contains as many fields as the number of arguments the data structure contains. Each field is given an automatic name \texttt{argC} where \texttt{C} is the index of the argument in the data structure definition. The data structure symbols used in the definition might be pre-processed and replaced in order to avoid illegal characters in the C\# class definition. The class contains an additional field that stores the original name of the data structure before the replacement is performed, used for its ``pretty print''. For example the data structure

\begin{lstlisting}
Data "$i" -> int : Value
\end{lstlisting}

\noindent
will be generated as

\begin{lstlisting}
public interface Value {  }

public class __opDollari : Value
{
  public string __name = "$i";
  public int __arg0;

  public override string ToString()
  {
    return "(" + __name + " " + __arg0 + ")";
  }
}
\end{lstlisting}

\subsection{Code Generation for Rules}
Each rule contains a set of premises that in general call different functions to produce a result, and a conclusion that contains the function evaluated by the current rule and the result it produces. The code generation for the rules follows the steps below:

\begin{enumerate}
	\item Generate a data structure for each function defined in the meta-program.
	\item For each function $f$ extract all the rules whose conclusion contains $f$.
	\item Create a \texttt{switch} statement with a case for each rule that is able to execute the function (the function is in its conclusion).
	\item In the case block of each rule, define the local variables defined in the rule.
	\item Apply pattern matching to the arguments of the function contained in the conclusion of the rule. If it fails, jump immediately to the next case (rule).
	\item Store the values passed to the function call into the appropriate local variables.
	\item Run each premise by instantiating the class for the function used by it and copying the values into the input arguments.
	\item Check if the premise outputs a result and, in the case of an explicit data structure argument, check the pattern matching. If the premise result is empty or the pattern matching fails for all the possible executions of the premise then jump to the next case.
	\item Generate the result for the current rule execution. 
\end{enumerate}

\noindent
In what follows, we use as an example the code generation for the following rule (which computes the sum of two integer expressions in a programming language):

\begin{lstlisting}
eval a -> $i c
eval b -> $i d
<< c + d >> -> e
----------------
eval (a + b) -> $i e
\end{lstlisting}

From now on we will refer to an argument as \textit{explicit data argument} when its structure appears explicitly in the conclusion or in one of the premises, as in the case of \texttt{a + b} in the example above.

\subsubsection{Data Structure for the Function}
\label{subsubsec:function_generation}

As first step the meta-compiler generates a class for each function defined in the meta-program. This class contains one field for each argument the function accepts. It also contains a field to store the possible result of its evaluation. This field is a \texttt{struct} generated by the meta-compiler defined as follows:

\begin{lstlisting}
public struct __MetaCnvResult<T> { public T Value; public bool HasValue; }
\end{lstlisting}

The result contains a boolean to mark if the rule actually returned a result or failed, and a value which contains the result in case of success.

For example, the function

\begin{lstlisting}
Func eval -> Expr : Value
\end{lstlisting}

\noindent
will be generated as

\begin{lstlisting}
public class eval
{
  public Expr __arg0;
  public __MetaCnvResult<Value> __res;
  ...
}
\end{lstlisting}

\subsubsection{Rule Execution}

The class defines a method \texttt{Run} that performs the actual code execution. The meta-compiler retrieves all the rules whose conclusion contains a call to the current function, which define all the possible ways the function can be evaluated with. It then creates a \texttt{switch} structure where each \texttt{case} represents each rule that might execute that function. The result of the rule is also initialized here (the \texttt{struct} will contain a default value and the boolean flag will be set to \texttt{false}). Each \texttt{case} defines a set of local variables, that are the variables used within the scope of that rule.

\subsubsection{Local Variables Definitions and Pattern Matching of the Conclusion}

At the beginning of each \texttt{case}, the meta-compiler defines the local variables initialized with their respective default values. It also generates then the code necessary for the pattern-matching of the conclusion arguments. Since variables always pass the pattern-matching, the code is generated only for arguments explicitly defining a data structure (see the examples about arithmetic operators in Section \ref{sec:semantics}) and literals. If the pattern matching fails then the execution jumps to the next \texttt{case} (rule). For instance, the code for the following conclusion

\begin{lstlisting}
...
-------------
eval (a + b) -> $i e
\end{lstlisting}

\noindent
is generated as follows

\begin{lstlisting}
case 0:
{
  Expr a = default(Expr);
  Expr b = default(Expr);
  int c = default(int);
  int d = default(int);
  int e = default(int);
  if (!(__arg0 is __opPlus)) goto case 1;
  ...
}
\end{lstlisting}

\noindent
Note that an explicit data argument, such in the example above, might contain other nested explicit data arguments, so the pattern-matching is recursively performed on the data structure arguments themselves.

\subsubsection{Copying the Input Values Into the Local Variables}
When each function is called by a premise, the local values are stored into the class fields of the function defined in Section \ref{subsubsec:function_generation}. These values must be copied to the local variables defined in the \texttt{case} block representing the rule. Particular care must be taken when one argument is an explicit data. In that case, we must copy, one by one, the content of the data into the local variables bound in the pattern matching. For example, in the rule above, we must separately copy the content of the first and second parameter of the explicit data argument into the local variables \texttt{a} and \texttt{b}. The generated code for this step, applied to the example above, will be:

\begin{lstlisting}
__opPlus __tmp0 = (__opPlus)__arg0;
a = __tmp0.__arg0;
b = __tmp0.__arg1;
\end{lstlisting}

Note that the type conversion from the polymorphic type \texttt{Expr} into \texttt{opPlus} is now safe because we have already\\ checked during the pattern matching that we actually have \texttt{opPlus}.

\subsubsection{Generation of Premises}
Before evaluating each premise, we must instantiate the class for the function that they are invoking. The input arguments of the function call must be copied into the fields of the instantiated object. If one of the arguments is an explicit data argument, then it must be instantiated and its arguments should be initialized, and then the whole data argument must be assigned to the respective function field. After this step, it is possible to invoke the \texttt{Run} method of the function to start its execution. The first premise of the example above then becomes (the generation of the second is analogous):

\begin{lstlisting}
eval a -> $i c
\end{lstlisting}

\begin{lstlisting}
eval __tmp1 = new eval();
__tmp1.__arg0 = a;
__tmp1.Run();
\end{lstlisting}

\subsubsection{Checking the Premise Result}
After the execution of the function called by a premise, we must check if a rule was able to correctly evaluate it. In order to do so, we must check that the result field of the function object contains a value, and if not the rule fails and we jump to the next case (rule), which is performed in the following way:

\begin{lstlisting}
if (!(__tmp1.__res.HasValue)) goto case 1;
\end{lstlisting}

If the premise was successfully evaluated by one rule, then we must check the structure of the result, which leads to the following three situations: (\textit{i}) the result is bound to a variable, (\textit{ii}) the result is constrained to be a literal, and (\textit{iii}) the result is an explicit data argument. In the first case, as already explained above, the pattern matching always succeeds, so no check is needed. In the second case, it is enough to check the value of the literal. In the last case, all the arguments of the data argument must be checked to see if they match the expected result. In general this process is recursive, as the arguments could be themselves other explicit data arguments. If the result passes the check, then the result is copied into the local variables, in a fashion similar to the one performed for the function premise. For instance, for the premise

\begin{lstlisting}
eval a -> $i c
\end{lstlisting}

\noindent
the meta-compiler generates the following code to check the result
\begin{lstlisting}
if (!(__tmp1.__res.Value is __opDollari)) goto case 1;
__MetaCnvResult<Value> __tmp2 = __tmp1.__res;
__opDollari __tmp3 = (__opDollari)__tmp2.Value;
c = __tmp3.__arg0;
\end{lstlisting}

\subsubsection{Generation of the Result}
When all premises correctly output the expected result, the rule can output the final result. In order to do that, the generated code must copy the right part of the conclusion (the result) into the \texttt{res} variable of the function class. If the right part of the conclusion is, again, an explicit data argument, then the data object must first be instantiated and then copied into the result. For example the result of the rule above is generated as follows:

\begin{lstlisting}
res = c + d;
__opDollari __tmp7 = new __opDollari();
__tmp7.__arg0 = res;
__res.HasValue = true;
__res.Value = __tmp7;
break;
\end{lstlisting}

\noindent
After this step, the rule evaluation successfully returns a result.

This implementation choice is due to the fact that we plan to support partial function applications, thus, when a function is partially applied, there is the need to store the values of the arguments that were partially given. This could still be implemented with static methods and lambdas in C\#, but not all programming languages natively support lambda abstractions, so we chose to have a set-up that allows us to change the target language without dramatically altering the logic of code generation.

\subsection{Discussion}
\label{subsec:code_generation_discussion}
Metacasanova has been evaluated in \cite{DiGiacomo2017} by re-building the DSL for game development Casanova \cite{abbadi2015casanova, abbadithesis2017}. Even though the size of the code required to implement the language has been drastically reduced (almost 1/5 shorter), performance dropped dramatically. We identified a main problem causing the performance decay that, if solved, will improve the performance of the generated code.

In order to encode a symbol table in the meta-compiler in the current implementation (used for example to store the variables defined in the local scope of a control structure or to model a class/record data structure), we are left with two options: (\textit{i}) define a custom data structure made of a list of pairs, containing the field/variable name as a string and its value, in the following way

\begin{lstlisting}
Data "table" -> List[Tuple[string, Value]] : SymbolTable
\end{lstlisting}

\noindent
or (\textit{ii}) use a dictionary data structure coming from .NET, such as \texttt{ImmutableDictionary}, which was the implementation choice for Casanova. In both cases, the behaviour of the language implemented in Metacasanova will be that of a dynamic language, because whenever the value of a variable or class field must be read, the evaluation rule must look up the symbol table at run time to retrieve the value, whose complexity will be $O(n)$ with the list implementation and $O(\log n)$ with the dictionary implementation. This issue is caused by the fact that, in the current state of Metacasanova, the meta-type system is unaware of the type system of the language that is being implemented in the meta-compiler. This is not a problem limited to Metacasanova but to all meta-compilers having a meta-type system that does not allow embedding of the host language type system. In the next section we propose an extension to Metacasanova to overcome this problem by embedding the type system of the implemented language in the meta-type system of Metacasanova and inlining the code to access the appropriate variable at compile time.

