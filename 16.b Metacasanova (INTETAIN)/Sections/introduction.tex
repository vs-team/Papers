In video games development it is often the case that  \textit{Domain
Specific languages} (DSL) are used, as they provide ad-hoc features
that simplify the coding process and yield to more concise and
readable code when dealing with  time management, synchronization, and
AI thanks to their little CPU and memory overhead
\cite{DSL_SURVEY_PAPER, GAME_SCRIPTING, CASANOVA2_PAPER} . A typical
synchronization problem arises when when the short distance of a player from a door enables
the player to open it. These scenarios usually happen in a heavily
concurrent system, where possibly hundreds of entities perform such
interactions within the same update of the game. In order to tackle
these problems, as a valuable alternative to the use of Threads,
Finite state machines, or Strategy patterns, developers make use of
Domain specific languages, like JASS \cite{JASS}, Unreal Script \cite{UNREAL_ENGINE},
or NWScript \cite{NW_SCRIPT}.

A first approach is implementing a DSL by
building an interpreter within the host language abstractions, such as
monads in a functional programming language \cite{DSL_MONAD_PAPER,
CASANOVA1_PAPER, SCRIPT_MONAD_PAPER}. Unfortunately the performance of
an interpreted DSL built with monads is not as high as that achieved
by compiled code, as monads make a large use of anonymous functions
(or lambda expressions) which are often implemented with virtual
method calls. Moreover functional languages are rarely employed in
game developments as games are highly stateful programs.

Another typical approach is to design a hard-coded compiler for the
DSL. This is a hard and time-consuming task, since a compiler is made
of several components which perform transformations from the source
code into machine code. The steps performed in this transformations
are often the same, regardless of the language for which the compiler
is being implemented, and they are not part of the creative aspect of
language design \cite{CWIC}. This is why metacompliers come into the
scene, with the ability to treat programs as data
\cite{GENERATIVE_PROGRAMMING_CZARNECKI}.

In this paper we present a novel solution to ease the development of a
compiler for a game DSL by developing a metacompiler, called Metacasanova, producing code
that is both clear and efficient, especially designed for games
development. We show that, with this approach, the code to generate the compiler is 5 times shorter than a hard-coded compiler.

In this work we briefly describe the most common techniques used to build DSL's for game and their drawbacks (Section \ref{sec:problem_statement}). We then propose a novel approach by introducing Metacasanova as a tool to develop a DSL (Section \ref{sec:formal_description}) and by re-emplementing Casanova DSL (Section \ref{sec:casanova3}). We then evaluate the result in terms of time performance and code length (Section \ref{sec:evaluation}) and draw the conclusion.

\begin{comment}
In Section \ref{sec:problem_statement} we describe the most
common techniques to solve the concurrency problems in games. In
Section \ref{sec:formal_description} we propose a novel approach to
defining DSL's for games by using a metacompiler called Metacasanova.
We will motivate the role of Metacasanova in easing the process of building a compiler and explain the structure of a program in Metacasanova. In Section
\ref{sec:casanova3} we will present as case study Casanova 2.5, a
re-implementation of Casanova, a language oriented to video game
development, in the metacompiler. We choose this particular language
for its high-performance and usability in the scope of game
development \cite{CASANOVA2_PAPER}. In Section \ref{sec:evaluation} we
compare the running time of the generated code of Casanova 2.5 for a
sample with an analogous implementation in Python. Moreover we compare
the code length of the implementation of Casanova 2.5 in Metacasanova
with the implementation of the current Casanova hard-coded compiler.
In Section \ref{sec:future_works} we will present the conclusions about this work.
\end{comment}

\begin{comment}
\subsection{Related work}
In this section we present the existing work on metacompilers and
Domain specific languages for games.

\paragraph{Metacompilers}
A first approach to metacompilers was first made with the
Schorre-metalanguages family of metacompilers, among which we consider
META-II \cite{META2}. A META-II program is a set of rules. Each rule
can either succeed or fail. The body of the rule is made of
\textit{test}, \textit{operators}, and \textit{output code}. The rules
allow to express the syntax of the language in a BNF-like syntax. Each
rule is evaluated, i.e. the tests are performed, and if the tests pass
then the output code is generated (in the specific case the output was
machine code for the IBM 1401). An improvement to this metacompiler
was TREE-META \cite{META3}, which allows the definition of
tree-building operators to generated an \textit{Abstract Syntax Tree}.
This feature allows the implementation of language optimizations on
the AST. Finally, CWIC \cite{CWIC} was a further refinement, employing
three sub-languages used to define the different stages of the
transformation process: \textit{Syntax} is used to defined the
transformation between the grammar definition and the associated AST.
\textit{Generator} is made of a series of transforming rules to output
IBM 360 binary code. \textit{MOL-360} is an intermediate language
which was introduced to write support libraries for CWIC and added
later as an intermediate layer in the code generation phase. These
compilers ease the process of defining and checking the grammar of the
language, but they do not allow to define semantics rule (i.e. a type
system). Indeed they simply parse the input language and then directly
output the target code.

The type semantics and operational semantics are often expressed in
the form of logic rules, which define transformations among AST's. For
this reason a metacompiler can be seen as a set of logic rules
defining these transformations. The RML metacompiler \cite{RML1, RML2,
MODELICA} follows this idea. A program in RML is made of a set of
rules. Each rule is made of a set of \textit{premises}, whose
evaluation produces a result, and of a \textit{conclusion} which
produces the final result of the rule. Each of the premises is
evaluated and can result in a fail or a success. Premises can be
combined using logical operators, assuming that a success corresponds
to \textit{true} and a fail to \textit{false}.

\paragraph{Domain specific languages for games}
A very common domain specific language is Unreal Script
\cite{UNREAL_ENGINE}, used to develop a variety of commercial games
such as Unreal, Unreal Tournament, and Deus Ex. It is an
object-oriented language interpreted in the Unreal Engine. In this
language it is possible to define particular functions called
\texttt{states} to model the concurrent behaviours described above.
These functions are executed only when the object is in that
particular state.

Another widely used language for games is the script language created
for the Neverwinter Nights game series, called NWScript. This language
is an extension of the C programming language with extended native
types to support in-game elements, such as graphics effects,
characters, etc. This language does not provide primitives to
explicitly interrupt the execution or synchronize the execution with
other events in the games, rather it allows to run a specific script
when a set of pre-defined events happen in the game.

The Game Maker Script language \cite{GAME_MAKER} is another widely
used script language based on a syntax similar to C. It is used in the
Game Maker game engine.

A different approach in literature is the Game Description Language
\cite{GAME_DESCRIPTION_LANGUAGE}, a logic language that represents the
game state as facts and the game mechanics as logical rules.
\end{comment}