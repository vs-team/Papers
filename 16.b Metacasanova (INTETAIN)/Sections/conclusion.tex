In this work we proposed an alternative technique to implement a DSL for games by using a metacompiler called Metacasanova. As a case study we re-implemented the Casanova language, a DSL for game development, in Metacasanova. Our results show that the code required to re-implement Casanova in Metacasanova is (\textit{i}) shorter, and (\textit{ii}) more readable with respect to the existing hard-coded compiler for the same language. Moreover we showed that the language behaviour can be expressed in a way that directly mimics the formal semantics definition of the language. Adding the layer of the meta-compiler to the language affects the performance of the generated code so that we cannot achieve the same performance as with the manual implementation. Despite this, we managed to achieve performance similar to Python, a language typically used as a scripting language to define the game logic in several commercial games.