%%%%%%%%%%%%%%%%%%%%%%%%%%%%%%%%%%%%%%%%
% THE CASANOVA LANGUAGE
%%%%%%%%%%%%%%%%%%%%%%%%%%%%%%%%%%%%%%%%

In this section we present the Casanova language syntax, typing rules and semantics. 

\begin{center}
\line(1,0){240}
\end{center}

\textit{Note on syntactic sugar:}

We will use the following syntactic sugar to increase source code readability:

Rather than write:

\begin{lstlisting}
let! _ = b1
in b2
\end{lstlisting}

we can write:

\begin{lstlisting}
do! b1
in b2
\end{lstlisting}

Rather than write:

\begin{lstlisting}
let x = t1
in let y = t2
in t3
\end{lstlisting}

we can write:

\begin{lstlisting}
let x = t1
let y = t2
t3
\end{lstlisting}

\begin{comment}
We may also directly iterate all the elements of a table \texttt{t} by writing:

\begin{lstlisting}
for x in t do
  action
\end{lstlisting}

rather than explicitly using indices or \texttt{head} and \texttt{tail}.
\end{comment}

\begin{center}
\line(1,0){240}
\end{center}

\subsection{Simple Example}

Before we start, we will give a general idea of how the language works with a small example. We will build a very simple game where asteroids enter the screen from the top, scroll down to the bottom at different velocities and then disappear.

A Casanova program is composed of two portions:
\begin{itemize}
\item the state of the game, a series of types arranged hierarchically (typically at least one for the global state and one for the state of each entity); each portion of the game state may contain exactly one rule (a function that computes the new value of the field)
\item the main behavior, which performs a series of instructions on all the mutable fields of the state (those marked as \texttt{Rule T} or \texttt{Var T}; behavior execution is suspended at the \texttt{yield} statement, and resumed at the next tick
\end{itemize}

The state of our simple program is defined as:

\begin{lstlisting}
type Asteroid =
  {
    Y     : Rule float
          :: \(self,y,dt) -> y + dt * self.VelY
        
    VelY  : float        
    X     : float
  }

type GameState =
  {
    Asteroids           
        : Rule(Table Asteroid)
        :: \asteroids -> [a | a <- asteroids && a.Y > 0]
  	    
    DestroyedAsteroids	
        : Rule<int>
        :: \(state,destroyed_asteroids) -> destroyed_asteroids + count([a | a <- !state.Asteroids && a.Y <= 0])
  }
\end{lstlisting}
  
In a type declaration, the \texttt{:} operator means ``has type'', while the \texttt{::} operator means ``has rule''. Rules can access the game state, the current entity and the time delta between the current and previous ticks.

In the state definition above we can see that the state is comprised by a set of asteroids which are removed when they reach the bottom. Removing these asteroids increments a counter, which is essentially the ``score'' of our game. Each asteroid moves according to its velocity.

The initial state is then provided:
\begin{lstlisting}
let state0 =
  {
    Asteroids               = []
    DestroyedAsteroids      = 0
  }
\end{lstlisting}

After defining the state we must give an initial behavior. As can be easily noticed, our game does not generate any asteroids and so the initial state will never change. Since creating asteroids is an activity that certainly must not be performed at every tick (otherwise we could generate in excess of 60 asteroids per second: clearly too many), we need a function that is capable of performing \textit{different} operations on the state depending on time. Since rules perform the \textit{same} operation at every tick, they are unsuited to this kind of processing. Behaviors are built exactly around this need. The behavior for our game is the following:

\begin{lstlisting}
let main state =
  let random = mk_random()
  let rec wait interval =
    {
      let! t0 = time
      do! yield
      let! t = time
      let dt = t - t0
      if dt > interval then
        return ()
      else
        do! wait (interval - dt)
    }  
  let rec behavior() =
    {
      do! wait (random.Next(1,3))
      do! state.Asteroids.Add
      	  {
      	    X     = random.Next(-1,+1)
      	    Y     = 1
      	    VelY  = random.Next(-0.1,-0.2)
      	  }
      if state.DestroyedAsteroids < 100 then
        do! behavior()
      else
        return ()
    }
  in behavior()
\end{lstlisting}
  
Our behavior declares a random number generator and then starts iterating a function that waits between 1 and 3 seconds and then creates a random asteroid. When the number of destroyed asteroids is greater than 100, the function stops and the game ends (games end when their main behavior terminates).

Notice that behaviors are expressed with two different syntaxes: an ML-style syntax for pure terms (those which read the state and simply perform computations) and an imperative-style syntax for impure terms (those which write the state and interact with time such as wait). The imperative syntax loosely follows the monadic syntax of the F\# language, where a monadic block is declared within \texttt{\{\}} parentheses, monadic operations are preceded by either \texttt{do!} or \texttt{let!} and returning a result is done with the \texttt{return} statement.

\subsection{Syntax}
In the remainder of this section we will adopt the following conventions:
\begin{itemize}
\item capitalized items such as \texttt{Program} and \texttt{StateDef} are grammatical elements
\item quoted items such as \texttt{`type'} and \texttt{`GameState'} are keywords that must appear as indicated
\item items surrounded by \texttt{[ ]} parentheses such as \texttt{[EntityName]} are user-defined strings
\end{itemize}

The program syntax starts with the definition of the state (a series of type definitions with rules) and is followed by the entry point (the initial state and the initial behavior):

\begin{lstlisting}
Program  ::= StateDef
             Main
StateDef ::= EntityDefs
\end{lstlisting}

A type definition is comprised of one of various primitive types such as integers, floating point numbers, two- or three- dimensional vectors, etc. combined into any of the usual composite types known to functional programmers such as tuples, functions, records and sum types. Also, type declarations may contain a rule (which is simply a term, even though with the limitation that only pure functional terms are allowed inside rules). Finally, there are two types for describing behaviors; \texttt{UnsafeBehavior T} which represents an unsafe behavior which may never \texttt{yield}, and \texttt{SafeBehavior T} which represents a safe behavior which never loops indefinitely without \texttt{yield}-ing.

\begin{lstlisting}
TypeDef  ::= TypeDef'
           | TypeDef' :: Rule
          
TypeDef' ::= `()' | `int' | `float' | `Vector2' | ...
           | TypeDef $\times$ TypeDef
           | TypeDef $\rightarrow$ TypeDef
           | `{' Labels `}'
           | TypeDef $+$ TypeDef
           | Modifier TypeDef
           | `UnsafeBehavior' TypeDef
           | `SafeBehavior' TypeDef
           | [EntityName]
           
Labels   ::= Label; Labels
           | Label
           
Label    ::= [Name] `:' TypeDef
                     
Rule     ::= Term
\end{lstlisting}

A \texttt{Modifier} for a type definition allows to make a field mutable (\texttt{Rule} or \texttt{Var}), or to use queries to manipulate that field (\texttt{Table}). Also, another important modifier is \texttt{Ref} which can be seen as a programmer annotation that tells the compiler how a certain field is just a pointer to another portion of the state and as such it must not be processed recursively:

\begin{lstlisting}
Modifier  ::= Rule
            | Var
            | Table
            | Ref
\end{lstlisting}

Entities are definied as a series of type definitions with a name which can be referenced anywhere in the state; the last entity to be defined is the game state itself:

\begin{lstlisting}
EntityDefs  ::= `type' `GameState' = TypeDef
              | EntityDef
                EntityDefs

EntityDef   ::= `type' [EntityName] `=' TypeDef
\end{lstlisting}

The various entity names are simply replaced with their type definition in the remainder of the program, according to the $\llbracket \bullet \rrbracket_{\mathtt{MAIN}}$ translation rule:

\begin{mathpar}
\llbracket \mathtt{type\ EntityName\ =\ TypeDef;\ EntityDefs; Main} \rrbracket_{\mathtt{MAIN}} = 
\llbracket \mathtt{EntityDefs; \ Main} \rrbracket_{\mathtt{MAIN}} \mathtt{[EntityName} \mapsto \mathtt{TypeDef]}

\and 

\llbracket \mathtt{TypeDef;\ Main} \rrbracket_{\mathtt{MAIN}} = \mathtt{TypeDef;\ Main}
\end{mathpar}

The actual type definition of the state may be extracted from the program with the $\sigma(\bullet)$ function, which extracts the type definition and erases all the rules from it; this means that two entities may have the same type with different sets of rules. The function inductively removes all rules from a type declaration:

\begin{mathpar}
\sigma(\mathtt{Entity; EntityDefinitions}) = \sigma(\mathtt{EntityDefinitions})
\and 

\sigma(\mathtt{`type GameState = ' TypeDefinition}) = \sigma(\mathtt{TypeDefinition})
\and 

\sigma(\mathtt{T :: rule}) = \sigma(T)
\and 

\sigma(\mathtt{T}_1 \stackrel{\times}{\stackrel{+}{\mathtt{...}}} \mathtt{T}_2) = 
  \sigma(\mathtt{T}_1) \stackrel{\times}{\stackrel{+}{\mathtt{...}}} \sigma(\mathtt{T}_2)
\and 


\and 

\\ (...) \\
\end{mathpar}

A term can be a simple, ML-style functional term (we do not give all these possible definitions because they are fairly known) or an imperative behavior. Functional terms can read variables with the \texttt{!} operator and can use a Haskell-style table-comprehension syntax:

\begin{lstlisting}
Term        ::= `let' [Var] `=' Term 
                `in' Term
              | `letrec' [Var] `=' Term 
                `in' Term
              | `if' Term `then' 
                   Term 
                `else' 
                   Term
              | Term Term
              | !Term
              | `add' Term Term
              | ... (* other ML-style terms: fun, types, head, tail for tables, etc. *)
              | [ Term | Predicates ]
              | `{' Behavior `}'
              
Predicates  ::= $\epsilon$
              | [Var] `<-' Term, Predicates
              | Term, Predicates
\end{lstlisting}

A behavior defines an imperative coroutine that is capable of reading and writing the state and manipulating time. Behaviors can be freely mixed with terms. The simplest behavior simply returns a result with \texttt{return}. The result of a behavior can be plugged inside another behavior with \texttt{let!}, which behaves like a monadic binding operator. A variable can be modified inside a behavior with \texttt{:=} or \texttt{add}; a behavior can suspend itself until the next tick (\texttt{yield}) and it may read the current time with \texttt{time}.

Behaviors can be combined into more complex behaviors with a small set of combinators. A behavior may spawn another behavior with \texttt{run}, be executed in parallel with another behavior with $\vee$ or $\wedge$, be suspended until another behavior completes ($\Rightarrow$), be repeated indefinitely (\texttt{repeat}) and be forced to execute in a single tick (\texttt{atomic}):

\begin{lstlisting}
Behavior    ::= `return' Term
              | `let!' [Var] `=' Term
                `in' Term
              | Term := Term
              | `yield'
              | `time'
              | `run' Term
              | Term $\vee$ Term
              | Term $\wedge$ Term
              | Term $\Rightarrow$ Term
              | `repeat' Term
              | `atomic' Term
\end{lstlisting}

The main program is comprised of two terms: the initial state and the initial behavior:

\begin{lstlisting}
Main   ::= `let' state0 = Term
           `let' main = Term
\end{lstlisting}


\subsection{Type System}
Our language is strongly typed. We will omit some type declarations when obvious, and our language will make use of type inference. Typing rules for ML-style terms are the usual ML-style typing rules; for example:

\begin{mathpar}
\inferrule
  {\Gamma \vdash t_1:U \\\\ \Gamma,x:U \vdash t_2 : V}
  {\Gamma \vdash \mathtt{let}\ x \mathtt{ = } t_1\ \mathtt{in}\ t_2\ :V}
\quad \textsc {LET}

\and

\inferrule
  {\Gamma \vdash c:bool \\\\ \Gamma \vdash t_1 : T \\\\ \Gamma \vdash t_2 : T}
  {\Gamma \vdash \mathtt{if}\ c\ \mathtt{then}\ t_1\ \mathtt{else}\ t_2 : T}
\quad \textsc {IF}

\and

\inferrule
  {\Gamma \vdash r:Var\ T}
  {\Gamma \vdash !r : T}
\quad \textsc {VAR-GET}

\and

\inferrule
  {\Gamma \vdash r:Rule\ T}
  {\Gamma \vdash !r : T}
\quad \textsc {RULE-GET}

\and

\\ (...)

\end{mathpar}

Table comprehensions are types thusly:

\begin{mathpar}
\inferrule
  {\mathtt{decls(} \Gamma \mathtt{,ts)} \vdash t_1:T}
  {\Gamma \vdash \mathtt{[}\ t_1\ \mathtt{|}\ t_s\ \mathtt{]}\ :Table\ T}
\quad \textsc {TABLE}

\and

\mathtt{decls(} \Gamma \mathtt{,} \epsilon \mathtt{)} = \Gamma

\and

\inferrule
  {\Gamma \vdash t:Table\ T}
  {\mathtt{decls(} \Gamma \mathtt{,(x} \leftarrow \mathtt{t, ts))} = \mathtt{decls((} \Gamma \mathtt{,x:T),ts)}}

\and

\inferrule
  {\Gamma \vdash t:bool}
  {\mathtt{decls(} \Gamma \mathtt{,(t, ts))} = \mathtt{decls(} \Gamma \mathtt{,ts)}}

\and

\\ (...)

\end{mathpar}


\paragraph{Linearity}

Our type system needs to ensure that updating each entity of the state is a safe operation, that is the same entity will be updated \textit{exactly} once. The semantics function that we give further in this Section guarantees that all entities are updated \textit{at least} once, but duplicate entities may be updated twice or more. Updating an entity more than once is dangerous because it may lead to unexpected behaviors, but there is another downside to duplicates: with duplicates, rules are no more order-independent. With duplicates, the same entity may be subject to more than one rule, and thus the same entity may be modified twice in (possibly) irreconcilable ways.

We now show a few examples of how we may produce these situations.

A common error is duplication of a field either inside a behavior or at initialization time; for example, given a record type \texttt{T} with two fields \texttt{Position:Rule U} and \texttt{Velocity:Rule U} we may erroneously write:

\begin{lstlisting}
let r:Rule U = mk_cell ...

let (x:T) = 
  {
    Position = r
    Velocity = r
  }
\end{lstlisting}

Another common error is adding to a table an element from the table itself; for example, given a variable \texttt{t} of type \texttt{Rule Table T} we may write:

\begin{lstlisting}
t.add (t.head)
\end{lstlisting}

The final, common error we show is duplicating data with two symmetric rules:

\begin{lstlisting}
Asteroids$_1$
   : Rule(Table Asteroid) 
   :: fun self -> [a | a <- self.Asteroids$_1$ $\cup$ self.Asteroids$_2$, p$_1$ a]

Asteroids$_2$
   : Rule(Table Asteroid) 
   :: fun self -> [a | a <- self.Asteroids$_1$ $\cup$ self.Asteroids$_2$, p$_2$ a]
\end{lstlisting}

unless $p_1 = \neg p_2$ there will be duplicates between $\mathtt{Asteroids}_1$ and $\mathtt{Asteroids}_2$.

To solve this problem we use two techniques; first, we make \texttt{Rule} a \textit{linear} type. Second, we give rules read-only (\texttt{Ref}) access to the state, so that copying an external portion of the state into another field will require a deep cloning of that portion of the state and not just variable sharing.

We do not discuss all the typing rules required to enforce; instead we show some of the most significant:

\begin{mathpar}
\inferrule
  {\Gamma \vdash t : \mathtt{Ref}\ \{l_1:T_1; ... l_n:T_n\}}
  {\Gamma \vdash t.l_i \ :\ \mathtt{Ref}\ T_i}
\quad \textsc {FOREIGN-LOOKUP}

\and

\inferrule
  {\Linear(U) \\\
   \Gamma \vdash t_1 : U \\\\   
   (\Gamma \setminus \LFV(t_1)),x:U \vdash t_2 : V}
  {\Gamma \vdash \mathtt{let}\ x\ \mathtt{=}\ t_1\ \mathtt{in}\ t_2\ :\ V}
\quad \textsc {LINEAR-LET}

\and

\mathtt{where}

\Linear(T) = (\exists \alpha\ |\ \mathtt{Rule}\ \alpha\ \in \range(T))

\and

\NonLinear(T) = \neg \Linear(T)

\LFV(t) = \{(v:\alpha) | (v:\alpha) \in \FV(t)\ \wedge \Linear(\alpha)\}
\end{mathpar}

\paragraph{Behaviors}

We also state another informal restriction, that is function types may not have rules; so, a type such as $(U :: Rule) \rightarrow V$ is forbidden and generates a compile-time error.

The first typing rules for behaviors are the monadic typing rules which allow to build and consume basic behaviors:

\begin{mathpar}
\inferrule
  {\Gamma \vdash x:T}
  {\Gamma \vdash \mathtt{return}\ x :\ UnsafeBehavior\ T}
\quad \textsc {RETURN}

\and

\inferrule
  {\Gamma \vdash t_1 : b_1 \ U \\\\ \Gamma,x:U \vdash t_2 : b_2\ V}
  {\Gamma \vdash \mathtt{let!}\ x\ \mathtt{=}\ t_1\ \mathtt{in}\ t_2 : b_1 \sqcap b_2 \ V}
\quad \textsc {BIND}

\and

\mathtt{where}

SafeBehavior \sqcap \_ = SafeBehavior
\and
\_ \sqcap SafeBehavior = SafeBehavior
\and
UnsafeBehavior \sqcap UnsafeBehavior = UnsafeBehavior

\end{mathpar}

Behaviors may also be suspended for a tick (to wait for an application of all rules or to synchronize between behaviors, for example), they may read the current time (in fractional seconds) or they may spawn other behaviors:

\begin{mathpar}
\inferrule
  {\\}
  {\mathtt{yield} : SafeBehavior\ ()}
\quad \textsc {YIELD}

\and

\inferrule
  {\\}
  {\mathtt{time} : UnsafeBehavior\ float}
\quad \textsc {TIME}

\and

\inferrule
  {\Gamma \vdash t : SafeBehavior\ ()}
  {\Gamma \vdash \mathtt{run}\ t : UnsafeBehavior\ ()}
\quad \textsc {RUN}
\end{mathpar}

Behaviors are the only places where unrestricted modification of the state may happen; behaviors may indiscriminately write any portion of the state:

\begin{mathpar}
\inferrule
  {\Gamma \vdash t_1 : Var\ (Table\ U) \\\\ \Gamma \vdash t_2 : U}
  {\Gamma \vdash \mathtt{add}\ t_1\ t_2 : UnsafeBehavior\ ()}
\quad \textsc {VAR-TABLE-ADD}

\and

\inferrule
  {\Gamma \vdash t_1 : Rule\ (Table\ U) \\\\ \Gamma \vdash t_2 : U}
  {\Gamma \vdash \mathtt{add}\ t_1\ t_2 : UnsafeBehavior\ ()}
\quad \textsc {RULE-TABLE-ADD}

\and

\inferrule
  {\Gamma \vdash t_1 : Var\ U \\\\ \Gamma \vdash t_2 : U}
  {\Gamma \vdash t_1 \mathtt{:=} t_2 : UnsafeBehavior\ ()}
\quad \textsc {VAR-SET}

\and

\inferrule
  {\Gamma \vdash t_1 : Rule\ U \\\\ \Gamma \vdash t_2 : U}
  {\Gamma \vdash t_1 \mathtt{:=} t_2 : UnsafeBehavior\ ()}
\quad \textsc {RULE-SET}
\end{mathpar}

Conditionals on behaviors propagate the least safe of the two behavior types:

\begin{mathpar}
\inferrule
  {\Gamma \vdash c : bool \\\\ \Gamma \vdash t_1 : b_1\ U \\\\ \Gamma \vdash t_2\ :\ b_2\ U}
  {\Gamma \vdash \mathtt{if}\ c\ \mathtt{then}\ t_1\ \mathtt{else}\ t_2\ :\ b_1 \sqcup b_2 U}
\quad \textsc {IF-BEHAVIOR}

\mathtt{where}

UnsafeBehavior \sqcup \_ = UnsafeBehavior
\and
\_ \sqcup UnsafeBehavior = UnsafeBehavior
\and
SafeBehavior \sqcup SafeBehavior = SafeBehavior
\end{mathpar}

We also support a small behavior calculus. Two behaviors may be executed concurrently (the first one that terminates returns its result while the other behavior is discarded), or they may be executed in parallel (when both terminate their results are returned together). A behavior may also act as a guard ($\Rightarrow$) for another behavior, that is until the first behavior does not terminate with a result the second behavior is kept waiting. Finally, a behavior may be repeated indefinitely or it may be forced to run inside a single tick:

\begin{mathpar}
\inferrule
  {\Gamma \vdash t_1 : b_1\ U \\\\ \Gamma \vdash t_2 : b_2\ V}
  {\Gamma t_1 \vee t_2 : b_1 \sqcup b_2\ (U + V)}
\quad \textsc {CONCURRENT}

\and

\inferrule
  {\Gamma \vdash t_1 : b_1\ U \\\\ \Gamma \vdash t_2 : b_2\ V}
  {\Gamma t_1 \wedge t_2 : b_1 \sqcup b_2\ (U \times V)}
\quad \textsc {PARALLEL}

\and

\inferrule
  {\Gamma \vdash t_1 : SafeBehavior\ (U + ()) \\\\ \Gamma \vdash t_2 : U \rightarrow b\ V}
  {\Gamma t_1 \Rightarrow t_2 : b\ V}
\quad \textsc {GUARD}

\and

\inferrule
  {\Gamma \vdash t : SafeBehavior\ ()}
  {\Gamma \vdash \mathtt{repeat}\ t : SafeBehavior\ ()}
\quad \textsc {REPEAT}

\and

\inferrule
  {\Gamma \vdash t :\ b ()}
  {\Gamma \vdash \mathtt{atomic}\ t : UnsafeBehavior \ ()}
\quad \textsc {ATOMIC}
\end{mathpar}

The typing rules on behaviors force them to be written in such a way as to guarantee statically that no behavior will run indefinitely without yielding. This is seen easily because the only two operators capable of looping are \texttt{repeat} and $\Rightarrow$, and both require a \texttt{Safebehavior} as input. 

We now show a small list of dangerous behaviors that are forbidden by our type system:

\begin{lstlisting}
repeat { r := v }

repeat {if cond then r := v else yield}

{ return None } => fun () -> yield
\end{lstlisting}

The inability to eliminate a behavior unless inside another behavior is important because it allows us to force rules to not contain behaviors; thanks to this limitation we can ensure that the execution of rules may only read from the state and never write to it, and so rules can be made to behave as if they are executed simultaneously without risking complex interdependencies. This simplifies many instances of game programming; for example, consider the rules seen in the example at the beginning of the section:

\begin{lstlisting}
Asteroids           
    : Rule(Table Asteroid)
    :: \asteroids -> [a | a <- asteroids && a.Y > 0]
  	    
DestroyedAsteroids	
    : Rule<int>
    :: \(state,destroyed_asteroids) -> destroyed_asteroids + count([a | a <- !state.Asteroids && a.Y <= 0])
\end{lstlisting}

If rules are executed sequentially from top to bottom, then when an asteroid is eliminated that same asteroid will not be available anymore when computing the number of asteroids waiting for deletion.


\subsection{Accepted Programs}

To determine if a game is correct or not we require for it to pass a static analyisis which is comprised of two steps: its state definition must be well-formed (the $\wf(\bullet)$ function) and its \texttt{main} must be typed correctly.

The $\wf(\bullet)$ function requires that a state definition's rules are typed correctly:

\begin{mathpar}
\wf(\mathtt{type [EntityName] = TypeDef;\ EntityDefinitions}) = \wf(\mathtt{TypeDef}, \mathtt{EntityName}) \wedge \wf(\mathtt{EntityDefinitions})
\and

\wf(\mathtt{Primitive}, \mathtt{EntityName}) = \mathtt{true}
\and

\wf(\mathtt{CompositeType(Types)}, \mathtt{EntityName}) = \bigwedge_{\mathtt{T} \in \mathtt{Types}}(\wf(\mathtt{T}, \mathtt{EntityName}))
\and

\wf(\mathtt{Rule\ T\ ::\ rule}) = \wf(\mathtt{T}, \mathtt{EntityName}) \wedge 

\inferrule
  {}
  {\vdash \mathtt{rule}\ : rule-fun(T,EntityName) }

\and

\wf(\mathtt{Var\ T}, \mathtt{EntityName}) = \wf(\mathtt{T}, \mathtt{EntityName})
\and

\wf(\mathtt{Table\ T}, \mathtt{EntityName}) = \wf(\mathtt{T}, \mathtt{EntityName})
\and

\wf(\mathtt{Foreign\ T}, \mathtt{EntityName}) = \mathtt{true}

\and

\\\\
\mathtt{where}

\rule-fun(T,\mathtt{EntityName}) = \mathtt{Foreign(GameState)} \times \mathtt{Foreign(EntityName)} \times \mathtt{T} \times \mathtt{float} \rightarrow \mathtt{T}
\end{mathpar}

Given a game source:
\begin{lstlisting}
StateDefinition

let state0 = $t_1$
let main   = $t_2$
\end{lstlisting}

its compilation succeeds if the following is satisfacted:

\begin{mathpar}
\wf(\mathtt{StateDefinition}) \wedge \\
   \vdash t_1 : \sigma(\mathtt{StateDefinition}) \wedge \\
   \vdash t_2 : \sigma(\mathtt{StateDefinition}) \rightarrow \mathtt{SafeBehavior()}
\end{mathpar}


\subsection{Semantics}

We now define the semantics function of a Casanova program.

We start by defining our memory model. Our memory is defined as a map from rules and variables into values:

\begin{lstlisting}
m = $\epsilon$
  | m[r $\rightarrow$ v]
  | m[r $\Rightarrow$ v]
\end{lstlisting}

\texttt{m[r $\rightarrow$ v]} means that \texttt{r}, which has type \texttt{Var T} or \texttt{Rule T} has value \texttt{v}, while \texttt{m[r $\Rightarrow$ v]} means that \texttt{r}, which has type \texttt{Rule T}, has a pending assignment of value \texttt{v}. The execution of rules does not modify assignments of the form \texttt{m[r $\rightarrow$ v]}, and it will only add assignments of the form \texttt{m[r $\Rightarrow$ v]}. After all rules have been executed, then we use the compacting function $\oplus$:

\begin{lstlisting}
$\oplus$($\epsilon$) = $\epsilon$
$\oplus$(m[r $\rightarrow$ v]) = ($\oplus$m)[r $\rightarrow$ v]
$\oplus$(m[r $\Rightarrow$ v]) = ($\oplus$m)[r $\rightarrow$ v]
\end{lstlisting}

The \texttt{!} operator (valid on both rules and variables) is defined as:

\begin{lstlisting}
!r (m[r' $\rightarrow$ v]) = if r = r' then v else !r m
!r (m[r' $\Rightarrow$ v]) = !r m
\end{lstlisting}

variables may never be null, since there is no way of declaring a variable without initializing it at the same time; this way we are assured that variable lookups will always succeed in finding a value.

The \texttt{:=} operator (valid on both rules and variables) is defined as:

\begin{lstlisting}
(r := v) m = m[r $\rightarrow$ v]
\end{lstlisting}

Notice that \texttt{:=} cannot be used inside the body of a rule, thanks to the limitation that the type system imposes that unrestricted assignments can only happen inside behaviors.

We can now define the \texttt{update\_rules} function, which builds the program that will run all the rules; this program takes as input the memory and after each imperative statement it returns the modified memory. The \texttt{update\_rules} function inductively processes the state and applies each rule it finds. The function does not recursively process those portions of the state with type \texttt{Foreign T} for some \texttt{T}.

The \texttt{update\_rules} function traverses all entity definitions starting from the beginning of the program; it also invokes the compacting function $\oplus$ to apply all the pending changes caused by all the rule executions:

\begin{lstlisting}
update_rules (StateDefinition; Main) m =
  $\oplus$(update_state StateDefinition m)

update_state (type EntityName = TypeDef; EntityDefinitions) =
  entities_update[(update_state EntityName EntityName) $\mapsto$ entity_update]
  where
    entities_update = update_state EntityDefinitions
  and
    entity_update = update_entity TypeDef TypeDef
    
update_state (type GameState = TypeDef) =
  update_entity TypeDef TypeDef
\end{lstlisting}

The \texttt{update\_state} function inductively processes type definitions; on each type definition it goes in search for rules and applies those rules with the \texttt{update\_entity} function:

\begin{lstlisting}
update_entity E T m (s,(e:E),(t:T),dt) = m (* when T is a primitive type such as (), int, ... *)

update_entity E (U $\rightarrow$ V) m (s,e,f,dt) = m

update_entity E (SafeBehavior T) m (s,e,f,dt) = m

update_entity E (UnsafeBehavior T) m (s,e,f,dt) = m

update_entity E (Foreign T) m (s,e,f,dt) = m

update_entity E (U $\times$ V) m (s,e,(u,v),dt) = 
  update_entity E U (update_entity E V m (s,e,v,dt)) (s,e,u,dt)

update_entity E (U $+$ V) m (s,e,(Left u),dt) = update_entity E U m (s,e,u,dt)

update_entity E (U $+$ V) m (s,e,(Right v),dt) = update_entity E V m (s,e,v,dt)

update_entity E ({l1:T1,...,ln:Tn}) m (s,e,r,dt) = 
  update_entity E T1 (... (update_entity E T1 m (s,e,r.ln,dt)) ... ) (s,e,r.l1,dt)

update_entity E ([EntityName]) m (s,e,e',dt) = (update_entity [EntityName] [EntityName]) m (s,e',e',dt)

update_entity E (Var T) m (s,e,r,dt) = update_entity E T m (s,e,!r m,dt)

update_entity E (Table T) m (s,e,t,dt) = [update_entity E T m (s,e,x,dt) | x $\leftarrow$ t]

update_entity E (Rule T :: rule) m (s,e,r,dt) = 
  update_entity E T (m[r $\Rightarrow$ ($\llbracket \mathtt{rule} \rrbracket_I$ m (s,e,!r m,dt)]) (s,e,!r m,dt)
\end{lstlisting}

Where $\llbracket \bullet \rrbracket_I\ \mathtt{m}$ is the well-known semantics of the pure lambda calculus, augmented with the rule that:

$\llbracket !r \rrbracket_I\ \mathtt{m}$ \texttt{ = !r m}.

Behavior semantics is simpler than rule semantics. We update a list of behaviors, where each behavior may modify the memory or run new behaviors. Each behavior may also perform regular functional computations which are processed according to $\llbracket \bullet \rrbracket_I \mathtt{m}$.

We define the \texttt{update\_behaviors} function which folds the \texttt{update\_behavior} over all behaviors; each single behavior update modifies the memory and returns a set of behaviors to run at the next update:

\begin{lstlisting}
update_behaviors ({b1,...,bn},m) =
  ((b1',bs'),m'')
  where 
    (),b1',m' = update_behavior m b1
  and
    bs',m'' = update_behaviors ({b2,...,bn},m')
\end{lstlisting}

The \texttt{update\_behavior} function executes a behavior until the next \texttt{yield}, immediately writing assignments to the memory; we omit the definition of this function for regular pure functional terms, and instead give it only for behavior terms:

\begin{lstlisting}
update_behavior m (return v) = v,{},m

update_behavior m (let! _ = yield in t) = (),(t),m

update_behavior m (let! _ = (run b) in t) = 
  v,(b,b'),m'
  where
    v,b',m' = update_behavior m (t)

update_behavior m (let! _ = (r := v) in t) = 
  v,b,m'
  where
    v,b,m' = update_behavior (m[r $\rightarrow$ v]) (t)

update_behavior m (let! x = !r in t) = 
  v,b,m'
  where
    v,b,m' = update_behavior (($\lambda$x . t) (!r m)) m

update_behavior m (let! x = !r in t) = 
  v,b,m'
  where
    v,b,m' = update_behavior (($\lambda$x . t) (!r m)) m

update_behavior m (let! x = t1 in t2) = 
  update_behavior (($\lambda$x . t2) t1) m
\end{lstlisting}

The two update functions, one for rules and one for behaviors are combined into the final update function which takes as input the game state, the set of current behaviors and the current memory and which returns the updated behaviors and memory. The state never changes, but the memory it points to may:

\begin{lstlisting}
update Program state bs m =
  bs',m''
  where
    bs',m' = update_behaviors (bs,m)
  and
    m'' = update_rules Program m' state
\end{lstlisting}


\subsection{Correctness}
We have stated in Sec. \ref{sec:game_model} that a correct game according to our model respects the following rules:
\begin{enumerate}
\item all rules of each entity are applied exactly once
\item rule application is order-independent
\item tick always terminates
\end{enumerate}

We will now briefly discuss why our system forces games to respect these rules:
\begin{enumerate}
\item is respected because the semantics function explores the entire state recursively and applies its rules making sure that each rule is applied at least once; the linearity of the \texttt{Rule} type makes sure that the same cell is never processed more than once with one or more rules
\item is respected because rule application cannot access the result of already completed rules because those results are stored in memory as $m[r \Rightarrow v]$ but each rule may only read memory cells of the form $m[r \rightarrow v]$; also, the linearity of the \texttt{Rule} type ensures that no two rules write the same cell with $m[r \Rightarrow v]$, because all $r$'s are distinct
\item is respected because the state is not cyclic (apart from \texttt{Foreign} declarations, which are not processed recursively) and because only behaviors with type \texttt{YieldBehavior} can be run, that is accepted behaviors never run indefinitely without yielding
\end{enumerate}

