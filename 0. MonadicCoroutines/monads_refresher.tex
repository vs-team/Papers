%%%%%%%%%%%%%%%%%%%%%%%%%%%%%%%%%%%%%%%%%%%%%%%%%%%%%%%%%%
% monads refresher.tex
%%%%%%%%%%%%%%%%%%%%%%%%%%%%%%%%%%%%%%%%%%%%%%%%%%%%%%%%%%

A monad \cite{MOGGI_MON,DECL_IMP,COMPR_MON,EFF_MON} is a kind of
abstract data type used to represent computations (instead of data in
the domain model). Monads allow the programmer to chain actions
together to build a pipeline; the monad stores and represents
additional processing rules which decorate each action. Monads in the
literature have been used to model stateful programs, to manipulate
sequences , asynchronous computations and many more; for example
implementations see (\cite{CSHARP_ASYNC} \cite{CSHARP_LINQ}). 

Formally, a monad is constructed by defining a type constructor $M$
and two operators (\texttt{bind} and \texttt{return}) that must
fulfill several properties to allow the correct composition of monadic
functions (i.e. functions that use values from the monad as their
arguments). The \texttt{return} operation takes a value from a plain
type $\alpha$ and puts it into a monadic container of type $M\ \alpha$. The
\texttt{bind} operation performs the reverse process, extracting the
original value from the container and passing it to the associated
next function in the pipeline. 

As a simple example, let us consider the simple ``maybe monad'', which
represents computations which may fail depending on runtime
conditions. The data constructor $M$ is the standard $\mathtt{Maybe}$
type defined below: 
\begin{lstlisting}
type Maybe<'a> = Some of 'a | None
\end{lstlisting}

In F\# we define a helper class (called a ``monad builder'') that has
the bind and return operators as methods. In the case of the maybe
monad, we represent operations that can either hold some result or no
result in case of error. Creating such an operation from a value $x$
simply packs $x$ into $\mathtt{Some}\ x$; binding an operation $p$ with another
operation $k$ \textit{which depends on the result of $p$} requires us
to inspect $p$; if $p$ contains some result $r$, then we return $k\
r$; otherwise we cannot invoke $k$ because $p$ gave an error, and so
we propagate the error by returning $\mathtt{None}$: 

\begin{lstlisting}
type MaybeBuilder() =
  member this.Bind(p:Option<'a>, k:'a->Option<'b>):Option<'b> =
    match p with
    | None -> None
    | Some r -> k r
  member this.Return(x:'a):Option<'a> = Some x

let maybe = MaybeBuilder()
\end{lstlisting}

F\# introduces syntactic sugar to hide the invocation of the bind and
return methods. In particular, writing: 

\medskip
\begin{lstlisting}
let maybe_sum (t1:Option<int>) (t2:Option<int>):Option<int> =
  maybe{
    let! x = t1
    let! y = t2
    return x+y
  }
\end{lstlisting}

is equivalent to writing:
\begin{lstlisting}
maybe.Bind(t1, fun x ->
  maybe.Bind(t2, fun y ->
    maybe.Return(x+y)))
\end{lstlisting}

The small sample above tries to evaluate $\mathtt{t1}$ and
$\mathtt{t2}$; if any of those returns an error, then the entire
computation returns an error. If both $\mathtt{t1}$ and
$\mathtt{t2}$ succeed, then the sum of their results is returned.