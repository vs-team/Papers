%%%%%%%%%%%%%%%%%%%%%%%%%%%%%%%%%%%%%%%%%%%%%%%%%%%%%%%%%%
% conclusions.tex
%%%%%%%%%%%%%%%%%%%%%%%%%%%%%%%%%%%%%%%%%%%%%%%%%%%%%%%%%%

Scripts are an important and pervasive aspect of computer games. Scripts simplify the interaction with computer game engines to the point that a designer or an end-user can easily customize gameplay. Scripting languages must support coroutines because these are a very recurring pattern when creating gameplay modules. Scripts should be fast at runtime because games need to run at interactive framerates. Finally, the scripting runtime should be as modular and as programmable as possible to facilitate its integration in an existing game engine.

In this paper we have shown how to use meta-programming facilities (in particular monads) in the functional language F\# to enhance the existing scripting systems which are based on Lua, the current state of the art, in terms of speed, safety and extensibility. We have also shown how having a typed representation of coroutines promotes building powerful libraries of combinators that abstract many common patterns found in scripts. As evidence of the capabilities of our proposed system we have outlined a series of applications of our scripts into an actual game that is under development.
