%%%%%%%%%%%%%%%%%%%%%%%%%%%%%%%%%%%%%%%%%%%%%%%%%%%%%%%%%%
% scripting_in_games.tex
%%%%%%%%%%%%%%%%%%%%%%%%%%%%%%%%%%%%%%%%%%%%%%%%%%%%%%%%%%

In this section we briefly illustrate a (heavily) simplified game architecture in order to describe the main problems that a scripting system faces when introduced inside a game engine.

A game engine is based on three fundamental components: 
({\em i}) the game state, which is a snapshot of the game world
and includes a description of all the various entities of the game; 
({\em ii}) the update  function, which computes the next value of the
game state, and ({\em iii}) the draw/rendering function, which draws
the game state to the screen.

The game loop is the code that defines how the game is run; it is a
recursive function that continuously invokes the update and draw
functions with the current state as input. The game loop also computes
the time delta between iterations, so that the update function will be
able to adjust its computations to cover the actual amount of time
elapsed between its invocations: 

\begin{lstlisting}
let run_game (game:Game) t =
  let t' = get_time()
  do game.Update (game.State, t'-t, t)
  do game.Draw (game.State)
  do run_game game t'
\end{lstlisting}

Central to our present concerns is the update function, which
implements all the functionalities that modify the game state. As 
discussed earlier, most of these functionalities, typically the
physics of the various entities, such as forces, 
collision detection, $\dots$ etc, the interaction with the
input/output and other devices are coded in low-level languages such
as C or C++ to guarantee the fast framerates  needed for a smooth play
experience. On the other hand, higher-level aspects of the game,
related to gameplay, are typically left outside the code of the 
update function, and made scriptable.  

The most important function of the scripts is to model the behaviors
of the computer characters and of the other in-game objects. To
illustrate, the following pseudo-code describes the behavior of  
a prince in a Role Playing Game: 

\begin{lstlisting}
prince:
  princess = find_nearest_princess()
  walk_to(princess)
  save(princess)
  take_to_castle(princess)
\end{lstlisting}

The main problem in coding this behavior with a script is to achieve a
smooth interaction between the discrete-time structure of the game
animation implemented by the simulation engine, and the behavior
implemented by the script, which spans multiple time slots of the
simulation engine. Specifically, in order to guarantee a smooth user
experience, each such script must be interruptible, so that at 
each discrete step of the simulation engine the script performs a
finite number of transitions and then suspend itself: failing to do 
so would slow down the simulation steps, hence the resulting framerate
of the game would decrease, thereby reducing the player immersion. 

The problem is sometimes addressed by coding scripts as state machines
(SMs), whose execution gets interrupted at each state transition. 
However, while SMs represent a viable design choice for simple
scripts, they are far less effective for modelling objects with
complex behavior, as their structure grows easily out of control and
becomes rather hard to maintain. Modern scripting languages adopt 
\textit{coroutines} as a mechanism to build state machines
implicitly, by way of their (the coroutines') built-in mechanisms to
suspend and resume execution. With coroutines the 
code for a SM is written ``linearly'' one statement after another, but
each action may suspend itself (an operation often called ``yield'')
many times before completing. The local state of the state machine
is stored in the continuation of the coroutine. Some of the most used 
scripting languages, which are Lua, Python and C\#, all offer some
suspension mechanisms similar to coroutines that game developers use
for scripting; for a detailed discussion of couroutines in these
languages, see \cite{PYTHON_COROUTINES,LUA_COROUTINES,CSHARP_YIELD}. 

\begin{comment}
\subsection{Coroutines in action}
In the remainder of this section we analyze coroutines in action in Lua. We will also briefly discuss how coroutines are emulated in Python and C\# with generators. For a more detailed discussion of the mechanisms of coroutines in Lua, Python and C\# see \cite{PYTHON_COROUTINES,LUA_COROUTINES,CSHARP_YIELD}.

Lua coroutines are based on the three functions \texttt{coroutine.yield, coroutine.resume} and \texttt{coroutine.create} which respectively pause execution of a coroutine, resume execution of a paused coroutine and create a coroutine from a function.

\begin{lstlisting}
function walk_to(self,target)
  return coroutine.create(
    function()
      while(dist(self,target) > self.reach) do
        self.Velocity = towards(self, target)
        coroutine.yield()
      end
    end)
end

...

function prince(self)
  return coroutine.create(
    function()
      princess = bind_co(find_nearest_princess(self))
      bind_co(walk_to(self, princess))
      bind_co(save(self, princess))
      bind_co(take_to_castle(self, princess))
    end)
end
\end{lstlisting}

Notice that to invoke a coroutine we need to explicitly bind it with the \texttt{bind\_co} function (we do not show here the source for reasons of space), which keeps resuming a coroutine until it yields for the last time, and then it returns the resulting value.

A mechanism to implement coroutines in Python makes use of generators. A generator is a special routine that returns a sequence of values. However, instead of building an array containing all the values and returning them all at once, a generator yields the values one at a time; yielding effectively suspends the execution of the generator until the next element of the sequence is requested by the caller. Python generators may appear as a way to return lazy sequences but they are powerful enough to implement coroutines. We can adopt the convention that a coroutine is actually a generator which yields a sequence of null (\texttt{None}) values until it is ready to return; the returned value will be yielded last.

\begin{lstlisting}
def walk_to(self,target):
  while(dist(self,target) > self.Reach):
    self.Velocity = towards(self,target)
    yield

...

def prince(self):
  for princess in find_nearest_princess(self):
    yield
  for x in walk_to(self,princess):
    yield
  for x in save(self,princess):
    yield
  for x in take_to_castle(self,princess):
    yield
\end{lstlisting}

As in Python, C\# supports generators. Since C\# is statically typed, we need to assign a type to our coroutines. We have two alternatives; a coroutine that returns nothing (\texttt{void}) has type \texttt{IEnumerable}, that is it returns a sequence of \texttt{Object}s that are all null (a similar strategy is used by Unity, even though with unsafe casts \cite{UNITY_YIELD}) and we can type a coroutine that returns a value of type \texttt{T} as \texttt{IEnumerable<T?>}, where \texttt{T?} is either \texttt{null} or an instance of \texttt{T}.

We omit the C\# sample for brevity, and also because of its similarity with Python. Moreover, when compared with coroutines, generators are quite cumbersome in a scripting language and indeed LUA is by far more used in games.

In the remainder of the paper we will present a different approach to coroutines, namely building a meta-programming abstraction (called \textit{monad}) to implement coroutines in F\#. We will discuss how our approach produces code which is faster and shorter than similar implementations in Lua, Python and C\#. We will also discuss how our approach is very customizable, thanks to the fact that coroutines are not \textit{wired} into the language runtime but rather we have defined them with our monad. Monads simplify the use of coroutines, making them completely invisibile to the user. Also, thanks to type inference the resulting scripts require no typing annotations. 
Finally (see Section \ref{sec:benchmarks} for the details), our system offers a good runtime performance and is type safe; this makes it suitable for large and complex scripts.
\end{comment}
