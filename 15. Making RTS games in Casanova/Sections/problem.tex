Implementing an RTS requires writing code for all of the game common elements such as: units, battles, movement, production, resources gathering, statistics, etc.

We identify the common game elements of an RTS by mean of a taxonomy \cite{abbadi2014resource}. In this paper the authors introduce a design pattern called \textit{REA} (resource, entity, action) for representing RTS's. In particular the design effectively describes any RTS game in terms of:
\begin{itemize}[noitemsep,nolistsep]
\item \textit{Resource}, which is any kind of game statistic. A statistic might represent a numerical value of a battle, or the cost to deploy a facility, etc.
\item \textit{Entity}, which is any kind of game element that contains resources. We distinguish different entities by mean of their interaction.
\item \textit{Action}, which describes an interaction, is used to specialize an entity. Precisely it describes the flow of resources among entities.
\end{itemize}


\noindent
\newline
Whereas the definition of action given above covers generic type of interaction (like the attack of a ship, or the percentage of construction, etc) a special attention should be given to some kind of actions that are common to all RTS's. We identified these special actions in terms of: \textit{creation}, \textit{deletion}, and \textit{strategy update}:

\noindent
\newline
\textbf{Creation}
An entity is crated after some conditions in the game world are met. A condition could be for example the player who decides to create a fleet to attack an enemy player; an automated spawner that after a certain amount of time creates a unit, etc. Furthermore, the creation of an entity typically consumes some game resources of the player. If the resources are not enough then the creation will be postponed or not allowed at all. 

\noindent
\newline
\textbf{Deletion}
Analogous to creation, an entity is deleted after some conditions in the game are met. A condition could be for example during a battle the life of the entity is lower or equal to zero. Entities removed from the game world are not able to interact with other entities.

\noindent
\newline
\textbf{Update}
During the life time of an entity often happens in an RTS that the entity changes its behavior. For example a resource gatherer unity mainly collects resources from all around the world, but if necessary it can also attack; a fleet moving around the world might eventually end up in the local fleets of a planet or take part of a battle. All the just mentioned actions differ from each other, indeed their logics affect different set of resources even though the entity remains the same.

Next, we discuss how the just described model effectively describes an RTS. 