Galaxy wars (GW) is an RTS game published in 2012 inspired by the popular board game Risk. Galaxy wars has been used as a case study in related research \cite{papers_galaxy_wars}. The gameplay revolves around your strategic choices, where timing, battles, and resource management are key elements to prevail against the opponents.

The map field is a connected graph where nodes represent planets and links are the physical connection between planets. Planets produce fleets that are controllable by the players. A fleet can either defend its containing planet or move battle against other planets. Fleets move only through links. To every player global statistics are assigned. Global statistics influence fleets attack/defence capabilities as well as fleets speed and planets production. Few special planets feature special ability to update the global statistics of their owners.

\subsection{Galaxy Wars with REA}
We show an informal idea of implementation of Galaxy Wars with the REA pattern. The elements of Galaxy Wars that follow the REA pattern are: \textit{fleet}, \textit{planet}, \textit{statistic}, and \textit{link}. Resources are \textit{statistics}, the entities are \textit{fleets}, \textit{planets}, and \textit{links}. The possible actions are movement, fight, and upgrade. In GW most of the entities are static. An entity that can be created and deleted is fleet. A fleet is spawned after a player decides to send some units to a planet. A fleet is disposed after either it has reached its destination, or it has lost a battle. Moreover, the fleet entity is the only entity which might change its strategy/behavior during its lifetime (a fleet can either travel along a link or fight in a battle).

In the next section we show how to implement the REA pattern for Galaxy Wars in Casanova 2.