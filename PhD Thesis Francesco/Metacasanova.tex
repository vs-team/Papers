\documentclass[9pt,a5paper,openright]{extbook}
\usepackage[utf8]{inputenc}
\usepackage[english]{babel}
\usepackage{amsthm}
\usepackage{amsmath}
\usepackage{amsfonts}
\usepackage{amssymb}
\usepackage{enumitem}
\usepackage{listings}
\usepackage{graphicx}
\usepackage{epigraph}
\usepackage{comment}
\usepackage{mathpartir}
\usepackage{subcaption}
\usepackage{multirow}
\usepackage{pdfpages}
\usepackage{colortbl}
\usepackage[chapter]{algorithm}
\usepackage{algpseudocode}
\usepackage{algorithmicx}
\usepackage{float}
\usepackage{rotating}
\usepackage{balance}
\usepackage[font=small,labelfont=bf]{caption}
\usepackage[top=2cm,bottom=1.5cm,left=2cm,right=2.5cm]{geometry}
\usepackage[pdfa]{hyperref}
\author{Francesco Di Giacomo}
\title{Metacasanova: a High-performance Meta-compiler for Domain-specific Languages}
\date { }

\newcommand{\tu}{\textunderscore}
\newcounter{rqCounter}
\setcounter{rqCounter}{1}

\newcommand{\researchQuestion}[1]{
  \textbf{Research question \therqCounter:} \textit{#1}
  \setcounter{rqCounter}{\therqCounter + 1}
}

\newcommand{\problemStatement}[1]{
  \textbf{Problem statement:} \textit{#1}
}

\newcommand{\psContent}{To what extent does a meta-compiler benefit the development of a domain-specific language for game development?}

\newcommand{\rqContentOne}{To what extent can a meta-compiler reduce the amount of code required to create a compiler for a domain-specific language for game development?}

\newcommand{\rqContentTwo}{How much is the performance loss introduced by the meta-compiler with respect to an implementation written in a language compiled with a traditional compiler and is this loss acceptable when considering game development?}

\newcommand{\rqContentThree}{What is the cause of the performance degradation when employing a meta-compiler and how can this be improved?}

\definecolor{scheduleGreen}{RGB}{1,168,1}

\lstset
{
	frame = single,
	breaklines = true,
	showstringspaces = false,
	basicstyle = \ttfamily\small,	
	tabsize=2,
	captionpos=b
}

\theoremstyle{definition}
\newtheorem{definition}{Definition}[chapter]

\newcommand{\AlgError}[1]{\State $\text{\textbf{error:} #1}$}
\newcommand{\AlgLet}[2]{\State \textbf{let} #1 \textbf{be} #2}

\hyphenation{ca-sa-no-va}
\hyphenation{me-ta-ca-sa-no-va}

\clubpenalty = 10000
\widowpenalty = 10000
\displaywidowpenalty = 10000

\makeindex
\begin{document}
\cleardoublepage\includepdf{Extra/cover_venice.pdf}
\thispagestyle{empty}
\frontmatter
\pagenumbering{gobble} %remove numbering
\maketitle
%\thispagestyle{empty}
\chapter*{Abstract}
\thispagestyle{empty}

Programming languages are at the foundation of computer science, as
they provide abstractions that allow the expression of the logic of a program independent
from the underlying hardware architecture. In particular scenarios,
it can be convenient to employ Domain-Specific Languages, which are
capable of providing an even higher level of abstraction to solve problems
which are common in specific domains. Examples of such domains are database programming, text editing,
3D graphics, and game development. The use of a domain-specific
language for the development of particular classes of software
may drastically increase the development speed and the maintainability
of the code, in comparison with the use of a general-purpose
programming language. While the idea of having a domain-specific language for a particular domain
may be appealing, implementing such a language tends to come at a heavy cost: as it is common to all programming
languages, domain-specific languages require a compiler which translates their
programs into executable code. Implementing
a compiler tends to be an expensive and time-consuming task, which may very well be a burden
which overshadows the advantages of having a domain-specific language.

To ease the process of developing
compilers, a special class of compilers called ``meta-compilers'' has been
created. Meta-compilers have the advantage of requiring only the definition of a language in order to
generate executable code for a program written in that language, thus
skipping the arduous task of writing a hard-coded compiler for the new language. A disadvantage of
meta-compilers is that they tend to generate slow executables, so they are usually only employed
for rapid prototyping of a new language. The main aim of this thesis is to create a meta-compiler
which does not suffer from the disadvantage of inefficiency. It presents a meta-compiler called
``Metacasanova'', which eases the development cost of a compiler while simultaneously
generating efficient executable code.

The thesis starts by analysing the recurring
patterns of implementing a compiler, to define a series of requirements for Metacasanova.
It then explains the architecture of the meta-compiler and provides examples of its
usage by implementing a small imperative language called C-{}-, followed by the
reimplementation of a particular, existing domain-specific language, namely Casanova,
which has been created for use in game development. The thesis presents a novel way to
optimize the performance of generated code by means of functors; it demonstrates the
effect of this optimization by comparing the efficiency of Casanova code generated with and without it.
Finally, the thesis demonstrates the advantages of having a meta-compiler like Metacasanova,
by using Metacasanova to extend the semantics of Casanova to allow the definition of multiplayer online games.
\newpage
%\thispagestyle{empty}
\tableofcontents

\mainmatter
\pagenumbering{arabic}
\chapter{Introduction}
\label{ch:introduction}
\epigraph{About the use of language: it is impossible to sharpen a pencil with a blunt axe. It is equally vain to try to do it with ten blunt axes instead.}{Edsger Dijkstra}
The number of programming languages available on the market has dramatically increased during the last years. The tiobe index \cite{tiobe2018}, a ranking of programming languages based on their popularity, lists 50 programming languages for 2018. This number is only a small glimpse of the real amount, since it does not take into account several languages dedicated to specific applications. This growth has brought a further need for new compilers that are able to translate programs written in those languages into executable code. The goal of this work is to investigate how the development speed of a compiler can be boosted by employing meta-compilers, programs that generalize the task performed by a normal compiler. In particular the goal of this research is creating a meta-compiler that significantly reduces the amount of code needed to define a language and its compilation steps, while maintaining acceptable performance.

This chapter introduces the issue of expressing the solution of problems in terms of algorithms in Section \ref{sec:ch1_algorithms}. Then we proceed by defining how the semi-formal definition of an algorithm must be translated into code executable by a processor (Section \ref{sec:ch1_programming_languages}). In this section we discuss the advantages and disadvantages of using different kinds of programming languages with respect to their affinity with the specific hardware architecture and the scope of the domain they target. In Section \ref{sec:ch1_compilers} we explain the reason behind compilers and we explain why building a compiler is a time-consuming task. In Section \ref{sec:ch1_metacompilers} we introduce the idea of meta-compilers as a further step into generalizing the task of compilers. In this section we also explain the requirements, benefits, and the relevance as a scientific topic. Finally in Section \ref{sec:ch1_problem_statement} we formulate the problem statement and the research questions that this work will answer.

\section{Algorithms and problems}
\label{sec:ch1_algorithms}
Since the ancient age, there has always been the need of describing the sequence of activities needed to perform a specific task \cite{barbin2012history}, to which we refer with the name of \textit{Algorithm}. The allegedly most ancient known example of this dates back to the Ancient Greek, when Hero invented an algorithm to perform the factorization and the approximation of the square root, discovered also by other civilizations \cite{ bailey2012ancient, smith1923history} . Regardless of the specific details of each algorithm, one needs to use some kind of language  to define the sequence of steps to perform. In the past people used natural language to describe such steps but, with the advent of the computer era, the choice of the language has been strictly connected with the possibility of its implementation. Natural languages are not suitable for the implementation, as they are known to be verbose and ambiguous \cite{church1982coping, resnik1999semantic}. For this reason, several kind of formal solutions have been employed, which are described below.

\subsubsection*{Flow charts}
A flow chart is a diagram where the steps of an algorithm are defined by using boxes of different kinds, connected by arrows to define their ordering in the sequence. The boxes are rectangular-shaped if they define an \textit{activity} (or processing step), while they are diamond-shaped if they define a \textit{decision}. A rectangle with rounded corners denotes the initial step. An example of a flow chart describing how to sum the numbers in a sequence is described in Figure \ref{fig:ch1_flow_chart}.

\begin{figure}
	\centering
	\includegraphics[width = \textwidth]{Figures/flow_chart}
	\caption{Flow chart for the sum of a sequence of numbers}
	\label{fig:ch1_flow_chart}
\end{figure}

\subsubsection*{Pseudocode}
Pseudocode is a semi-formal language that might contain also statements expressed in natural language and omits system specific code like opening file writers, printing messages on the standard output, or even some data structure declaration and initialization. It is intended mainly for human reading rather than machine reading. The pseudocode to sum a sequence of numbers is shown in Algorithm \ref{alg:ch1_pseudocode}.

\begin{algorithm}
	\caption{Pseudocode to perform the sum of a sequence of integer numbers}
	\label{alg:ch1_pseudocode}
	\begin{algorithmic}
		\Function{SumIntegers}{$l \text{ list of integers}$}
			\State $sum \gets 0$
			\ForAll {$x \text{ in } l$}
				\State $sum \gets sum + x$
			\EndFor
			\State \Return $sum$
		\EndFunction
	\end{algorithmic}
\end{algorithm}

\subsubsection*{Advantages and disadvantages}
Using flow charts or pseudo-code has the advantage of being able to define an algorithm in a way which is very close to the abstractions employed when using natural language: a flow chart combines both the use of natural language and a visual interface to describe an algorithm, pseudo-code allows to employ several abstractions and even define some steps in terms of natural language. The drawback of these two formal representations is that, when it comes to the implementation, the definition of the algorithm must be translated by hand into code that the hardware is able to execute. This could be done by implementing the algorithm in a low-level or high-level programming language. This process affects at different levels how the logic of the algorithm is presented, as explained further.

\section{Programming languages}
\label{sec:ch1_programming_languages}
A programming language is a formal language that is used to define instructions that a machine, usually a computer, must perform in order to produce a result through computation \cite{mordechai1996, narasimhan1967programming, oxford2008}. There is a wide taxonomy used to classify programming languages depending on their use \cite{kelleher2005lowering, myers1986visual, myers1990taxonomies}, but all can be grouped according to two main characteristics: the level of abstraction, or how close to the specific targeted hardware they are, and the domain, which defines the range of applicability of a programming language. In the following sections we give an exhaustive explanation of the aforementioned characteristics.

\subsection{Low-level programming languages}
\label{subsec:ch1_ll_languages}
A low-level programming language is a programming language that provides little to no abstraction from the hardware architecture of a processor. This means that it is strongly connected with the instruction set of the targeted machine, the set of instructions a processor is able to execute. These languages are divided into two sub-categories: \textit{first-generation} and \textit{second-generation} languages:

\subsubsection*{First-generation languages}
\textit{Machine code} falls into the category of first-generation languages. In this category we find all those languages that do not require code transformations to be executed by the processor. These languages were used mainly during the dawn of computer age and are rarely employed by programmers nowadays. Machine code is made of stream of binary data, that represents the instruction codes and their arguments \cite{guide2011intel, seal2001arm}. Usually this stream of data is treated by programmers in hexadecimal format, which is then remapped into binary code. The programs written in machine code were once loaded into the processor through a front panel, a controller that allowed the display and alteration of the registers and memory (see Figure \ref{fig:ch1_front_panel}). An example of machine code for a program that computes the sum of a sequence of integer numbers can be seen in Listing \ref{lst:ch1_machine_code}.

\begin{figure}
	\centering
	\includegraphics[width = \textwidth]{Figures/ch1_front_panel}
	\caption{Front panel of IBM 1620}
	\label{fig:ch1_front_panel}
\end{figure}

\begin{minipage}{\linewidth}
\begin{lstlisting}[numbers = left, caption = Machine code to compute the sum of a sequence of numbers, label = lst:ch1_machine_code]
 00075	c7 45 b8 00 00
 00 00
 0007c	eb 09	
 0007e	8b 45 b8
 00081	83 c0 01
 00084	89 45 b8
 00087	83 7d b8 0a
 0008b	7d 0f
 0008d	8b 45 b8
 00090	8b 4d c4
 00093	03 4c 85 d0
 00097	89 4d c4
 0009a	eb e2
\end{lstlisting}
\end{minipage}

\subsubsection*{Second-generation languages}
The languages in this category provides an abstraction layer over the machine code by expressing processor instructions with mnemonic names both for the instruction code and the arguments. For example, the arithmetic sum instruction \texttt{add} is the mnemonic name for the instruction code \texttt{0x00} in \texttt{x86} processors. Among these languages we find \textit{Assembly}, that is mapped with an \textit{Assembler} to machine code. The Assembler can load directly the code or link different \textit{object files} to generate a single executable by using a \textit{linker}. An example of assembly \texttt{x86} code corresponding to the machine code in Listing \ref{lst:ch1_machine_code} can be found in Listing \ref{lst:ch1_assembly_code}. You can see that the code in the machine code \texttt{00081	83 c0 01} at line 5 has been replaced by its mnemonic representation in Assembly as \texttt{add	eax, 1}.

\begin{minipage}{\linewidth}
\begin{lstlisting}[numbers = left, caption = Assembly x86 code to compute the sum of a sequence of numbers, label = lst:ch1_assembly_code]
mov	DWORD PTR _i$1[ebp], 0
jmp	SHORT $LN4@main
$LN2@main:
mov	eax, DWORD PTR _i$1[ebp]
add	eax, 1
mov	DWORD PTR _i$1[ebp], eax
$LN4@main:
cmp	DWORD PTR _i$1[ebp], 10			; 0000000aH
jge	SHORT $LN3@main
mov	eax, DWORD PTR _i$1[ebp]
mov	ecx, DWORD PTR _sum$[ebp]
add	ecx, DWORD PTR _numbers$[ebp+eax*4]
mov	DWORD PTR _sum$[ebp], ecx
jmp	SHORT $LN2@main
\end{lstlisting}
\end{minipage}

\subsubsection*{Advantages and disadvantages}
Writing a program in low-level programming languages might produce programs that are generally more efficient than their high-level counterparts, as ad-hoc optimizations are possible. However, the high-performance comes at great costs: (\textit{i}) the programmer must be an expert of the underlying architecture and of the specific instruction set of the processor, (\textit{ii}) the program loses portability because the low-level code is tightly bound to the specific hardware architecture it targets, (\textit{iii}) the logic and readability of the program is hidden among the details of the instruction set itself, and (\textit{iv}) developing a program in assembly requires a considerable effort in terms of time and debugging \cite{frampton2009demystifying}: assembly lacks any abstraction from the concrete hardware architecture, such as a type system, that partially ensures the correctness of the program or high-level constructs that allow to manipulate the execution of the program.

\subsection{High-level programming languages}
\label{subsec:ch1_hl_languages}
A high-level programming language is a programming language that offers a high level of abstraction from the specific hardware architecture of the machine. Unlike machine code (and in some way also assembly), high-level languages are not directly executable by the processor and they require some kind of translation process into machine code. The level of abstraction offered by the language defines how high level the language is. Several categories of high-level programming language exist, but the main one are described below.

\subsubsection*{Imperative programming languages}
\textit{Imperative programming languages} model the computation as a sequence of statements that alter the state of the program (usually the memory state). A program in such languages consists then of a sequence of \textit{commands}. Notable examples are FORTRAN, C, and PASCAL. An example of the program used in Listing \ref{lst:ch1_machine_code} and \ref{lst:ch1_assembly_code} written in C can be seen in Listing \ref{lst:ch1_c_code}. Line 5 to 9 corresponds to the Assembly code in Listing \ref{lst:ch1_assembly_code}.

\begin{lstlisting}[numbers = left, caption = C code to compute the sum of a sequence of numbers, label = lst:ch1_c_code]
int main()
{
  int numbers[10] = { 1, 6, 8, -2, 4, 3, 0, 1, 10, -5 };
  int sum = 0;
  for (int i = 0; i < 10; i++)
  {
    sum += numbers[i];
  }
  printf("%d\n", sum);
}
\end{lstlisting}

\subsection*{Declarative programming languages}
\textit{Declarative programming languages} are antithetical to those based on imperative programming, as they model computation as an evaluation of expressions and not as a sequence of commands to execute. Declarative programming languages are called as such when they are side-effects free or referentially transparent. The definition of referential transparency varies \cite{quine2013word}, but it is usually explained with the substitution principle, which states that a language is referentially transparent if any expression can be replaced by its value without altering the behaviour of the program \cite{mitchell2003concepts}. For instance, the following sentences in natural language are both true

\begin{lstlisting}
Cicero = Tullius

''Cicero`` contains six letters
\end{lstlisting} 

\noindent
but they are not referentially transparent, since replacing the last name with the middle name falsifies the second sentence.

A similar situation in programming languages is met when considering variable assignments: the statement

\begin{lstlisting}
x = x + 5
\end{lstlisting}

\noindent
is not referentially transparent. Let us assume this statement appears twice in a program and that at the beginning x = 0. Clearly the expression \texttt{x + 5} results in the value 5 the first time, but the second time the same statement is executed the expression has value 10. Thus replacing all the occurrences of \texttt{x + 5} with 5 is wrong, which is why imperative languages are not referentially transparent. A more rigorous definition of referential transparency can be found in \cite{sondergaard1990referential}.

Declarative programming languages are often compared to imperative programming languages by stating that declarative programming defines \textit{what} to compute and not \textit{how} to compute it. This family of languages include \textit{functional programming}, \textit{logic programming}, and \textit{database query languages}. Notable examples are F\#, Haskell, Prolog, SQL, and Linq (which is a query language embedded in C\#). Listing \ref{lst:ch1_fsharp_code_rec} shows the code to perform the sum of a sequence of integer numbers in F\# with a recursive function. Higher-order functions, such as \texttt{fold}, allow even to capture the same recursive pattern into a single function as shown in Listing \ref{lst:ch1_fsharp_code_fold}. Both implementations are referentially transparent.

\begin{lstlisting}[caption = Recursive F\# code to compute the sum of a sequence of numbers, label = lst:ch1_fsharp_code_rec]
let rec sumList l =
  match l with
  | [] -> 0
  | x :: xs -> x + (sumList xs)
\end{lstlisting}

\begin{lstlisting}[caption = F\# code to compute the sum of a sequence of numbers using higher-order functions, label = lst:ch1_fsharp_code_fold]
let sumList l = l |> List.fold (+) 0
\end{lstlisting}

\subsection{General-purpose vs Domain-specific languages}
\label{sec:ch1_dsl}
\textit{General-purpose languages} are defined as languages that can be used across different application domains and lack abstractions that specifically target elements of a single domain. Example of these are languages such as C, C++, C\#, and Java. Although several applications are still being developed by using general-purpose programming languages, in several contexts it is more convenient to rely on \textit{domain-specific languages}, because they offer abstractions relative to the problem domain that are unavailable in general-purpose languages \cite{van2000domain, voelter2013dsl}. Notable examples of the use of domain-specific languages are listed below.

\subsubsection*{Graphics programming}
Rendering a scene in a 3D space is often performed by relying on dedicated hardware. Modern graphics processors rely on shaders to create various effects that are rendered in the 3D scene. Shaders are written in domain-specific languages, such as GLSL or HLSL \cite{glhl2014, hlsl2018, hlslref2018}, that offer abstractions to compute operations at GPU level that are often used in computer graphics, such as vertices and pixel transformations, matrix multiplications, and interpolation of textures. Listing \ref{lst:ch1_hlsl_code} shows the code to implement light reflections in HLSL. At line 4 you can, for example, see the use of matrix multiplication provided as a language abstraction in HLSL.

\begin{lstlisting}[numbers = left, caption = HLSL code to compute the light reflection, label = lst:ch1_hlsl_code]
VertexShaderOutput VertexShaderSpecularFunction(VertexShaderInput input, float3 Normal : NORMAL)
{
  VertexShaderOutput output;
  float4 worldPosition = mul(input.Position, World);
  float4 viewPosition = mul(worldPosition, View);
  output.Position = mul(viewPosition, Projection);
  float3 normal = normalize(mul(Normal, World));
  output.Normal = normal;
  output.View = normalize(float4(EyePosition,1.0f) - worldPosition);
  return output;
}
\end{lstlisting}

\subsubsection*{Game programming}
Computer games are a field where domain-specific languages are widely employed, as they contain complex behaviours that often require special constructs to model timing event-based primitives, or to execute tasks in parallel. These behaviours cannot be modelled, for performance reasons, by using threads. Therefore, in the past, domain-specific languages which provide these abstractions have been implemented \cite{nwnlexicon2018, jass2011, unrealscript2018, sqf2018}. In Listing \ref{lst:ch1_sqf_code} an example of the SQF domain-specific language for the game ArmA2 is shown. This language offers abstractions to wait for a specific amount of time, to wait for a condition, and to spawn scripts that run in parallel to the callee, that you can respectively see at lines 18, 12, and 10.

\begin{lstlisting}[numbers = left, caption = ArmA 2 scripting language, label = lst:ch1_sqf_code]
"colorCorrections" ppEffectAdjust [1, pi, 0, [0.0, 0.0, 0.0, 0.0], [0.05, 0.18, 0.45, 0.5], [0.5, 0.5, 0.5, 0.0]];  
"colorCorrections" ppEffectCommit 0;  
"colorCorrections" ppEffectEnable true;

thanatos switchMove "AmovPpneMstpSrasWrflDnon";
[[],(position tower) nearestObject 6540,[["USMC_Soldier",west]],4,true,[]] execVM "patrolBuilding.sqf";
playMusic "Intro";

titleCut ["", "BLACK FADED", 999];
[] Spawn 
{
	waitUntil{!(isNil "BIS_fnc_init")};
	[
	  localize "STR_TITLE_LOCATION" ,
	  localize "STR_TITLE_PERSON",
	  str(date select 1) + "." + str(date select 2) + "." + str(date select 0)
	] spawn BIS_fnc_infoText;
	sleep 3;
	"dynamicBlur" ppEffectEnable true;   
	"dynamicBlur" ppEffectAdjust [6];   
	"dynamicBlur" ppEffectCommit 0;     
	"dynamicBlur" ppEffectAdjust [0.0];  
	"dynamicBlur" ppEffectCommit 7;
	titleCut ["", "BLACK IN", 5];
};
\end{lstlisting}

\subsubsection*{Shell scripting languages}
Shell scripting languages, such as the \textit{Unix Shell script}, are used to manipulate files or user input in different ways. They generally offer abstractions to the operating system interface in the form of dedicated commands. Listing \ref{lst:ch1_shell_code} shows an example of a program written in Unix shell script to convert an image from JPG to PNG format. At line 3 you can see the use of the statement \texttt{echo} to display a message in the standard output.

\begin{lstlisting}[numbers = left, caption = Unix shell code, label = lst:ch1_shell_code]
for jpg; do                                  
  png="${jpg%.jpg}.png"                    
  echo converting "$jpg" ...               
  if convert "$jpg" jpg.to.png ; then      
    mv jpg.to.png "$png"                 
  else                                     
    echo 'jpg2png: error: failed output saved in "jpg.to.png".' >&2
    exit 1
  fi                                       
done                                         
echo all conversions successful              
exit 0
\end{lstlisting}

\subsubsection*{Advantages and disadvantages}
High-level programming languages offer a variety of abstractions over the specific hardware the program targets. The obvious advantage of this is that the programmer does not need to be an expert of the underlying hardware architecture or instruction set. A further advantage is that the available abstractions are closer to the semi-formal description of the underlying algorithm as pseudo-code. This produces two desirable effects: (\textit{i}) the readability of the program is increased as the available abstractions are closer to the natural language than the equivalent machine code, and (\textit{ii}) that being able to mimic the semi-formal version of an algorithm, which is generally how the algorithm is presented and on which its correctness is proven, grants a higher degree of correctness in the specific implementation.

The use of a high-level programming language might, in general, not achieve the same high-performance as writing the same program with a low-level programming language  \cite{chatzigeorgiou2002evaluating}, but modern code-generation optimization techniques can generally mitigate this gap \cite{amarasinghe1993communication, wang2007code}. A further major issue in using high-level programming languages is that the machine cannot directly execute the code, thus the use of a compiler that translates the high-level program into machine code is necessary.

The portability of a high-level programming language depends on the architecture of the underlying compiler, thus some languages are portable and the same code can be run on different machines (for example Java), while others might require to be compiled to target a specific architecture (for example C++).

\section{Compilers}
\label{sec:ch1_compilers}
A compiler is a program that transforms source code defined in a programming language into another computer language, which usually is object code but can also be code in a high-level programming language \cite{aho2007compilers, appel2002javacompiler}. Writing a compiler is a necessary step to implementing a high-level programming language. Indeed, a high-level programming language, unlike low-level ones, are not executable directly by the processor and need to be translated into machine code, as stated in Section \ref{subsec:ch1_ll_languages} and \ref{subsec:ch1_hl_languages}.

The first complete compiler was developed by IBM for the FORTRAN language and required 18 person-years for its development \cite{backus1957fortran}. This clearly shows that writing a compiler is a hard and time-consuming task.

A compiler is a complex piece of software made of several components that implement a step in the translation process. The translation process performed by a compiler involves the following steps:

\begin{enumerate}
	\item \textit{syntactical analysis:} In this phase the compiler checks that the program is written according to the grammar rules of the language. In this phase the compiler must be able to recognize the \textit{syntagms} of the language (the ``words'') and also check if the program conforms to the syntax rules of the language through a grammar specification.
	\item \textit{type checking:} In this phase the compiler checks that a \textit{syntactically correct program} performs operations conform to a defined \textit{type system}. A type system is a set of rules that assign properties called types to the constructs of a computer program \cite{pierce2002types}. The use of a type system drastically reduces the chance of having bugs in a computer program \cite{cardelli1996type} . This phase can be performed at compile time (\textit{static typing}) or the generated code could contain the code to perform the type checking at runtime (\textit{dynamic typing}). 
	\item \textit{code generation:} In this phase the compiler takes the \textit{syntactically and type-correct program} and performs the translation step. At this point an equivalent program in a target language will be generated. The target language can be object code, another high-level programming language, or even a bytecode that can be interpreted by a virtual machine.
\end{enumerate}

All the previous steps are always the same regardless of the language the compiler translates from and they are not part of the creative aspect of the language design \cite{book1970cwic}. Approaches to automating the construction of the syntactical analyser are well known in literature \cite{mcpeak2004elkhound, nivre2006maltparser, parr1995antlr}, to the point that several lexer/parser generators are available for programmers, for example all those belonging to the \texttt{yacc} family such as \texttt{yacc} for C/C++, \texttt{fsyacc} for F\#, \texttt{cup} for Java, and \texttt{Happy} for Haskell. On the other hand, developers lack a set of tools to automate the implementation of the last two steps, namely the type checking and the code generation.

For this reason, when implementing a compiler, the formal type system definition and the operational semantics, which is tightly connected to the code generation and defines how the constructs of the language behave, must be translated into the abstractions provided by the host language in which the compiler will be implemented. Other than being a time-consuming activity itself, this causes that (\textit{i}) the logic of the type system and operational semantics is lost inside the abstraction of the host-language, and (\textit{ii}) it is difficult to extend the language with new features.

\section{Meta-compilers}
\label{sec:ch1_metacompilers}
In Section \ref{sec:ch1_compilers} we described how the steps involved in designing and implementing a compiler do not require creativity and are always the same, regardless of the language the compiler is built for. The first step, namely the syntactical analysis, can be automated by using one of the several lexer/parser generators available, but the implementation of a type checker and a code generator still relies on a manual implementation. This is where meta-compilers come into play: a meta-compiler is a program that takes the source code of another program written in a specific language and the language definition itself, and generates executable code. The language definition is written in a programming language, referred to as \textit{meta-language}, which should provide the abstractions necessary to define the syntax, type system, and operational semantics of the language, in order to implement all the steps above.

\subsection{Requirements}
As stated in Section \ref{sec:ch1_metacompilers}, a meta-compiler should provide a meta-language that is able to define the syntax, type system, and operational semantics of a programming language. In Section \ref{sec:ch1_compilers} we discussed how methods to automate the implementation of syntactical analyser are already known in scientific literature. For this reason, in this work, we will focus exclusively on automating the implementation of the type system and of the operational semantics. Given this focus, we formulate the following requirements:

\begin{itemize}
	\item The meta-language should provide abstractions to define the constructs of the language. This includes the possibility of defining control structures, operators with any form of prefix or infix notation, and the priority of the constructs that is used when evaluating their behaviour. Furthermore, it must be possible to define the equivalence of language constructs. For instance, an integer constant might be considered both a value and a basic arithmetic expression.
	
	\item The meta-language must be able to mimic as close as possible the formal definition of a programming language. This will bring the following benefits: (\textit{i}) Implementing the language in the meta-compiler will just involve re-writing almost one-to-one the type system or the semantics of the language with little or no change; (\textit{ii}) the correctness and soundness \cite{cardelli1996type, milner1972proving} of the language formal definition will be directly reflected in the implementation of the language; indeed if a meta-program allows to mimic directly the type system and semantics of the language their correctness is transferred also in the implementation, while this might not be trivial when translating them in the abstractions of a high-level programming language; (\textit{iii}) any extension of the language definition can be just added as an additional rule in the type system or the semantics.
	
	\item The meta-compiler must be able to embed libraries from external languages, so that they can be used to implement specific behaviours such as networking transmission or specific data structure usage.
\end{itemize}

\subsection{Benefits}
\label{sec:ch1_benefits}
%list the benefits first and then explain. Add a paragraph also about correctness
Programming languages usually are released with a minimal (but sufficient to be Turing-complete) set of features, and later extended in functionality in successive versions. This process tends to be slow and often significant improvements or additions are only seen years after the first release. For example, Java was released in 1996 and lacked an important feature such as Generics until 2004, when J2SE 5.0 was released. Furthermore, Java and C++ lacked constructs from functional programming, which is becoming more and more popular with the years \cite{thompson1995miranda}, such as lambda abstractions until 2016, while a similar language like C\# 3.0 was released with such capability in 2008. The slow rate of change of programming languages is due to the fact that every abstraction added to the language must be reflected in all the modules of its compiler: the grammar must be extended to support new syntactical rules, the type checking of the new constructs must be added, and the appropriate code generation must be implemented. Given the complexity of compilers, this process requires a huge amount of work, and it is often obstructed by the low flexiblity of the compiler as piece of software. Using a meta-compiler would speed up the extension of an existing language because it would require only to change on paper the type system and the operational semantics, and then add the new definitions to their counterpart written in the meta-language. This process is easier because the meta-language should mimic as close as possible their behaviour. Moreover, backward compatibility is automatically granted because an older program will simply use the extended language version to be compiled by the meta-compiler.

To this we add the fact that, in general, for the same reasons, the development of a new programming language is generally faster when using a meta-compiler. This could be beneficial to the development of a high variety of domain-specific languages. Indeed, such languages are often employed in situations where the developers have little or no resources to develop a fully-fledged hard-coded compiler by hand. For instance, it is desirable for game developers to focus on aspects that are strictly tied to the game itself, for example the development of an efficient graphics engine or to improve the game logic. At the same time they would need a domain-specific language to express some behaviours typical of games, things that could be achieved by using a meta-compiler rather than on a hand-made implementation.

\subsection{Scientific relevance} %change the tile
\label{sec:ch1_scientific_relevance}
Meta-compilers have been researched since the 1960's \cite{schorre1964meta} and several implementations have been proposed \cite{ braborovansky1998overview, venboer2008stratego, klint2009rascal, pettersson1996compiler, verdejo2006executable}. In general meta-compilers perform poorly compared to hard-coded compilers because they add the additional layer of abstraction of the meta-language. Moreover, a specific implementation of a compiler opens up the possibility of implementing language-specific optimizations during the code generation phase. Meta-compilers have been used in a wide range of applications, such as source code analysis and manipulation and physical simulations \cite{kaagedal1998generating}, but no use up to our knowledge was made in the field of domain-specific languages for games. Since games are pieces of software that are very demanding in terms of performance, we think that it could be of interest to investigate the applicability of meta-compilers in the scope of domain-specific languages for games and the development speed up introduced by the use of such a tool. In this work we present Metacasanova, a meta-compiler based on natural semantics that was born from the intent of easing the development of the domain-specific language for game development Casanova, and we analyse the benefit of using it for a re-implementation and extension of Casanova.

\section{Problem statement}
\label{sec:ch1_problem_statement}
In Section \ref{sec:ch1_programming_languages} we showed the advantages of using high-level programming languages when implementing an algorithm. Among such languages, it is sometimes desirable to employ domain-specific languages that offer abstractions relative to a specific application domain (Section \ref{sec:ch1_dsl}). In Section \ref{sec:ch1_compilers} we described the need of a compiler for such languages, and that developing one is a time-consuming activity despite the process being, in great part, non-creative. In Section \ref{sec:ch1_metacompilers} we introduced the role of meta-compilers to speed up the process of developing a compiler and we listed the requirements and the benefits that one should have. In Section \ref{sec:ch1_scientific_relevance} we explained why we believe that meta-compilers are a relevant scientific topic if coupled with the problem of of developing domain-specific languages in response to the their increasing need. We can now formulate our problem statement:

\vspace{0.5cm}
\noindent
\textbf{Problem statement: } \textit{\psContent}

\vspace{0.5cm}
\noindent
The first parameter we need to evaluate in order to answer this question is the size of the code reduction needed to implement the domain-specific language. At this purpose, the following research question arises:

\vspace{0.5cm}
\noindent
\textbf{Research question 1: } \textit{\rqContentOne}

\vspace{0.5cm}
\noindent
The second parameter we need to evaluate is the eventual performance loss caused by introducing the abstraction layer provided by the meta-compiler. This leads to the following research question:

\vspace{0.5cm}
\noindent
\textbf{Research question 2: } \textit{\rqContentTwo}

\vspace{0.5cm}
\noindent
In case of a performance loss, we need to identify the cause of this performance loss and if an improvement is possible. This leads to the following research question:

\vspace{0.5cm}
\noindent
\textbf{Research question 3: } \textit{\rqContentThree}

\vspace{0.5cm}
\noindent

\section{Thesis structure}
This thesis describes the architecture of Metacasanova, a meta-compiler whose meta-language is based on operational semantics, and a possible optimization for such meta-compiler. It also shows its the capabilities by implementing a small imperative language and re-implementing the existing domain-specific language for games \textit{Casanova 2}, extending it with abstractions to express network operations for multiplayer games.

In Chapter \ref{ch:background} we provide background information in order to understand the choices made for this work. The chapter presents the state of the art in designing and implementing compilers and existing research on meta-compilers.

In Chapter \ref{ch:metacasanova} we present the architecture of Metacasanova by extensively describing the implementation of all its modules.

In Chapter \ref{ch:languages} we show how to use Metacasanova to implement two languages: a small imperative language, and \textit{Casanova 2}, a language for game development. At the end of the chapter we provide an evaluation of the performance of the two languages and their implementation length with respect to existing compilers, thus answering to Research Question 1 and 2.

In Chapter \ref{ch:functors} we discuss the performance loss of the implementation of the presented languages and we propose an extension of Metacasanova that aims to improve the performance of the generated code, thus answering Research Question 3.

In Chapter \ref{ch:functor_languages} we show how to use functors to improve the performance of Casanova implemented in Metacasanova, comparing this approach and the one presented in Chapter \ref{ch:languages} with respect to the execution time of a sample in Casanova. 

In Chapter \ref{ch:networking} we propose an extension of Casanova 2 for multiplayer game development. We first provide its hard-coded compiler solution and then we show how to extend the implementation in the meta-compiler to include the same extension. In this chapter we evaluate the performance of a multiplayer game implemented in Casanova with this extension with respect to the same game implemented in C\#, and we measure the effort of realising such extension in the hard-coded compiler of Casanova versus the implementation with Metacasanova.

In Chapter \ref{ch:discussion} we discuss the result and answer the research questions.
	

\chapter{Background}
\label{ch:background}
\epigraph{Trying to outsmart a compiler defeats much of the purpose of using one.}{Kernighan and Plauger - \textit{The Elements of Programming Style}.}
\section{Architectural overview of a compiler}
\label{sec:ch_background_compiler_architecture}
Compilers are software that read as input a program written in a programming language, called \textit{source language}, and translate it into an equivalent program expressed with another programming language, called \textit{target language}. Usually the target language is machine code, but this is not mandatory. A special kind of compilers are interpreted, that directly execute the program written in the source language rather than translating it into a target language. Some languages, like Java, use a hybrid approach, that is they compile the program into an intermediate language that is later interpreted by a \textit{virtual machine}. Another approach involves the translation into a target high-level language [...].

Although the architecture of a compiler may slightly vary depending on the specific implementation, the translation process usually consists of the following steps:

\begin{enumerate}
	\item \textbf{Lexical analysis:} this phase is performed by a module called \textit{lexer} that is able to process the text and identify the syntactical elements of the language, called \textit{tokens}.
	\item \textbf{Syntactical analysis:} this phase is performed by a module called \textit{parser}, that checks whether the program written in the source language is compliant to the formal syntax of the language. The parser is tightly coupled with the lexer, as it needs to identify the tokens of the language to correctly process the syntax rules. The parser outputs a representation of the program, called \textit{Abstract Syntax Tree}, for later use.
	\item \textbf{Type checking:} this phase is performed by the \textit{type checker} that uses the rules defined by a \textit{type system} to assign a property to the elements of the language called \textit{type}. The types are used to determine whether the abstractions of the language, in a program that is syntactically correct, are used in a meaningful way.
	\item \textbf{Code generation:} the code generation phase requires to choose one or more target languages to emit. In the latter case, the code generator must have a modular structure to allow to interchange the output language. For this reason this step is usually preceded by an \textit{intermediate code generation} step, that converts the source program into an intermediate representation close to the target language. This phase can later be followed by different kinds of code optimization phases.
\end{enumerate}

In what follows we extensively describe each module that was summarized above.

\section{Lexer}
\label{sec:ch_background_compiler_lexer}
As stated above, the lexer task is to recognize the \textit{words} or \textit{tokens} of the source language. In order to perform this task the token structure must be expressed in a formal way. Below we present such formalization and we describe the algorithm that actually recognizes the token.

Let us consider a finite alphabet $\Sigma$, a \textit{language} is a set of strings, intended as sequences of characters in $\Sigma$.

\begin{definition}
	A string in a language $L$ in the alphabet $\Sigma$ is a tuple of characters $\mathbf{a} \in \Sigma^{n}$.
\end{definition}

 A notable difference between  languages in this context and human-spoken languages is that, in the former, we do not associate a meaning to the words but we are only interested to define which words are part of the language and which are not. Regular expressions are a convenient formalization to define the structure of sets of strings:

\begin{definition}
	\label{def:ch_background_regexp}
	 The following are the possible ways to define regular expressions:
	\begin{itemize}[noitemsep]
		\item \textit{Empty:} The regular expression $\epsilon$ is a language containing only the empty string.
		\item \textit{Symbol:} $\forall a \in \Sigma$, $\mathbf{a}$ is a string containing the character $a$.
		\item \textit{Alternation:} Given two regular expressions $M$ and $N$, a string in the language of $M | N$, called alternation, is the sets of strings in the language of $M$ or $N$.
		\item \textit{Concatenation:} Given two regular expressions $M$ and $N$, a string in the language of $M \cdot N$ is the language of strings $\mathbf{\alpha \cdot \beta}$ such as $\mathbf{\alpha} \in M$ and $\mathbf{\alpha} \in N$.
		\item \textit{Repetition:} Given a regular expression $M$, its Kleene Closure $M^{*}$ is formed by the concatenation of zero or more strings in the language $M$.
	\end{itemize}
\end{definition}

The regular expressions defined in Definition \ref{def:ch_background_regexp} can be combined to define tokens in a language.

Regular expressions can be processed by using a finite state automaton. Informally a finite state automaton is made of a finite set of states, an alphabet $\Sigma$ of which it is able to process the symbols, and a set of symbol-labelled edges that connect two states and define how to transition from one state to another. Automata can be divided into two categorise: \textit{non-deterministic finite state automata (NFA)} and \textit{deterministic finite state automata (DFA)}. Formally we have the following definitions:

\begin{definition}
	A non-deterministic finite state automaton (NFA) is made of:
	
	\begin{itemize}[noitemsep]
		\item A finite set of states S.
		\item An alphabet $\Sigma$ of input symbols.
		\item A state $s_{0} \in S$ that is the starting state of the automaton.
		\item A set of states $F \subset S$ called final or accepting states.
		\item A set of transitions $\mathcal{T} \subseteq S \times (\Sigma \cup \lbrace \epsilon \rbrace) \times S$.
	\end{itemize}
\end{definition}

\begin{definition}
	A deterministic finite state automaton (DFA) is a NFA where the transition is a function, i.e.
	\begin{equation*}
		\begin{array}{l}
			\tau : S \times \Sigma \rightarrow S\\
			\tau(s_{i},c) = s_{j}
		\end{array}
	\end{equation*}
	and $\nexists \; \tau(s,c_{i}),\tau(s,c_{j}) \; | \; c_{i} = c_{j} \; \forall i,j$.
\end{definition}

Informally, in NFA's there might be two transitions from the same state that can process the same symbol, while in DFA's for the same state there exists one and only one transition able to process a symbol and no transition processes the empty string. Regular expressions can be converted in NFA by using translation rules. The formalization of the algorithm can be found in \cite{mcnaughton1960regular}, here we just show an informal overview for brevity.

\subsection{Finite state automata for regular expressions}
\label{subsec:ch_background_automata}
In this section we present an informal overview of the translation rules for regular expressions into NFA's, and an algorithm to convert an NFA into a DFA.

\paragraph{Conversion for Symbols}
A regular expression containing just one symbol $a \in \Sigma$ can be converted by creating a transition $\tau(s_{i},a) = s_{j}$.

\paragraph{Conversion for concatenation}
The conversion for concatenation is recursive: the base case of the recursion is the symbol conversion. The conversion of a concatenation of $n$ symbols $a_{1}a_{2}, ..., a_{n}$ is obtained by adding a transition from the last state of the conversion for the first $n - 1$ symbols into a new state through a transition processing the n-th symbol, $\tau(s_{n - 1},a_{n}) = s_{n}$.

\paragraph{Conversion for alternation}
The alternation $M | N$ is obtained by creating an automata with a $\epsilon$-transition into a new state, that we call $s_{\epsilon}$. From $s_{\epsilon}$ we recursively generate the automata for both $M$ and $N$. Both automata can finally reach the same state through an $\epsilon$-transition.

\paragraph{Conversion for Kleene closure}
The Kleene Closure $M^{*}$ is obtained by initially creating an $\epsilon$-transition into a state $s_{\epsilon}$. $s_{\epsilon}$ can recursively transition to the automaton for $M$, which in turn transitions through an $\epsilon$-transition to $s_{\epsilon}$.

\paragraph{Conversion for $\mathbf{M^{+}}$}
The regular expression $M^{+}$ contains the concatenation of one or more strings in $M$. This can be translated by translating $M \cdot M^{*}$.

\paragraph{Conversion for $\mathbf{M?}$}
The regular expression $M?$ is a shortcut for $M|\epsilon$, thus it can be translated by using the conversion rule for the alternation.

\subsection{Conversion of a NFA into a DFA}
As stated in Section \ref{sec:ch_background_compiler_lexer}, a NFA might have, for the same state, a set of transitions that process the same symbol (including the empty string since $\epsilon$-transitions are allowed). This means that a NFA must be able to guess which transition to follow when trying to process a token. This is not efficient to implement in a computer, thus it is better to use a DFA where there can be only one way of processing a symbol for a given state. An algorithm to automate such conversion exists and is presented in \cite{aho2007compilers} but there exists an algorithm to directly convert regular expressions into DFA's, as shown in \cite{aho1986compilers}. Below we present the algorithm to convert NFA's into DFA's.

The informal idea behind the algorithm is the following: since a DFA cannot contain $\epsilon$-transitions or transitions from one state into another containing the same symbols, we have to construct an automaton that skips the $\epsilon$-transitions and pre-calculates the calculation of the sets of states in advance. In order to do so, we need to be able to compute the \textit{closure} of a set of states. Informally the closure of a set of states $S$ is the sates that can be reached by one of the states of $S$ through an $\epsilon$-transition. The formal definition is given below:

\begin{definition}
	The closure $\mathcal{C}(S)$ of a set of states $S$ is defined as
	\begin{itemize}[noitemsep]
		\item 
				$\displaystyle \mathcal{C}(S) = S \cup \left(\bigcup_{s \in T} \tau(s,\epsilon)	\right)$
		\item if $\exists \; \mathcal{C}'(S) \; | \; \mathcal{C}(S) \subseteq \mathcal{C}'(S) \Rightarrow \mathcal{C}'(S) = \mathcal{C}(S)$.
	\end{itemize}	
	
\end{definition}

\begin{algorithm}
	\caption{Closure of $S$}
	\label{alg:ch_background_closure}
	\begin{algorithmic}
		\State $T \gets S$
		\Repeat
			\State $T' \gets T$
			\State $T \gets \cup \left( \bigcup_{s \in  T'}\tau(s,\epsilon) \right)$
		\Until {$T = T'$}
	\end{algorithmic}
\end{algorithm}

Algorithm \ref{alg:ch_background_closure} computes the closure of a set of states. Note that the algorithm termination is granted because we are considering finite-state automata.

At this point we can build the set of all possible states reachable by consuming a specific character. We call this set \textit{edge} of a set of states $d$. 

\begin{definition}
	Let $d$ be a set of states, then the \textit{edge} of $d$ is defined as
	\begin{equation*}
		\mathcal{E}(d,c) = \mathcal{C}\left(\bigcup_{s \in d}\tau(s,c)\right)
	\end{equation*}
\end{definition}

\noindent
Now we can use the \textit{closure} and \textit{edge} to build the DFA from a NFA.

\begin{algorithm}
	\caption{NFA into DFA conversion}
	\label{alg:ch_background_dfa}
	\begin{algorithmic}
		\State $states[0] \gets \emptyset$
		\State $states[1] \gets \mathcal{C}(s_{1})$
		\State $p \gets 1$
		\State $j \gets 0$
		\While {$j \leq p$}
			\ForAll {$c \in \Sigma$}
				\State $e \gets \mathcal{E}(states[j],c)$
				\If {$\exists \; i \leq p \; | \; e = states[i]$}
					\State $trans[j,c] \gets i$
				\Else
					\State $p \gets p + 1$
					\State $states[p] \gets e$
					\State $trans[j,c] \gets p$
				\EndIf
			\EndFor
			\State $j \gets j + 1$
		\EndWhile
	\end{algorithmic}
\end{algorithm}

\noindent
Algorithm \ref{alg:ch_background_dfa} performs the conversion into a DFA but we need to adjust it in order to mark the final states of the automaton. A state $d$ is final in the DFA if it is final if any of the states in $state[d]$ is final. In addition to marking final states, we must also keep track of what token is produced in that final state.

\section{Parser}
\label{sec:ch_background_parser}
Regular expressions are a concise declarative way to define the lexical structure of the terms of a language, but they are insufficient to describe its syntax, i.e. how to combine tokens together to  make ``sentences''. A compiler uses the parser module to check the syntactical structure of a program. As we will see more in depth below, the parser is tightly coupled with the lexer, which is used by it to recognize tokens. In order to present the structure of the parser, it is first necessary to introduce \textit{context-free grammars}.

As before we consider a language as a set of tuples of characters taken from a finite alphabet $\Sigma$. Informally, a context-free grammar is a set of productions of the form $symbol \rightarrow symbol_{1} \; symbol_{2} \; ... symbol_{n}$, where the left argument can be replaced by the sequence of symbols contained in the right argument. Some productions are \textit{terminal}, meaning that they cannot be replaced any longer, while the others are \textit{non-terminal}. Terminal symbols can only appear on the right side, while non-terminals can appear on both sides. Formally a context free grammar is defined as follows

\begin{definition}
	\label{def:ch_background_grammar}
	A \textit{context-free grammar} is made of the following elements:
	
	\begin{itemize}[noitemsep]
		\item A set of non-terminal symbols $N$.
		\item A finite set of terminal symbols $\Sigma$, called \textit{alphabet}.
		\item A non-terminal symbol $S \in N$ called \textit{starting symbol}.
		\item A set of productions $P$ in the form $N \rightarrow (N \cup \Sigma)^{*}$.
	\end{itemize}
\end{definition}

\noindent
Note that Definition \ref{def:ch_background_grammar} allows \textit{context-free} grammars to process also regular expression, thus context-free grammars are more expressive then regular expressions. In what follows we assume that the terminal symbols are treated as tokens with regular expressions that can be processed by a lexer, but in general a context-free grammar does not require a lexer DFA to process terminal symbols.

In order to check if a sentence is valid in the grammar defined for a language, we perform a process called \textit{derivation}: starting from the symbol $S$ of the grammar, we recursively replace non-terminal symbols with the right side of their production. The derivation can be done in different ways: we can start expanding the leftmost non-terminal in the production or the rightmost one. The result of the derivation usually generates a data structure called \textit{parse tree} or \textit{abstract syntax tree}, which connects a non-terminal symbol to the symbols obtained through the derivation; the leaves of the tree are terminal symbols.

\subsection{LR(k) parsers}
Simple grammars can be parsed by using \textit{left-to-right parse, leftmost-derivation, k-tokens lookahead}, meaning that the parser processes a symbol by performing a derivation starting from the leftmost symbol of the production, and looking at the first \textit{k} tokens of a string of the language. The weakness of this technique is that the parser must predict which production to use only knowing the first k tokens of the right side of the production. For instance, consider the two expression

\begin{equation*}
	\begin{array}{l}
		(15 * 3 + 4) - 6\\
		(15 * 3 + 4)
	\end{array}
\end{equation*}

\noindent
and the grammar

\begin{equation*}
	\begin{array}{l}
		S \rightarrow E \; eof\\
		E \rightarrow E + T\\
		E \rightarrow E - T\\
		E \rightarrow T * F\\
		E \rightarrow T / F\\
		E \rightarrow T\\
		T \rightarrow F\\
		F \rightarrow id\\
		F \rightarrow num\\
		F \rightarrow ( E )	
	\end{array}
\end{equation*}

\noindent
In the first case the parser should use the production $E \rightarrow E - T$ while in the second it should use the production $E \rightarrow T$. This grammar cannot be parsed by a LL(k) parser because it is not possible to decide which of the two productions must be used just by looking at the first k leftmost tokens. Indeed expressions of that form could have arbitrary length and the lookahead is, in general, insufficient. In general LL(k) grammars are context-free, but not all context-free grammars are LL(k), so such a parser is unable to parse all context-free grammars.

A more powerful parser is the \textit{left-to-right parse, rightmost-derivation, k-tokens lookahead} or LR(k). This parse maintains a \textit{stack} and an \textit{input} (which is the sentence to parse). The first k tokens of the input are the \textit{lookahead}. The parser uses the stack and the lookahead to perform two different actions:

\begin{itemize}
	\item \textit{Shift}: The parser moves the first input token to the top of the stack.
	\item \textit{Reduce}: The parser chooses a grammar production $N_{i} \rightarrow s_{1} \; s_{2} \; ... s_{j}$ and pop $s_{j}, s_{j - 1}, ... , s_{1}$ from the top of the stack. It then pushes $N_{i}$ at the top of the stack.
\end{itemize}

\noindent
The parser uses a DFA to know when to apply a shift action or a reduce action. The DFA is insufficient to process the input, as DFA's are not capable of processing context-free grammars, but it is applied to the stack. The DFA contains edges labelled by the symbols that can appear in the stack, while states contain one of the following actions:

\begin{itemize}[noitemsep]
	\item $s_{n}$: shift the symbol and go to state $n$.
	\item $g_{n}$: go to state $n$.
	\item $r_{k}$: reduce using the production $k$ in the grammar.
	\item $a$: accept, i.e. shift the end-of-file symbol.
	\item $error$: invalid state, meaning that the sentence is invalid in the grammar.
\end{itemize}

\noindent
The automaton is usually represented with a tabular structure, which is called \textit{parsing table}. The element $p_{i,s}$ in the table represents the transition from state $i$ when the symbol at the top of the stack is $s$.

In order to generate the parsing table (or equivalently the DFA for the parser) we need two support functions, one to generate the possible states the automaton can reach by using grammar productions, and one to generate the actions to advance past the current state. We introduce an additional notation to represent the situation where the parser has reached a certain position while deriving a production.

\begin{definition}
	An \textit{item} is any production in the form $N \rightarrow \alpha.X\beta$, meaning that the parser is at the position indicated by the dot where $X$ is a grammar symbol.
\end{definition}

At this point we are able to define the \textit{Closure} function, that adds more items to a set of items when the dot is before a non-terminal symbol, which is shown in Algorithm \ref{alg:ch_background_parser_closure}. Note that, for brevity, we present the version to generate a LR(0) parser, for a LR(1) parser a minor adjustment must be made.

\begin{algorithm}
	\caption{Closure function for a LR(0) parser}
	\label{alg:ch_background_parser_closure}
	\begin{algorithmic}
		\Function {Closure} {$I$}
			\Repeat
				\ForAll {$N \rightarrow \alpha.X\beta \; \in I$}
					\ForAll {$X \rightarrow \gamma$}
						\State $I \gets I \cup \left\lbrace X \rightarrow .\gamma \right\rbrace$
					\EndFor
				\EndFor
			\Until  {$I' \neq I$}
			\State \Return $I$
		\EndFunction
	\end{algorithmic}
\end{algorithm}

The algorithm starts with an initial set of items $I$ and adds all grammar productions that contain $X$ as left argument as items with the dot at the beginning of their right argument, meaning that the symbols of the production must still be completely parsed.
 
Now we need a function that, given a set of items, is able to advance the state of the parser past the symbol $X$. This is shown in Algorithm \ref{alg:ch_background_parser_goto}.

\begin{algorithm}
	\caption{Goto function for a LR(0) parser}
	\label{alg:ch_background_parser_goto}
	\begin{algorithmic}
		\Function {Goto} {$I, X$}
			\State $J \gets \emptyset$
			\ForAll {$N \rightarrow \alpha.X\beta \; \in I$}
				\State $J \gets J \cup \left\lbrace N \rightarrow \alpha X.\beta \right\rbrace$
			\EndFor
			\State \Return \Call {Closure} {$J$}
		\EndFunction
	\end{algorithmic}
\end{algorithm}

The algorithm starts with a set of items and a symbol $X$ and creates a new set of items where the parser position has been moved past the symbol $X$. It then compute the closure of this new set of items a returns it.

We can now proceed to define the algorithm to generate the LR(0) parser, which is shown in Algorithm \ref{alg:ch_background_lr0_parser}. The initial state is made of all the productions where the left side is the starting symbol, which is equivalent to compute the closure of $S' \rightarrow .S \; eof$. It then proceeds to expand the set of states and the set of actions to perform. Note that we never compute GOTO$(I,eof)$ but we simply generate an \textit{accept} action. Now, for all actions in $E$ where $X$ is a terminal, we generate a shift action at position $(I,X)$, for all actions where $X$ is non-terminal we put a goto action at position $(I,X)$, and finally for a state containing an item $N_{k} \rightarrow \gamma.$ (the parser is at the end of the production) we generate a $r_{k}$ action at $(I,Y)$ for every token $Y$. 

In general parsing tables can be very large, for this reason it is usually wise to implement a variant of LR(k) parsers called LALR(k) parsers, where all states that contain the same actions but different lookaheads are merged into one, thus reducing the size of the parsing table. LR(1) and LALR(1) parsers are very common, since most of the programming languages can be defined by a LR(1) grammar. For instance, the popular family of parser generators \texttt{Yacc} produces LALR(1) parsers.

\begin{algorithm}
	\caption{LR(0) parser generation}
	\label{alg:ch_background_lr0_parser}
	\begin{algorithmic}
		\State $T \gets $ \Call {Closure} {$\left\lbrace S' \rightarrow .S \; eof \right\rbrace$}
		\State $E \gets \emptyset$
		\Repeat
			\State $T' \gets T$
			\State $E' \gets E$
			\ForAll {$I \in T$}
				\ForAll {$N \rightarrow \alpha.X\beta \; \in I$}
					\State $J \gets $ \Call {Goto} {$I,X$}
					\State $T \gets T \cup \lbrace J \rbrace$
					\State $E \gets E \cup \lbrace I \xrightarrow{X} J \rbrace$
				\EndFor
			\EndFor
		\Until {$E' = E \text{\textbf{ and }} T' = T$}
	\end{algorithmic}
\end{algorithm}

\subsection{Parser generators}


\subsection{Monadic parsers}
\label{sec:ch_background_parser_monad}
Monadic parsing is an alternative to traditional parsers, such as LR(k) and LALR(k) presented above. Monadic parsers have inferior performance with respect to LR(k) and LALR(k) \cite{hutton1998monadic} parsers but they are extensible, i.e. they do not rely on a limited set of combinators to describe the grammar of language as for parser generators. Monadic parsers were extensively explained in \cite{hutton1998monadic, wadler1995monads}, here we present a variation that can deal also with error handling. Before explaining how to implement a monadic parser, we introduce the concept of Monad:

\begin{definition}
	\label{def:ch_background_monad}
	A \textit{Monad} is a tern made of the following elements:
	\begin{itemize}[noitemsep]
		\item A type constructor $M$.
		\item A unary operation $Return \; :: \; a \rightarrow M \; a$.
		\item A binary operation $Bind \; :: \; M \; a \rightarrow (a \rightarrow M \; b) \rightarrow M \; b$. The bind can also be written by using the symbol $>>=$.
	\end{itemize}
	where both operations satisfy the following properties:
	\begin{itemize}[noitemsep]
		\item $a >>= return \equiv a$.
		\item $(a >>= f) >>= g \equiv a >>= (\lambda x.f x >>= g)$.
	\end{itemize}
\end{definition}
We now proceed to define a parser monad by defining (\textit{i}) the type constructor for the parser, (\textit{ii}) the unary operator, (\textit{iii}) the binary operator, and (\textit{iv}) parser combinators as an example of the extensibility of the parser monad. Note that below we provide an implementation in F\#, which does not have type classes as Haskell, so the parser monad does not use any type argument and directly defines the operators for this specific instance of monad.

\paragraph{Parser type constructor and monadic operations}

A parser is defined in literature as a function that takes as input a text and returns a list of pairs made of the parsing result and the rest of the text to process. The parsing result is usually the syntax tree generated by the parser. The result is a list because the same syntactical structure might be processed in different ways. By convention, an empty list denotes a parser failure. Here we propose a variation of this traditional implementation in order to provide a better error report.

In this alternative implementation, the parser is a function that takes as input the text to process, a \textit{parsing context} that might hold auxiliary information necessary for the parsing, the current position of the parser in the text, and returns either a tuple containing the parsing result, the text left to process, an updated context, and the updated position, or an error in case of a parser failure.

\begin{lstlisting}
type Parser<'a, 'ctxt> = { Parse : List<char> -> 'ctxt -> Position -> Either<'a * List<char> * 'ctxt * Position, Error>}

static member Make(p:List<char> -> 'ctxt -> Position -> Either<'a * List<char> * 'ctxt * Position, Error>) : Parser<'a,'ctxt> = { Parse = p }
\end{lstlisting}

The \textit{return} operation should take as input a generic value of type \texttt{'a} and return a \texttt{Parser<'a,'ctxt>}. The return simply creates the parser function for the given input:

\begin{lstlisting}
member this.Return(x:'a) : Parser<'a,'ctxt> =
  (fun buf ctxt pos -> First(x, buf, ctxt, pos)) |> Parser.Make
\end{lstlisting}

According to the Definition \ref{def:ch_background_monad}, the bind operator must take as input a \texttt{Parser<'a>}, a function \texttt{'a -> Parser<'b>} and return \texttt{Parser<'b>}. The bind generates a function that runs the input parser on the text. The result of the input parser can, according to its definition, contain a parsing result or an error in case of failure. The function generated by the bind must be able to handle these two situations: in case of a correct result the function creates a new parser using the parsing result and runs it on the remaining portion of the text, while in case of an error it simply outputs the error. In this way, when parsing fails, the error will be propagated ahead.

\begin{lstlisting}
member this.Bind(p:Parser<'a,'ctxt>, k:'a->Parser<'b,'ctxt>) : Parser<'b,'ctxt> =
(fun buf ctxt pos ->
  let all_res = p.Parse buf ctxt pos
  match all_res with
  | First p1res ->
      let res, restBuf, ctxt', pos' = p1res
      (k res).Parse restBuf ctxt' pos'
  | Second err -> Second err ) |> Parser.Make
\end{lstlisting}

\paragraph{Parser combinators}
With the parser monad implemented above, we can implement several parser combinators that can be used to define the grammar of a language. Here we show only a small glimpse of the possible combinators that can be implemented.\\\\
The first parser combinator that we present is the \textit{choice}. The choice takes as input two parsers and runs the first. If the first parser succeeds than its result is returned, otherwise the second is run. If it succeeds its result is return, otherwise the whole parser outputs an error. This combinator is useful, for instance, when there might be two possible choices for a token in a statement. For instance, in either Java or C\# is possible to exchange the order of the access modifier and the static modifier in the method declaration, thus both \texttt{public static} or \texttt{static public} are valid combinations. This combinator would try to parse the declaration in the first way, and if it fails it will try also the second option. Of course if the syntax of both combinations is wrong the parser will fail completely. The code for the combinator is shown below:

\begin{lstlisting}
static member (++) (p1:Parser<'a,'ctxt>, p2:Parser<'a,'ctxt>) : Parser<'a,'ctxt> = 
  (fun buf ctxt p ->
  match p1.Parse buf ctxt p with
  | Second err1 ->
      match p2.Parse buf ctxt p with
      | Second err2 -> Second err2
      | p2res -> p2res
  | p1res -> p1res) |> Parser.Make
\end{lstlisting}

\noindent
A useful variation of this combinator, is the one that executes two parsers with different generic types and returns a \texttt{Either} data type, containing either the result of the first or the second.
\begin{lstlisting}
static   member (+) (p1:Parser<'a,'ctxt>, p2:Parser<'b,'ctxt>) : Parser<Either<'a,'b>,'ctxt> = 
  (fun buf ctxt p ->
   match p1.Parse buf ctxt p with
   | Second err1 ->
       match p2.Parse buf ctxt p with
       | Second err2 -> Second(err2)
       | First p2res -> 
           let res,restBuf,ctxt',pos = p2res 
           First(Second res, restBuf, ctxt', pos)
   | First p1res ->
       let res, restBuf, ctxt', pos = p1res
       First((First res), restBuf, ctxt', pos)) |> Parser.Make
\end{lstlisting}

\noindent
Other combinators are possible, but for brevity we have only shown two. It should appear clear how this approach is completely extensible with no limitations. Any combinator would take as input two parsers and define the type of the resulting parser. The implementation will contain the logic to combine two parsers together. For example, another parser combinator is the application of 0 or more times of the same parser.

To complete this discussion, we now show how to parse a specific character and a keyword. The parser for a character takes as input the text to process and the character to match. If the input text is empty of course the parser immediately fails because no character will ever be matched. Otherwise if the first character of the text matches the one provided then we return the matched character as result and the rest of the text to process, otherwise we output an error. The function also takes care of updating the position of the parser accordingly and to skip line breaks.

\begin{lstlisting}
let character(c:char) : Parser<char, 'ctxt> = 
  (fun buf ctxt (pos:Position) ->
   match buf : List<char> with
   | x::cs when x = c -> 
       let pos' = 
       if x = '\n' then 
         pos.NextLine 
       else 
         pos.NextCol
         First( c, cs, ctxt, pos')
   | _ -> 
      Second (Error(pos, sprintf "Expected character %A" c))) |> Parser.Make
\end{lstlisting}

\noindent
The word parser takes as input the text to process and the word to match. It then applies the character parser to the word until it has all been processed. In the code below the syntax \texttt{let! x = y} is a syntactical sugar for \texttt{y >>= fun x -> ...} in the fashion of Haskell \texttt{do} notation.

\begin{lstlisting}
let rec word (w:List<char>) : Parser<List<char>, 'ctxt> =
  p{
    match w with
    | x::xs ->
        let! c = character x
        let! cs = word xs
        return c::cs
    | [] -> 
        return []
  }
\end{lstlisting}
\section{Type systems and type checking}
\label{sec:ch_background_type_checking}

\section{Operational semantics}
\label{sec:ch_background_semantics}

\section{Meta-compilers}

\begin{itemize}[noitemsep]
	\item General overview
	\item META-languages overview.
	\item RML overview.
	\item Possibly other meta-compilers (?)
\end{itemize}

Describe what it is existing in literature.




	
\chapter{Metacasanova}
\label{ch:metacasanova}
\epigraph{Typing is no substitute for thinking}{Dartmouth Basic manual, 1964}
\section{Repetitive steps in compilers development}
In Section \ref{sec:introduction} we briefly stated that the process of developing a compiler includes several steps that are repetitive, i.e. their behaviour is always the same regardless of the language for which the compiler is built. In this section we show in what way this process is repetitive and what is the common pattern 

\subsection{Type checking}
Type systems are generally expressed in the form of logical rules \cite{cardelli1996type}, made of a set of premises, that must be verified in order to assign to the language construct the type defined in the conclusion. For example the following rule defines the typing of an \texttt{if-then-else} statement in a functional programming language:\footnote{Note that the type rule of \texttt{if-then-else} in an imperative programming language is different.}

\begin{mathpar}
	\mprset{flushleft}
	\inferrule*{\Gamma \vdash c : bool \quad \Gamma \vdash t : \tau \quad \Gamma \vdash e : \tau}
	{\Gamma \vdash \text{if \textit{c} then \textit{t} else \textit{e}} : \tau}
\end{mathpar}

\noindent
In this rule $\Gamma$ is the environment. The type rule first evaluates the premises, which means that if the condition of the \texttt{if-then-else} has type \texttt{bool} and both \texttt{then} and \texttt{else} blocks have the same type, then the whole \texttt{if-then-else} has the type of either blocks.

Typing a construct of the language requires to evaluate its corresponding typing rule. In order to do so, the behaviour of each typing rule must be implemented in the host language in which the compiler is defined. Independently of the chosen language, the behaviour will always be the following : (\textit{i}) evaluate a premise, (\textit{ii}) if the evaluation of the premise fails then the construct fails the type check and an error is returned, (\textit{iii}) repeat step 1 and 2 until all the premises have been evaluated, and (\textit{iv}) assign the type to the construct that is defined in the rule conclusion.

\subsection{Semantics}
Semantics define how the language abstractions behave and can be expressed in different ways, for example with a term-rewriting system \cite{klop1992term} or with the operational semantics \cite{plotkin1981}. For the scope of this work, we choose to rely on the operational semantics. The definition of the operational semantics of a language abstraction is, again, in the form of a logical rule where the conclusion (which is the final behaviour of the construct) is achieved if the evaluation of the premises lead to the desired results. For instance, the operational semantics of a while loop could be the following:

\begin{mathpar}
	\mprset{flushleft}
	\inferrule*
	{\langle c \rangle \Rightarrow \text{\texttt{true}}}
	{\langle \text{while \textit{c} do \textit{L} ; \textit{k}} \rangle \Rightarrow \langle \text{\textit{L} ; while \textit{c} do \textit{L} ; \textit{k}} \rangle}
	
	\inferrule*
	{\langle c \rangle \Rightarrow \text{\texttt{false}}}
	{\langle \text{while \textit{c} do \textit{L} ; \textit{k}} \rangle \Rightarrow \langle k \rangle}
\end{mathpar}

Again, the behaviour of the semantics rule must be encoded in the host language in which the compiler is being developed, but the pattern it follows is always the same. This step, depending on the implementation choice, might also require to translate this behaviour into an \textit{intermediate language} representation that is more suitable for the subsequent code generation phase.

\subsection{Discussion}
The examples above show how the behaviour of the type checking and semantics rules must be hard-coded in the language chosen for the compiler implementation, regardless of the fact that their pattern is constantly repeated in every rule. This pattern can be captured in a meta-language that is able to process the type system and operational semantics definition of the language and produce the code to execute the behaviour of the rules. In this work we describe the meta-language for \textit{Metacasanova}, a meta-compiler that is able to read a program written in terms of type system/operational semantics rules defining a programming language, a program written in that language, and output executable code that mimics the behaviour of the semantics. Such a language relieves the programmer from writing boiler-plate code when implementing a compiler for a (Domain-Specific) language. For this reason we formulate the following research question:

\vspace{0.2cm}
\noindent
\textbf{Research question 1:} \textit{To what extent Metacasanova eases the development speed of a compiler for a Domain-Specific Language, in terms of code length compared to the hard-coded implementation, and how much does the abstraction layer of the Metacompiler affect the performance of the generated code?}

\vspace{0.2cm}
Another problem that arises when using meta-compilers is the performance decay given by the introduction of their additional abstraction layer. One of the reasons for this performance decay (see Section \ref{subsec:code_generation_discussion}) is that the meta-language (and thus the meta-type system) is unaware of the type system and the memory model of the language implemented in the meta-compiler. For this reason, checking the types and accessing the memory requires to dynamically look up a symbol table defined with the abstractions provided by the meta-language. The need for performance is for Metacasanova important because it is being used to extend the DSL for games \textit{Casanova} \cite{abbadi2015casanova, abbadithesis2017}. Thus, we formulate a second research question:

\vspace{0.2cm}
\noindent
\textbf{Research question 2:} \textit{In what way can we embed the type system of the implemented language in Metacasanova in order to get rid of the dynamic lookups at runtime and what is the performance gain of this optimization?}

\vspace{0.2cm}
We try to answer these two research questions by using a two-steps methodology: (\textit{i}) we present an architecture for Metacasanova aimed to automate the process of code generation, and then (\textit{ii}) we propose a language extension to embed the implemented language type system in the meta-type system of Metacasanova.

\subsection{Related work}
\textit{RML} \cite{pettersson1996compiler} is a meta-compiler based on operational semantics that is similar to Metacasanova. Its syntax is very close to that of ML and it generates C code. A notable effort was done to optimize the tail calls in the generated code for the rules, but the problem arisen by Research Question 2 is not addressed.

\textit{Stratego} \cite{bravenboer2008stratego} is a meta-compiler based on a transformation system. A transformation language consists of a series of constructor calls to construct the terms of the grammar and functions that specify how to evaluate the terms. Stratego is not a typed language, so it does not ensure that the terms and transformation functions are used consistently.

A language extension for Haskell involving \textit{template meta-programming} exists \cite{sheard2002template}. Although a valuable and elegant approach, using Haskell language extensions is not suitable for domain-specific languages for games due to the wide use of monads a lambda abstractions, which greatly affect the performance, and the lazy nature of Haskell that affects the memory usage. In Section \ref{sec:code_generation} we underline how this project was born to ease the extension of a domain-specific language for game development, thus this was not a suitable choice for our initial goals.

Syntax Macro meta-programming \cite{campbell1978compiler} is an approach that operates during the parsing phase. Macros are used to produce an abstract syntax tree that is replaced when the macro is invoked. One notable example of this kind of \\meta-programming can be found in the language Lisp. Macros guarantee syntactic safety \cite{weise1993programmable} but not semantics safety, since no meta-type system is available for macros.

\subsection{Requirements of Metacasanova}
In order to relieve programmers of manually defining the behaviour described in Section \ref{sec:problem} in the back-end of the compiler, we propose the following features for Metacasanova:

\begin{itemize}
	\item It must be possible to define custom operators (or functions) and data containers. This is needed to define the syntactic structures of the language we are defining.
	\item It must be typed: each syntactic structure can be associated to a specific type in order to be able to detect meaningless terms (such as adding a string to an integer) and notify the error to the user.
	\item It must be possible to have polymorphic syntactical structures. This is useful to define equivalent ``roles'' in the language for the same syntactical structure; for instance we can say that an integer literal is both a \textit{Value} and an \textit{Arithmetic expression}.
	\item It must natively support the evaluation of semantics rules, as those shown above.
\end{itemize}

We can see that these specifications are compatible with the definition of meta-compiler, as the software takes as input a language definition written in the meta-language, a program for that language, and outputs runnable code that mimics the code that a hard-coded compiler would output.

\subsection{General overview}

A Metacasanova program is made of a set of \texttt{Data} and \texttt{Function} definitions, and a sequence of rules. A data definition specifies the constructor name of the data type (used to construct the data type), its field types, and the type name of the data. Optionally it is possible to specify a priority for the constructor of the data type. For instance this is the definition of the sum of two arithmetic expression

\begin{lstlisting}
Data Expr -> "+" -> Expr : Expr
\end{lstlisting}

\noindent
Note that Metacasanova allows you to specify any kind of notation for data types in the language syntax, depending on the order of definition of the argument types and the constructor name. In the previous example we used an infix notation. The equivalent prefix and postfix notations would be:

\begin{lstlisting}
Data "+" -> Expr -> Expr : Expr
Data Expr -> Expr -> "+" : Expr
\end{lstlisting}

\noindent
A function definition is similar to a data definition but it also has a return type. For instance the following is the evaluation function definition for the arithmetic expression above:

\begin{lstlisting}
Func "eval" -> Expr : Value
\end{lstlisting}

\noindent
In Metacasanova it is also possible to define polymorphic data in the following way:

\begin{lstlisting}
Value is Expr
\end{lstlisting}

\noindent
In this way we are saying that an atomic value is also an expression and we can pass both a composite expression and an atomic value to the evaluation function defined above.

Metacasanova also allows to embed C\# code \footnote{See Section \ref{sec:code_generation} for the motivation.} into the language by using double angular brackets. This code can be used to embed .NET types when defining data or functions, or to run C\# code in the rules. For example in the following snippets we define a floating point data which encapsulates a floating point number of .NET to be used for arithmetic computations:

\begin{lstlisting}
Data "$f" -> <<float>> : Value
\end{lstlisting}

\noindent
A rule in Metacasanova, as explained above, may contain a sequence of function calls and clauses. In the following snippet we have the rule to evaluate the sum of two floating point numbers:

\begin{lstlisting}
eval a => $f c
eval b => $f d
<<c + d>> => res
------------------------
eval (a + b) => $f res
\end{lstlisting}

\noindent
Note that if one of the two expressions does not return a floating point value, then the entire rule evaluation fails. Also note that we can embed C\# code to perform the actual arithmetic operation. Metacasanova selects a rule by means of pattern matching (in order of declaration of rules) on the function arguments. This means that both of the following rules will be valid candidates to evaluate the sum of two expressions:

\begin{lstlisting}
...
---------------
eval expr => res

...
----------------
eval (a + b) => res
\end{lstlisting} 

Finally the language supports expression bindings with the following syntax:

\begin{lstlisting}
x := $f 5
\end{lstlisting}

\begin{comment}
\subsection{Syntax in BNF}
The following is the syntax of Metacasanova in Backus-Naur form. Note that, for brevity, we omit the definitions of typical syntactical elements of programming languages, such as literals or identifiers:

\begin{lstlisting}[basicstyle = \ttfamily\tiny]
<program> ::= 
{<include>} {<import>} {<data>} <function> {<function>} {<alias>} <rule> {<rule>}
<premise> ::= 
<clause> | <functionCall> | <binding>
<binding> ::= 
id ":=" <constructor>
<rule> ::= 
{premise} "-" {"-"} <functionCall>
<clause> ::= //typical boolean expression
<functionCall> ::= 
<id> {<argument>} <arrow> <argument> | 
{<argument>} <id> {<argument>} <arrow> <argument> | 
<id> {<argument>} <arrow> <argument>
<arrow> ::= "=>" | "==>"
<constructor> ::= 
<id> {<argument>} | 
{<argument>} <id> {<argument>} | 
{<argument>} <id>
<external> ::= "<<" <csharpexpr> ">>"
<csharpexpr> ::= //all available C# expressions
<argument> ::= 
["("] 
(<id> | 
<external> | 
<literal> | 
<constructor>) 
[")"]
<literal> ::= //typical literals such as integer, float, string, ...
<import> ::= import id {"." id}
<include> ::= include id {.id}
<alias> ::= <typeDef> is <typeDef>
<typeDef> ::= id | "<<" id ">>"
<typeArguments> :: = 
'"' <id> '"' {"->" <typeDef>} ":" <typeDef> |
<typeDef> {"->" <typeDef>} "->" '"' <id> '"' {"->" <typeDef>} ":" <typeDef> |
<typeDef> {"->" typeDef} "->" '"' <id> '"' ":" <typeDef> 
<function> ::= Func <typeArguments> "=>" <typeDef> [Priority <literal>]
<data> ::= Data <typeArguments> [Priority <literal>]
\end{lstlisting}
\end{comment}

\subsection{Formalization}
In what follows we assume that the pattern matching of the function arguments in a rule succeeds, otherwise a rule will fail to return a result.
The informal semantics of the rule evaluation in Metacasanova is the following:
\begin{enumerate}
	\item[R1] A rule with no clauses or function calls always returns a result.
	\item[R2] A rule returns a result if all the clauses evaluate to \texttt{true} and all the function calls in the premise return a result.
	\item[R3] A rule fails if at least one clause evaluates to \texttt{false} or one of the function calls fails (returning no results).
\end{enumerate}
We will express the semantics, as usual, in the form of logical rules, where the conclusion is obtained when all the premises are true.
In what follows we consider a set of rules defined in the Metacasanova language $R$. Each rule has a set of function calls $F$ and a set of clauses (boolean expressions) $C$. We use the notation $f^{r}$ to express the application of the function $f$ through the rule $r$. We will define the semantics by using the notation $\langle expr \rangle$ to mark the evaluation of an expression, for example $\langle f^{r} \rangle$ means evaluating the application of $f$ through $r$. The following is the formal semantics of the rule evaluation in Metacasanova, based on the informal behaviour defined above:


\begin{mathpar}
	\mprset{flushleft}
	\inferrule*[left=R1:]
	{C = \emptyset \\\\ F = \emptyset}
	{\langle f^{r} \rangle \Rightarrow \lbrace x \rbrace} \\
	
	\mprset{flushleft}
	\inferrule*[left=R2:]
	{\forall c_{i} \in C \;, \langle c_{i} \rangle \Rightarrow true \\\\
		\forall f_{j} \in F \; , \exists r_{k} \in R \; | \; \langle f_{j}^{r_{k}} \rangle \Rightarrow \lbrace x_{r^{k}} \rbrace}
	{\langle f^{r} \rangle \Rightarrow \lbrace x_{r} \rbrace} \\
	
	\mprset{flushleft}
	\inferrule*[left=R3(a):]
	{\exists c_{i} \in C \; | \; \langle c_{i} \rangle \Rightarrow false}
	{\langle f^{r} \rangle \Rightarrow \emptyset} \\
	
	\mprset{flushleft}
	\inferrule*[left=R3(b)]
	{\forall r_{k} \in R \; , \exists f_{j} \in F \; | \; \langle f_{j}^{r_{k}} \rangle \Rightarrow \emptyset}
	{\langle f^{r} \rangle \Rightarrow \emptyset}
\end{mathpar}

R1 says that, when both $C$ and $F$ are empty (we do not have any clauses or function calls), the rule in Metacasanova returns a result. R2 says that, if all the clauses in $C$ evaluates to true and, for all the function calls in $F$ we can find a rule that returns a result (all the function applications return a result for at least one rule of the program), then the current rule returns a result. R3(a) and R3(b) specify when a rule fails to return a result: this happens when at least one of the clauses in $C$ evaluates to false, or when one of the function applications does not return a result for any of the rules defined in the program.

\vspace{0.2cm}
\noindent
In the following section we describe how the code generation process works, namely how the \texttt{Data} types of Metacasanova are mapped in the target language, and how the rule evaluation is implemented.

In Section \ref{sec:semantics} we defined the syntax and semantics of Metacasanova. In this section we explain how the abstractions of the language are compiled into the generated code. We chose C\# as target language because the development of Metacasanova started with the idea of expanding the DSL for game development Casanova with further functionalities. Casanova hard-coded compiler generated C\# code as well because it is compatible with game engines such as Unity3D and Monogame. At the same time, C\# grants decent performance without having to manually manage the memory such as for lower-level languages like C/C++. Code generation in different target languages is possible but still an ongoing project (see Section \ref{sec:conclusion}).

\section{Parsing}

\section{Type checking}

\section{Code generation}

\subsection{Data structures code generation}
The type of each data structure is generated as an interface in C\#. Each data structure defined in Metacasanova is mapped to a \texttt{class} in C\# that implements such interface. The class contains as many fields as the number of arguments the data structure contains. Each field is given an automatic name \texttt{argC} where \texttt{C} is the index of the argument in the data structure definition. The data structure symbols used in the definition might be pre-processed and replaced in order to avoid illegal characters in the C\# class definition. The class contains an additional field that stores the original name of the data structure before the replacement is performed, used for its ``pretty print''. For example the data structure

\begin{lstlisting}
Data "$i" -> int : Value
\end{lstlisting}

\noindent
will be generated as

\begin{lstlisting}
public interface Value {  }

public class __opDollari : Value
{
public string __name = "$i";
public int __arg0;

public override string ToString()
{
return "(" + __name + " " + __arg0 + ")";
}
}
\end{lstlisting}

\subsection{Code generation for rules}
Each rule contains a set of premises that in general call different functions to produce a result, and a conclusion that contains the function evaluated by the current rule and the result it produces. The code generation for the rules follows the steps below:

\begin{enumerate}
	\item Generate a data structure for each function defined in the meta-program.
	\item For each function $f$ extract all the rules whose conclusion contains $f$.
	\item Create a \texttt{switch} statement with a case for each rule that is able to execute the function (the function is in its conclusion).
	\item In the case block of each rule, define the local variables defined in the rule.
	\item Apply pattern matching to the arguments of the function contained in the conclusion of the rule. If it fails, jump immediately to the next case (rule).
	\item Store the values passed to the function call into the appropriate local variables.
	\item Run each premise by instantiating the class for the function used by it and copying the values into the input arguments.
	\item Check if the premise outputs a result and, in the case of an explicit data structure argument, check the pattern matching. If the premise result is empty or the pattern matching fails for all the possible executions of the premise then jump to the next case.
	\item Generate the result for the current rule execution. 
\end{enumerate}

\noindent
In what follows, we use as an example the code generation for the following rule (which computes the sum of two integer expressions in a programming language):

\begin{lstlisting}
eval a -> $i c
eval b -> $i d
<< c + d >> -> e
----------------
eval (a + b) -> $i e
\end{lstlisting}

From now on we will refer to an argument as \textit{explicit data argument} when its structure appears explicitly in the conclusion or in one of the premises, as in the case of \texttt{a + b} in the example above.

\subsubsection{Data structure for the function}
\label{subsubsec:function_generation}

As first step the meta-compiler generates a class for each function defined in the meta-program. This class contains one field for each argument the function accepts. It also contains a field to store the possible result of its evaluation. This field is a \texttt{struct} generated by the meta-compiler defined as follows:

\begin{lstlisting}
public struct __MetaCnvResult<T> { public T Value; public bool HasValue; }
\end{lstlisting}

The result contains a boolean to mark if the rule actually returned a result or failed, and a value which contains the result in case of success.

For example, the function

\begin{lstlisting}
Func eval -> Expr : Value
\end{lstlisting}

\noindent
will be generated as

\begin{lstlisting}
public class eval
{
public Expr __arg0;
public __MetaCnvResult<Value> __res;
...
}
\end{lstlisting}

\subsubsection{Rule execution}

The class defines a method \texttt{Run} that performs the actual code execution. The meta-compiler retrieves all the rules whose conclusion contains a call to the current function, which define all the possible ways the function can be evaluated with. It then creates a \texttt{switch} structure where each \texttt{case} represents each rule that might execute that function. The result of the rule is also initialized here (the \texttt{struct} will contain a default value and the boolean flag will be set to \texttt{false}). Each \texttt{case} defines a set of local variables, that are the variables used within the scope of that rule.

\subsubsection{Local variables definitions and pattern matching of the conclusion}

At the beginning of each \texttt{case}, the meta-compiler defines the local variables initialized with their respective default values. It also generates then the code necessary for the pattern-matching of the conclusion arguments. Since variables always pass the pattern-matching, the code is generated only for arguments explicitly defining a data structure (see the examples about arithmetic operators in Section \ref{sec:semantics}) and literals. If the pattern matching fails then the execution jumps to the next \texttt{case} (rule). For instance, the code for the following conclusion

\begin{lstlisting}
...
-------------
eval (a + b) -> $i e
\end{lstlisting}

\noindent
is generated as follows

\begin{lstlisting}
case 0:
{
Expr a = default(Expr);
Expr b = default(Expr);
int c = default(int);
int d = default(int);
int e = default(int);
if (!(__arg0 is __opPlus)) goto case 1;
...
}
\end{lstlisting}

\noindent
Note that an explicit data argument, such in the example above, might contain other nested explicit data arguments, so the pattern-matching is recursively performed on the data structure arguments themselves.

\subsubsection{Copying the input values into the local variables}
When each function is called by a premise, the local values are stored into the class fields of the function defined in Section \ref{subsubsec:function_generation}. These values must be copied to the local variables defined in the \texttt{case} block representing the rule. Particular care must be taken when one argument is an explicit data. In that case, we must copy, one by one, the content of the data into the local variables bound in the pattern matching. For example, in the rule above, we must separately copy the content of the first and second parameter of the explicit data argument into the local variables \texttt{a} and \texttt{b}. The generated code for this step, applied to the example above, will be:

\begin{lstlisting}
__opPlus __tmp0 = (__opPlus)__arg0;
a = __tmp0.__arg0;
b = __tmp0.__arg1;
\end{lstlisting}

Note that the type conversion from the polymorphic type \texttt{Expr} into \texttt{opPlus} is now safe because we have already checked during the pattern matching that we actually have \texttt{opPlus}.

\subsubsection{Generation of premises}
Before evaluating each premise, we must instantiate the class for the function that they are invoking. The input arguments of the function call must be copied into the fields of the instantiated object. If one of the arguments is an explicit data argument, then it must be instantiated and its arguments should be initialized, and then the whole data argument must be assigned to the respective function field. After this step, it is possible to invoke the \texttt{Run} method of the function to start its execution. The first premise of the example above then becomes (the generation of the second is analogous):

\begin{lstlisting}
eval a -> $i c
\end{lstlisting}

\begin{lstlisting}
eval __tmp1 = new eval();
__tmp1.__arg0 = a;
__tmp1.Run();
\end{lstlisting}

\subsubsection{Checking the premise result}
After the execution of the function called by a premise, we must check if a rule was able to correctly evaluate it. In order to do so, we must check that the result field of the function object contains a value, and if not the rule fails and we jump to the next case (rule), which is performed in the following way:

\begin{lstlisting}
if (!(__tmp1.__res.HasValue)) goto case 1;
\end{lstlisting}

If the premise was successfully evaluated by one rule, then we must check the structure of the result, which leads to the following three situations:
\begin{enumerate}
	\item The result is bound to a variable.
	\item The result is constrained to be a literal.
	\item The result is an explicit data argument.
\end{enumerate}

In the first case, as already explained above, the pattern matching always succeeds, so no check is needed. In the second case, it is enough to check the value of the literal. In the last case, all the arguments of the data argument must be checked to see if they match the expected result. In general this process is recursive, as the arguments could be themselves other explicit data arguments. If the result passes the check, then the result is copied into the local variables, in a fashion similar to the one performed for the function premise. For instance, for the premise

\begin{lstlisting}
eval a -> $i c
\end{lstlisting}

\noindent
the meta-compiler generates the following code to check the result
\begin{lstlisting}
if (!(__tmp1.__res.Value is __opDollari)) goto case 1;
__MetaCnvResult<Value> __tmp2 = __tmp1.__res;
__opDollari __tmp3 = (__opDollari)__tmp2.Value;
c = __tmp3.__arg0;
\end{lstlisting}

\subsubsection{Generation of the result}
When all premises correctly output the expected result, the rule can output the final result. In order to do that, the generated code must copy the right part of the conclusion (the result) into the \texttt{res} variable of the function class. If the right part of the conclusion is, again, an explicit data argument, then the data object must first be instantiated and then copied into the result. For example the result of the rule above is generated as follows:

\begin{lstlisting}
res = c + d;
__opDollari __tmp7 = new __opDollari();
__tmp7.__arg0 = res;
__res.HasValue = true;
__res.Value = __tmp7;
break;
\end{lstlisting}

\noindent
After this step, the rule evaluation successfully returns a result.

This implementation choice is due to the fact that we plan to support partial function applications, thus, when a function is partially applied, there is the need to store the values of the arguments that were partially given. This could still be implemented with static methods and lambdas in C\#, but not all programming languages natively support lambda abstractions, so we chose to have a set-up that allows us to change the target language without dramatically altering the logic of code generation.

\subsection{Discussion}
\label{subsec:code_generation_discussion}
Metacasanova has been evaluated in \cite{DiGiacomo2017} by re-building the DSL for game development Casanova \cite{abbadi2015casanova, abbadithesis2017}. Even though the size of the code required to implement the language has been drastically reduced (almost 1/5 shorter), performance dropped dramatically. We identified a main problem causing the performance decay that, if solved, will improve the performance of the generated code.

In order to encode a symbol table in the meta-compiler in the current implementation (used for example to store the variables defined in the local scope of a control structure or to model a class/record data structure), we are left with two options: (\textit{i}) define a custom data structure made of a list of pairs, containing the field/variable name as a string and its value, in the following way

\begin{lstlisting}
Data "table" -> List[Tuple[string, Value]] : SymbolTable
\end{lstlisting}

\noindent
or (\textit{ii}) use a dictionary data structure coming from .NET, such as \texttt{ImmutableDictionary}, which was the implementation choice for Casanova. In both cases, the behaviour of the language implemented in Metacasanova will be that of a dynamic language, because whenever the value of a variable or class field must be read, the evaluation rule must look up the symbol table at run time to retrieve the value, whose complexity will be $O(n)$ with the list implementation and $O(\log n)$ with the dictionary implementation. This issue is caused by the fact that, in the current state of Metacasanova, the meta-type system is unaware of the type system of the language that is being implemented in the meta-compiler. This is not a problem limited to Metacasanova but to all meta-compilers having a meta-type system that does not allow embedding of the host language type system. In the next section we propose an extension to Metacasanova to overcome this problem by embedding the type system of the implemented language in the meta-type system of Metacasanova and inlining the code to access the appropriate variable at compile time.

\section{The C-{}- language}

\section{Casanova 2.5 in Metacasanova}
In this section we will briefly introduce the Casanova language, a domain specific language for games. We then show a re-implementation, which we call Casanova 2, of the Casanova 2 language hard-coded compiler as an example of use of Metacasanova.


%expand with the full implementation of Casanova 2
\subsection{The Casanova language}
Casanova 2.5 is a language oriented to video game development which is based on Casanova 2 \cite{CASANOVA2_PAPER}. A program in Casanova is a tree of \textit{entities}, where the root is marked in a special way and called \textit{world}. Each entity is similar to a \textit{class} in an object-oriented programming language: it has a constructor and some fields. The fields do not have access modifiers because they are not directly modifiable from the code except with a specific statement. Each entity also contains a list of \textit{rules}, that are methods that are ticked in order with a specific refresh rate called \texttt{dt}. Each rule takes as input four elements: \texttt{dt}, \texttt{this}, which is a reference to the current entity, \texttt{world} that is a reference to the world entity, and a subset of entity fields called \textit{domain}. A rule can only modify the fields contained in the domain. The rules can be paused for a certain amount of seconds or until a condition is met by using the \texttt{wait} statement. It is possible to modify the values of the fields in the domain by using the \texttt{yield} statement which takes as input a tuple of values to assign to the fields. When the \texttt{yield} statement is executed the rule is paused until the next frame. Also the body of control structures (\texttt{if-then-else}, \texttt{while}, \texttt{for}) is interruptible. In the following section we show the implementation of Casanova 2.5 in Metacasanova.

\subsection{Casanova 2.5}
The memory in Casanova 2.5 is represented using three maps, where the key is the variable/field name, and the value is the value stored in the variable/field. The first dictionary represents the global memory (the fields of the \texttt{world} entity or \textit{Game State}), the second dictionary represents the current entity fields, and the third the variable bindings local to each rule.

The core of the entity update is the \texttt{tick} function. This function evaluates in order each rule in the entity by calling the \texttt{evalRule} function. This function executes the body of the rule and returns a result depending on the set of statements that has been evaluated. This result is used by \texttt{tick} to update the memory and rebuild the rule body to be evaluated at the next frame. The result of \texttt{tick} is a \texttt{State} containing the rules updated so far, and the updated entity and global fields. Since a rule must be restarted after the whole body has been evaluated, we need to store a list containing the original rules, which will be restored when evaluation returns \texttt{Done} (see below). At each step the function recursively calls itself by passing the remaining part of original rules (the rules which body was not altered by the evaluation of the statements) and modified rules (which body has been altered by the evaluation of the statements) to be evaluated. The function stops when all the rules have been evaluated, and this happens when both the original and the modified rule lists are empty.

Interruption is achieved by using \textit{Continuation passing style}: the execution of a sequence of statements is seen as a sequence of steps that returns the result of the execution and the remaining code to be executed. Every time a statement is executed we rebuild a new rule whose body contains the continuation which will be evaluated next. 

\begin{comment}
For example, consider the following rule:

\begin{lstlisting}
rule X,Y =
while X > 0 do
wait 1.0f
yield X - 1,Y + 1
\end{lstlisting}

The code is executed atomically until the \texttt{wait} statement (assuming that the \texttt{while} condition is true). At that point we rebuild a new rule containing the code to execute at the next iteration:

\begin{lstlisting}
rule X,Y =
wait (1.0f - dt)
yield X - 1, Y + 1
while X > 0 do
wait 1.0f
yield X - 1,Y + 1
\end{lstlisting}
Note that the \texttt{while} is placed at the end of the continuation because it must be re-evaluated after the first iteration is complete, and that we have decreased the waiting time by \texttt{dt} (the time elapsed between one frame and the previous one).
\end{comment}

The possible results returned by the \texttt{tick} function are the following: (\textit{i}) \texttt{Suspend} contains a \texttt{wait} statement with the updated timer, the continuation, and a data structure called \texttt{Context} which contains the updated local variables, the entity fields, and the global fields. The function rebuilds a rule which body is the sequence of statements contained by the \texttt{Suspend} data structure. (\textit{ii}) \texttt{Resume} is returned when the timer must resume after the last waited frame. In order not to skip a frame we must still re-evaluate the rule at the next frame and not immediately. In this case the argument of \texttt{Resume} is only the remaining statements to be executed. (\textit{iii}) \texttt{Yield} stops evaluation for one frame. We use the continuation to rebuild the rule body. Memory is updated by \texttt{evalRule}. (\textit{iv}) \texttt{Done} stops the evaluation for one frame and rebuilds the original rule body by taking it from the original rules list.

For brevity we write only the code for \texttt{Suspend}. A full implementation can be found at \cite{CASANOVA_SOURCE_CODE}. You can see a schematic representation of the tick function in Figure \ref{fig:tick}.

\begin{lstlisting}
evalRule (rule dom body k locals delta) fields globals => Suspend (s;cont) (Context newLocals newFields newGlobals)
r := rule dom s cont newLocals dt
tick originals rs newFields newGlobals dt => State updatedRules updatedFields updatedGlobals
st := State (r::updatedRules) updatedFields updatedGlobals
------------------------------------------------------
tick (original::originals) ((rule dom body k locals delta)::rs) fields globals dt => st
\end{lstlisting}


\begin{figure}
	\centering
	\includegraphics[scale = 0.25]{Figures/tick2}
	\caption{Casanova 2.5 rule evaluation}
	\label{fig:tick}
\end{figure}

The function \texttt{evalRule} calls \texttt{evalStatement} to evaluate the first statement in the body of the rule passed as argument. The result of the evaluation of the statement is processed in the following way: (\textit{i}) if the result is \texttt{Done}, \texttt{Suspend} or \texttt{Resume} then it is just returned to the caller function. We omit the code for this case, since it is trivial; (\textit{ii}) if the result is \texttt{Atomic} it means that the evaluated statement was uninterruptible and the remaining statements of the rule must be re-evaluated immediately; (\textit{iii}) if the result is \texttt{Yield} then the fields in the domain are updated recursively in order and then the updated memory is encapsulated in the \texttt{Yield} data structure and passed to the caller function.

\vspace{0.1cm}
\begin{lstlisting}
evalStatement b k ctxt dt => Atomic z c    
evalRule (rule dom z nop c dt) => res
-------------------------------
evalRule (rule dom b k ctxt dt)  => res
\end{lstlisting}

\begin{lstlisting}
evalStatement b k (Context locals fields globals) dt => Yield ks values context
updateFields dom values context  => updatedContext
--------------------------------------------------------
evalRule (rule dom b k locals dt) fields globals => Yield ks values updatedContex
\end{lstlisting}

Note that, in case of a rule containing only atomic statements, we will eventually return \texttt{Done} after having recursively called \texttt{evalStatement} for all the statements, and the rule will be paused for one frame.

\begin{comment}
\begin{figure}
\centering
\includegraphics[scale=0.15]{Pictures/statement_evaluation}
\caption{Statement evaluation}
\label{fig:statement_evaluation}
\end{figure}
\end{comment}

\noindent
The \texttt{evalStatement} function is used both to evaluate a single statement and a sequence of statements. When evaluating a sequence of statements, the first one is extracted. A continuation is built with the following statement and passed to a recursive call to \texttt{evalStatement} which evaluates the extracted statement. If the existing continuation is non-empty, then it is added before the current continuation. If both the continuation and the body are empty (situation represented by the \texttt{nop} operator) then it means the rule evaluation has been completed and we return \texttt{Done}.

\begin{lstlisting}
a != nop
---------------------                           ----------------------- 
addStmt a b => a;b                              addStmt nop nop => nop   

addStmt b k => cont
evalStatement a cont ctxt dt => res
-------------------------------                 -----------------------------------       
evalStatement (a;b) k ctxt dt => res            evalStatement nop nop ctxt dt => Done ctxt


\end{lstlisting}

\noindent
We will now present, for brevity, only the evaluation of the \texttt{wait} and \texttt{yield} statements. Both the evaluation of the control structures and the variable bindings always return \texttt{Atomic} because they do not, by definition, pause the execution of the rule.

The \texttt{wait} statement has two different evaluations, based on the rules defined in Section \ref{sec:problem_statement}: (\textit{i}) the timer has elapsed: in this case we return \texttt{Resume} which contains the code to execute after the \texttt{wait} statement, or (\textit{ii}) the timer has not elapsed: in this case we return \texttt{Suspend} which contains the \texttt{wait} statement with the updated timer followed by the continuation.


\begin{lstlisting}
<<t <= dt>> == false
----------------------------------
evalStatement (wait t) k ctxt dt => Suspend wait <<t - dt>>;k ctxt

<<t <= dt>> == true
----------------------------------
evalStatement (wait t) k ctxt dt => Resume k ctxt
\end{lstlisting}

\noindent
The \texttt{yield} statement takes as argument a list of expressions whose values are used to update the corresponding fields in the rule domain. The evaluation rule recursively evaluates the expressions and stores them into a list passed as argument of the \texttt{Yield} result. Those arguments are used later by \texttt{evalRule} to update the corresponding fields.

\begin{lstlisting}
eval expr ctxt => v
evalYield exprs ctxt => vs
-------------------------------------------          ----------------------------
evalYield (expr :: exprs) ctxt => v :: vs            evalYield nil ctxt => nil
\end{lstlisting}

In this section we provide an implementation of a patrol script for an entity in a game. The sample is made up of an entity, representing a guard, and a couple of checkpoints. The guard continuously moves between the two checkpoints. We choose this sample because this is a typical behaviour implemented in several games, where the user is able to set up a patrol route for a unit. We show the comparison between the sample implemented in Casanova 2.5 and an equivalent implementation in Python with respect to the running time. We then show a comparison between the hard-coded compiler of Casanova 2.0 and the implementation of Casanova 2.5 in Metacasanova with respect to the code length. 

\begin{comment}
We want to underline that the main goal of this work is \textbf{to ease the process of building a compiler for a DSL for games, thus our main objective is decreasing the code length and complexity necessary to implement a hard-coded compiler for the language}. At the same time we show that the compiled program in Casanova 2.5 \textbf{has performance similar to that of a language used in game development, and thus Casanova 2.5 is usable in a real scenario}.
\end{comment}

\subsection{Chosen languages}
We compared the running time of the sample in metacompiled Casanova with an equivalent implementation in Python. This language was chosen based on its use in game development: Python has been used extensively in several games such as Civlization IV \cite{CIV4} or World in Conflict \cite{WIC} because of the native support for coroutines. We deliberately ignore C++ and C\# implementations, although they are widely used in the industry, because we knew in advance \cite{CASANOVA2_PAPER} that the current version of the code generated by the meta-compiler would not match the high performance of these languages: the main goal of this work is to reduce the effort of writing a compiler for a DSL for games while having acceptable performance.


\subsection{Performance}
The performance results are shown in Table \ref{tab:evaluation}. We see that the generated code has performance on the same order as Python. This is mainly due to the fact that the memory, in the metacompiled implementation of Casanova, is managed through a map, and because of the virtuality of the implemented operators. Each time Casanova accesses a field in an entity this must be looked up into the map. To this we add the complexity of dynamic lookups when we must deal with polymorphic results into the rules. 

From Table \ref{tab:compiler_comparison} we see that the implementation of Casanova 2.0 language in Metacasanova is almost 5 times shorter in terms of lines of code than the previous Casanova implementation in F\#. We believe it is worthy noticing that structures with complex behaviours, such as \textit{wait} or \textit{when}, require hundreds of lines of codes with a standard approach (the code lines to define the behaviour of the structure plus the support code to correctly generate the state machine), while in the meta-compiler we just need tens of lines of codes to implement the same behaviour. Moreover we want to point out that the previous Casanova compiler was written in a functional programming language: these languages tend to be more synthetic than imperative languages, so the difference with the same compiler implemented in languages such as C/C++ might be even greater.

The readability with respect to the hard-coded compiler code is also improved: we managed to implement the behaviour of synchronization and timing primitives almost imitating one to one the formal semantics of the language definition (see the semantics rules in Section \ref{sec:formal_description} and their implementation in Section \ref{sec:casanova3}). In the hard-coded compiler implementation for Casanova 2.0 the semantics are lost in the code for generating finite state machines.

\begin{table}[!h]
	\centering
	\tiny	
	\begin{tabular}{|c|c|c|}
		\hline
		\multicolumn{3}{|c|}{\textbf{Casanova 2.5}} \\
		\hline
		Entity \# & Average update time (ms) & Frame rate \\
		\hline
		100 & 0.00349 & 286.53 \\
		\hline
		250 & 0.00911 & 109.77 \\
		\hline
		500 & 0.01716 & 58.275 \\
		\hline
		750 & 0.02597 & 38.506 \\
		\hline
		1000 & 0.03527 & 28.353 \\
		\hline
		\multicolumn{3}{|c|}{\textbf{Python}} \\
		\hline
		Entity \# & Average update time (ms) & Frame rate \\
		\hline
		100 & 0.00132 & 756.37 \\
		\hline
		250 & 0.00342 & 292.05 \\
		\hline
		500 & 0.00678 & 147.54 \\
		\hline
		750 & 0.01087 & 91.988 \\
		\hline
		1000 & 0.01408 & 71.002 \\
		\hline
	\end{tabular}
	\caption{Patrol sample evaluation}
	\label{tab:evaluation}
\end{table}

\begin{table}[!h]
	\centering
	\tiny
	\begin{tabular}{|c|c|}
		\hline
		\multicolumn{2}{|c|}{\textbf{Casanova 2.5 with Metacasanova}} \\
		\hline
		Module & Code lines \\
		\hline
		Data structures and function definitions & 40 \\
		\hline
		Query Evaluation & 16 \\
		\hline
		While loop & 4 \\
		\hline
		For loop & 5 \\
		\hline
		If-then-else & 4 \\
		\hline
		When & 4 \\
		\hline
		Wait & 6 \\
		\hline
		Yield & 10 \\
		\hline
		Additional rules for Casanova program evaluation & 40 \\
		\hline
		Additional rules for basic expression evaluation & 201 \\
		\hline
		\multicolumn{2}{|l|}{\textbf{Total: } 300} \\
		\hline
		\multicolumn{2}{|c|}{\textbf{Casanova 2.0 compiler}} \\
		\hline
		Module & Code lines \\
		\hline
		While loop & 10 \\
		\hline
		For-loop and query evaluation & 44 \\
		\hline
		If-Then-Else & 15 \\
		\hline
		When & 11 \\
		\hline
		Wait & 24 \\
		\hline
		Yield & 29 \\
		\hline
		Additional structures for rule evaluation & 63 \\
		\hline
		Structures for state machine generations & 754 \\
		\hline
		Code generation & 530 \\
		\hline
		\multicolumn{2}{|l|}{\textbf{Total: } 1480} \\
		\hline			
	\end{tabular}	
	\caption{meta-compiler vs standard compiler}
	\label{tab:compiler_comparison}
\end{table}

\subsection{Discussion}
\label{subsec:code_generation_discussion}
Metacasanova has been evaluated in \cite{DiGiacomo2017} by re-building the DSL for game development Casanova \cite{abbadi2015casanova, abbadithesis2017}. Even though the size of the code required to implement the language has been drastically reduced (almost 1/5 shorter), performance dropped dramatically. We identified a main problem causing the performance decay that, if solved, will improve the performance of the generated code.

In order to encode a symbol table in the meta-compiler in the current implementation (used for example to store the variables defined in the local scope of a control structure or to model a class/record data structure), we are left with two options: (\textit{i}) define a custom data structure made of a list of pairs, containing the field/variable name as a string and its value, in the following way

\begin{lstlisting}
Data "table" -> List[Tuple[string, Value]] : SymbolTable
\end{lstlisting}

\noindent
or (\textit{ii}) use a dictionary data structure coming from .NET, such as \texttt{ImmutableDictionary}, which was the implementation choice for Casanova. In both cases, the behaviour of the language implemented in Metacasanova will be that of a dynamic language, because whenever the value of a variable or class field must be read, the evaluation rule must look up the symbol table at run time to retrieve the value, whose complexity will be $O(n)$ with the list implementation and $O(\log n)$ with the dictionary implementation. This issue is caused by the fact that, in the current state of Metacasanova, the meta-type system is unaware of the type system of the language that is being implemented in the meta-compiler. This is not a problem limited to Metacasanova but to all meta-compilers having a meta-type system that does not allow embedding of the host language type system. In the next section we propose an extension to Metacasanova to overcome this problem by embedding the type system of the implemented language in the meta-type system of Metacasanova and inlining the code to access the appropriate variable at compile time.


		

\chapter{Language Design in Metacasanova}
\label{ch:languages}
\epigraph{A language that doesn't affect the way you think about programming is not worth knowing.}{Alan J. Perlis}
\section{The C-{}- language}
In this section we present the implementation of a small imperative language called C-{}-. Note that, although the name might suggest this, we do not claim any resemblance with the C programming language, as it lacks several features such as pointer arithmetic, arrays, and functions.

C-{}- allows the use of four built-in values: integer, strings, booleans, and floating-point numbers in double precision. The memory is represented using a dictionary that pairs variable names with their value. In what follows we omit the details of the lookup of entries in the dictionary for brevity. Suffice to say that the meta-program makes use of the \texttt{ImmutableDictionary} data structure available in .NET. Also note that C-{}- defines scopes for variables, so that if a variable is declared inside the scope of a code block in a control structure, that is usable only within the scope itself.

The core of the meta-program is made of the evaluation of both expressions and statements. We proceed below to present the details of both kinds of evaluation.

\subsection{Expression Semantics}
\label{subsec:ch_mcnv_languages_expression_semantics}
As explained above C-{}- supports boolean, string, integer, and floating-point values. These are represented through the following meta-data structures in the meta-program.

\begin{lstlisting}
Data "$i" -> <<int>> : Value Priority 300
Data "$d" -> <<double>> : Value Priority 300
Data "$s" -> <<string>> : Value Priority 300
Data "$b" -> <<bool>> : Value Priority 300
\end{lstlisting}

\noindent
Note that we are using the .NET data types to represent the actual values stored in the meta-data structures. We also define the following type equivalence, since values are atomic cases of expressions and can be used as such:

\begin{lstlisting}
Value is Expr
\end{lstlisting}

\noindent
Expressions can also contain variables, thus we need a meta-data structure to represent them as well.

\begin{lstlisting}
Data "$" -> <<string>> : Id Priority 300
\end{lstlisting}

\noindent
Variables can be used as atomic expressions as well, so we need an additional type equivalence

\begin{lstlisting}
Id is Expr
\end{lstlisting}

\noindent
We now define a data structure to represent the state of the program. The state is simply a map where the key is a variable name and the stored element a valid value in C-{}-. In the declaration we will define the meta-type \texttt{SymbolTable}, and from now on we will refer use the term ``symbol table'' as a synonym of ``state''.

\begin{lstlisting}
Data "$m" << ImmutableDictionary<Id, Value> >> : SymbolTable 
\end{lstlisting}

\noindent
Since we want to allow variable scoping, the state of the program is not represented by a single map, but by a list of maps. Each time the program enters a different scope context, an empty map is added to this list, and removed when the program exits the scope. This process will be further explained below. We define a meta-data structure to represent this list of states (note that the operator for the construction of the list is infix).

\begin{lstlisting}
Data SymbolTable -> "::" -> TableList : TableList
Data "[]" -> TableList
\end{lstlisting}

\noindent
We can finally proceed to define a meta-data structure to represent the operations for expressions. First we define the arithmetic operators in the language:

\begin{lstlisting}
Data Expr -> "+" -> Expr : Expr
Data Expr -> "-" -> Expr : Expr
Data Expr -> "*" -> Expr : Expr
Data Expr -> "/" -> Expr : Expr
\end{lstlisting}

\noindent
then we can define operators for boolean expressions:
\begin{lstlisting}
Data Expr -> "&&" -> Expr : Expr
Data Expr -> "||" -> Expr : Expr
Data "!" -> Expr : Expr
\end{lstlisting}

\noindent
and finally comparison operators:
\begin{lstlisting}
Data Expr -> "equals" -> Expr : Expr
Data Expr -> "neq" -> Expr: Expr
Data Expr -> "ls" -> Expr : Expr
Data Expr -> "leq" -> Expr : Expr
Data Expr -> "grt" -> Expr : Expr
Data Expr -> "geq" -> Expr : Expr
\end{lstlisting}

We now have to define the function that evaluates an expression through rules in the program. This function takes as input the list of symbol tables (needed to read possible variables), an expression, and returns the value after computing the expression.

\begin{lstlisting}
Func "evalExpr" -> TableList -> Expr : Value
\end{lstlisting}

\noindent
Now we have to proceed to define the rules to compute the actual evaluation of an expression. Clearly the base cases of the evaluation are the atomic values, where we immediately return the value itself.

\begin{lstlisting}
-----------------------------
evalExpr tables ($i v) -> ($i v)

-----------------------------
evalExpr tables ($d v) -> ($d v)

-----------------------------
evalExpr tables ($s v) -> ($s v)

-----------------------------
evalExpr tables ($b v) -> ($b v)
\end{lstlisting}

Evaluating variables is more complex: we have to look at the table currently in the head of the list of tables (which is the one relative to the current scope). If we do not find the required variable we have to recursively look it up in the tail of the list, since we could have an arbitrary amount of nested scopes. When the variable is found we return its value. This behaviour is implemented by the following code:

\begin{lstlisting}
symbols contains ($name) -> Yes
symbols lookup ($name) -> val
-------------------------------------------
evalExpr (symbols :: tables) ($name) -> val

symbols contains ($name) -> No
evalExpr tables ($name) -> val
-------------------------------------------
evalExpr (symbols :: tables) ($name) -> val

\end{lstlisting}

We proceed now to define the evaluation of arithmetic operators. We show only the example of the sum for brevity, the other rules differ only in the operator. Evaluating the arithmetic expression requires to recursively call \texttt{evalExpr} on the right and left argument. These recursive calls will eventually return two values that are the result of the two evaluations. After we obtain these values, we can compute their sum and return it as result.

\begin{lstlisting}
evalExpr tables expr1 -> ($i val1)
evalExpr tables expr2 -> ($i val2)
<<val1 + val2>> -> v
---------------------------------------
evalExpr tables expr1 + expr2 -> ($i v)
\end{lstlisting}

\noindent
Note that we have used .NET external code in the third premise to compute the result of the arithmetic operation. Evaluating arithmetic operations involving floating-point expressions can be done in an analogous way, except in the premises we expect to have the meta-data structure for floating-point values as result of \texttt{evalExpr}:

\begin{lstlisting}
evalExpr tables expr1 -> ($d val1)
evalExpr tables expr2 -> ($d val2)
<<val1 + val2>> -> v
---------------------------------------
evalExpr tables expr1 + expr2 -> ($d v)
\end{lstlisting}

\noindent
The same can be said for the string concatenation.
The evaluation of boolean expression is analogous: we show again only the evaluation for AND as the other rules are analogous:

\begin{lstlisting}
evalExpr tables expr1 -> ($b val1)
evalExpr tables expr2 -> ($b val2)
<<val1 && val2>> -> b
----------------------------------------------------------------------
evalExpr tables expr1 && expr2 -> b
\end{lstlisting}

\noindent
Again we rely on external code to compute the actual boolean value.
As for the comparison operators, we can use a clause in the premise to avoid using external code in the following way:

\begin{lstlisting}
evalEpxr tables expr1 -> val1
evalExpr tables expr2 -> val2
val1 == val2
------------------------------------------------
evalExpr tables (expr1 equals expr2) -> $b true

evalEpxr tables expr1 -> val1
evalExpr tables expr2 -> val2
val1 != val2
------------------------------------------------
evalExpr tables (expr1 equals expr2) -> $b false
\end{lstlisting}

\noindent
The first rule checks that the values computed by evaluating the left and right argument of the equality comparison are the same. If this happens then the rule returns a meta-data structure containing the boolean representation of \texttt{true}. Otherwise the first rule fails and the second one is executed. This one will return a boolean representation of \texttt{false} when the values are different.

For inequality operators we must rely on external code for the computation is Metacasanova only allows equality comparisons in clauses:

\begin{lstlisting}
evalExpr tables expr1 -> ($i val1)
evalExpr tables expr2 -> ($i val2)
<< val1 < val2 >> -> boolResult
---------------------------------------------------------
evalExpr tables (expr1 ls expr2) -> ($b boolResult)
\end{lstlisting}

\noindent
The evaluation of the other comparison operators is implemented through analogous rules, which differ only in the operators.

\subsection{Statement Semantics}
\label{subsec:ch_mcnv_languages_statement_semantics}

Statement evaluation requires the definition of a different function, \texttt{eval}, that processes each statement and returns the result of the statement evaluation and the updated state. Note that, even if the evaluation of statements does not always change the state, in general we have to assume that this will happen.

The function \texttt{eval} takes as input a statement to process and the current state (list of symbol tables), and returns the updated list of symbol tables

\begin{lstlisting}
Func "eval" -> TableList -> Stmt : TableList
\end{lstlisting}

We now proceed to define the meta-data structures necessary to represent the statements of the language: C-{}- supports (\textit{i}) variable declarations, (\textit{ii}) variable assignment, (\textit{iii}) if-then-else, (\textit{iv}) while loops, and (\textit{v}) for loops.

Variable declarations follow the same structure of standard C, that is a type name followed by an identifier. Thus, the corresponding meta-data structure can be defined as:

\begin{lstlisting}
"variable" -> Type -> Id : Stmt
\end{lstlisting}

\noindent
Analogously, variable assignment follows the C convention and uses the \texttt{=} symbol.

\begin{lstlisting}
Id -> "=" -> Expr : Stmt
\end{lstlisting}

The control structure if-then-else does not follow the standard C representation, rather we use the keywords \texttt{then} and \texttt{else} to delimit its code blocks. Note that nothing would prevent to implement the conventional C syntax, but we prefer this ``lightweight'' representation. The keywords \texttt{then} and \texttt{else} are meta-data structures that take no arguments and do not have any functional utility other than syntactical mark-ups.

\begin{lstlisting}
Data "then" : Then
Data "else" : Else
Data "if" -> Expr -> Then -> Stmt -> Else -> Stmt : Stmt
\end{lstlisting}

Analogously we can define the meta-data structure for \texttt{While} and \texttt{For}

\begin{lstlisting}
Data "do" : Do
Data "while" -> Expr -> Do -> Stmt : Stmt
Data "for" -> Expr -> Expr -> Epxr -> Do -> Stmt : Stmt
\end{lstlisting}

\noindent
Up to this point we are able to define single statements in the language, but we need a way to concatenate a sequence of statements to form code blocks, in the fashion of C. This is done by introducing an additional meta-data structure, which is the ";" symbol. For convenience, we also introduce a \texttt{nop} statement, which does not do anything, but it will be useful to express the semantics of statements evaluation.

\begin{lstlisting}
Data Stmt -> ";" -> StmtList : StmtList
Data "nop" -> :Stmt

StmtList is Stmt

\end{lstlisting}

We now proceed to define the semantics of statement evaluation.

\subsubsection{Evaluating a sequence of statement}
The evaluation of a sequence of statements require to evaluate the first statement in a sequence and then recursively evaluate the rest of the sequence. The recursive evaluation returns the final program state. The base case of the recursion is met when the sequence contains only \texttt{nop}. In this case we terminate the evaluation and return the unchanged program state.

\begin{lstlisting}
--------------------------
eval tables nop -> tables

eval tables a -> tables'
eval table' b -> res
---------------------------
eval tables (a;b) -> res
\end{lstlisting}

\subsubsection{Variable Declarations and Assignments}
Evaluating a variable declaration simply adds the variable to the symbol table of the current scope. Note that we allow variable shadowing, so it is possible to redefine the same symbol in different scopes.

\begin{lstlisting}
symbols defineVariable id -> symbols'
-------------------------------------
eval (symbols nextTable tables) (variable t id) -> symbols' nextTable tables
\end{lstlisting}

\noindent
This rule is executed whenever the processed statement matches the structure of a variable declaration statement. The premise adds the symbol to the symbol table of the current scope (we omit the details for brevity), and returns an updated symbol table. The list of symbol 
tables is rebuilt to include the updated table and returned as result.

Variable assignment is more complex, since the variable we are trying to use might not be in the symbol table of the current scope. We must then define two lookups functions, that behave differently depending on whether the variable is in the symbol table in the head of the symbol table list or not. We declare the function \texttt{updateTable} that performs this lookup and updates the table list accordingly. 

\begin{lstlisting}
Func "updateTables" -> TableList -> TableList -> Id -> Expr : EvaluationResult
\end{lstlisting}

\noindent
In the case that the variable is in the symbol table in the head of the list of tables we have the following rule:

\begin{lstlisting}
symbols contains id -> Yes
evalExpr vars expr -> val
symbols add id val -> symbols'
-------------------------------
updateTables vars (symbols :: tables) id expr -> symbols' :: tables
\end{lstlisting}

\noindent
The first premise checks if the symbol is contained in the table in the head of the list. If the answer is \texttt{Yes} (a meta-data structure returned by the function \texttt{contains}, not described here again for brevity), then the second premise proceeds to evaluate the expression in the right hand-side of the assignment. The third premise adds the value obtained as result of the second premise to the current symbol table and returns the modified table. The new table is than placed in the head of the table list and the whole list is returned. Note that, at this point, all the tables in the list remain unchanged except the one that was in the head. Note that \texttt{updateTables} carries two copies of the list of tables. One of them is passed to \texttt{eval} because the right-hand side of the assignment might contain other variables. The process of looking up the left hand-side variable pops symbol tables from the head of list (see next rule) but the original list of tables is necessary when assigning the values of variables located in inner scopes. For instance, consider the following program in C-{}-:

\begin{lstlisting}[caption = C-{}- sample program, label = code:ch_mcnv_languages_c--_sample_program]
int x;
...
if (x > 0) then
	int y;
	y = 4
	x = y;
else
  x = x - 1;	
\end{lstlisting}

\noindent
assume that before the if-then-else $x > 0$. The program will enter the \texttt{then} block and declare \texttt{y}. In the current state we have to symbol tables, one for the scope of the \texttt{if-then-else} and one for the outer scope. When assigning \texttt{y} to \texttt{x} the symbol table tries to look up \texttt{x} in the table of the current scope and fails. This will pop the head of the list of tables (which is the table of \texttt{if-then-else}) and recursively look in the tail. During the second attempt \texttt{x} is retrieved but now we do not have the symbol table where \texttt{y} is defined anymore to evaluate the right hand-side. We thus need the original list of tables to be able to retrieve \texttt{y}. In general, if we call \texttt{$d_l$} the depth of scoping of the left hand-side and $d_r$ the depth of scoping of the right hand-side, the process pops the table of the right hand-side whenever $d_l > d_r$ and this is when we need the original list of tables to retrieve the value of the right hand-side.

If the variable is not contained in the head of the list, i.e. it has not been declared in the current scope, we have the following rule:

\begin{lstlisting}
symbols contains id -> No
updateTables vars tables id expr -> tables'
----------------------------------------------
updateTables vars (symbols :: tables) id expr -> symbols :: tables'
\end{lstlisting}

The first premise tries to lookup the variable in the symbol table of the current scope and does not find it. Thus we recursively call \texttt{updateTables} with the tail of the list. The recursive call will eventually find the variable in one of the symbol tables associated with outer scopes. This process will produce an updated list of tables that is returned as a new tail for the current list.

At this point, the rule for the evaluation of the variable assignment simply class the \texttt{updateTables} function in its premise:

\begin{lstlisting}
updateTables tables tables id expr -> res
----------------------------------------
eval tables (id = expr) -> res
\end{lstlisting}

\subsubsection{Conditionals}

Evaluating \texttt{if-then-else} requires two rules, depending on the result of the evaluation of its condition. The following rule implements the semantics when the condition is false:

\begin{lstlisting}
evalExpr tables condition -> $b true
emptyDictionary -> table
eval (table :: tables) thenBlock -> table' :: tables''
-------------------------------------------------------------------
eval tables (if condition then thenBlock else elseBlock) -> tables''
\end{lstlisting}

The first premise evaluates the condition of the control structures and succeeds if the result is a meta-data structure containing the boolean value \texttt{true}. The second premise uses an utility function to initialize an empty symbol table. This is required to define a new table for the scope of the conditional. The third premise evaluates the statements contained in the then block after pushing the symbol table for the current scope in the list of symbol tables. This process will eventually produce a new list of symbol tables. The result returns only the tail of this list, since when we exit the scope of the conditional we must pop its symbol tables.

For instance, consider again the program in Listing \ref{code:ch_mcnv_languages_c--_sample_program} and again assume that $x > 0$. After executing the then block, the state of the program is made of the symbol tables shown in Table \ref{tab:ch_mcnv_languages_tables_ifthenelse}

\begin{table}
	\centering
	\begin{tabular}{|l|l|}
		\hline
		\textbf{Variable} & \textbf{Value} \\
		\hline
		x & undefined\\
		\hline
	\end{tabular}\\
	\vspace{0.2cm}
	\begin{tabular}{|l|l|}
		\hline
		\textbf{Variable} & \textbf{Value} \\
		\hline
		y & 4\\
		\hline
	\end{tabular}
	\begin{tabular}{|l|l|}
		\hline
		\textbf{Variable} & \textbf{Value} \\
		\hline
		x & 4\\
		\hline
	\end{tabular}
	\caption{Symbol table at the beginning and after the execution of the program in Listing \ref{code:ch_mcnv_languages_c--_sample_program} with $x > 0$}
	\label{tab:ch_mcnv_languages_tables_ifthenelse}
\end{table}

\noindent
After exiting the then block, variable \texttt{y} exits the scope, thus we have to pop the symbol table for the current scope. However, the symbol table of the outer scope has been changed because \texttt{x} got the value of \texttt{y}. Thus the evaluation returns the list containing this updated table. In general, the process should consider that an arbitrary amount of symbol tables for each outer scope have been changed, thus we return this updated list.

The rule that evaluates conditionals when the condition is false is analogous, except this time we evaluate the else block:

\begin{lstlisting}
evalExpr tables condition -> $b false
emptyDictionary -> table
eval (table :: tables) elseBlock -> table' :: tables''
--------------------------------------------------------------------
eval tables (if condition then thenBlock else elseBlock) -> tables''
\end{lstlisting}

\subsubsection{While Loops}
Evaluating the while loops require to check its condition first. When the condition is false we simply skip the loop without changing the state. The rule to implement this behaviour is thus straightforward:

\begin{lstlisting}
evalExpr tables condition -> $b false
--------------------------------------------
eval tables (while condition do block) -> tables
\end{lstlisting}

\noindent
The semantics when the condition is true is more complex:

\begin{lstlisting}
evalExpr tables condition -> $b true
emptyDictionary -> table
eval (table :: tables) block -> table' :: tables''
eval tables'' (while condition do block) -> res
---------------------------------------------------
eval tables (while condition do block) -> res
\end{lstlisting}

\noindent
The first premise succeeds when the evaluation of the condition returns a true boolean value in C-{}-. Analogously to what we did for conditionals, we initialize an empty symbol table for the current scope and we push it into the list of symbol tables. We then evaluate the body of the loop. This process will, in general, produce an updated list of symbol tables. Again we pop the symbol table for the current scope because we are exiting the loop. We then evaluate again the whole loop to test its condition again.

\subsubsection{For Loops}
For loops follow the C convention and are made of four parts: (\textit{i}) an initialization, (\textit{ii}) a condition (\textit{ii}), (\textit{iii}) a step, and (\textit{iv}) a block of code. The initialization is evaluated once before entering the loop, the condition is tested before each iteration of the loop, and the step is evaluated at the end of each iteration. In order to implement this behaviour we make use of an additional support function called \texttt{loopFor}:

\begin{lstlisting}
Func "loopFor" -> TableList -> Expr -> Stmt -> Stmt : TableList
\end{lstlisting}

\noindent
The evaluation of the \texttt{for-loop} evaluates the initialization in its premise. It then calls \texttt{loopFor} after the initialization has been evaluated. Again the initialization might define additional variables that enter the scope of the loop, so the updated table of the current scope is pushed into the list of symbol tables. Note that possible variables defined in the initialization part of the loop might be used after the loop itself, according to the semantics of C, so we have to insert them into the symbol table of the current scope and not the one of the loop itself.

\begin{lstlisting}
eval tables init => tables'
loopFor tables' condition step block => res
-------------------------------------------------------
eval tables (for init condition step do block) => res
\end{lstlisting}

The rules for \texttt{loopFor} are two, since we must consider the case when the condition is false and the one where it is true. When the condition is false the loop is completely skipped, thus we simply return the current state without any changes, in the same fashion of the \texttt{while-loop}:

\begin{lstlisting}
evalExpr tables condition -> $b false
------------------------------------------------------------------
loopFor tables condition step block -> tables
\end{lstlisting}

\noindent
When the condition is true, we create as usual a symbol table for the scope of the loop and we push it into the list of symbol tables. The third premise evaluates the block of the loop returning an updated list of tables. As usual we pop the table of the scope of the loop and we evaluate the step. This again might change the list of symbol tables. We then run again the loop with the updated list of tables.

\begin{lstlisting}
evalExpr tables condition -> $b true
emptyDictionary -> table
eval (table :: tables) block -> table' :: tables''
eval tables'' step -> tables3
loopFor tables3 condition step expr -> res 
-------------------------------------------------
loopFor tables condition step block -> res
\end{lstlisting}

\subsection{Type Checker}
Type checking can be performed by using a representation of the type system of C-{}- in terms of rules, in the same fashion of the semantics. In this section we explain the details of how each language construct is type-checked according to its type rule. We begin by defining an alternative version of the symbol table defined in Section \ref{subsec:ch_mcnv_languages_expression_semantics} that contains a mapping between variable names and types:

\begin{lstlisting}
Data "$m" << ImmutableDictionary<Id, Type> >> : TypeTable 
\end{lstlisting}

\noindent
and a constructor for the meta-data representing a sequence of type tables.

\begin{lstlisting}
Data TypeTable -> "::" -> TypeTableList : TypeTableList
Data "[]" : TypeTable
\end{lstlisting}

\noindent
We now start by defining the meta-data structures for the types in C-{}-:

\begin{lstlisting}
Data "t_int" : Type
Data "t_double" : Type
Data "t_string" : Type
Data "t_bool" : Type
Data "t_unit" : Type
\end{lstlisting}

\noindent
We also defined a special meta-data representing a type error to correctly report errors if the program contains invalid types:

\begin{lstlisting}
Data "error" -> <<string>> : Type
\end{lstlisting}

\subsubsection{Typing expressions}

We now proceed to define the type rules for expressions. We initially need to define a function to use in the conclusion of a type rule that is able to evaluate type of an expression:

\begin{lstlisting}
Func "typeExpr" -> TypeTableList -> Expr : Type
\end{lstlisting}

The axioms of expression typing are those that return the type of a literal. In this case the rule immediately returns the type associated to the specific literal.

\begin{lstlisting}
-----------------------------
typeExpr tables ($i v) -> t_int

-----------------------------
typeExpr tables ($d v) -> t_double

-----------------------------
typeExpr tables ($s v) -> t_string

-----------------------------
typeExpr tables ($b v) -> t_bool
\end{lstlisting}

\noindent
Type checking variables require to perform a lookup for the variable name in the list of type tables that we carry along during the typing process. The variable could be in the table associated with the current scope or in the table of an outer scope. Therefore, we start by first looking in the table of the current scope, and if we do not find the variable we recursively look it up in the subsequent table. If we traverse the whole list of tables without finding the variable, then it means that the program contains an undefined variable and an appropriate error notifying the problem should be returned.

\begin{lstlisting}
-------------------------------------
typeExpr [] ($ name) -> error <<"Undefined variable:" + name>>

types contains ($ name) -> Yes
types lookup ($ name) -> varType
------------------------------------------------
typeExpr (types :: tables) ($ name) -> varType

types contains ($ name) -> No
typeExpr tables ($ name) -> error msg
------------------------------------------
typeExpr (types :: tables) ($ name) -> error msg 

types contains ($ name) -> No
typeExpr tables ($ name) -> varType
------------------------------------------
typeExpr (types :: tables) ($ name) -> varType 
\end{lstlisting}

Note that we had to include a rule in whose premise we check whether the recursive lookup returned an error. If this is the case the entire rule returns the error message rather than the type of the variable.

Type-checking expression operators require to perform the following steps:

\begin{enumerate}[noitemsep]
	\item Type-check the left and right argument.
	\item Check that the types obtained at the previous step are compatible with the operator definition.
	\item Return the type of the operator.
\end{enumerate}

The process fails when the type-checking of one of the two expressions fails or when the types are incompatible with the operator definition.

\noindent
For brevity we only present the case of the sum, the rules for the other operators are analogous:

\begin{lstlisting}
typeExpr tables expr1 -> error msg
----------------------------------
typeExpr tables expr1 + expr2 -> error msg

typeExpr tables expr2 -> error msg
----------------------------------
typeExpr tables expr1 + expr2 -> error msg

typeExpr tables expr1 -> t_int
typeExpr tables expr2 -> t_int
--------------------------------------
typeExpr tables expr1 + expr2 -> t_int

typeExpr tables expr1 -> t_double
typeExpr tables expr2 -> t_double
--------------------------------------
typeExpr tables expr1 + expr2 -> t_double

typeExpr tables expr1 -> t_string
typeExpr tables expr2 -> t_string
--------------------------------------
typeExpr tables expr1 + expr2 -> t_string

-----------------------------------
typeExpr tables expr -> 
  error << "Incompatible types given to operator +" >>
\end{lstlisting}

\noindent
Note that the last rule is executed only if all the previous failed, so when the recursive check did not fail or when the returned types were incompatible with the sum operator.

\subsubsection{Typing a sequence of statements}
Typing a sequence of statements requires to type check the first statement in the sequence and then recursively type check the remaining statements in the sequence. We also need a different type-checking function that is able to process statements and meta-data structure for its result.

\begin{lstlisting}
Data TypeTableList -> "," -> Type : TypeResult
Func "typeStmt" -> TypeTableList -> Stmt : TypeResult
\end{lstlisting}

\noindent
This function in general returns an updated list of type tables and a type, since variable declarations might change them. We use \texttt{t\textunderscore unit} for the type of statements, which is a place holder for language constructs that just change the state of the program.

The base case of the recursion is when the sequence contains only \texttt{nop}, which returns immediately the same type tables.

\begin{lstlisting}
------------------------------------
typeStmt tables nop -> tables,t_unit
\end{lstlisting}

\noindent
Type-checking a sequence of statements initially checks the first statement. This might return an updated list of tables. Then it recursively checks the other statements with the result of the first step and returns the final type tables. If either of the process returns an error we just propagate the error.

\begin{lstlisting}
typeStmt tables a -> tables',error msg
------------------------------------------
typeStmt tables (a;b) -> tables',error msg

typeStmt tables a -> tables',t_unit
typeStmt tables' b -> finalTables,error msg
----------------------------------------------
typeStmt tables (a;b) -> finalTables,error msg


typeStmt tables a -> tables'
typeStmt tables b -> finalTables,t_unit
--------------------------------
typeStmt tables (a;b) -> finalTables,t_unit
\end{lstlisting}

\noindent
Note that we are sure that the final rule succeeds because the type-checking of a statement always returns \texttt{unit} if the type-checking succeeds, according to the type rules of the language; this is further explained in the sections below.

\subsubsection{Typing variable declarations and assignments}
When we encounter a variable declaration we have to add the variable name and its type to the table of the current scope, unless the variable is already defined in the current scope, in which case we return an error. We must also prevent the declaration of variable with type \texttt{unit}, because that is a reserved type for statements. This is implemented with the following rules:

\begin{lstlisting}

-----------------------------------
typeStmt types (variable t_unit id) -> [],error << "The type unit cannot be used as a variable type" >>

types contains id -> Yes
------------------------------------
typeStmt (types :: tables) (variable t ($ name)) -> [],error << "Variable " + name " already defined" >>

types add id t -> types'
------------------------------------------
typeStmt (types :: tables) (variable t id) -> types' :: tables
\end{lstlisting}

In the case of a variable assignment, the type checker must first look up in the type tables for the variable type. If the variable cannot be found then an error is returned because the program is trying to use an undefined variable. Otherwise we check the type of the right expression, and if it is compatible with the type of the variable then the declaration succeeds. Note that the process of checking the right side of the assignment might fail and, in this case, we have to propagate the error.

\begin{lstlisting}
-------------------------------
typeStmt [] (($ name) = expr) -> [],error << "Variable " + name + " undefined" >>

typeExpr tables expr -> error msg
-------------------------------------------
typeStmt tables (id = expr) -> [],error msg

types contains id -> No
typeStmt tables (id = expr) -> res
-------------------------------------
typeStmt (types :: tables) (id = expr) -> res

types getValue id -> tvar
typeExpr (types :: tables) expr -> te
tvar <> te
-------------------------------------
typeStmt (types :: tables) (($ name) = expr) -> [],error << "Trying to assign an incompatible value to " + name >>

types getValue id -> tvar
---------------------------------------------
typeStmt (types :: tables) (id = expr) -> (types :: tables),tvar
\end{lstlisting}

\subsubsection{Typing conditionals}
Type-checking \texttt{if-then-else} requires to first check the type of the expression provided as condition. This process might fail and in this case we propagate the returned error. If the type checking of the expression succeeds but the returned type is not boolean, we have to return an error as well. Otherwise we can proceed to type-check the body of \texttt{then} and \texttt{else}. This process can again fail and we must again propagate a possible error. If no errors are returned after this step we return a possible updated list of type tables and the type \texttt{unit}.

\begin{lstlisting}
typeExpr tables condition -> error msg
--------------------------------------------------------------------
typeStmt tables (if condition then thenBlock else elseBlock) -> 
  [],error msg

emptyDictionary -> table
typeStmt (table :: tables) thenBlock -> t,error msg
------------------------------------------------------------
typeStmt tables (if condition then thenBlock else elseBlock) -> 
  [],error msg

emptyDictionary -> table
typeStmt (table :: tables) elseBlock -> t,error msg
------------------------------------------------------------
typeStmt tables (if condition then thenBlock else elseBlock) -> 
  [],error msg

typeExpr tables condition -> tc
tc <> t_bool
------------------------------------------------------------
typeStmt tables (if condition then thenBlock else elseBlock) ->
  [],error << "The condition of an if-then-else must be boolean" >>

--------------------------------------------------------
typeStmt tables (if condition then thenBlock else elseBlock) ->
  t_unit,tables
\end{lstlisting}

Note that the last rule does not type check again the code blocks of \texttt{if-then-else} because the only statement that can change a type table is a variable declaration, but after we exit the scope of the block the local declarations are removed. At this point we are sure that the type-checking of the blocks has succeeded, otherwise we would have triggered one of the rules above returning an error, thus we can immediately return the result.

\subsubsection{Typing while-loops}
Type-Checking a while loop is similar to the procedure of evaluating a conditional statement. We must first check that the provided condition is boolean. This might fail either because type-checking the condition itself returns an error or because the type of the expression is not boolean. After this step we have to check the body of the loop, which might fail as well. If no error is reported then we can safely return the correct result.

\begin{lstlisting}
evalExpr tables condition -> error msg
------------------------------------------------
typeStmt tables (while condition do block) -> [],error msg

evalExpr tables condition -> tc
tc <> t_bool
------------------------------------------------
typeStmt tables (while condition do block) ->
  [],error << "The condition of a while loop must be boolean" >>
  
evalExpr tables condition -> tc
emptyDictionary -> table
typeStmt (table :: tables) condition -> t,error msg
-----------------------------------------------------
typeStmt tables (while condition do block) -> [],error msg


-------------------------------------------
typeStmt tables (while condition do block) -> tables,t_unit
\end{lstlisting}

\noindent
Again note that the last rules can immediately return the result because we know that, at this point, we cannot have any error and we do not need to keep the type table of the scope of the code block.

\subsubsection{Typing for loops}
Type-checking a for loops requires to first type-check the initialization. This might fail and we must propagate the error. We must then type-check the condition and the step. This process can fail either because of an error in the condition or in the statement in the step, or because the condition is not boolean.  If this succeeds we then proceed to type-check the body of the loop.
\begin{lstlisting}

typeStmt tables init -> t,error msg
-----------------------------------------------
typeStmt tables (for init condition step do block) -> [],error msg

typeExpr tables condition -> error msg
---------------------------------------------------
typeStmt tables (for init condition step do block) -> [],error msg

typeExpr tables condition -> tc
tc <> t_bool
---------------------------------------------------
typeStmt tables (for init condition step do block) -> 
  [],error << "The condition of a for loop must be boolean" >>
  
typeExpr tables condition -> tc
tc <> t_bool
---------------------------------------------------
typeStmt tables (for init condition step do block) -> 
  [],error << "The condition of a for loop must be boolean" >>

emptyDictionary -> table
typeStmt (table :: tables) step -> t,error msg  
-----------------------------------------------------
typeStmt tables (for init condition step do block) -> [],error msg

emptyDictionary -> table
typeStmt (table :: tables) block -> t,error msg  
-----------------------------------------------------
typeStmt tables (for init condition step do block) -> [],error msg

-----------------------------------------------------
typeStmt tables (for init condition step do block) -> tables,t_unit
\end{lstlisting}

\section{The Casanova language}
In the previous section we have shown how to implement a small imperative language using Metacasanova. In this section we show the implementation in Metacasanova of Casanova, a Domain-Specific Language for game development. We first give an informal explanation about how the language works and then we show an implementation of the language semantics.

%expand with the full implementation of Casanova 2
\subsection{The structure of a Casanova program}
In this section we give an informal overview of a program in Casanova, leaving aside for brevity many of the details about the language itself, which can be found in \cite{abbadi2014resource, abbadi2015casanova, abbadi2015high, abbadithesis2017}.

A program in Casanova is structured as a tree of \textit{entities} that represent the dynamic elements of a game, where the root entity is special and called \textit{world}. An entity is similar to a class in an object-oriented programming language, containing fields and a constructor. However, the difference lies in how the language implements the dynamic behaviour of an entity: each entity defines a set of \textit{rules} that describe the temporal evolution of an entity instance. A rule operates on a set of fields of an entity called \textit{domain}, and it is allowed changed only the values of the fields in its domain. A rule can write in a field of the domain only through a dedicated statement called \texttt{yield}. On the other hand, reading fields outside the domain is always possible. Each rule in an entity is run periodically up to a maximum refresh rate, which is usually set to 60Hz. One update cycle is called \textit{frame}. Each rule is automatically passed two special identifiers, \texttt{this} and \texttt{dt}, where the former is a reference to the current instance of the entity and the latter the time elapsed between the last and the current frame.

Rules have mechanics similar to threads: they can be paused for a specific amount of time or until a certain condition is met. Furthermore, every time the rule executes a \texttt{yield} statement (thus changing the values of the fields in the entity) it is suspended until the next frame. Casanova also features interruptible control structures, such as \texttt{if-then-else}, \texttt{while-do}, and list comprehensions in a syntax similar to SQL or Linq (\texttt{from-where-select}).
 
The Casanova compiler generates the code to simulate the rule suspension and restart in the form of states machines. In the following section we show how to implement the same behaviour in the form of natural semantics in Metacasanova by using continuation-passing style.

\subsection{Casanova 2.5}
The memory in Casanova 2.5 is represented using three maps, where the key is the variable/field name, and the value is the value stored in the variable/field. The first dictionary represents the global memory (the fields of the \texttt{world} entity or \textit{Game State}), the second dictionary represents the current entity fields, and the third the variable bindings local to each rule.

The core of the entity update is the \texttt{tick} function. This function evaluates in order each rule in the entity by calling the \texttt{evalRule} function. This function executes the body of the rule and returns a result depending on the set of statements that has been evaluated. This result is used by \texttt{tick} to update the memory and rebuild the rule body to be evaluated at the next frame. The result of \texttt{tick} is a \texttt{State} containing the rules updated so far, and the updated entity and global fields. Since a rule must be restarted after the whole body has been evaluated, we need to store a list containing the original rules, which will be restored when evaluation returns \texttt{Done} (see below). At each step the function recursively calls itself by passing the remaining part of original rules (the rules which body was not altered by the evaluation of the statements) and modified rules (which body has been altered by the evaluation of the statements) to be evaluated. The function stops when all the rules have been evaluated, and this happens when both the original and the modified rule lists are empty.

Interruption is achieved by using \textit{Continuation passing style}: the execution of a sequence of statements is seen as a sequence of steps that returns the result of the execution and the remaining code to be executed. Every time a statement is executed we rebuild a new rule whose body contains the continuation which will be evaluated next. 

\begin{comment}
For example, consider the following rule:

\begin{lstlisting}
rule X,Y =
while X > 0 do
wait 1.0f
yield X - 1,Y + 1
\end{lstlisting}

The code is executed atomically until the \texttt{wait} statement (assuming that the \texttt{while} condition is true). At that point we rebuild a new rule containing the code to execute at the next iteration:

\begin{lstlisting}
rule X,Y =
wait (1.0f - dt)
yield X - 1, Y + 1
while X > 0 do
wait 1.0f
yield X - 1,Y + 1
\end{lstlisting}
Note that the \texttt{while} is placed at the end of the continuation because it must be re-evaluated after the first iteration is complete, and that we have decreased the waiting time by \texttt{dt} (the time elapsed between one frame and the previous one).
\end{comment}

The possible results returned by the \texttt{tick} function are the following: (\textit{i}) \texttt{Suspend} contains a \texttt{wait} statement with the updated timer, the continuation, and a data structure called \texttt{Context} which contains the updated local variables, the entity fields, and the global fields. The function rebuilds a rule which body is the sequence of statements contained by the \texttt{Suspend} data structure. (\textit{ii}) \texttt{Resume} is returned when the timer must resume after the last waited frame. In order not to skip a frame we must still re-evaluate the rule at the next frame and not immediately. In this case the argument of \texttt{Resume} is only the remaining statements to be executed. (\textit{iii}) \texttt{Yield} stops evaluation for one frame. We use the continuation to rebuild the rule body. Memory is updated by \texttt{evalRule}. (\textit{iv}) \texttt{Done} stops the evaluation for one frame and rebuilds the original rule body by taking it from the original rules list.

For brevity we write only the code for \texttt{Suspend}. A full implementation can be found at \cite{CASANOVA_SOURCE_CODE}. You can see a schematic representation of the tick function in Figure \ref{fig:tick}.

\begin{lstlisting}
evalRule (rule dom body k locals delta) fields globals => Suspend (s;cont) (Context newLocals newFields newGlobals)
r := rule dom s cont newLocals dt
tick originals rs newFields newGlobals dt => State updatedRules updatedFields updatedGlobals
st := State (r::updatedRules) updatedFields updatedGlobals
------------------------------------------------------
tick (original::originals) ((rule dom body k locals delta)::rs) fields globals dt => st
\end{lstlisting}


\begin{figure}
	\centering
	\includegraphics[scale = 0.25]{Figures/tick2}
	\caption{Casanova 2.5 rule evaluation}
	\label{fig:tick}
\end{figure}

The function \texttt{evalRule} calls \texttt{evalStatement} to evaluate the first statement in the body of the rule passed as argument. The result of the evaluation of the statement is processed in the following way: (\textit{i}) if the result is \texttt{Done}, \texttt{Suspend} or \texttt{Resume} then it is just returned to the caller function. We omit the code for this case, since it is trivial; (\textit{ii}) if the result is \texttt{Atomic} it means that the evaluated statement was uninterruptible and the remaining statements of the rule must be re-evaluated immediately; (\textit{iii}) if the result is \texttt{Yield} then the fields in the domain are updated recursively in order and then the updated memory is encapsulated in the \texttt{Yield} data structure and passed to the caller function.

\vspace{0.1cm}
\begin{lstlisting}
evalStatement b k ctxt dt => Atomic z c    
evalRule (rule dom z nop c dt) => res
-------------------------------
evalRule (rule dom b k ctxt dt)  => res
\end{lstlisting}

\begin{lstlisting}
evalStatement b k (Context locals fields globals) dt => Yield ks values context
updateFields dom values context  => updatedContext
--------------------------------------------------------
evalRule (rule dom b k locals dt) fields globals => Yield ks values updatedContex
\end{lstlisting}

Note that, in case of a rule containing only atomic statements, we will eventually return \texttt{Done} after having recursively called \texttt{evalStatement} for all the statements, and the rule will be paused for one frame.

\begin{comment}
\begin{figure}
\centering
\includegraphics[scale=0.15]{Pictures/statement_evaluation}
\caption{Statement evaluation}
\label{fig:statement_evaluation}
\end{figure}
\end{comment}

\noindent
The \texttt{evalStatement} function is used both to evaluate a single statement and a sequence of statements. When evaluating a sequence of statements, the first one is extracted. A continuation is built with the following statement and passed to a recursive call to \texttt{evalStatement} which evaluates the extracted statement. If the existing continuation is non-empty, then it is added before the current continuation. If both the continuation and the body are empty (situation represented by the \texttt{nop} operator) then it means the rule evaluation has been completed and we return \texttt{Done}.

\begin{lstlisting}
a != nop
---------------------                           ----------------------- 
addStmt a b => a;b                              addStmt nop nop => nop   

addStmt b k => cont
evalStatement a cont ctxt dt => res
-------------------------------                 -----------------------------------       
evalStatement (a;b) k ctxt dt => res            evalStatement nop nop ctxt dt => Done ctxt


\end{lstlisting}

\noindent
We will now present, for brevity, only the evaluation of the \texttt{wait} and \texttt{yield} statements. Both the evaluation of the control structures and the variable bindings always return \texttt{Atomic} because they do not, by definition, pause the execution of the rule.

The \texttt{wait} statement has two different evaluations, based on the rules defined in Section \ref{sec:problem_statement}: (\textit{i}) the timer has elapsed: in this case we return \texttt{Resume} which contains the code to execute after the \texttt{wait} statement, or (\textit{ii}) the timer has not elapsed: in this case we return \texttt{Suspend} which contains the \texttt{wait} statement with the updated timer followed by the continuation.


\begin{lstlisting}
<<t <= dt>> == false
----------------------------------
evalStatement (wait t) k ctxt dt => Suspend wait <<t - dt>>;k ctxt

<<t <= dt>> == true
----------------------------------
evalStatement (wait t) k ctxt dt => Resume k ctxt
\end{lstlisting}

\noindent
The \texttt{yield} statement takes as argument a list of expressions whose values are used to update the corresponding fields in the rule domain. The evaluation rule recursively evaluates the expressions and stores them into a list passed as argument of the \texttt{Yield} result. Those arguments are used later by \texttt{evalRule} to update the corresponding fields.

\begin{lstlisting}
eval expr ctxt => v
evalYield exprs ctxt => vs
-------------------------------------------          ----------------------------
evalYield (expr :: exprs) ctxt => v :: vs            evalYield nil ctxt => nil
\end{lstlisting}

In this section we provide an implementation of a patrol script for an entity in a game. The sample is made up of an entity, representing a guard, and a couple of checkpoints. The guard continuously moves between the two checkpoints. We choose this sample because this is a typical behaviour implemented in several games, where the user is able to set up a patrol route for a unit. We show the comparison between the sample implemented in Casanova 2.5 and an equivalent implementation in Python with respect to the running time. We then show a comparison between the hard-coded compiler of Casanova 2.0 and the implementation of Casanova 2.5 in Metacasanova with respect to the code length. 

\begin{comment}
We want to underline that the main goal of this work is \textbf{to ease the process of building a compiler for a DSL for games, thus our main objective is decreasing the code length and complexity necessary to implement a hard-coded compiler for the language}. At the same time we show that the compiled program in Casanova 2.5 \textbf{has performance similar to that of a language used in game development, and thus Casanova 2.5 is usable in a real scenario}.
\end{comment}

\subsection{Chosen languages}
We compared the running time of the sample in metacompiled Casanova with an equivalent implementation in Python. This language was chosen based on its use in game development: Python has been used extensively in several games such as Civlization IV \cite{CIV4} or World in Conflict \cite{WIC} because of the native support for coroutines. We deliberately ignore C++ and C\# implementations, although they are widely used in the industry, because we knew in advance \cite{CASANOVA2_PAPER} that the current version of the code generated by the meta-compiler would not match the high performance of these languages: the main goal of this work is to reduce the effort of writing a compiler for a DSL for games while having acceptable performance.


\subsection{Performance}
The performance results are shown in Table \ref{tab:evaluation}. We see that the generated code has performance on the same order as Python. This is mainly due to the fact that the memory, in the metacompiled implementation of Casanova, is managed through a map, and because of the virtuality of the implemented operators. Each time Casanova accesses a field in an entity this must be looked up into the map. To this we add the complexity of dynamic lookups when we must deal with polymorphic results into the rules. 

From Table \ref{tab:compiler_comparison} we see that the implementation of Casanova 2.0 language in Metacasanova is almost 5 times shorter in terms of lines of code than the previous Casanova implementation in F\#. We believe it is worthy noticing that structures with complex behaviours, such as \textit{wait} or \textit{when}, require hundreds of lines of codes with a standard approach (the code lines to define the behaviour of the structure plus the support code to correctly generate the state machine), while in the meta-compiler we just need tens of lines of codes to implement the same behaviour. Moreover we want to point out that the previous Casanova compiler was written in a functional programming language: these languages tend to be more synthetic than imperative languages, so the difference with the same compiler implemented in languages such as C/C++ might be even greater.

The readability with respect to the hard-coded compiler code is also improved: we managed to implement the behaviour of synchronization and timing primitives almost imitating one to one the formal semantics of the language definition (see the semantics rules in Section \ref{sec:formal_description} and their implementation in Section \ref{sec:casanova3}). In the hard-coded compiler implementation for Casanova 2.0 the semantics are lost in the code for generating finite state machines.

\begin{table}[!h]
	\centering
	\tiny	
	\begin{tabular}{|c|c|c|}
		\hline
		\multicolumn{3}{|c|}{\textbf{Casanova 2.5}} \\
		\hline
		Entity \# & Average update time (ms) & Frame rate \\
		\hline
		100 & 0.00349 & 286.53 \\
		\hline
		250 & 0.00911 & 109.77 \\
		\hline
		500 & 0.01716 & 58.275 \\
		\hline
		750 & 0.02597 & 38.506 \\
		\hline
		1000 & 0.03527 & 28.353 \\
		\hline
		\multicolumn{3}{|c|}{\textbf{Python}} \\
		\hline
		Entity \# & Average update time (ms) & Frame rate \\
		\hline
		100 & 0.00132 & 756.37 \\
		\hline
		250 & 0.00342 & 292.05 \\
		\hline
		500 & 0.00678 & 147.54 \\
		\hline
		750 & 0.01087 & 91.988 \\
		\hline
		1000 & 0.01408 & 71.002 \\
		\hline
	\end{tabular}
	\caption{Patrol sample evaluation}
	\label{tab:evaluation}
\end{table}

\begin{table}[!h]
	\centering
	\tiny
	\begin{tabular}{|c|c|}
		\hline
		\multicolumn{2}{|c|}{\textbf{Casanova 2.5 with Metacasanova}} \\
		\hline
		Module & Code lines \\
		\hline
		Data structures and function definitions & 40 \\
		\hline
		Query Evaluation & 16 \\
		\hline
		While loop & 4 \\
		\hline
		For loop & 5 \\
		\hline
		If-then-else & 4 \\
		\hline
		When & 4 \\
		\hline
		Wait & 6 \\
		\hline
		Yield & 10 \\
		\hline
		Additional rules for Casanova program evaluation & 40 \\
		\hline
		Additional rules for basic expression evaluation & 201 \\
		\hline
		\multicolumn{2}{|l|}{\textbf{Total: } 300} \\
		\hline
		\multicolumn{2}{|c|}{\textbf{Casanova 2.0 compiler}} \\
		\hline
		Module & Code lines \\
		\hline
		While loop & 10 \\
		\hline
		For-loop and query evaluation & 44 \\
		\hline
		If-Then-Else & 15 \\
		\hline
		When & 11 \\
		\hline
		Wait & 24 \\
		\hline
		Yield & 29 \\
		\hline
		Additional structures for rule evaluation & 63 \\
		\hline
		Structures for state machine generations & 754 \\
		\hline
		Code generation & 530 \\
		\hline
		\multicolumn{2}{|l|}{\textbf{Total: } 1480} \\
		\hline			
	\end{tabular}	
	\caption{meta-compiler vs standard compiler}
	\label{tab:compiler_comparison}
\end{table}

\subsection{Discussion}
\label{subsec:code_generation_discussion}
Metacasanova has been evaluated in \cite{DiGiacomo2017} by re-building the DSL for game development Casanova \cite{abbadi2015casanova, abbadithesis2017}. Even though the size of the code required to implement the language has been drastically reduced (almost 1/5 shorter), performance dropped dramatically. We identified a main problem causing the performance decay that, if solved, will improve the performance of the generated code.

In order to encode a symbol table in the meta-compiler in the current implementation (used for example to store the variables defined in the local scope of a control structure or to model a class/record data structure), we are left with two options: (\textit{i}) define a custom data structure made of a list of pairs, containing the field/variable name as a string and its value, in the following way

\begin{lstlisting}
Data "table" -> List[Tuple[string, Value]] : SymbolTable
\end{lstlisting}

\noindent
or (\textit{ii}) use a dictionary data structure coming from .NET, such as \texttt{ImmutableDictionary}, which was the implementation choice for Casanova. In both cases, the behaviour of the language implemented in Metacasanova will be that of a dynamic language, because whenever the value of a variable or class field must be read, the evaluation rule must look up the symbol table at run time to retrieve the value, whose complexity will be $O(n)$ with the list implementation and $O(\log n)$ with the dictionary implementation. This issue is caused by the fact that, in the current state of Metacasanova, the meta-type system is unaware of the type system of the language that is being implemented in the meta-compiler. This is not a problem limited to Metacasanova but to all meta-compilers having a meta-type system that does not allow embedding of the host language type system. In the next section we propose an extension to Metacasanova to overcome this problem by embedding the type system of the implemented language in the meta-type system of Metacasanova and inlining the code to access the appropriate variable at compile time.

\chapter{Metacasanova Optimization}
\label{ch:functors}
\epigraph{First you learn the value of abstraction, then you learn the cost of abstraction, then you're ready to engineer.}{Kent Beck}
In Chapter \ref{ch:metacasanova} and \ref{ch:languages} we have presented the Metacasanova metacompiler and its meta-language and shown how to implement with it a small imperative language, C-{}-, and a DSL for game development, Casanova. The performance analysis showed that, although the development effort for the language compilers was greatly reduced by using Metacasanova, this has come to the cost of performance. The performance decay is due to the fact that the meta-type system of Metacasanova is unaware of the type system of C-{}- or Casanova. This requires all the type checking and access to data structures being performed at runtime, thus making a statically-typed language exhibit the behaviour and performance of dynamically typed languages. In this Chapter we propose a language extension \cite{DiGiacomo2017SLE} for Metacasanova that is thought to overcome the problem of performance decay and dynamic checks. In this context we use the term \textit{embedded language} to refer to a language that is being implemented in Metacasanova and \textit{embedded program} for a program implemented in an embedded language.

\section{Language extension idea}
\label{sec:ch_functors_idea}
The experimental results from Chapter \ref{ch:languages} showed that the performance of Metacasanova is strongly affected by the dynamic type checks and symbol table access at runtime. This is due to the fact that Metacasanova generates the code necessary to evaluate the semantics of accessing the value of a variable in the symbol table that mimics the behaviour of rules in natural semantics, but such evaluation is performed at runtime. However the runtime evaluation is only due to the limitations of the language presented so far, which is not able to build a symbol table while while compiling the meta-program, since

\begin{enumerate}
	\item The symbol table of a statically-typed language does not grow at runtime because it is built during the compilation.
	\item The position of an entry for a variable in the symbol table does not change during the program execution, thus every time we perform an access to the same variable, we access the very same element in the symbol table.
\end{enumerate}

\noindent
Analogously type checking in a statically-typed language is performed at compilation time rather than at runtime, which is a behaviour typical of dynamic languages such as Python. Metacasanova is forced to do runtime type checking because, at compilation time, the metacompiler only checks for the meta-types, i.e. the types of the language abstractions defined in the meta-language, but not for the program structures of the embedded program itself. This would require to be able to embed the type system of the embedded language into the meta-type system of Metacasanova. In this way the type checker of Metacasanova would be able to check at the same time the types of both the meta-program and of the embedded program. 

To better clarify what stated so far we show in the following section an example of what happens when accessing the field of a Casanova entity with the implementation given in Chapter \ref{ch:languages}. We then proceed to show the idea of a possible solution to overcome the performance decay.

\subsection{Field access in Casanova}
\label{subsec:ch_functors_casanova_example}
As we showed in Section \ref{subsec:ch_mcnv_languages_casanova_semantics}, an entity in Casanova embedded in Metacasanova is represented through a map where the key is the field name and the value is the value currently stored in the field. This representation is very similar to that of records or classes. Let us consider the following entity representing a physical body consisting of a \texttt{Position} and a \texttt{Velocity} in a 2D space:

\begin{lstlisting}
type PhysicalBody = {
  Position        : Vector2
  Velocity        : Vector2
}
\end{lstlisting}

\noindent
and the following rules for \texttt{PhysicalBody}

\begin{lstlisting}
rule Position = Position + Velocity * dt

rule Position =
  if Position.X > 500f then
    yield new Vector2(500f,Position.Y)
  elif Position.X < 0f then
    yield new Vector2(0f,Position.Y)
  elif Position.Y < 0f then
    yield new Vector2(Position.X,0f)
  elif Position.Y > 500f then
    yield new Vector2(Position.X,500f)
\end{lstlisting}

The first rule simply updates the position using the Euler approximation of the differential equation for the velocity

\begin{equation*}
v(t) = \dfrac{ds(t)}{dt}
\end{equation*}

\noindent
while the second rule ensures that the physical body does not exit a specific area, which could represent the playable area in a 2D game.

Assuming that the physical body is in position $(10,10)$, it is represented in Metacasanova through a map as shown in Table \ref{tab:ch_functors_physical_body}.

\begin{table}
	\centering
	\begin{tabular}{|c|c|}
		\hline
		\textbf{Field} & \textbf{Value} \\
		\hline
		Position	& 10 \\
		\hline
		Velocity & 10 \\
		\hline
	\end{tabular}
	\caption{Meta-representation of the physical body}
	\label{tab:ch_functors_physical_body}
\end{table}

\noindent
The Metacasanova semantics rule that evaluates the first Casanova rule will evaluate the expression in its body by accessing respectively the field \texttt{Position} and \texttt{Velocity} to compute the expression value. It then stores the expression value in \texttt{Position} as shown in Table \ref{tab:ch_functors_physical_body_access1_1}.

\begin{table}
	\centering
	\begin{tabular}{c|c|c|}
		\cline{2-3}
		& \textbf{Field} & \textbf{Value} \\
		\cline{2-3}
		$\Rightarrow$ & \cellcolor{green}{Position}	& \cellcolor{green}{10,10} \\ 
		\cline{2-3}
	  & Velocity & 10,0 \\
		\cline{2-3}
	\end{tabular}
	
	\vspace{0.15cm}
	\begin{tabular}{c|c|c|}
		\cline{2-3}
		& \textbf{Field} & \textbf{Value} \\
		\cline{2-3}
	  & Position	& 10,10 \\ 
		\cline{2-3}
		$\Rightarrow$ & \cellcolor{green}{Velocity} & \cellcolor{green}{10,0} \\
		\cline{2-3}
	\end{tabular}
	
	\vspace{0.15cm}
	\begin{tabular}{c|c|c|}
		\cline{2-3}
		& \textbf{Field} & \textbf{Value} \\
		\cline{2-3}
		$\Rightarrow$ & \cellcolor{green}{Position}	& \cellcolor{green}{11,10} \\ 
		\cline{2-3}
		& Velocity & 10,0 \\
		\cline{2-3}
	\end{tabular}

	\caption{Memory access in the first rule of the Physical Body. We assume \texttt{dt = 0.1} and \texttt{Velocity = (10,0)}}
	\label{tab:ch_functors_physical_body_access1_1}
\end{table}

\begin{table}
	\centering
	\begin{tabular}{c|c|c|}
		\cline{2-3}
		& \textbf{Field} & \textbf{Value} \\
		\cline{2-3}
		$\Rightarrow$ & \cellcolor{green}{Position}	& \cellcolor{green}{\fbox{501},10} \\ 
		\cline{2-3}
		& Velocity & 10,10 \\
		\cline{2-3}
	\end{tabular}
	
	\vspace{0.15cm}
	\begin{tabular}{c|c|c|}
		\cline{2-3}
		& \textbf{Field} & \textbf{Value} \\
		\cline{2-3}
		$\Rightarrow$ & \cellcolor{green}{Position}	& \cellcolor{green}{501,\fbox{10}} \\ 
		\cline{2-3}
		& Velocity & 10,10 \\
		\cline{2-3}
	\end{tabular}
	
	\vspace{0.15cm}
	\begin{tabular}{c|c|c|}
		\cline{2-3}
		& \textbf{Field} & \textbf{Value} \\
		\cline{2-3}
		$\Rightarrow$ & \cellcolor{green}{Position}	& \cellcolor{green}{500,10} \\ 
		\cline{2-3}
		& Velocity & 10,10 \\
		\cline{2-3}
	\end{tabular}
	\caption{Memory access in the second rule of the Physical Body. We assume \texttt{Position.X = 501}}
\end{table}

\noindent
As for the second rule, assuming that \texttt{Position.Y > 500f}, the rule will access \texttt{Position} three times: (\textit{i}) to evaluate the expression in the conditional, (\textit{ii}) to read \texttt{Position.Y} when instantiating a new vector, and (\textit{iii}) to write the new vector in \texttt{Position}. This situation is shown in Table

It should now appear clear that every time we need to read or write \texttt{Position} we access the first element of the table, while for \texttt{Velocity} we always access the second. In the following snippet we provide an alternative version of the code for the Casanova rules above that shows what really happens in Casanova embedded in Metacasanova :

\begin{lstlisting}
rule Position = PhysicalBodyTable[0] + PhysicalBodyTable[1] * dt
  
rule Position =
	if PhysicalBodyTable[0].X > 500f then
		yield new Vector2(500f,PhysicalBodyTable[0].Y)
	elif PhysicalBodyTable[0].X < 0f then
		yield new Vector2(0f,PhysicalBodyTable[0].Y)
	elif PhysicalBodyTable[0].Y < 0f then
		yield new Vector2(PhysicalBodyTable[0].X,0f)
	elif PhysicalBodyTable[0].Y > 500f then
		yield new Vector2(PhysicalBodyTable[0].X,500f)
\end{lstlisting}

Let us now assume that the program provides an invalid value for the update of \texttt{Position}:

\begin{lstlisting}
rule Position = "(10,10)"
\end{lstlisting}

\noindent
what would happen in embedded Casanova is that the type checker evaluates the type of the expression in the rule body, obtaining \texttt{string}. This type is then compared with that of \texttt{Position}, which is \texttt{Vector2}, and at this point an error would be reported. Again, this would require at runtime to access the first element of a symbol table containing type information about the entity fields. Note that all these lookups are not array accesses but rather dictionary accesses.

\subsection{Inlining the entity fields}
\label{subsec:ch_functors_inlining}
From the example above we can notice that, when the program runs, the symbol table used to represent a Casanova entity does not change, nor its entries change position. This means that every time we read or write the same field we perform the same access in the table. In the implementation provided in Section \ref{subsec:ch_mcnv_languages_casanova_semantics} this access requires to evaluate a Metacasanova rule that is able to traverse the dictionary used for the entity symbol table and return the stored value. The traverse is performed every time, regardless of the fact that the field we are trying to access is indeed the same. Moreover, as it was also stated in Section \ref{ch:mcnv_languages_evaluation}, we are looking at the very optimistic scenario where we make use of external .NET dictionaries to actually model the entity. If one had to rely solely on language abstractions defined in Metacasanova the symbol table should be modelled as a list of pairs containing field names, represented as strings, and meta-data structures representing values in the embedded language, introducing even a greater overhead. The physical body modelled in such way would then look like

\begin{lstlisting}
[("Position",(10,10)),("Velocity",(10,0)]
\end{lstlisting}

Accessing \texttt{Position} would then be performed by a Metacasanova rule that looks for the correct field name and stops when the field in this tuple has been reached:

\begin{lstlisting}
name = fieldName
----------------------------------
getField name ((fieldName,value) :: t) -> value

name <> fieldName
getField name t -> v
----------------------------------
getField name ((fieldName,value) :: t) -> v
\end{lstlisting}

\noindent
However the traversal of the tuple would always be the same when looking for a specific field, namely for \texttt{Position} the first Metacasanova rule will always be executed, while for \texttt{Velocity} the first time the second rule will be executed, which in turn recursively evaluates the remaining part of the list. The recursive call will then trigger the first rule at the second step. That being said, since the table does not grow and the access patterns are always the same, we could represent an entity as a nested tuple of pairs, in the fashion of Church encoding \cite{pierce2002types, kleene1935theory}, and inline in the code \texttt{fst PhysicalBodyTable} for \texttt{Position} and \texttt{fst(snd PhysicalBodyTable)} for \texttt{Velocity} whenever we require to access the respective fields, without repeating the same traversal every time. In this way the entity would look like:

\begin{lstlisting}
PhysicalBodyTable = ("Position",(10,10)),(("Velocity",(10,0)),())
\end{lstlisting}

\noindent
and thus \texttt{fst PhysicalBodyTable} (access to \texttt{Position}) would return\\ \texttt{("Position",(10,10))} and \texttt{fst(snd PhysicalBodyTable)} (access to\\ \texttt{Velocity}) would return \texttt{("Velocity",(10,0))}.

In the following sections we present the language extension required to allow this form of inlining and we show their usage implementing the example above.

\section{Modules and Functors}
\label{sec:ch_functors_modules_functors}
In order to implement the idea about inlining symbol table access and embed the type system of a language inside Metacasanova type system we extend the language with \textit{functors} and \textit{modules}. Functors are a concept borrowed form category theory that here are used in a more narrow sense. Formally a category is defined as follows \cite{asperti1991categories, mitchell1965theory, pierce1991basic}:

\begin{definition}
	A category $\mathcal{C}$ is made of
	
	\begin{itemize}[noitemsep]
		\item A collection of \textit{objects}.
		\item A collection of \textit{arrows} or \textit{morphism} between objects. Each morphism starts from a source object and ends into a target object.
		\item For every triplet of objects, there exists a composition operation $\circ$, such that, given the morphisms $f:a \rightarrow b$ and $g:b \rightarrow c$ then $g \circ f: a \rightarrow c$.
		\item The composition operation is associative, i.e. $f \circ (g \circ h) = (f \circ g) \circ h$.
		\item For each object $x$ There exists a morphism $1_x: x \rightarrow x$, called \textit{identity}, such that for every morphism $f:a \rightarrow x$ and $g: x \rightarrow b$ we have that $f \circ 1_x = f$ and $g \circ 1_x = g$.
	\end{itemize}
\end{definition}

\noindent
Functors are mapping between two categories defined as follows:

\begin{definition}
	Given two categories $\mathcal{C}_1$ and $\mathcal{C}_2$, a \textit{functor} $\mathcal{F}$ from $\mathcal{C}_1$ to $\mathcal{C}_2$ is a mapping such that:
	
	\begin{itemize}
		\item Each object $x$ of $\mathcal{C}_1$ is mapped to an object $\mathcal{F}(x)$ of $\mathcal{C}_2$.
		\item Each morphism $f: a \rightarrow b$ of $\mathcal{C}_1$ is mapped to a morphism $\mathcal{F}(f): \mathcal{F}(a) \rightarrow \mathcal{F}(b)$ such that
		\begin{itemize}
			\item $\mathcal{F}(1_x) = 1_{\mathcal{F}(x)}$.
			\item For all morphism $f: a \rightarrow b$ and $g: b \rightarrow c$ of $\mathcal{C}_1$ we have that $\mathcal{F}(g \circ f) = \mathcal{F}(g) \circ \mathcal{F}(f)$.
		\end{itemize}
	\end{itemize}
\end{definition}

\noindent
Informally, functors are transformations between categories that preserve the identity and the associativity properties. In the scope of programming languages the term functor is used with a more narrow sense: they usually define transformations between types. These transformations are functors (actually \textit{endofunctors} since they transform elements of the category of types in elements of the same category) at all effects but not all functors from category theory coincide with functors in a programming language. Popular programming languages that provide functors in this sense are Haskell with \textit{Type Classes} \cite{jones1995functional, kiselyov2004strongly, mcbride2002faking, thompson1999haskell, wadler1989make} and Caml with \textit{Modules} \cite{leroy2000modular, paulson1996ml, wehr2008ml}. Functors in Metacasanova are no different: they define transformations between types. Modules are simply collection of function and functor declarations grouped together under the same name that can be used as types themselves.

\subsection{Language Extension}
\label{subsec:ch_functors_language_extension}
Modules can be defined through the keyword \texttt{Module} followed by a module name and series of construction parameters that are used to create an instance of the module. Constructions parameters have the same form of parameters in normal functions, so they are defined through an identifier and a type. The special symbol \texttt{*} (\textit{kind}) can be used if any type is suitable for that specific argument. Elements of a module can be accessed with the \texttt{.} access operator.

\begin{lstlisting}
Module "M" => ma1 : t1 => ma2 : t2 => ... => ma_k : tk : M {
  Func "f1" -> ...
  Func "f2" -> ...
  Func "f_k" -> ...
  
  ...
} 
\end{lstlisting}

\noindent
Functors are defined similarly to function but using the double arrow instead of the single arrow:

\begin{lstlisting}
Functor "foo" a1 => a2 => ... => an : T
\end{lstlisting}

\noindent
Moreover, since the result of calling a functor is a type, functors can be used wherever a type annotation is required, for example in the declaration of a function

\begin{lstlisting}
Func "bar" b1 -> b2 -> ... -> (foo a1 a2 ... an) -> ... : U
\end{lstlisting}

\noindent
Functors are evaluated through rules whose behaviour is identical to those used to evaluate functions. The difference lies in the fact that results of functors are evaluated at compile time rather than runtime. Functors results are evaluated by an interpreter that mimics the semantics of rules in natural semantics, in the fashion of the semantics used in the code generation explained in Section \ref{sec:ch_metacasanova_code_generation}. Since the evaluation is performed at compile time, all the values passed to a functor call must be known when compiling the meta-program. This means that the arguments of a functor call can be either types or constants. When an evaluation rule for a functor is called, this is run through the interpreter and a module instance is returned as result. Figure \ref{fig:ch_functors_compiler_architecture} shows the steps performed by the new compiler architecture to include functors interpretation. Functors can be called both in the premises of rules for functors and for rules that evaluate regular functions. In the latter case, the premise will simply instantiate the module that can then be used within the rule itself. This process is shown in Figure \ref{fig:ch_functors_functor_processing}: the functor call is processed by selecting the possible candidate rules to execute it, in the same fashion of what is done for regular functions. At this point the interpreter run the rule and the result of the first one that succeeds is taken. The result of such rule is a module instantiation. The module instantiation is bound to the variable contained in the result of the premise. From that point on, the module instance can be referred by the caller rule.

In the following sections we show how to implement the mechanism of inlining for the record getter and setter described in Section \ref{sec:ch_functors_idea} that makes use of the compile-time interpretation of functors.

\begin{figure}
	\centering
	\includegraphics[width=\textwidth]{Figures/chapter_functors/compiler_architecture_functors}
	\caption{Compiler architecture with functor interpreter}
	\label{fig:ch_functors_compiler_architecture}
\end{figure}

\begin{figure}
	\centering
	\includegraphics[width=\textwidth]{Figures/chapter_functors/functor_rule_processing}
	\caption{Functor processing}
	\label{fig:ch_functors_functor_processing}
\end{figure}

\section{Record implementation with modules}
\label{sec:ch_functors_record_implementation}
In Section \ref{subsec:ch_functors_inlining} we showed how Casanova entities can be expressed, at meta-language level, as a tuple of field names and values. We also showed that getters and setters always perform the same steps when looking up for the same field because the entity structure does not change at runtime. In this section we proceed to give an implementation based on functors to implement a Casanova entity. We refer to this implementation as ``Record'', since a Casanova entity is simply a record from the point of view of the data representation and since this solution works in general for any data structure that is isomorphic to a record. From now on we also use, as example, the physical body entity described in Section \ref{subsec:ch_functors_casanova_example}.

A module for records simply contains a functor that returns the type of the record. This functor, in general, can return any type since the type of the record can be ``customized'' and depends on the specific definition given by the programmer (thus it cannot be known beforehand). For this reason we use \textit{kind} as return type for this functor. The functor itself is parameterless since nothing is required to generate the type of a record.

\begin{lstlisting}
Module "Record" : Record {
  Functor "RecordType" : * }
\end{lstlisting}

The data representation of the record will be a tuple as shown in Section \ref{subsec:ch_functors_inlining}. For this purpose, we need two functors that are able to represent the type of a record in a recursive way with one being the type of an empty record (a record with no fields) and another a record field followed by the rest of the record representation. The functor for the empty record simply returns the type of the record module, while the functor to represent a record field takes as input a \texttt{string}, representing the name of the field, \textit{kind} because a record field can have any type, and a \texttt{Record} which represents all the other fields coming after the current one. 

\begin{lstlisting}
Functor "EmptyRecord" : Record
Functor "RecordField" => string => * => Record : Record
\end{lstlisting}

After declaring the functors necessary to build a record, we proceed to define their implementation in the form of rules. The functor for an empty record simply generates a module containing a function \texttt{cons}, that is the constructor for the record, that simply returns unit (because an empty record does not contain any field). Consistently, the functor \texttt{RecordType} implemented by the module will simply return \texttt{unit} as type. Note that a module instantiation must implement \textbf{at least} all the declarations of the module (like for an interface), but can add other declarations and implementations that are not shared by all the module instantiations. For example \texttt{cons} for an empty record is different than the one for a non-empty one.

\begin{lstlisting}
-------------------
EmptyRecord => Record {

  Func "cons" : unit
  
  ------------------
  RecordType => unit
  
  ------------------
  cons -> ()

}
\end{lstlisting}

A record field must be constructed through a functor that takes the field name, the type of the field, and the type of the rest of the record. This functor will construct the type of a record as a \texttt{Tuple}, where the first element is the type of the current field and the second the type of the rest of the record. The constructor of the record field will be a function that takes as input an argument of the type of the current field, a tuple representing the remaining part of the record and returns a tuple combining the current field and the rest of the record.

\begin{lstlisting}
------------------
RecordField name type r = Record {
  Func "cons" -> type -> r.RecordType : RecordType

  ---------------------------------------
  RecordType => Tuple[type,r.RecordType]

  -------------------
  cons x xs -> (x,xs)}
\end{lstlisting}

Consider now the physical body representation given above. We show how to use the functors we have just defined to build an instance of a physical body. First of all we defined a functor \texttt{PhysicalBodyType} that returns a \texttt{Record}.

\begin{lstlisting}
Functor "PhysicalBodyType" : Record
\end{lstlisting}

The final representation of the type that should be returned by\\ \texttt{PhysicalBodyType} is \texttt{Tuple[Vector2,Tuple[Vector2,unit]]} because the field \texttt{Position} and \texttt{Velocity} have type \texttt{Vector2}. Note that \texttt{Vector2} can be trivially implemented in Metacasanova as a tuple containing two floating point values. Here we use this type assuming that has already been defined above. The same appiles to \texttt{unit}, which can be defined as a meta-data with no arguments.

The rule to evaluate \texttt{PhysicalBodyType} will call in its premises \texttt{EmptyRecord} and \texttt{RecordField} to generate the type of the tuple appropriately:

\begin{lstlisting}
EmptyRecord => empty
RecordField "Velocity" Vector2 empty => velocity
RecordField "Position" Vector2 velocity => body
----------------------------
PhysicalBodyType => body
\end{lstlisting}

Let us now analyse in detail what the premises generate: the first premise will generate an instance of \texttt{EmptyRecord} and bind it to the variable \texttt{empty}. The instance of this module is parameterless and thus will always be the same every time the functor is invoked. The second premise will instantiate \texttt{RecordField} by using the string \texttt{"Velocity"} as field name, \texttt{Vector2} as field type, and \texttt{empty} as argument for the remaining part of the record (there is no other field after \texttt{Velocity} in the physical body). The instantiation of \texttt{RecordField} produces a rule for the functor \texttt{RecordType}. According to the definition above this functor generates \texttt{Tuple[type,r.RecordType]}. By replacing the argument values provided in the premise, we have that \texttt{type := Vector2} and \texttt{r := empty := EmptyRecord}. Thus \texttt{r.RecordType} uses the functor \texttt{RecordType} in the instance of \texttt{EmptyRecord} which returns the type \texttt{unit} (the call can be seen as \texttt{empty.RecordType}). Thus \texttt{r.RecordType} can be replaced with \texttt{unit}, thus leading to \texttt{Tuple[Vector2, unit]}. Thus the rule for the functor \texttt{RecordType} generated in the module returned by the second premise will be.

\begin{lstlisting}
-----------------------
RecordType => Tuple[Vector2,unit]
\end{lstlisting}

By replacing the argument variables with the values provided in the second premises we can also get the declaration and rule for \texttt{cons}. By replacing again \texttt{type} and \texttt{r.RecordType} as done before, we have that the declaration for \texttt{cons} in the current instance of the module becomes:

\begin{lstlisting}
Func "cons" -> Vector2 -> unit: Tuple[Vector2,unit]
\end{lstlisting}

\noindent
while the corresponding rule will be generated as

\begin{lstlisting}
--------------------
cons x xs -> (x,xs)
\end{lstlisting}

\noindent
The complete module instance will then look like:

\begin{lstlisting}
velocity := Record {
  Func "cons" -> Vector2 -> unit: Tuple[Vector2,unit]
  
  -----------------------
  RecordType => Tuple[Vector2,unit]
  
  --------------------
  cons x xs -> (x,xs)
}
\end{lstlisting}

The third premise calls \texttt{RecordField} with \texttt{name := "Position"}, \texttt{type := Vector2}, and \texttt{r := velocity}. Now in the definition of the \texttt{RecordField} module again the functor \texttt{RecordType} returns \texttt{Tuple[type,r.RecordType]}. Now \texttt{r.RecordType} can be rewritten as \texttt{velocity.RecordType} that returns (see the instantiation of \texttt{velocity} above) \texttt{Tuple[Vector2,unit]}. Thus\\ \texttt{RecordType} for the field \texttt{Position} will be instantiated as

\begin{lstlisting}
-----------------------
RecordType => Tuple[Vector2,Tuple[Vector2,unit]]
\end{lstlisting}

\noindent
Analogously the declaration of \texttt{cons} will be instantiated as

\begin{lstlisting}
Func "cons" -> Vector2 -> Tuple[Vector2,unit]: Tuple[Vector2,Tuple[Vector2,unit]]
\end{lstlisting}

\noindent
while its rule is the same. The full module instance will then be

\begin{lstlisting}
body := Record {
  Func "cons" -> Vector2 -> Tuple[Vector2,unit]: Tuple[Vector2,Tuple[Vector2,unit]]
  
  -----------------------
  RecordType => Tuple[Vector2,Tuple[Vector2,unit]
  
  --------------------
  cons x xs -> (x,xs)
}
\end{lstlisting}

\noindent
which is returned by the functor \texttt{PhysicalBodyType}. In order to build an instance of the physical body, we define a function that returns a value of type \texttt{PhysicalBodyType} (which in turn is simply\\ \texttt{Tuple[Vector2,Tuple[Vector2,unit]}):

\begin{lstlisting}
Func "PhysicalBody" : PhysicalBodyType.RecordType

-----------------------
PhysicalBody -> PhysicalBodyType.cons((10.0,10.0),((10.0,0.0),()))
\end{lstlisting}

\noindent
The rule creates a physical body in position $(10,10)$ moving at velocity $(10,0)$.\\

One of the main arguments in favour of using functors was that they should allow to embed the type system of the embedded language in the meta-type system of Metacasanova. This means that, at compile time, the meta-compiler should be able to detect a physical body that is constructed in the wrong way. Let us then assume that we define another function to build a physical body where the programmer uses a scalar for the velocity instead of a vector:

\begin{lstlisting}
Func "WrongPhysicalBody" : PhysicalBodyType.RecordType

-------------------------------------
WrongPhysicalBody ->  PhysicalBodyType.cons((10.0,10.0),(10.0,()))
\end{lstlisting}

\noindent
What happens is that \texttt{PhysicalBodyType.RecordType} is equal to\\ \texttt{Tuple[Vector2,Tuple[Vector2,unit]]}. At this point the type checker of Metacasanova will successfully match the first element of the tuple returned by the rule, which is correctly provided as a value of type \texttt{Vector2}, but will fail to check the second, which is \texttt{double} where it expects a \texttt{Vector2}. With the implementation based on dictionaries given in Section \ref{subsec:ch_mcnv_languages_casanova_semantics} this check happens dynamically at runtime by means of type rules defined in the meta-program, rather than statically like in this case.

\section{Using Modules and Functors in Metacasanova}
\label{subsec:record_implementation}
A module definition in Metacasanova is parametric with respect to types, in the sense that a module definition might contain some type parameters, and can be instantiated by passing the specific types to use. A module can contain the definition of data structures, functions, or functors.

\begin{lstlisting}
Module "Record" : Record {
  Functor "RecordType" : * }
\end{lstlisting}

The symbol \texttt{*} reads \textit{kind} and means that the functor might return any type. Indeed the type of a record (or class) in a programming language can be ``customized'' and depends on its specific definition, thus it is not possible to know it beforehand.

We the define two modules for the \textit{getter} and \textit{setter} of a field of a record. In this example, we use type parameters in the module definitions.

\begin{lstlisting}
Module "Getter" => (name : string) => (r : Record) {
  Functor "GetType" : *
  Func "get" -> (r.RecordType) : GetType }
  
Module "Setter" => (name : string) => (r : Record ) {
  Functor "SetType" : *
  Func "set" -> (r.RecordType) -> SetType : (r.RecordType) }
\end{lstlisting}

\noindent
These two modules respectively define a functor to retrieve the type of the record field, and a function to get or set its value. Note that in the function definitions \texttt{get} and \texttt{set} we are calling the functor of the \texttt{Record} module to generate the appropriate type for the signature. This is allowed, since the result of a functor is indeed a type.

A record meta-type (i.e. its representation at meta-language level) is recursively defined as a sequence of pairs $(field,type)$, whose termination is given by \texttt{EmptyField}. We thus define the following functors:

\begin{lstlisting}
Functor "EmptyRecord" : Record
Functor "RecordField" => string => * => Record : Record
\end{lstlisting}

\noindent
The first functor defines the end point of a record, which is simply a record without fields. The second functor defines a field as the pair mentioned above followed by other field definitions.

Moreover, we must define two functors that are able to dynamically build the \textit{getter} and \textit{setter} for the field.

\begin{lstlisting}
Functor "GetField" => string => Record : Getter
Functor "SetField" => string => Record : Setter
\end{lstlisting}

The behaviour of functor is expressed, as for normal functions, through a rule in the meta-program. A rule that evaluates a functor returns an instantiation of a module. Note that, inside a module instantiation, it is possible to define and implement functions other than those in the module definition, i.e. the module instantiation must implement \textit{at least} all the functors and functions of the definition. For instance, the following is the type rule instantiating the module for \texttt{EmptyRecord}:

\begin{lstlisting}
-------------------
EmptyRecord => Record {

  Func "cons" : unit

  ------------------
  RecordType => unit

  ------------------
  cons -> ()

}
\end{lstlisting}

\noindent
The function \texttt{cons} defines a constructor for the record, which, in the case of an empty record, returns nothing. The module instantiation for a record field evaluates as well \texttt{RecordType}, and has a different definition and evaluation of the function \texttt{cons} (because it is constructed in a different way):

\begin{lstlisting}
------------------
RecordField name type r = Record {
  Func "cons" -> type -> r.RecordType : RecordType
  
  ---------------------------------------
  RecordType => Tuple[type,r.RecordType]
  
  -------------------
  cons x xs -> (x,xs)} 
\end{lstlisting}

\noindent
Note that the return type of \texttt{cons} is to be intended as calling \texttt{RecordType} of the current module, so as it were \\ \texttt{this.RecordType}.
The getter of a field must be able to lookup the record data structure in search of the field and generate a function to get the value from it. For this reason, the functor instantiates two separate modules, depending on the name of the field that we are currently examining.

\begin{lstlisting}[caption = Module instantiations for getters, label = code:getters]
//Rule 1
name = fieldName
thisRecord := RecordField name type r
-----------------
GetField fieldName (RecordField name type r) => Getter name thisRecord {
  GetType => type
  
  ---------------
  get (x,xs) -> x}

//Rule 2
name <> fieldName
thisRecord := RecordField name type r
------------------
GetField fieldName (RecordField name type r) => Getter name type thisRecord{
  Functor "GetAnotherField" : Getter
  
  ---------------
  GetAnotherField => GetField fieldName r
  
  GetAnotherField => g
  ---------------
  GetType => g.GetType
  
  GetAnotherField => getter
  getter.get xs -> v
  -------------------
  get (x,xs) -> v }
\end{lstlisting}

\noindent
Analogously, the setter of a field instantiates two separate modules whether the current field is the one we want to set or not.

\begin{lstlisting}[caption = Module instantiations for setters, label = code:setters]
name = lt
thisRecord := RecordField name type r
----------
SetField lt (RecordField name type r) => Setter name thisRecord{
  
  -----------------
  SetType => type
  
  -------------------
  set (x,xs) v -> (v,xs)}

name <> lt
thisRecord := RecordField name type r
------------
SetField lt (RecordField name type r) => Setter name thisRecord{
  TypeFunc "SetAnotherField" : Setter
  
  -------------------------
  SetAnotherField => SetField lt r
  
  ----------------------------
  SetType => type
  
  SetAnotherField => setter
  setter.set xs v -> xs'
  ----------------------------------
  set (x,xs) v -> (x,xs') }
\end{lstlisting}

\section{Functor result inlining}
If a premise or a conclusion contains a call to a functor, this call is evaluated at compile time, rather than at runtime. Metacasanova has been extended with an interpreter which is able to evaluate the result of the functor calls. The behaviour of the interpreter follows the same logic explained when presenting the code generation steps in Section \ref{sec:code_generation}, thus here we do not present the details for brevity. When a rule outputs the instantiation of the module, the generated code will contain only rules of the modules which conclusion contains a function (i.e. functions that output values, not functors). In this way the generated code will contain a different version of those functions depending on the instantiation parameters of the module.

We now show how to use the implementation of the records given in Section \ref{subsec:record_implementation} for the physical body presented as a case study.
The definition of the record type for the physical body is done through a functor

\begin{lstlisting}
Functor "PhysicalBodyType" : Record

EmptyRecord => empty
RecordField "Velocity" Vector2 empty => velocity
RecordField "Position" Vector2 velocity => body
--------------------------
PhysicalBodyType => body
\end{lstlisting}

This rule is evaluated at compile time by the interpreter that generates one module for each field of the \texttt{PhysicalBody}, containing the constructor. For example, for the field \texttt{Velocity} the interpreter will generate\footnote{Note that here we give a high-level representation of the generated rules that are actually directly generated as C\# code.}

\begin{lstlisting}
Func "cons" -> Vector2 -> unit : Tuple[Vector2,unit]

------------------------
cons x xs -> (x,xs)
\end{lstlisting}

This because the functor will call the evaluation rule for \texttt{RecordField} with the argument \texttt{(Recordfield "Velocity" Vector2 (EmptyRecord))}. This rule generates the function \texttt{cons} by evaluating the result of the functors\\ \texttt{EmptyRecord.RecordType} and \texttt{RecordField.RecordType}, which respectively produce \texttt{unit} and \texttt{Tuple[Vector2,unit]}.

Instantiating a physical body will just require to build a function that returns the type of the physical body, which is obtained by calling the functor \texttt{PhysicalBodyType}.

\begin{lstlisting}
Func "PhysicalBody" : PhysicalBodyType.RecordType

-----------------------
PhysicalBody -> PhysicalBodyType.cons((Vector2.Zero,(Vector2.Zero,())))
\end{lstlisting}

Defining the setter and getter of a field, requires to use the functor \texttt{GetField} to generate the appropriate getter function. After the module has been correctly generated, we can use the getter for the field. For example, in order to get the position field, we use the following function.

\begin{lstlisting}
Func "getPos" -> PhysicalBodyType : Vector2

GetField "Position" PhysicalBodyType => getter
getter.get PhysicalBody -> p
-------------------------------
getPos -> p
\end{lstlisting}

The result of the premise \texttt{GetField} will be evaluated at compile time through the code in Listing \ref{code:getters} and will instantiate a module containing the following function definition and rule.

\begin{lstlisting}
Func "get" -> Tuple[Vector2,Tuple[Vector2,unit]] : Vector2

-------------------------
get (x,xs) -> x
\end{lstlisting}

\noindent
Note that the second premise of \texttt{getPos} will immediately call the \texttt{get} generated in this step. The case of \texttt{setPos} is analogous except the setter takes an additional argument.

Reading \texttt{Velocity} analogously uses a functor call to generate a getter:

\begin{lstlisting}
Func "getVel" -> PhysicalBodyType : Vector2

GetField "Velocity" PhysicalBodyType => getter
getter.get PhysicalBody -> p
-------------------------------
getVel -> p
\end{lstlisting}

\noindent
This time the functor will generate two different functions in two separate modules. The first time the record is processed, \texttt{Rule 2} in Listing \ref{code:getters} will be activated (because the first field in the Record is \texttt{Position}). This rule will instantiate an additional module when evaluating the functor call in its premise, which in turn is able to get the \texttt{Velocity} field. The rule for \texttt{get} in the first module will contain in its premise a call to  \texttt{get} of the second module.

\begin{lstlisting}
//Code for module1
Func "get" -> Tuple[Vector2,Tuple[Vector2,unit]] : Vector2

module2.get xs -> v
-------------------------
get (x,xs) -> v

//Code for module2 generated by evaluating the functor in the premise of Rule 2
Func "get" -> Tuple[Vector2,unit] : Vector2

------------------
get (x,xs) -> x
\end{lstlisting}

We want to point out that this optimization has been presented on the specific case of records, but can be generalized for any situations where you would use a symbol table. Indeed any symbol table can be expressed with the representation above as a sequence of pair where the first item is the value of the current variable, and the second item is the continuation of the symbol table.

\section{Evaluation}
An extensive evaluation of Casanova implemented in Metacasanova, which we omit for brevity, can be found in \cite{DiGiacomo2017}. The implementation of Casanova operational semantics in Metacasanova is almost 5 times shorter than the corresponding F\# implementation in the hard-coded compiler. In addition to Casanova, we have implemented a subset of the C language called C-{}-. This language supports \texttt{if-then-else}, \texttt{while-loop}, and \texttt{for} statements, as well as local scoping of variables. The total length of the language definition in Metacasanova is 353 lines of code. The corresponding C\# code to implement the operational semantics of the language is 3123 lines, thus the code reduction with Metacasanova is roughly 8.84 times. For comparison, in Table \ref{tab:cmm} it is possible to see the code length to implement three different statements, both in Metacasanova and C\#. We tested C-{}- against Python by computing the average running time to compute the factorial of a number. C-{}- results to be 50 times slower than Python. This result is worse than what we obtained with Casanova, because in order to emulate the interruptible rule mechanism of Casanova in Python you must rely on coroutines that are slower than a program containing simple statements. Moreover, we tested the performance improvement of the optimization using Functors to represent records against the standard one using dynamic symbol tables. The test was run using records with a number of fields ranging from 1 to 10 and updating from 10000 to 1000000 instances of such records. In Table \ref{tab:functors}, we can see that the optimization using Functors leads to a performance increase on average of about 11 times, with peaks of 30 times. The gain increases with the number of fields, thus Functors are particularly effective for records with high number of fields. Figure \ref{fig:chart} shows a chart of the overall performance of the two techniques (the data points are taken from Table \ref{tab:functors}). The horizontal axis contains the amount of fields per record, while the vertical axis contains the number of records that are being updated. We can see that the performance of the dynamic table degrades considerably when increasing the number of fields, and that the higher the amount of records is, the steeper the curve is. On the other hand, the performance of the implementation with Functors is almost constant, regardless of the amount of fields or records that are being updated. Moreover, note that the performance of the dynamic table is improved by the fact that we are using a dictionary implemented in .NET, which can access the entries in $O(\log n)$. If the symbol table were represented as a meta-data structure in the language the performance would be even worse, since it would have to be encoded as a list of pairs with the field name and its value, and its manipulation would be affected by the evaluation rules that should implement this behaviour. Furthermore, the dynamic lookup should be done also to ensure that the types of the record fields are used consistently (for example to prevent that a record is constructed with incompatible values for its fields), while using the functors in Metacasanova embeds the type system of the language in the meta-type system, whose type safety is checked at compile-time rather than at runtime, and this contributes to further increase the performance.

\begin{table}	
	\caption{Running time with the functor optimization and the dynamic table with 10000, 100000, and 1000000 records.}
	\begin{tabular}{|c|c|c|c|}
		\hline
		\textbf{FIELDS}& \textbf{Functors (ms)}&\textbf{Dynamic Table (ms)} & \textbf{Gain}\\ \hline
		1&	1.00E-05&	5.00E-06&	0.50\\ \hline
		2&	9.00E-06&	1.30E-05&	1.44\\ \hline
		3&	9.00E-06&	2.70E-05&	3.00\\ \hline
		4&	9.00E-06&	4.50E-05&	5.00\\ \hline
		5&	9.00E-06&	7.00E-05&	7.78\\ \hline
		6&	9.00E-06&	9.90E-05&	11.00\\ \hline
		7&	9.00E-06&	1.33E-04&	14.78\\ \hline
		8&	9.00E-06&	1.75E-04&	19.44\\ \hline
		9&	9.00E-06&	2.20E-04&	24.44\\ \hline
		10&	9.00E-06&	2.70E-04&	30.00\\ \hline
		\multicolumn{2}{c|}{} & \textbf{Average gain} & 11.74\\ \cline{3-4}			
	\end{tabular}
	
	\vspace{0.15cm}
	\begin{tabular}{|c|c|c|c|}
		\hline
		\textbf{FIELDS}& \textbf{Functors (ms)}&\textbf{Dynamic Table (ms)} & \textbf{Gain}\\ \hline
		1&	9.60E-05&	6.30E-05&	0.66\\ \hline
		2&	9.40E-05&	1.59E-04&	1.69\\ \hline
		3&	9.50E-05&	3.04E-04&	3.20\\ \hline
		4&	9.60E-05&	5.03E-04&	5.24\\ \hline
		5&	9.60E-05&	7.52E-04&	7.83\\ \hline
		6&	9.60E-05&	1.05E-03&	10.95\\ \hline
		7&	9.70E-05&	1.41E-03&	14.57\\ \hline
		8&	9.80E-05&	1.82E-03&	18.59\\ \hline
		9&	9.90E-05&	2.29E-03&	23.17\\ \hline
		10&	1.00E-04&	2.81E-03&	28.05\\ \hline
		\multicolumn{2}{c|}{} & \textbf{Average gain} & 11.39\\ \cline{3-4}						
	\end{tabular}
	
	\vspace{0.15cm}
	\begin{tabular}{|c|c|c|c|}
		\hline
		\textbf{FIELDS}& \textbf{Functors (ms)}&\textbf{Dynamic Table (ms)} & \textbf{Gain}\\ \hline
		1&	9.47E-04&	7.29E-04&	0.77\\ \hline
		2&	9.51E-04&	1.78E-03&	1.87\\ \hline
		3&	9.50E-04&	3.33E-03&	3.51\\ \hline
		4&	9.60E-04&	5.43E-03&	5.66\\ \hline
		5&	9.65E-04&	8.03E-03&	8.32\\ \hline
		6&	9.71E-04&	1.11E-02&	11.44\\ \hline
		7&	9.75E-04&	1.47E-02&	15.12\\ \hline
		8&	9.82E-04&	1.89E-02&	19.28\\ \hline
		9&	9.92E-04&	2.37E-02&	23.86\\ \hline
		10&	1.00E-03&	2.87E-02&	28.62\\ \hline
		\multicolumn{2}{c|}{} & \textbf{Average gain} & 11.84\\ \cline{3-4}						
	\end{tabular}
	\label{tab:functors}
\end{table}

\begin{figure}
	\includegraphics[width = \columnwidth]{Figures/functor_chart.jpg}
	\caption{Execution time of the different memory models}
	\label{fig:chart}
\end{figure}

	
\chapter{Language Design with Functors}
\label{ch:functor_languages}
\epigraph{A monad is just a monoid in the category of endofunctors, what's the problem?}{James Iry}
In Chapter \ref{ch:languages} we showed an implementation in Metacasanova of Casanova, a domain-specific language for game development and we discussed the reason of the poor performance of that implementation. In Chapter \ref{ch:functors} we extended Metacasanova with functors and modules to allow to embed the type system of an embedded  language \footnote{See the introduction of Chapter \ref{ch:functors} for a definition of this term} in the meta-compiler to overcome the problem of dynamic lookups at runtime. We then showed an implementation of records with modules and functors that significantly improved the performance of memory accesses, as shown in Section \ref{sec:ch_functors_evaluation}. In this chapter we show how this language extension can be used to improve the performance of the implementation in Metacasanova of the domain-specific language for game development Casanova. In what follows we start by describing how entities are updated in Casanova to make their dynamics evolve with respect to time. We then proceed to discuss how functors can be used to describe the semantics of entity updates in Casanova, and we further refine it to support the semantics of interruption of Casanova rules. We conclude with an evaluation about the performance gain achieved by using this implementation over the previous one presented in Chapter \ref{ch:languages}.

\section{Casanova entity update}
\label{subsec:ch_networking_casanova_update}
In Section \ref{subsec:ch_mcnv_languages_casanova_semantics} we described the memory representation of a Casanova entity in Metacasanova and how the rules of an entity are updated. What was skipped for brevity was a description of how the system behind Casanova updates the entities of a Casanova program. As briefly described in Section \ref{sec:ch_mcnv_languages_casanova_language}, the structure of a program in Casanova is a tree, whose root is a special entity called \textit{World}. The world entity can contain fields that are instances of other entities as well, thus creating an additional level in the program tree. This is, of course, allowed also for regular entities, thus the height of the tree is arbitrary. Each entity might contain a set of rules that describe its dynamic behaviour with respect to time, thus they are updated by considering the time difference between the current and the previous update (\textit{frames}). Updating a rule is enforced by traversing the entity tree, thus when the field of an entity is an entity itself, the system will first update the entity instance contained in the field and then update the current entity. Casanova also natively supports lists and tuples as valid data types, and this requires to handle their update as well: a tuple or a list might themselves contain instances of entities that must be updated accordingly. In the case of a list of entities, we must run the update on each element, while in the case of a tuple we must examine each element and check whether it requires an update or not. This process is called \textit{update traversal} and might become very complex, as lists and tuples can be combined together in infinite many ways, thus the process recursively calls the proper update depending on the type of the field.

For instance, let us consider a simulation consisting of an arbitrary number of physical bodies, in the fashion of what was used in Section \ref{subsec:ch_functors_casanova_example}. The world entity will contain a list of physical bodies that are updated during the simulation. The Casanova code that described such a simulation is the following:

\begin{lstlisting}
worldEntity World {
	PhysicalBodies : [PhysicalBody]
}

entity PhysicalBody {
	Position				: Tuple<float,float>
	Velocity				: Tuple<float,float>
	Acceleration		: Tuple<float,float>
	
	rule Position = Position + Velocity * dt
	rule Velocity = Velocity + Acceleration * dt
}
\end{lstlisting}

\noindent
In this simulation, the update starts from \texttt{World}. This entity contains only one field, which is a list of physical bodies. Since \texttt{PhysicalBody} is an entity, the update must be run individually for each element of the list. The world contains no rules, thus after updating its only field we complete its update. At this point the update of each of the physical body examines each fields. All fields are represented as a point in a 2D space with a tuple containing two floating-point values. The update will examine each value of the tuple and find that they do not require any update (again because the only language abstractions that exhibit dynamic behaviours are entities). The update will then move on to run the rules that will update the content of \texttt{Position} and \texttt{Velocity}. The update process is sketched in Figure \ref{fig:ch_networking_simulation_update} and can thus be seen as a process that consists of the following steps:

\begin{enumerate}[noitemsep]
	\item An \textit{entity update} that traverses all the fields and rules of the entity and calls the appropriate updater.
	\item A \textit{field update} that updates (or not) the field depending on its type. The fields that will be updated have type \texttt{List}, \texttt{Tuple}, or \texttt{Entity}.
\end{enumerate}

\begin{figure}
	\centering
	\includegraphics[width=\textwidth]{Figures/chapter_networking/update_traversal}
	\caption{Entity update for the simulation of physical bodies}
	\label{fig:ch_networking_simulation_update}
\end{figure}

\section{Update in Metacasanova}
\label{subsec:ch_networking_update_metacasanova}
The update mechanism described in Section \ref{subsec:ch_networking_casanova_update} can of course be integrated in the implementation of Casanova described in Chapter \ref{ch:languages}. In order to do so, we should dynamically look into the dictionary representing the entity fields at each update, extract the field and perform an update according to the following cases:

\begin{itemize}[noitemsep]
	\item If the field is a list, then we must examine each element and choose for each one whether it needs to be updated or not. This is done by recursively applying these cases (being a dynamic check we have to perform this check for each element).
	\item If the field is a tuple, then we behave as above.
	\item If the field is an entity, then we must run an update on it.
	\item In all the other cases the field is not updated.
\end{itemize}

\noindent
The cases above are translated into four rules in Metacasanova. The first three will use pattern-matching to decide whether the examined field is a list, a tuple, or an entity. The fourth one is a default rule that simply returns the field as it is. Moreover, each entity should store a list of rules that are updated as well, where all the get and set operations require dynamic lookups in the symbol table of the entity.

Repeating the traversal of the entity tree at each update at runtime is unnecessary since

\begin{itemize}[noitemsep]
	\item The structure of a Casanova entity cannot change at runtime. Its fields and field types will always remain the same.
	\item The fields affected by an entity rule and the rules of an entity do not change during the program execution.
\end{itemize}

\noindent
This means that, by exploiting modules and functors, we are able to specify the structure of the update at compile time and generate directly the function that performs the update at runtime, in the same fashion as what has been done for the record setter and getter. In the following sections we will describe extensively the implementation of the update using modules and functors in Metacasanova that generates at compile time the functions necessary to perform the update of a Casanova program. Note that we will refer to the implementation of records given in Section \ref{sec:ch_functors_record_implementation}.

\section{Updater Modules}
\label{subsec:ch_networking_updater_modules}
As explained above, Casanova needs to recursively update fields that are lists, tuples, or entity instances. For this purpose, we define a module that represents an \textit{updatable element} in the Casanova language. The module constructor takes as only argument the type of the element to update. This module contains a function \texttt{update} that is able to update a value of this particular type and uses an additional parameter \texttt{dt} that contains the time difference between the current and the previous update. The function returns the updated value of the element. It also contains an utility functor to return the type of the element.
 
\begin{lstlisting}
Module "ElementUpdater" => (elementType : *) : ElementUpdater {
	Functor "GetType" : *
	Func "update" -> elementType -> float : elementType
}
\end{lstlisting}

\noindent
The updater for a field is a module constructed by providing the record of the fields, its name as a string, and contains: (\textit{i}) an utility functor that returns the record used in the field updater, and (\textit{ii}) an update function that takes as input an instance of the record, \texttt{dt}, and returns the updated value of the field. We also define an external utility functor \texttt{GetFieldType} that can retrieve the type of a record field given the record it belongs to and its name. The rule for the functor calls the field getter and its \texttt{GetType} functor to retrieve the type of the field. This functor is used by the module constructor to correctly generate the return type of the \texttt{update} function.

\begin{lstlisting}
Functor "GetFieldType" => Record => string : *

GetField r name => getter
getter.GetType => type
---------------------------
GetFieldType r name => type

Module "FieldUpdater" => (r : Record) => (name : string) : FieldUpdater {
  Functor "GetRecord" : Record
  Func "update" -> r.RecordType -> float : (GetFieldType r name)
}
\end{lstlisting}

\noindent
Finally, the updater for a record is a module constructed by providing the record itself and contains: (\textit{i}) an utility functor that returns the type of the record, and (\textit{ii}) a function \texttt{update} that takes the instance of the record, \texttt{dt}, and returns an updated instance of the record.

\begin{lstlisting}
Module "RecordUpdater" => (r : Record) : RecordUpdater {
  Functor "RecordType" : *
  Func "update" -> r.RecordType -> float : r.RecordType
}
\end{lstlisting}

\section{Updatable elements}
\label{subsec:ch_networking_updatable_elements}
As explained above, the elements for which the update is needed can be lists, tuples, or entity instances. For this reason we have to create separately three different instances of the module \texttt{ElementUpdater} each one dedicated to updating one of those updatable elements. The first updatable element that we consider is an \textit{entity instance}. The module to update such updatable element uses a \texttt{RecordUpdater} to define how the entity instance should be updated. Updating a field containing an entity instance requires the application of the specific record updater for that entity, which in turn returns the updated instance of the entity itself. Thus the declaration for the functor that constructs the proper instance of the module for the entity updater is the following:

\begin{lstlisting}
Functor "UpdateEntity" => RecordUpdater : ElementUpdater
\end{lstlisting}

\noindent
The rule for this functor extracts in its premise the type of the record by calling the utility functor \texttt{RecordType} in the record updater passed as parameter to \texttt{UpdateEntity}. The \texttt{update} function uses the record updater to recursively update the entity instance in its premise and then returns the result of this update.

\begin{lstlisting}
recordUpdater.RecordType => recordType
--------------------------
UpdateEntity recordUpdater => ElementUpdater recordType {

	----------------------
	GetType => recordType
 
  recordUpdater.update entity dt -> entity'
  ------------------------------
  update entity dt -> entity'
}
\end{lstlisting}

\noindent
The second updatable element is the list. An updater for a list must take the updater for its elements. Since a list contains elements of the same type, only one updater is required to instantiate its updater module. The functor \texttt{UpdateList} used to generate this module takes one argument which is an \texttt{ElementUpdater}. This is done because the elements of a list could be themselves other lists, entities, or tuples, so we must be able to use their updaters as arguments for this function. The declaration for this functor is thus:

\begin{lstlisting}
Functor "UpdateList" => ElementUpdater : ElementUpdater
\end{lstlisting}

\noindent
The rule for \texttt{UpdateList} extracts in its premise the type of the elements of the list by calling the functor \texttt{GetType} in the element updater provided as input. It then instantiates an \texttt{ElementUpdater} with the type \texttt{List} using as argument for the generic type the type of the element extracted in its premise. The \texttt{update} function for the list is recursive: its base case is the empty list, for which it simply returns an empty list. For a non-empty list the rule for this function uses the element updater in its premise to update the head of the list and then recursively calls the \texttt{update} of the list on the tail to update the remaining part.

\begin{lstlisting}
updater.GetType => elementType
---------------------------------
UpdateList updater => ElementUpdater List[elementType] {

  -----------------
  GetType => List[elementType]

  --------------------
  update nil dt -> nil

  updater.update x dt -> x'
  update xs dt -> xs'
  -------------------
  update (x :: xs) dt -> (x' :: xs')
}
\end{lstlisting}

\noindent
The updater for tuples is built by defining a functor that takes as input two element updaters, one for the current element of the tuple, and one for the second one. Note that it is possible to recursively provide a tuple updater as a second updater to support the update of tuples containing more than two elements. For example, the updater for \texttt{Tuple[PhysicalBody,Tuple[PhysicalBody,PhysicalBody]]} would require the passing of an entity updater and recursively a tuple updater. The declaration of this fuctor is thus:

\begin{lstlisting}
Functor "UpdateTuple" => ElementUpdater => ElementUpdater : ElementUpdater
\end{lstlisting}

\noindent
The rule for \texttt{UpdateTuple} uses in its premises \texttt{GetType} from the first updater and the second updater to obtain the types of the first and second element of the tuple. It then instantiates \texttt{ElementUpdater} with the tuple type called with the type of the first and second element as arguments for the generics. The \texttt{update} function runs the \texttt{update} of the first updater on the first element of the tuple and the second updater on the second element.

\begin{lstlisting}
updater.GetType => firstType
nextUpdater.GetType => nextType
---------------------------------------------
UpdateTuple updater nextUpdater => ElementUpdater Tuple[firstType,nextType] {

  -------------------
  GetType => Tuple[firstType,nextType]

  updater.update x dt -> x'
  nextUpdater.update x' dt -> xs'
  ----------------------
  update (x,xs) dt -> (x',xs')
}
\end{lstlisting}

Finally, we need a \texttt{ZeroUpdate} that is required for fields whose values do not change with respect to time, namely all those that do not fall in the three categories above. The functor \texttt{ZeroUpdate} takes as input any type and builds an \texttt{ElementUpdater} with that type. The rule for \texttt{update} simply returns the value of the field as it is.

\begin{lstlisting}
Functor "ZeroUpdate" => * : ElementUpdater

-----------------------
ZeroUpdate type => ElementUpdater type {

  ----------------
  GetType => type

  ----------------
  update v dt -> v
}
\end{lstlisting}

\section{Updatable Fields and Records}
The field updater is instantiated by a functor that takes as input an element updater, a record containing the field, and the name of the field to update. Its declaration is the following:

\begin{lstlisting}
Functor "UpdateField" => ElementUpdater => Record => string : FieldUpdater
\end{lstlisting} 

\noindent
The rule for the \texttt{update} function creates in its premises a field getter through the record and the field name passed as input. It then call the function \texttt{get} of the getter with the record instance taken as input to get the value of the field. It then uses the \texttt{update} function from the element updater taken as input from the functor to update the field.

\begin{lstlisting}
----------------------------------------
UpdateField elementUpdater r name => FieldUpdater r name {

  ---------------
  GetRecord => r

  GetField r name => getter
  getter.get rec -> field
  elementUpdater.update field dt -> field' 
  -----------------------------
  update rec dt -> field'
}
\end{lstlisting}

\noindent
The record updater is built by a functor \texttt{Update} that takes as input a field updater, a record updater to update the next part of the record, and returns an instance of the \texttt{RecordUpdater} module. The rule that evaluates the functor extracts the record from the field updater in its premise and passes it to the module constructor for the record updater. The rule for \texttt{update} generates a setter for the field by using the record and the field name. It then calls the field updater passing the record instance and \texttt{dt} as input. This premise will return the updated value for the field. The following premise uses the \texttt{set} function from the previously generated setter to update the record with the new value of the field. After this step it calls the \texttt{update} function of the record updater passed as function argument, which is recursively able to update the remaining part of the record. The result of this update is then returned as final result. Both the functor declaration and the rule for it are provided below

\begin{lstlisting}
Functor "Update" => FieldUpdater => RecordUpdater : RecordUpdater

fieldUpdater.GetRecord => r
---------------------------
Update fieldUpdater nextUpdater => RecordUpdater r {

  r.RecordType => recordType
  ------------------------
  RecordType => recordType

  SetField r name => setter
  fieldUpdater.update rec dt -> v
  setter.set rec v -> rec'
  nextUpdater.update rec' dt -> updatedRecord
  ----------------------------
  update rec dt -> updatedRecord
}
\end{lstlisting}

\noindent
Note that it is possible to provide different field updaters for the same field, as it is possible that, besides the standard Casanova traversal, one wants to define a custom way of updating the field through a Casanova rule.

In order to stop this otherwise infinite recursive process, we must also generate a record updater that simply returns the record as it is. We build such updater through the functor \texttt{NoUpdate}. This functor takes as input a record and instantiates its updater with it. The updater contains a rule for the \texttt{update} function that simply returns the record as it is. The implementation for this updater is provided below:

\begin{lstlisting}
Functor "NoUpdate" => Record : RecordUpdater

---------------------
NoUpdate r => RecordUpdater r {

  r.RecordType => recordType
  ----------------------
  RecordType => recordType
  
  ----------------
  update r dt -> r
}
\end{lstlisting}

\noindent
Finally, rules can be implemented as a field updater that is instantiated by a functor taking as input the record and the field name. The \texttt{update} function will contain the specific code that the rule will perform. In the following section we will provide the implementation of the physical body simulation and show how to use functors to generate the field updater for rules.

\section{Physical Body Simulation with Functors}
\label{subsec:ch_networking_simulation}
In this section we present the implementation with functors of the simulation in Casanova presented in Section \ref{subsec:ch_networking_casanova_update}. The simulation consists of a set of bodies that moves according to their physical properties. As previously done in Section \ref{sec:ch_functors_record_implementation}, we create a functor that builds the record module instance for the physical body:

\begin{lstlisting}
Functor "PhysicalBodyType" : Record

RecordField "Acceleration" Tuple[float,float] EmptyRecord => acceleration
RecordField "Velocity" Tuple[float,float] acceleration => velocity
RecordField "Position" Tuple[float,float] velocity => body
---------------------------
PhysicalBodyType => body
\end{lstlisting}

\noindent
At this point, we define the updaters for the physical body fields. Its fields consist of a tuple with two floating point values. Since floating-point values do not require to be updated in Casanova, we create an updater for the floating-point numbers by using the \texttt{ZeroUpdate} functor that instantiates \texttt{ElementUpdater} with an \texttt{update} function that simply returns the input value.

\begin{lstlisting}
Functor "FloatUpdater" : ElementUpdater

ZeroUpdate float => zero
--------------------------
FloatUpdater => zero
\end{lstlisting}

\noindent
\texttt{ZeroUpdate} calls \texttt{ElementUpdater} with \texttt{elementType := float}. The instance of this module will then contain the following functor rule and function declaration\footnote{Note that the evaluation rules in a functor are always the same for each instance of a module, so from now on we omit them for brevity}.

\begin{lstlisting}
Func "update" -> float -> float : float

-----------------
GetType => float
\end{lstlisting}

\noindent
At this point we can define the element updater for the \texttt{Tuple} that contains the floating point values. This time we use the functor \texttt{UpdateTuple} to instantiate the \texttt{ElementUpdater} module by passing twice \texttt{FloatUpdater} to it. When we do so, we have that (see the definition of the rule for this functor):

\begin{lstlisting}
updater := FloatUpdater
nextUpdater := FloatUpdater
firstType := updater.GetType := float
nextType := nextUpdater.GetType = float
\end{lstlisting}

\noindent
Thus \texttt{ElementUpdater} will be called with \texttt{elementType := \\Tuple[float,float]}. This module instance will then contain the following functor rule and function declaration:

\begin{lstlisting}
Func "update" -> Tuple[float,float] -> float : Tuple[float,float]

-----------------------------
GetType => Tuple[float,float]
\end{lstlisting}

\noindent
We now build the field updaters for the two Casanova rules of the physical body. In order to do so, we define two functors that build their field updaters:

\begin{lstlisting}
Functor "PositionRule" : FieldUpdater
Functor "VelocityRule" : FieldUpdater
\end{lstlisting}

\noindent
\texttt{PositionRule} will instantiate \texttt{FieldUpdater} in the following evaluation rule:

\begin{lstlisting}
--------------------------------
PositionRule => FieldUpdater PhysicalBodyType "Position" {

  ---------------------
  GetRecord => PhysicalBodyType

  getPos body -> (xp,yp)
  getVel body -> (xv,yv)
  <<xp + xv * dt>> -> xp'
  <<yp + yv * dt>> -> yv'
  ---------------------------
  update body dt -> (xp',yp')
}
\end{lstlisting}

\noindent
Note that \texttt{getPos} and \texttt{getVel} are functions able to retrieve respectively the position and velocity from a physical body, analogously to what was done in Section \ref{sec:ch_functors_record_getter}. The \texttt{update} function uses these two functions in its premises to retrieve the value of the position and velocity and then updates the position according to the differential equation described in Section \ref{subsec:ch_functors_casanova_example}. The update for the velocity field is done analogously:

\begin{lstlisting}
--------------------------------
VelocityRule => FieldUpdater PhysicalBodyType "Velocity" {

  --------------------
  GetRecord => PhysicalBodyType

  getVel body -> (xv,yv)
  getAcc body -> (xa,ya)
  << xv + xa * dt >> -> xv'
  << yv + ya * dt >> -> yv'
  ---------------------------
  update body dt -> (xv',yv')
}
\end{lstlisting}

\noindent
We now have all the necessary tools to create the whole updater for a physical body. This updater is built by calling \texttt{UpdateTuple} to generate the updater for the tuple element representing the vector. This updater is used in all three field updaters for the physical body. We also use \texttt{PositionRule} and \texttt{VelocityRule} to create the correct updater for the two rules of the physical body.

\begin{lstlisting}
UpdateTuple FloatUpdater FloatUpdater => vectorUpdater
UpdateField vectorUpdater PhysicalBodyType "Position" => posUpdate  
UpdateField vectorUpdater PhysicalBodyType "Velocity" => velUpdate  
UpdateField vectorUpdater PhysicalBodyType "Acceleration" => accUpdate
NoUpdate PhysicalBodyType => zero
Update VelocityRule zero => velRule
Update PositionRule velRule => posRule
Update accUpdate posRule => accFieldUpdate
Update velUpdate accFieldUpdate => velFieldUpdate
Update posUpdate velFieldUpdate => bodyUpdater
--------------------------
BodyUpdater => bodyUpdater
\end{lstlisting}

\noindent
The first premise of this functor rule creates the updater for the vector. From premise 2 to premise 4 we create the updater for the three fields of the physical body. Premise 5 calls \texttt{NoUpdate} to build the module that terminates the update of the record. From Premise 6 on we build the record updaters necessary to update all the fields and rules of the physical body and then we assemble them together.
Let us now consider the following physical body instance

\begin{lstlisting}
(1.0,1.0),((0,0,0.0),((3.0,3.0),()))
\end{lstlisting}

\noindent
and let us see what happens when we call the \texttt{update} function of the \texttt{BodyUpdater}. The function will invoke the tuple updater which returns the tuple as it is, set the field to this value (which does not change), and recursively call \texttt{update} from the next updater. The following two updaters are the same, so the effect is identical. The updater for the position rule will instead run the \texttt{update} code of the module instance generated by the rule functor and update the field of the record accordingly. This will generate a record instance containing the field with the updated value. This new record instance is then recursively passed to the next \texttt{update} call where the \texttt{update} of the module instance generated by the rule functor for velocity is invoked. The updated record is then returned in an analogous way. At this point the \texttt{update} of the module instance generated by \texttt{NoUpdate} is called, which simply returns the record as it is.

We now repeat the same process to define the world entity. We thus define a functor to build the record for the world, which contains a single field that is a list of physical bodies.

\begin{lstlisting}
Functor "WorldType" : Record

RecordField "PhysicalBodies" List[PhysicalBodyType] EmptyRecord => world
---------------------------------
WorldType => world
\end{lstlisting}

\noindent
The updater for the world simply uses the \texttt{BodyUpdater} functor generated above to build a record instance that contains the update function for a physical body. It then builds a list updater passing as argument \texttt{BodyUpdater} (note that this is correct as this functor accepts a record updater as parameter).

\begin{lstlisting}
Functor "WorldUpdater" : RecordUpdater

UpdateEntity BodyUpdater => bodyUpdater
UpdateList bodyUpdater => listUpdater
UpdateField listUpdater WorldType "PhysicalBodies" => fieldUpdater
NoUpdate WorldType => zero
Update fieldUpdater zero => worldUpdater
--------------------------------------
WorldUpdater => worldUpdater
\end{lstlisting}

\noindent
The rule for the functor creates in its first premise an entity updater by passing the updater for the physical body. This updater allows to update each element in the list of physical bodies stored in the world entity. The second premise creates an updater for the whole list by passing the entity updater created at the previous step. This updater instantiates a module that is able to traverse the whole list and update each element by means of the entity updater. The third premise creates a field updater for \texttt{PhysicalBodies} by using the list updater, and the fourth creates as usual a \texttt{NoUpdate} to stop the update process. Finally, the last premise assembles the two field updaters into a record updater for the world entity. At this point, in order to update the world entity, it is enough to call this functor and access the \texttt{update} function for the world record.

We think it is worthy of note that all the updaters presented so far are built at compile time and that the only component that will be generated in the target code is the \texttt{update} function. This means that we get rid of all the dynamic lookups, described in Section \ref{subsec:ch_networking_update_metacasanova} in the entity field to inspect the type of the field itself and decide whether or not we require to perform the recursive update process on it. The update traversal with functors is instead generated at compile time, thus the structure of the update is pre-computed during the compilation step, and its execution delegated at runtime. This is possible because the structure of the update does not change with the execution of the program.

\section{Interruptible rules with functors}
\label{subsec:ch_networking_interruptible_rules}
With what shown so far, we can implement the update traversal of the fields of a Casanova entity and we can implement Casanova rules as updaters that act on the fields of an entity. However, we have not described yet how to implement the mechanism of rule interruption described in Section \ref{subsec:ch_mcnv_languages_rule_evaluation}. For this purpose, we have to refactor the implementation of the updaters seen so far: we assume that now the record field of a Casanova entity contains not only the value but a list of statements that represent the continuation of its rule, which represents the code left to execute after the rule is paused. The continuation will have type \texttt{stmt}, where \texttt{stmt} is a meta-data structure representing a statement in Casanova like shown in Section \ref{subsec:ch_mcnv_languages_rule_evaluation}. The reader should take into account that we can compose a sequence of statements through the \texttt{;} operator introduced in the same section. The field updater must be refactored as well: its update function now does not only return the updated field value but also the continuation of the rule:

\begin{lstlisting}
Module "FieldUpdater" => (r : Record) => (name : string) : FieldUpdater {
  Functor "GetRecord" : Record
  Func "update" -> r.RecordType -> float : Tuple[(GetFieldType r name),stmt]
}
\end{lstlisting}

\noindent
In this way we are correctly able to generate the declaration of the update function depending on the type of the field and, at the same time, to store the updated continuation of the rule. We now define a new functor called \texttt{Coroutine} that generates an instance of a field updater. The instantiation of the module should also contain a function \texttt{tick} that is able to correctly process the continuation of the rule and, when its body has been fully evaluated, to restart from the beginning. It should also contain a definition of the evaluation rules of all the Casanova statements introduced in Section \ref{subsec:ch_mcnv_languages_rule_evaluation}. For brevity here we show only how to re-implement \texttt{wait} and \texttt{yield}, all the others can be adjusted analogously to those. The following is the declaration of the Coroutine functor:

\begin{lstlisting}
Functor "Coroutine" => Record => string => stmt : FieldUpdater

----------------------------
Coroutine r name stmts => FieldUpdater r name {
  ... 
  // see the implementation below
}
\end{lstlisting}

\noindent
This functor takes the record and the name of the fields the rule is updating, as well as a list of statements that represents the body of the coroutine and produces a field updater enriched with the utility functions mentioned above (remember that a module instance must contain the implementation of at least all the declarations provided in the module declaration). From now on we provide the snippets of the implementations in the module in isolation, but the reader should keep in mind that they are defined within the scope of the module instance. 

The first function that we implement is \texttt{update}. This function is almost identical to the version described in Section \ref{subsec:ch_networking_simulation}, but this time the getter of the field will return both the value and the continuation of the rule built so far. We then call a \texttt{tick} function (see below) that is able to process the continuation of the rule. This function in general produces a pair containing the updated field value (when we encounter a \texttt{yield} statement) and the new continuation produced by the current execution of the rule. Note that the implementation of \texttt{update} is correctly able to return a pair because it has been redefined above in the new version of \texttt{FieldUpdater}.

\begin{lstlisting}
GetField r name => getter
getter.get body -> (v,k)
tick entity k dt -> (v',k')
-------------------------
update entity dt -> (v',k')
\end{lstlisting}

\noindent
The \texttt{tick} function takes a record instance as input, a list of statements, and \texttt{dt} and returns the pair of value and continuation produced by the evaluation of the rule body. The function calls \texttt{eval\tu s} that is similar to the homonym function presented in Section \ref{subsec:ch_mcnv_languages_rule_evaluation}, with the difference that it now returns a pair of value field and list of statements compatible with the required result.

\begin{lstlisting}
Func "tick" -> r.RecordType -> List[stmt] -> float : Tuple[r.RecordType,List[stmt]]
Func "eval_s" -> r.RecordType -> stmt -> float : Tuple[r.RecordType,stmt]


eval_s entity stmts dt -> res
--------------------------
tick entity nop dt -> res

eval_s entity statements dt -> (v,(atomic;k))
tick entity k dt -> res
-----------------------------------
tick entity statements dt -> res

eval_s entity statements dt -> res
-----------------------
tick entity statements dt -> res
\end{lstlisting}

\noindent
The function \texttt{tick} comes in three versions: the first one is executed when the rule has completed its execution and the body of the original rule should be rebuilt. In this case the function simply calls \texttt{eval\tu s} with the statements provided as argument of the functor \texttt{Coroutine}. The second one is when we evaluate an atomic statement: for this purpose we introduce a placeholder statement \texttt{atomic} that is returned in the continuation after an atomic statement has been evaluated. This case forces \texttt{tick} to be immediately re-evaluated without interrupting the rule execution. The third case happens when the rule evaluation has previously produced a continuation. In this case we pass the continuation instead of the original body of the rule to \texttt{eval\tu s}.

The function \texttt{eval\tu s} is very similar to its old counterpart, but this time it returns the pair of value and continuation resulting from the evaluation of the first statement in the current rule continuation. In the case of an empty continuation (the only statement is \texttt{nop}) then we return an empty continuation. The field of the value is unchanged so we use its getter to retrieve the value an return it in the result.

\begin{lstlisting}
GetField r name => getter
getter.get entity -> (v,cont)
-------------------------------
eval_s entity nop dt -> (v,nop)
\end{lstlisting}

We now proceed to describe how \texttt{wait} and \texttt{yield} behave. \texttt{wait} as usual simply checks whether the timer has elapsed. If that is the case, then it returns the continuation preceded by an \texttt{atomic} statement to force the immediate re-evaluation in \texttt{tick}. Otherwise it updates the timer by subtracting \texttt{dt} seconds and builds another \texttt{wait} statement that is placed in the continuation. In both cases the statement returns the current value of the field the rule is updating because it is untouched in the semantics of \texttt{wait}.

\begin{lstlisting}
t <= 0.0
GetField r name => getter
getter.get entity -> (v,cont)
---------------------------------------------
eval_s entity (wait t;k) dt => (v,(atomic;k))

t > 0.0
GetField r name => getter
getter.get entity -> (v,cont)
<<t - dt>> -> t'
---------------------------------------------
eval_s entity (wait t;k) dt => (v,(wait t';k))
\end{lstlisting}

\noindent
Note that the correct \texttt{getter} is generated at compile time, so the overhead of accessing the field value is minimal as shown in Section \ref{sec:ch_functors_evaluation}. The statement \texttt{when} behaves in the very same way, except that this time 


Finally \texttt{yield} simply evaluates the expression whose value is used to set the field and then returns it in the result of the evaluation. Note that we use the function \texttt{eval} already described in the first implementation of Casanova in Metacasanova. The behaviour of this function is exactly the same, except that now, if we need to retrieve the value of a specific field for the computation of the expression result, we can use the \texttt{GetField} functor to build the appropriate getter and thus improve the performance. Also note that the statement evaluation does not set the field itself, but as seen before it delegates this operation to the record updater. This is because the result of calling the setter on a record returns the updated record instance and not a value compatible with the field. Note also that the evaluation of \texttt{yield} does not produce \texttt{atomic} like for \texttt{wait} because according to Casanova semantics the \texttt{yield} stops the rule execution for one frame.

\begin{lstlisting}
eval entity expr -> v
--------------------------------
eval_s (yield expr;k) dt -> (v,k)
\end{lstlisting}

\noindent
It is worthy of note that, having placed the semantics of Casanova in a module instantiation, the language is able to build ad-hoc semantics for each specific field that we need to update through the rule. In other words, calling the coroutine functor with a specific field produces a different version of the language semantics at compile time, where the statements that need to access the value of the field contain directly the getter of that field generated at run-time. This allows us to incorporate the benefits of the record lookup optimization described in Chapter \ref{ch:functors} in the language semantics.

The final modification that we need to implement is on the record updater. The record updater now receives the pair of field value and rule continuation that must be stored in the field after its update. The updater uses the new generated pair to update both the field value and its rule continuation. This continuation will be used at the next update to evaluate the remaining part of the rule

\begin{lstlisting}
fieldUpdater.GetRecord => r
---------------------------
Update fieldUpdater nextUpdater => RecordUpdater r {

  r.RecordType => recordType
  ------------------------
  RecordType => recordType

  SetField r name => setter
  fieldUpdater.update rec dt -> (v,k)
  setter.set rec (v,k) -> rec'
  nextUpdater.update rec' dt -> updatedRecord
  ----------------------------
  update rec dt -> updatedRecord
}
\end{lstlisting}

\subsection{Multiple rules updating the Same Field and Local variables}
To conclude this section we want to point out that, in the implementation of interruptible rules described above, we implicitly make the assumption that only one rule is updating each field of the record. Indeed the rule continuation is saved in the field itself, thus if multiple rules are affecting the same field we would need to store their continuations separately, which is not possible in the current implementation. A naive approach would be to allow to store a list of statements, one for each rule acting on that field, where each element is the continuation of a specific rule. This approach affects the performance because we would need to iterate the whole list every time we need to update a rule. Since the number of rules updating a field does not change at run time, we can instead use a record to store their continuations whose structure is provided at compile time. In this way it will be possible to retrieve the continuation of a rule just by using a getter that is generated at compile time. Here we just briefly sketch the implementation. A schematic representation of the implementation can also be seen in Figure \ref{fig:ch_networking_interruptible_rules}.

\begin{figure}
  \centering
  \includegraphics[width=\textwidth]{Figures/chapter_networking/interruptible_rules}
  \caption{Schematic representation of the implementation of the interruptible rules}
  \label{fig:ch_networking_interruptible_rules}
\end{figure}

A field of the entity record must be adapted now to contain not only the field value, but a record instance used to store the continuations of the Casanova rules affecting that field. Since a record requires a name for each field, we can expand the coroutine functor to take a string representing an identifier for each rule and the continuation record itself:

\begin{lstlisting}
Functor "Coroutine" => string => Record => Record => string => stmt : FieldUpdater


---------------------------
Coroutine ruleId continuation r name => FieldUpdater r name {
  ...
}
\end{lstlisting}

\noindent
Now the first string in the declaration of the functor represents the rule identifier, while the other arguments have the same semantics (record and field of the record the rule can modify). When a Casanova statement requires to store the continuation it can use \texttt{ruleId} to build the setter for the record field of the continuation record. It then calls the function \texttt{set} from the setter module instance to save the continuation of each rule. In this way every rule acting on the record is able to store separately its continuation in the continuation record. As an example, we provide below the evaluation rule for the \texttt{wait} statement that updates the continuation in this implementation:

\begin{lstlisting}
t > 0.0
GetField r name => getter
getter.get entity -> (v,cont)
SetField cont ruleId => continuationSetter
continuationSetter.set (wait(t - dt);k) -> cont'
---------------------------------------------
eval_s entity (wait t;k) dt => (v,cont')
\end{lstlisting}

\noindent
Another aspect that has not been considered yet is how to define variables local to the rule (local bindings). Since the set of local bindings is known at compile time, we can modify the continuation record to store not only the continuation itself, but also the state of the local bindings as record of bindings. In this way an element of the continuation record, that we can now call rule state, stores not only the statements of the rule left to evaluate but also the state of the local bindings. When we need to read the value of a binding or update it, we can again use a getter or setter by accessing the rule state and getting or setting the appropriate field for the binding from the binding record. A schematic representation of this implementation can be seen in Figure \ref{fig:ch_networking_interruptible_rules_with_state}.

\begin{figure}
  \centering
  \includegraphics[width=\textwidth]{Figures/chapter_networking/interruptible_rules_with_state}
  \caption{Schematic representation of the implementation of interruptible rules with local bindings}
  \label{fig:ch_networking_interruptible_rules_with_state}
\end{figure}

As final remark, we point out that the use of records to store the rule continuations and local bindings show how the record optimization introduced in Chapter \ref{ch:functors} can also be adapted to implement a generic symbol table to store various information regarding the language elements that are needed during the execution of the generated code, thus making this approach extremely flexible for different situations.

\section{Evaluation}
\label{subsec:ch_functor_languages_evaluation}
In the previous sections we showed how to use functors to implement the entity update traversal of the domain-specific language for game development Casanova. Based on the preliminary analysis performed in Chapter \ref{ch:functors}, we claimed that using functors would improve the performance of the implementation of Casanova in Metacasanova given in Chapter \ref{ch:languages} by, at the same time, inlining the access to the entity fields and pre-building the traversal for the Casanova program at compile time, instead of dynamically accessing the fields from a dictionary and inspecting their type to perform the update traversal at every update. In this section we show the experimental results that show the performance of this implementation in comparison to the first dynamic implementation presented in Chapter \ref{ch:languages}.

\subsection{Experimental Setup}
For this evaluation we have implemented the physical body simulation that was presented in the previous sections. The simulation has been run for 10000 frames, which roughly correspond to 3 minutes assuming an average update rate of 60 frames/second, with a number of physical bodies ranging from 100 to 1000. Each physical body is randomly generated, that is, its initial position, velocity, and acceleration is randomly generated. We measured the time at the beginning and at the end of the execution of the whole simulation and we averaged the total time by the number of frames the simulation has been running for. We then compared the result with what obtained for the implementation shown in Chapter \ref{ch:languages}.

\subsection{Results}
In Table \ref{tab:ch_networking_evaluation} we can see that the update time is in the order of milliseconds or one tenth of milliseconds where the dynamic implementation was in the order of one hundredth of seconds with 1000 entities. This corresponds roughly to a frame rate of 939 frames/second for the functor implementation versus 28 frames/second. The performance gain ranges from a maximum of 55.397 to a minimum of 33.117 times with an avarage gain of 42.508 times. This comes at no surprise, since in Section \ref{sec:ch_functors_evaluation} we tested the gain of accessing record fields with the functor implementation compared to the dynamic tables, and we had an average gain of roughly 11 times. The gap with the dynamic implementation here is even greater because, to the cost of accessing dynamic tables at runtime to retrieve the values of the entity fields, we have to add the performance loss of performing the update traversal and the rule execution dynamically. Figure \ref{fig:ch_networking_chart} shows a chart where the horizontal axis represents the number of entities in the simulation, while the horizontal axis represents the average frame update time with that number of entities in seconds.

To conclude, we want to point out that this evaluation is a worst-case scenario, since the implementation shown in this Chapter makes use exclusively of Metacasanova meta-data structures to represent the values of the entity fields while the simulation shown in Chapter \ref{ch:languages} uses \texttt{Vector2} from the Monogame library. This means that this simulation has an additional overhead due to accessing the components of a tuple via pattern matching, and due to the use of value types versus reference types. The performance shown here could be improved by using \texttt{Vector2} from an external library instead of \texttt{Tuple[float, float]} to store the position, velocity, and acceleration of a physical body.
\begin{table}
  \resizebox{\textwidth}{!}{
  \begin{tabular}{|c|c|c|}
  \hline
  \textbf{Language implementation} & \textbf{Entity number} &	\textbf{Update time}\\
  \hline
  \multirow{5}{*}{Functors}
  & 100 &	0.000063\\
  & 250 &	0.000173\\
  & 500 &	0.000428\\
  & 750 &	0.000777\\
  & 1000 & 0.001065\\
  \hline
  \multirow{5}{*}{Dynamic}
  & 100 & 0.00349\\
  & 250 & 0.00911\\
  & 500 & 0.01716\\
  & 750 & 0.02597\\
  & 1000 & 0.03527\\
  \hline
  \end{tabular}}\\
  
  \vspace{0.5cm}
  \begin{tabular}{|c|c|}
  \hline
  \textbf{Entity number} & \textbf{Performance Gain}\\
  \hline
  100 &	55.397\\
  250 &	52.659\\
  500 &	40.093\\
  750 &	33.423\\
  1000 & 33.117\\
  \hline
  \textbf{Average gain} & 42.938 \\
  \hline
  \end{tabular}
 	\caption{Update time for one frame of the functor implementation of Casanova and the dynamic implementation shown in Chapter \ref{ch:languages}. The time is measured in seconds}
  \label{tab:ch_networking_evaluation}
\end{table}

\begin{figure}[!h]
  \centering
  \includegraphics[width=\textwidth]{Figures/chapter_networking/chart}
  \caption{Execution time of Casanova implemented with functors vs the dynamic implementation}
  \label{fig:ch_networking_chart}
\end{figure}

\section{Summary}
In this chapter we proposed a new implementation of the semantics of Casanova based on the language extension with functors and modules presented in Chapter \ref{ch:functors}. We showed that functors and modules are expressive enough to implement the logic of the entity update in Casanova and at the same time to allow rule interruption. At the same time, functors grant static typing and the inlining of ad-hoc update functions depending on the structure of the entity we need to update. This improvement increases the performance of this new implementation of Casanova on average by roughly 42 times. This improvement makes the implementation of Casanova suitable for game development, as the generated code is now able to process at more than 900 frames/second versus the 28 of the previous implementation.


\chapter{Networking in Casanova}
\label{ch:networking}
\epigraph{The Internet is not just one thing, it's a collection of things - of numerous communications networks that all speak the same digital language.}{Jim Clark}
In this section we introduce the basic concepts of the implementation of multiplayer game development for Casanova 2. This implementation aims to relieve the programmer of the complexity of hard-coding the network implementation for an online game, while preserving encapsulation in code. We show that code analysis is required to generate the appropriate network primitives to send and receive data. Finally, we present a simple multiplayer game to show a concrete example.

\section{Introduction}
Adding multi-player support to games is a highly desirable feature. By letting players interact with each other, new forms of gameplay, cooperation, and competition emerge without requiring any additional design of game mechanics \cite{granberg2014david}. This allows a game to remain fresh and playable, even after the single player content has been exhausted. For example, consider any modern AAA (AAA refers to games with the highest development budgets\cite{wolf2008video}) game such as \textit{Halo 4}. After months since its initial release, most players have exhausted the single player, narrative-driven campaign. Nevertheless the game remains heavily in use thanks to multiplayer modes, which in effect extended the life of the game significantly. This phenomenon is even more evident in games such as \textit{World of Warcraft} or \textit{EVE}, where multiplayer is the only modality of play and there is no single-player experience.

\paragraph{Challenges}
Multi-player support in games is a very expensive piece of software to build. Multiplayer games are under strong pressure to have very good \textit{performance} \cite{claypool2006latency}. Performance is both expressed in terms of CPU time and in bandwidth used. Also, games need to be very \textit{robust} with respect to transmission delays, packets lost, or even clients disconnected. To make matters worse, players often behave erratically. It is widespread practice among players to leave a competitive game as soon as their defeat is apparent (a phenomenon so common to even have its own name: ``rage quitting'' \cite{rage_quitting}), or to try to abuse the game and its technical flaws to gain advantages or to disrupt the experience of others.

Networking code reuse is quite low across titles and projects. This comes from the fact that the requirements of every game vary significantly: from turn-based games that only need to synchronize the game world every few seconds, and where latency is not a big issue, to first-person-shooter games where prediction mechanisms are needed to ensure the smooth movement of synchronized entities, to real-time strategy games where thousands of units on the screen all need to be synchronized across game instances \cite{smed2002aspects}. In short, previous effort is substantially inaccessible for new titles. 

Encapsulation suffers from this ad-hoc nature of the implementation of the networking layer in multiplayer games. Indeed managing the information about game updates over a network requires each game entity to interface the game logic code with network connection and socket objects, data transmission method calls such as send and receive, and support data structures to manage traffic and track the status of common protocols. This happens because each game entity must provide the following functionality in order to work in a multiplayer game:

\begin{itemize}
	\item Update the logic in the fashion of a singleplayer counterpart.
	\item Choose what data is necessary to send over the network and create the message containing this information.
	\item Choose what data can be lost and what data must always be received by the other clients.
	\item Periodically check if incoming messages contain information that needs to be read and to perform specific updates.
\end{itemize}

Combining these requirements together within the same entity breaks encapsulation because now the logic of the entity and lots of spurious details only relevant to the networking implementation are mixed together, resulting in a highly noisy program. Maintenance then becomes very hard, as every change in the game logic must also be reflected in the networking implementation.

\paragraph{Existing approaches}
Networking in games is usually built with either very low-level or very high-level mechanisms. Very low-level mechanisms are based on manually sending streams of bytes and serializing only the essential bits of the game world, usually incrementally, on unreliable channels (UDP). This coding process is highly expensive because building by hand such a low-level protocol is difficult to get right, and debugging subtle protocol mismatches, transmission errors, etc. will take lots of development resources. Low-level mechanisms must also be very robust, making the task even harder.

High-level protocols such as RDP, reflection-based serialization, frameworks (such as Pastry, netty.io), etc. can also be used. These methods greatly simplify networking code, but are rarely used in complex games and scenarios. The requirements of performance mean that many high-level protocols or mechanisms are insufficient, either because they are too slow computationally (especially when they rely on reflection or events) or because they transmit too much data across the network.

\section{Motivation}

To avoid the problems of both existing approaches, we propose a middle ground. We observe that networking fundamental abstractions upon which the actual code and protocols are built do not vary substantially between games, even though the code that needs to be written to implement them does. The similarity comes from the fact that the ways to serialize, synchronize, and predict the behaviour of entities are relatively standard and described according to a limited series of general ideas. The difference, on the other hand, comes from the fact that low-level protocols need to be adapted to the specific structure of the game world and the data structures that make it up. Until now, common primitives have not been syntactically and semantically captured inside existing domain-specific languages for game development \cite{bhatti2009domain}. Using the right level of abstraction, these general patterns of networking can be captured, while leaving full customization power in the hand of the developer (to apply such primitives to any kind of game).

\section{Related work}
In the following we discuss some existing networking tools used in game development and we highlight some issues that arise from their use.

\paragraph{The Real time framework (RTF)} RTF \cite{glinka2007rtf} is a middleware built for C++ to relieve the programmer from dealing with data compression. It is more flexible than solutions based on game engines or hand-made implementations, since it automates the process of data transmission. Moreover, it supports distributed server management. Unfortunately, this solution has several flaws:
\begin{itemize}
	\item All entities must inherit from the class \texttt{Local} and the semantics of the position is pre-determined, often clashing with rendering or physics.
	\item Platform independence requires that the programmer uses RTF primitive	types.
	\item Data transmission automation requires that all game entities inherit the class \texttt{Serializable}.
	\item Being a middleware, RTF is not aware of what games are going to use it for (every game comes with different data structures). Thus, the developer is tasked to include in his code also logic to update the RTF layer, in order to keep the game updated over the network.
\end{itemize}


\paragraph{Network scripting language (NSL)} NSL \cite{russell2008tackling} provides a language extension based on a send-receive mechanism. Moreover it provides a built-in client side prediction (a feature missing in existing highly concurrent and distributed languages such as Stackless Python \cite{kalogirou2005multithreaded} and Erlang \cite{armstrong1993concurrent}), which is periodically corrected by the server. 

\paragraph{Unreal Engine/Unity Engine} Unreal Engine \cite{games2006unreal} and Unity Engine \cite{engine9unity} are commercial game engines supporting networking.  Both Unity and Unreal Engine use a client-server approach. In Unreal Engine, the server contains the ``true'' game state, and the clients contain a ``dirty'' copy, which is validated periodically. It is possible to define entities (actors in Unreal Engine jargon) that are replicated on the clients. Whenever a replicated actor changes on the server, this change is also reflected on the clients. Additional customization can be achieved through Remote procedure calls (RPCs) of three kinds.
\begin{itemize}
	\item The function is called on the server and executed on the client. This is used for game elements that do not affect gameplay, such as creating a particle effect when a weapon is fired.
	\item The function is called on the client and executed on the server. This is useful for events that affect the other clients and should be validated by the server.
	\item The function is executed in multi-cast, meaning that the server calls the function and that it is executed on both the server and all the clients.
\end{itemize}

The Unity Engine uses a similar approach based on networking components, synchronized at every frame, and RPC's to define custom synchronization events.

Unfortunately, customization comes at the cost of the level of detail that developers must face. Using RPC's require a deep knowledge of the engine and writing lots of code, as discussed in Section \ref{Common issues}.

In this section we introduce a small example that addresses the requirements of designing a multiplayer game. We then present an architecture that aims to fulfil these requirements.

\section{The master/slave network architecture}

We chose to implement the networking layer in Casanova 2 by using a peer-to-peer architecture for the following reasons:

\begin{itemize}
	\item Server-client architectures are more reliable but suitable only for specific genres of games (mostly Shooter games), while other genres, such as Real-time strategy games or Online Role Playing Games, use P2P architectures.
	\item We do not have to write a separate logic for an authoritative game server, which has to validate the actions of clients.
\end{itemize}

Casanova will provide a generic tracking server, which is run separately from the main program. The tracking server is a thin service that connects players participating in a single game, and helps with forwarding the network traffic through NATs (Network Address Translation).

Each client maintains a local copy of the \texttt{world} entity and has direct control over a single portion of it. Instances belonging to such as portion are seen as \textit{master} by this player, who is always allowed to directly change the state of the master instances without having to validate this state change by synchronizing with other players through the network.

Each client also maintains a portion of the world that is not directly under his control. Instances belonging to such as portion are seen as \textit{slave} by this player, who is only allowed to \textit{predict} the local state of the instances and, whenever he receives an update from their masters, must correct this prediction according to the data contained in the received messages. The slave part of the world is thus maintained passively by the client: the only active part is predicting the evolution of the entity state and correcting it whenever he receives an update by its master.

For this purpose, we extend the syntax of Casanova rules by allowing them to be marked with the modifiers \texttt{master} and \texttt{slave}. These rules are executed respectively on master and slave entities. Note that it is still possible not to mark a rule with these modifiers, which means that the rule is always executed independently of the fact that the entity is either master or slave on that particular client. We also allow to mark a rule as \texttt{connecting} and \texttt{connected}. These rules are triggered only once respectively when a new client connects and when the clients detect a new connection.

Casanova also provides primitives to send (reliably or unreliably) and receive data. A schematic representation of this architecture can be seen in Figure \ref{fig:masterslave}.

\begin{figure}[h!]
	\centering
	\caption{Representation of the game world in a networking scenario}
	\label{fig:network_world}
	\begin{subfigure}[t]{0.3\linewidth}
		\centering
		\includegraphics[width=1\linewidth]{Figures/networking2}
		\caption{Unknown correct game state when P3 joins the game.\\}
		\label{subfig:networking_ideal}
	\end{subfigure}
	\begin{subfigure}[t]{0.3\linewidth}
		\centering
		\includegraphics[width=1\linewidth]{Figures/networking1}
		\caption{Networking game state seen from the point of view of P1. P2 is partially synchronized, P4 is fully synchronized, and P3 is a new client that is late and is still sending its data}
		\label{subfig:networking_relative}
	\end{subfigure}
	
	
\end{figure}

\begin{figure}
	\centering
	\includegraphics[width = \textwidth]{Figures/masterslave}
	\caption{master/slave architecture}
	\label{fig:masterslave}
\end{figure}

Note the aim of this architecture is to provide language-level primitives to describe the networking logic. This means that the compiler will be able to generate code compatible with the low-level network libraries that provide transmission functions over the network channel without having to change Casanova code in the program. In our implementation, we chose the .NET library \texttt{Lidgren}, which is widely used also in commercial game engines such as Unity3D and MonoGame, but nothing prevents the compiler to be expanded in order to target other similar libraries for other languages, such as jgroups \cite{ban2002jgroups}.

\section{Case study}
Let us consider a simple shooter game where each player controls a space ship. Players can move forward, backward, and rotate the ship to change direction. Moreover, they can use the ship lasers to shoot other players. If a laser hits an enemy ship, we increase the player's score. Designing such a game requires to address the following issues, depicted by the schematic representation in Figure \ref{fig:network_world}:

\begin{enumerate}
	\item Each player must maintain a local version of the game state (world). In order to avoid to flood the network with messages, all the copies are not fully synchronized at each frame, thus they are slightly different and each client knows the latest version of only part of the copy.
	\item A player \texttt{connecting} to an existing game must be able to receive the latest update of the game state and send the new ship he will control to existing players in the game.
	\item A player already \texttt{connected} to the game must detect a new connection and send his master portion of the game state.
	\item Each player must be able to control only one ship at a time. This means that the part of the game logic that processes the input and modifies the spatial data of the ship (position and rotation) should only be executed on the ship controlled by the player and not on the local copies of other players' ships. This means that each player sees as \texttt{master} only one ship instance.
	\item Each player must send the updated state of the ship he controls to the other players after executing the local update. To achieve better performance over the network, the data is not sent at every update, but with a lower frequency.
	\item Each player must receive the updated state of \texttt{slave} ships controlled by other players. In this phase, we must take into account that, as explained above, not every update is sent, so the player should ``predict'' what will happen during the game frames in which he does not receive an update.
\end{enumerate}

\section{Implementation}
Each of the scenarios described above requires specific language extensions. These extensions identify connection, ownership (master/slave), and various send and receive primitives. In this section, we introduce each primitive by using a multiplayer game example \footnote{The game source code and executable can be found at \url{https://github.com/vs-team/casanova-mk2/wiki/Networking-extension}}. We now give an implementation of the shooter game presented above, using the extended version of Casanova 2 with network primitives. 

The \texttt{world} contains a list of ships controlled by each player.
\begin{lstlisting}
world Shooter = {
Ships  : [Ship]
...
}
\end{lstlisting}

Each \texttt{Ship} contains a position, a rotation, a collection of shot projectiles, and the score.
\begin{lstlisting}
entity Ship = {
Position   : Vector2
Rotation   : float32
Projectiles : [Projectile]
Score		  : int
...
}

\end{lstlisting}

Each \texttt{Projectile} contains its position and velocity.

\begin{lstlisting}
entity Projectile = {
Position : Vector2
Velocity : Vector2
...
}
\end{lstlisting}

\subsubsection{Connection}
When a player connects, we must consider two different situations: (\textit{i}) a player is already in the game and must send the current game state to the connecting players, and (\textit{ii}) the player who is connecting needs to send the ship he will instantiate and control (its initial state). Both the players in the game and the connecting one must receive the game states that are sent. For this purpose we introduce two additional modifiers, \texttt{connecting} and \texttt{connected}, that can be added to rule declarations to mark their role in the multiplayer logic.

\paragraph{Connecting} A rule marked with \texttt{connecting} is executed once when a player joins the game for the first time. In our example, the player should send his initial state (the created ship) to the other players. We use the primitive \texttt{send\_reliable} because we must be sure that eventually all players will be notified of the ship creation.
\begin{lstlisting}
world Shooter = {
...
rule connecting Ships =
yield send_reliable Ships
}
\end{lstlisting}

\paragraph{Connected} A rule marked with \texttt{connected} is run whenever a new player joins the game by all existing players. When this occurs, each player sends its ship. The system will take care to send only the ship controlled locally by the player itself for each player. The rule will use the \texttt{send\_reliable} primitive for the same reason explained in the previous point.

\begin{lstlisting}
world Shooter = {
...
rule connected Ships =
yield send_reliable Ships
}
\end{lstlisting}

Note that even if the code is the same, the semantics of the two rules are different. The first one is executed by the player joining the game, who locally instantiates its \texttt{Ship} and must send its list of \texttt{Ships} (containing only the local instance) to the other players. The second one is executed by all existing players who must share with the joining player the list of existing ships.


\subsubsection{Master updates}
As explained above, each client manages a series of local game objects (called \textit{master objects}) that are under its direct control. The other clients read passively any update done on those instances and update their remote copy  (\textit{slave objects}) accordingly. We mark rules affecting the behaviour of master objects as \texttt{master}. In our example, the following situations are run as master: (\textit{i}) synchronizing the ships among players, (\textit{ii}) updating the ship and projectiles spatial data, and (\textit{iii}) creating and destroying projectiles.

\begin{enumerate}
	\item Each player is tasked to maintain the list of Ships in the world. This requires to receive the updated list from other players and to store the new value in a master rule. Indeed the world is a special case of an entity that is shared among players, and not directly owned by somebody. Each ship contained in that list and received from other players will be treated appropriately as slaves, while the only one owned by the current player will be under his direct control. In this rule we use \texttt{let!}, which is an operator that waits until the argument expression returns a result and then binds it to the variable. The symbol \texttt{@} stands for list concatenation. The rule uses \texttt{receive\_many}, which receives and collects the list of sent ships by the other players.
	
	\begin{lstlisting}
	world Shooter = {
	...
	rule master Ships =
	let! ships = receive_many()
	yield Ships @ ships
	}
	\end{lstlisting}
	
	\item The master version of the ship update reads the input of the player and moves (or rotates) the ship if the appropriate key is pressed. Note that this part must be executed only on a master object, because we want to allow each player to control only the ship he owns and instantiates at the beginning of the game. Below we show just the rule to move forward; the other movement and rotation rules are analogous. We use an \textit{unreliable send} because it is acceptable to lose an update of the position during a certain frame: shortly after, there will be a new update.
	
	\begin{lstlisting}
	entity Ship = {
	...
	rule master Position =
	wait world.Input.IsKeyDown(Keys.W)
	let vp = new Vector2(Math.Cos(Rotation), 
	Math.Sin(Rotation)) * 300.0f
	let p = Position + vp * dt
	yield send p
	}
	\end{lstlisting}
	
	We do the same for projectiles, except the projectile position is continuously updated and synchronized over the network without having to wait that a key is pressed.
	
	\item Creating a new projectile happens when the player shoots. A ship keeps track of the projectiles it has shot so far, and adds a new one to the list of the existing projectiles. The updated list is sent to all players with the new instance of the projectile (which is added as a new head of the list with the operator \texttt{::}). Here it is better to precise the semantics of the \texttt{yield} in conjunction with the use of networking primitives. A \texttt{yield} requires that the written value is type-compatible with the domain of the rule. Thus, when used with a \texttt{send} primitive, we must pass as argument a list. The system will ensure, for performance reasons, that the generated code only sends the new items added to the list. This semantics is defined as such for two main reasons: (\textit{i}) when sending the new projectiles we must also update the list in local (and given the immutability of Casanova we must replace the existing one), and (\textit{ii}) because in this way the programmer can focus on the logic of the game as if it were a single-player game without worrying of network-specific details. Note that the last \texttt{wait} forces the player to release the key before shooting again (semi-automatic fire). Removing that check would spawn multiple projectiles consecutively, which is not a wanted behaviour.
	
	\begin{lstlisting}
	entity Ship = {
	...
	rule master Projectiles =
	wait world.Input.IsKeyDown(Keys.Space)
	let vp = new Vector2(Math.Cos(Rotation), 
	Math.Sin(Rotation)) * 500.0f
	let projs = new Projectile(Position, vp) :: Projectiles
	yield send_reliable projs
	wait not world.Input.IsKeyDown(Keys.Space)
	}
	\end{lstlisting}
	
	Filtering the colliding projectiles and updating the score is run as a master rule. The rule computes the set difference between the ship projectiles and the colliding projectiles and updates the list of projectiles, sending them through the network as well. Even in this case, the network layer sends only the information about the projectiles to remove. Note that the score is managed by each player locally, as it does not require to be synchronized (we do not print the other players' scores. Doing so would indeed require to also send the score).
	
	\begin{lstlisting}
	entity Ship = {
	...
	rule master Projectiles, Score =
	let collidingProjs =
	[for p in Projectiles do
	let ships =
	[for s in Ships do
	where 
	s <> this and 
	Vector2.Distance(p.Position,s.Position) < 100.0f
	select s]
	where ships.Count > 0
	select p]
	let newProjectiles = Projectiles - collidingProjs
	yield send_reliable newProjectiles, 
	Score + collidingProjs.Count 
	}
	\end{lstlisting}
\end{enumerate}

\subsubsection{Managing remote instances}
The game objects that were not instantiated by a client, but received from another client, are \textit{slave objects} and must be synchronized differently than master objects. For this purpose, a rule can be marked as \texttt{slave}. In our example, we use slave rules in the following situations: (\textit{i}) synchronizing other players' ships and projectiles spatial data, and (\textit{ii}) projectiles instantiated by other players.

\begin{enumerate}
	\item Every remote projectile and ship is synchronized locally by a rule, which tries to \texttt{receive} a message containing updated spatial data. Below we provide the code to update the position of the ship; the synchronization of other spatial data is analogous.
	
	\begin{lstlisting}
	entity Ship = {
	...
	rule slave Position = yield receive()
	}
	\end{lstlisting}
	
	\item When a projectile is instantiated remotely, we have to receive it and add it to the list of projectiles. We use \texttt{receive\_many} because the new projectiles are added to a list. This case also supports the situation where a ship could shoot multiple projectiles at the same time.
	
	\begin{lstlisting}
	entity Ship = {
	...
	rule slave Projectiles =
	let! projs = receive_many()
	yield projs @ Projectiles
	}
	\end{lstlisting}
\end{enumerate}

In this scenario is important to discuss the atomicity of these transmissions: in the context of network games, reliability is often sacrificed for better network performance, so most of the data transmissions are unreliable (like in the case of the ship position). This means that we have no guarantee that the message will be received. Several issues can arise from this situation: for example, if a player fails to receive the position of the ship, then it might miss a collision with a projectile. This is a well-known issue in several shooter games and out-of-sync errors might happen during a multiplayer game. However, ensuring that all the data transmissions are reliable might affect network performance to the point that the game would become unplayable because of the network overload. 

Casanova 2 allows the programmer to decide whether the transmission should be reliable or not and experiment with the effect of a reliable transmission versus an unreliable one that does not overload the network. For example, the updated list of projectiles, after a collision, is always sent in a reliable way. This is acceptable because collisions are not so frequent. This is not true for the ship position, since movements are very frequent and mostly happen at every frame, thus it is something that should not be sent reliably at every frame.

Furthermore, we want to focus the attention on the implicit relationship between this networking architecture and the encapsulation: as shown for instance in the examples where the ship shoots a projectile, we ensure encapsulation by keeping a semantics that filters completely the details about networking. The programmer only worries about the logic of adding a new projectile, while the details of the network transmission are hidden. A hand-made implementation is usually prone to break this separation of concerns because the transmission logic is tightly coupled within the game logic itself.

\chapter{Discussion and Conclusion}
\label{ch:discussion}
This chapter provides an answer to the problem statement and research questions presented in Section \ref{sec:ch1_problem_statement}. The goal of the first research question is measuring the benefits of using a Metacompiler in terms of development speed when used to implement a domain-specific language for game development with respect to the implementation measured in code length. The goal of the second research question is aimed to determine the trade-off between a manual implementation of the language and an implementation with Metacasanova. The goal of the third research question is to identify reasons for this trade-off and propose an optimization to reduce it. The last part of this chapter answers the problem statement, provides an overview of future work and adds final remarks for this thesis.

\section{Answer to research questions}
\label{sec:ch_conclusion_answer_research_questions}
The three research questions stated in Section \ref{sec:ch1_problem_statement} are now answered  in Sections \ref{subsec:ch_conclusion_rq1}, \ref{subsec:ch_conclusion_rq2}, and \ref{subsec:ch_conclusion_rq3} respectively.

\subsection{Ease of development}
\label{subsec:ch_conclusion_rq1}

The first research question reads:\\

\researchQuestion{To what extent can a meta-compiler reduce the amount of code required to create a compiler for a given programming language?}\\

The answer to this research question is derived from the results shown in Chapter \ref{ch:languages}: in Section \ref{sec:ch_mcnv_languages_evaluation} we showed how the use of Metacasanova reduces the effort in term of code writing for the compiler of Casanova as the code required for the definition of the language semantics is roughly 5 times shorter in Metacasanova than the hard-coded version of the compiler written in F\#. This improvement is due to the fact that, in Metacasanova, it is possible to express the semantics of the language by mimicking almost directly the definition of Casanova written in natural semantics. 

\subsection{Performance trade-off}
\label{subsec:ch_conclusion_rq2}

\researchQuestion{How much is the performance loss introduced by the meta-compiler with respect to an implementation written in a language compiled with a traditional compiler and is this loss acceptable when considering game development?}

\subsection{Optimization}
\label{subsec:ch_conclusion_rq3}

\researchQuestion{What is the cause of the performance degradation when employing a meta-compiler and how can this be improved?}


\appendix
\chapter{List Operations with Templates}
\label{app:template}
In this appendix we will show in detail some operations on lists built on top of what presented in Section \ref{sec:ch_background_template_metaprogramming} that can be built by using meta-programming in C++ templates. The goal of this appendix is to convince the reader about the level of complexity of using C++ templates to express meta-programming and why it is preferable to use a dedicated meta-compiler.

\section{Element Getter}
\label{sec:app_templates_getter}
Accessing the n-th element of a list defined with templates mimics the behaviour of the its definition in a functional programming languages given below:

\begin{lstlisting}
let rec nth (n : int) (l : List<'a>) : 'a =
  match l,n with
  | x :: xs,0 -> x
  | x :: xs,_ -> nth (n - 1) xs
\end{lstlisting}

\noindent
The recursion base case is when the index we want to access is 0, which means that we want to access the head of the list. In this case we simply return the head by decomposing the list through pattern matching. In the other case we simply make a recursive call by passing the index decreased by 1 and the tail of the list. In template meta-programming, this is translated into a template that performs the same task:

\begin{lstlisting}
template <typename List> struct Nth<LST, 0> 
{
    typedef typename List::Head result;
};
\end{lstlisting}

\noindent
As shown in Section \ref{sec:ch_background_template_metaprogramming}, the arguments of the function are passed as arguments of the template itself. This version of the template is specialized for the integer 0, which corresponds to the base case of the recursion. The general case of the recursion has a dedicated template as follows:

\begin{lstlisting}
template <typename List, int N> struct Nth 
{
    typedef typename List::Tail Tail;
    typedef typename Nth<Tail, N - 1>::result result;
};
\end{lstlisting}

\noindent
The template contains a type definition for the parameter corresponding to the list tail and another type definition corresponding to the recursive call to another \texttt{Nth} template, this time containing only the tail of the list and the counter decreased by 1. To test this we can use the following sample:

\begin{lstlisting}
template <int N> struct Int 
{
  static const int result = N;
};

typedef List<Int<1>, List<Int<2>, List<Int<3>>>> testList;

int main()
{
  cout << Nth<testList, 2>::result::result << endl;
}
\end{lstlisting}

\noindent
Note that we need to access \texttt{result} twice, because the first \texttt{result} is the type of the head of the list generated by template, which is \texttt{Int}. So calling 

\begin{lstlisting}
Nth<testList, 2>::result
\end{lstlisting}

\noindent
returns \texttt{Int}, that is a type. If we want to access the value stored in \texttt{Int} then we must access the constant integer \texttt{result} contained in it. Note that if we try to access an invalid index in the list, the compiler will complain because it will try to generate a template with the tail of a list that does not exist. In this way something that in a normal program becomes a runtime error is here treated as a compilation error.

\section{Element Existence}
\label{sec:app_templates_existance}
The code that tests the existence of an element within a list is recursive as well and mimics the behaviour of its functional counterpart:

\begin{lstlisting}
let exists (element: 'a) (l : List<'a>) : 'a =
  match l with
  | [] -> false
  | x :: xs when element = x -> true
  | x :: xs -> exists element xs
\end{lstlisting}

\noindent
The function returns \texttt{false} as a base case when the list is empty, because it means that the whole list has been examined and the element has not been found. The second case is when the head of the list matches the element, which returns \texttt{true}. The last case is used when the head of the list does not match the element, thus we call recursively \texttt{exists} on the tail. In order to implement this function with C++ templates, we need to define two utility templates able to compare two elements:

\begin{lstlisting}
template <class X, class Y> struct Eq { static const bool result = false; };
template <class X> struct Eq<X, X> { static const bool result = true; };
\end{lstlisting}

\noindent
The first template has a result set to \texttt{false} when its arguments are different, while the second template is a specialization of the first one where both the first template argument and the second are the same and its result is \texttt{true}. With this utility templates we can correctly compare the values of a list defined with templates and define the recursive template for the existence function:

\begin{lstlisting}
template <class Element, class List> struct Exists
{
  static const bool result = 
    Eq<Element, typename List::Head>::result || Exists<Element, typename List::Tail>::result;
};

template <class Element> struct Exists<Element, NIL>
{
  static const bool result = false;
};
\end{lstlisting}

\noindent
The first template is the general case of the recursion. It uses \texttt{Eq} to test the value of the searched element against the head of the list. It then combines this result with the logical or on \texttt{Exists} run with the remaining tail of the list. The second template is the base case and contains a constant set to \texttt{false}. This corresponds to the base case of the recursive function above.

\chapter{Metacasanova Grammar in BNF}
\label{app:metacasanova_grammar}
In this section we provide the grammar of Metacasanova in Backus-Naur Form \cite{knuth1964backus}. For brevity we provide only the grammar productions and not the tokens. Note that this version includes the language extension described in Chapter \ref{ch:functors}.

\begin{lstlisting}

\end{lstlisting}

\backmatter

\bibliographystyle{plain}
\bibliography{references}

\end{document}