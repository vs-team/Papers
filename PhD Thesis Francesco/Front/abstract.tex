\chapter*{Abstract}
\thispagestyle{empty}

Programming languages are at the foundation of computer science, as
they provide abstractions that allow the expression of the logic of a program independent
from the underlying hardware architecture. In particular scenarios,
it can be convenient to employ Domain-Specific Languages, which are
capable of providing an even higher level of abstraction to solve problems
which are common in specific domains. Examples of such domains are database programming, text editing,
3D graphics, and game development. The use of a domain-specific
language for the development of particular classes of software
may drastically increase the development speed and the maintainability
of the code, in comparison with the use of a general-purpose
programming language. While the idea of having a domain-specific language for a particular domain
may be appealing, implementing such a language tends to come at a heavy cost: as it is common to all programming
languages, domain-specific languages require a compiler which translates their
programs into executable code. Implementing
a compiler tends to be an expensive and time-consuming task, which may very well be a burden
which overshadows the advantages of having a domain-specific language.

To ease the process of developing
compilers, a special class of compilers called ``meta-compilers'' has been
created. Meta-compilers have the advantage of requiring only the definition of a language in order to
generate executable code for a program written in that language, thus
skipping the arduous task of writing a hard-coded compiler for the new language. A disadvantage of
meta-compilers is that they tend to generate slow executables, so they are usually only employed
for rapid prototyping of a new language. The main aim of this thesis is to create a meta-compiler
which does not suffer from the disadvantage of inefficiency. It presents a meta-compiler called
``Metacasanova'', which eases the development cost of a compiler while simultaneously
generating efficient executable code.

The thesis starts by analysing the recurring
patterns of implementing a compiler, to define a series of requirements for Metacasanova.
It then explains the architecture of the meta-compiler and provides examples of its
usage by implementing a small imperative language called C-{}-, followed by the
reimplementation of a particular, existing domain-specific language, namely Casanova,
which has been created for use in game development. The thesis presents a novel way to
optimize the performance of generated code by means of functors; it demonstrates the
effect of this optimization by comparing the efficiency of Casanova code generated with and without it.
Finally, the thesis demonstrates the advantages of having a meta-compiler like Metacasanova,
by using Metacasanova to extend the semantics of Casanova to allow the definition of multiplayer online games.