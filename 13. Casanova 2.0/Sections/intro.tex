Computer and video games are a big business, and they have grown to the point that their sales are higher than those of music and movies \cite{ESAreport}. For mobile games alone, predicted sales for 2017 exceed 100 billions of dollars \cite{GLOBAL:GAMES:INVESTMENT:REVIEW:2014} . The relevance of games as a social phenomenon is at an all-times high. %  \cite{huizinga1949homo}. PS: You cannot cite a work from 1949 to refer to a recent phenomenon!

In the wake of this widespread adoption of games, their application areas have expanded. They are no longer used exclusively for entertainment. They have found applications in education, training, research, social interaction, and even raising awareness  \cite{bogost2007persuasive,seriousgameslist}. These so-called \textit{serious games} are used in schools, hospitals, industries, and by the military and the government. Researchers use games to simulate environments and evaluate their results. These games do not enjoy the same rich market of entertainment games, but their social impact is nevertheless high \cite{michael2005serious}.

An obstacle to the widespread use of serious games is that they are expensive to build. Since the resources of those who want to use serious games are usually quite limited, many serious games (and game-based research projects) fall short of their technological mark or fail altogether. This constitutes a high threshold that may cause innovative projects to be shut down prematurely.

The expenses of building a game are tightly related to the complexity of the structure of the game itself. At its core, a game features a state which describes the game world, plus the game loop that describes how the state changes over time. To avoid duplication of substantial effort in these areas, the game industry often uses ready-made components (also called ``engines'') \cite{gregory2009game}, for example for graphics and physics, which \textit{encapsulate} functionality in a modular way. As the number of components increases, complex concurrent \cite{bilas2002data} interactions between the components need to be realized. This yields additional difficulty.

%\subsection{Problem statement}
Our main goal in this paper is to reduce the complexity of game code. We introduce \textit{proper abstractions for game development, and build programming tools that implement those abstractions}. These abstractions would help developers in reaching their goals by substantially reducing development efforts, with a special benefit for smaller development teams that work on serious games. Since games are interactive applications, the proposed abstractions should not compromise the performance of the programs.


%\subsection{Structure of the paper}
We begin with a discussion of games and their complexity, introducing a case study. We use our case study to identify issues in the way games are traditionally expressed (Section \ref{sec:problem statement}). We propose a tiny, concurrency-oriented, game-centered language (Casanova 2) for describing game logic, and show how the case study is expressed in this language (Section \ref{sec:casanova 2}). We then evaluate the effectiveness of Casanova 2 for creating games in terms of performance and simplicity (Section \ref{sec:evaluation}). Sections \ref{sec:future_work} and \ref{sec:conclusions} cover future work and conclusions. 