In this section we introduce a short example which we will use to explain the problem of encapsulation in games. We then discuss the advantages and disadvantages of using encapsulation when designing a game.

\subsection{Running example}

Let us consider a game consisting of a set of planets linked together by routes. In what follows we call \textit{frame} a single update cycle of the game data structures. The player can move fleets from its planets to attack and conquer enemy planets. Fleets reach other planets by using the provided routes. Whenever a fleet gets close enough to an enemy planet it starts fighting the defending fleets orbiting around the planet. In our example we will assume that a \texttt{Route} is represented by a data structure containing (\textit{i}) the start and end point as references to \texttt{Planets}, and (\textit{ii}) a list of \texttt{Fleets} travelling through the route. The \texttt{Planet} is represented by a data structure containing (\textit{i}) a list of defending \texttt{Fleets}, (\textit{ii}) a list of attacking \texttt{Fleets}, and (\textit{iii}) an \texttt{Owner}. We also assume that each fleet has an owner as well. Each data structure contains a method called \texttt{Update} which updates the state of its class at every frame. Furthermore, we assume that all the game objects have direct access to the global game state which contains the list of all routes in the game scenario.

\subsection{Comparing design techniques}

According to the definition of encapsulation it is required that data and operations on them must be isolated within a module and a precise interface is provided \cite{ENCAPSULATION}. 

In our example the modules are \texttt{Planet} and \texttt{Route} class defined above, data are their fields, and operations are the following:

\begin{itemize}
	\item \textbf{Planet:} Take the enemy fleets travelling along its incoming routes which are close to the planet, and move them into the attacking fleets list,
	\item \textbf{Route:} Remove the travelling fleets which have placed in the attacking fleets of the destination planet from the list of travelling fleets.
\end{itemize}


\begin{lstlisting}
class Route
	private Planet Start, Planet End, 
	        List<Fleet> TravellingFleets,
	        Player Owner
	
	public void Update()
		foreach fleet in TravellingFleets
			if End.AttackingFleets.Contains(fleet)
				this.TravellingFleets.Remove(fleet)
				
class Planet
	private List<Fleet> DefendingFleets, 
	        List<Fleet> AttackingFleets
	
	public void Update()
		foreach route in GetState().Routes
			if route.End = this then
				foreach fleet in route.TravellingFleets
					if $\vert\vert$ fleet.Position - this.Position $\vert\vert$ < $\delta$ && 
				  		fleet.Owner != this.Owner then
				   			this.AttackingFleets.Add(fleet)

\end{lstlisting}

An alternative design not using encapsulation allows the route to move directly the fleets close to the destination planet into the attacking fleets by writing into the planet fields. In this scenario the route is modifying data related to the planet and the route is writing into a reference to a planet. Below we provide the pseudo-code for the update functions implemented with referential opaqueness \footnote{The code of the planet is missing because the fleet transfer is managed entirely by the \texttt{Route} class. In this case we assume that either the fields of \texttt{Planet} are public or a setter method exists.}.

\begin{lstlisting}
class Route
	private Planet Start, Planet End, 
	        List<Fleet> TravellingFleets
	
	public void Update()
		foreach fleet in this.TravellingFleets
			if $\vert\vert$ fleet.Position - End.Position $\vert\vert$ < $\delta$ &&
			   fleet.Owner != End.Owner then
				this.TravellingFleets.Remove(fleet)
				End.AttackingFleets.Add(fleet)

\end{lstlisting}

\subsection{Discussion}

In the fleet sample a programmer is left with the choice of either using the paradigm of encapsulation which improves the understandability of programs and ease their modification \cite{ENCAPSULATION_AND_INHERITANCE_IN_OOP}, or breaking encapsulation by writing directly into the planet fields from an external class, which, as we will show below, is more efficient.

In the encapsulated version the planet queries the game state to obtain all the travelling fleets to retrieve the links whose endpoints are the planet itself. At the same time, a \texttt{Route} checks the list of attacking fleets of its endpoints and removes the fleets which are contained in both lists from the travelling fleets. If we consider a scenario containing $m$ planets, $n$ routes, and at most $k$ travelling fleets per link, each planet should check the distance condition for $O(nk)$ ships, thus the overall complexity is $O(mnk)$.

On the other hand, we can break encapsulation in such way that each link checks the distance for a maximum of $k$ ships and then directly move those close to the planet, and this is done in time $O(nk)$.

\vspace{0.5cm}
\noindent
In the following section we define the idea behind a transformation from encapsulated code to a high-performance implementation. 