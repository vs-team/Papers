%%%%%%%%%%%%%%%%%%%%%%%%%%%%%%%%%%%%%%%%%%%%%%%%%%%%%%%%%%
% motivating example.tex
%%%%%%%%%%%%%%%%%%%%%%%%%%%%%%%%%%%%%%%%%%%%%%%%%%%%%%%%%%

To better understand the problem, we will refer to a typical situation encountered when trying to build videogames in a functional language. Let us suppose that in our game the player must stop the asteroids that come toward the bottom of the screen by shooting them down with plasma projectiles. We will have to update the simulation at fixed (small) intervals exactly as we would with any physical simulation. The update function takes the amount of time $dt$ that elapsed between this and the previous call to update, a description of the state of the game at time $T$ ($game_{-}state$) and returns the new state of the game at time $T+dt$. The code that achieves this is the following:

\begin{lstlisting}
type GameState = { asteroids : [Asteroid]; plasma : [Plasma] }

update_state dt (game_state:GameState) =
  let asteroids' = game_state.asteroids
                   |> map (update_asteroid dt)
                   |> filter (fun a -> plasma 
                                       |> map (fun p -> p.location) 
                                       |> exists (collides a.location))
  let plasma' = game_state.plasma 
                |> map (update_plasma dt)
                |> map (fun p -> p.location)
                |> filter (fun p -> asteroids
                                    |> map (fun a -> a.location) 
                                    |> exists (collides p.location))
  in { asteroids = asteroids'; plasma = plasma' }
\end{lstlisting}

Where 

\begin{lstlisting}
x |> f = f x
\end{lstlisting}

The above code can be quite optimized by:
\begin{itemize}
\item removing and inlining the unnecessary maps that extract the location of each asteroid and plasma
\item grouping the asteroids and the projectiles in spatial clusters (for example with a Quad Tree []) so that the two joins are faster
\end{itemize}

While this is not the ``definitive example" that shows all possible optimizations in action (there are many more optimizations that we will discuss) it is nevertheless quite useful. This code allows us to understand how easy it is to write clean, readable and concise code that uses list processing HOFs but which executes slower than it could and which is quite hard to optimize. Optimizing the above code would most likely result in a total rewriting: hardly convenient. Moreover, in our example the $update_{-}state$ function will be called at intervals of $\frac{1}{60}^{th}$ of a second: this notion only serves to stress the importance of being able to optimize said code in such scenarios.

In the rest of the paper we will discuss how these and many other optimizations can be achieved statically through the smart use of types.
