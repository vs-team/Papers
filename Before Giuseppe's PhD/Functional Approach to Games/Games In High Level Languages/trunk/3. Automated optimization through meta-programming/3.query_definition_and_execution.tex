%%%%%%%%%%%%%%%%%%%%%%%%%%%%%%%%%%%%%%%%%%%%%%%%%%%%%%%%%%
% query definition and execution.tex
%%%%%%%%%%%%%%%%%%%%%%%%%%%%%%%%%%%%%%%%%%%%%%%%%%%%%%%%%%

Query operators are not defined as a variant type. Rather, we define query operators as all those types which can be converted to a list. The type predicate that characterizes queries is the following:

\begin{lstlisting}
class Query q where
  type Elem q :: *
  run_query :: q -> [Elem q]
\end{lstlisting}

A type $q$ that represents a query that returns elements of type $Elem\ q$ can be converted (``executed" or ``run") by producing a list of elements of type $Elem\ q$. $Elem$ is a type function (a function from types to types []) while $run_{-}query$ is an actual function that is defined for all those types $q$ for which the predicate $Query\ q$ holds.

The first query operators that we support are:
\begin{itemize}
\item projection (``select")
\item filtering (``filter")
\item joining (``join")
\end{itemize}.

All these operators will have their instances of the $Query$ predicate. An additional instance will be given for relations, that is for simple lists:

\begin{lstlisting}
instance Query [a] where
  type Elem [a] = a
  run_query l = l
\end{lstlisting}

A projection simply takes a source query and the fields to be selected.
\begin{lstlisting}
type Select source fields = Select source fields
\end{lstlisting}

A projection is a query when its source is a query and its fields represent a proper subset of the element of the source query:
\begin{lstlisting}
instance (Query source, Subset fields (Elem source)) => 
          Query (Select source fields) where
  type Elem (Select source fields) = 
       Project fields (Elem source)
  run_query (Select source fields) = 
       run_query source |> map (project fields)
\end{lstlisting}

The projection operation requires a valid fields parameter; such a field must be a tuple of type-level naturals. When the fields tuple is empty, then we say that the projection of no fields of any tuple is the empty tuple:
\begin{lstlisting}
instance Tuple value => Subset () value where
  type Project () value = ()
  project () value = ()
\end{lstlisting}

When we have at least one field index to pick (the inductive case of projecting just one element), then we can say:
\begin{lstlisting}
instance (Natural n, Tuple value, 
          n < Length value, Subset ns value) => 
          Subset (n, ns) value where
  type Project (n, ns) value = 
       Nth n value , Project ns value
  project (n, ns) value = 
       nth n value, project ns value
\end{lstlisting}

Notice that we are assuming the existence of a series of predicates for determining whether or not a type variable is bound to:
\begin{itemize}
\item a type-level numeral ($Natural$ predicate)
\item a tuple ($Tuple$ predicate)
\end{itemize}.
We also assume a predicate for comparing two type-level naturals ($<$); the $Length$ type function is available on $value$ since it is true that $Tuple\ value$. Finally, we freely use the comma operator for building and decomposing inductively defined tuples.

Filtering queries require a source to be filtered and a statically known condition:
\begin{lstlisting}
type Filter source pred = Filter source pred
\end{lstlisting}

A filter is a proper query when its source is a query and its condition is a valid predicate on the elements of the source query:
\begin{lstlisting}
instance (Query source, 
          Predicate pred (Elem source)) => 
          Query (Filter source pred) where
  type Elem (Filter source pred) = 
       Elem source
  run_query (Filter source pred) = 
       run_query source |> filter (predicate pred)
\end{lstlisting}

A predicate is a boolean expression where each possible node has a different type; let us consider a small set of operators which can be easily extended:
\begin{lstlisting}
type True = True
type And a b = And a b
type Or a b = Or a b
type Equals a b = Equals a b
\end{lstlisting}

The simplest instances that we give are for combining together the $True$, $Or$ and $And$ predicates:
\begin{lstlisting}
instance Predicate True value where
  predicate True value = true

instance (Predicate a value, 
          Predicate b value) => 
          Predicate (And a b) value where
  predicate (And a b) value = 
       predicate a value && predicate b value

instance (Predicate a value, Predicate b value) => 
          Predicate (Or a b) value where
  predicate (Or a b) value = 
       predicate a value || predicate b value
\end{lstlisting}

Another important type that we define is the item lookup; this type allows us to statically represent the lookup of the $n^{th}$ item of a tuple for comparing it with a specific value in the $Equals$ predicate:

\begin{lstlisting}
type Item n = Item n

instance (Natural n, Tuple value, 
          n < Length value, Nth n value == b, 
          Predicate b value) => 
          Predicate (Equals (Item n) b) value where
  predicate (Equals (Item n) b) value = 
       nth n value == b
\end{lstlisting}

Now we can give predicates such as:
\begin{lstlisting}
And(Equals(Item(0), "Joe"), Equals(Item(2), 21))
\end{lstlisting}

which can be applied to any tuple where the first item is a string and the third item is an integer.

To make it easier writing conditions, we could even define type synonyms for our various predicates, such as:

\begin{lstlisting}
:&&: = And
:||: = Or
:==: = Equals
\end{lstlisting}

These synonyms would allow us to rewrite the above example as:

\begin{lstlisting}
(Item(0) :=: "Joe") :&&: (Item(2) :=: 21)
\end{lstlisting}

Which is quite more readable.

The final query operator we define is the join operator. A join simply is constructed with two queries $q_1$ and $q_2$:

\begin{lstlisting}
type Join q1 q1 = Join q1 q2
\end{lstlisting}

A $Join$ is a query when both its parameters are queries and the elements of its sub-queries can be flattened into just one element:
\begin{lstlisting}
instance (Query source1, Query source2, 
          CanFlatten (Elem source1) (Elem source2)) => 
          Query (Join source1 source2) where
  type Elem (Join source1 source2) = 
       Flatten (Elem source1) (Elem source2)
  run_query (Join source1 source2) = 
       [flatten x y | x <- run_query source1, 
                      y <- run_query source2]
\end{lstlisting}

The $CanFlatten$ predicate allows us to flatten the two tuples coming from the queries $source1$ and $source2$ into a single tuple:
\begin{lstlisting}
class CanFlatten value1 value2 where
  type Flatten value1 value2 :: *
  flatten :: value1 -> value2 -> Flatten value1 value2
\end{lstlisting}

We need just two predicates for flattening tuples: one for the base case of flattening an empty tuple with a non-empty tuple, and one for the inductive case:
\begin{lstlisting}
instance (Tuple value2) => CanFlatten () value2 where
  type Flatten () value2 = value2
  flatten () value2 = value2

instance CanFlatten vs value2 => CanFlatten (v,vs) value2 where
  type Flatten (v,vs) value2 = (v, Flatten vs value2)
  flatten (v,vs) value2 = (v, flatten vs value2)
\end{lstlisting}

