%%%%%%%%%%%%%%%%%%%%%%%%%%%%%%%%%%%%%%%%%%%%%%%%%%%%%%%%%%
% plain_imperative.tex
%%%%%%%%%%%%%%%%%%%%%%%%%%%%%%%%%%%%%%%%%%%%%%%%%%%%%%%%%%

The game we will implement is a simple vertical shooter. In the game the player controls a shooting ship with the directional arrows and asteroids fall down from the top of the screen. The player must destroy the asteroids before they fall out of the screen, and must also avoid colliding with an asteroid. After too many asteroids fallen out of the screen or after too many hits the game is lost, while after destroying enough asteroids the game is won.

The game starts with a menu. The menu features a simple animation: the various entries slide in from the left side of the screen. When the user selects the ``New Game" button, the menu is closed and the actual game begins. At the end of the game either the ``victory screen" or the ``defeat screen" is launched, depending on how the game ended; these two screens flash some animated text before closing and then launching the menu.

We can see the game as a finite state machine; the menu, the victory and defeat screens, and the game itself are the states of the machine. The transitions occur when the user selects the ``New Game" button, or when the victory or defeat conditions are met. Each state of the finite state machine is implemented in XNA by inheriting from the $Microsoft.Xna.Framework.DrawableGameComponent$ class. This class offers three virtual methods:
\begin{itemize}
\item Initialize (and LoadContent) for the initialization of the internal state of the class
\item Update for the update portion of the main loop
\item Draw for the drawing portion of the main loop
\end{itemize}

Thanks to this mechanism of game components we do not have to worry about managing the main loop and invoking the initialize, update and draw functions by hand; we just create a component, activate it and its functions will be invoked at the appropriate time.

The menu component is the first we will see. This component performs a simple animation so that the various items of the menu enter the screen from the left one after another.

The first structure we need for support is a container for a single menu item; a menu item has some text to be displayed on screen and an Action (a callback function) to be invoked when this item is selected:
\begin{lstlisting}
struct MenuEntry
{
  public string Name;
  public Action OnSelect;
}
\end{lstlisting}

The menu component itself inherits $DrawableGameComponent$:
\begin{lstlisting}
public class MenuScreen : Microsoft.Xna.Framework.DrawableGameComponent
\end{lstlisting}

The menu component stores the amount of time $T$ since its creation (to correctly perform the animation) and an array of entries:
\begin{lstlisting}
float T = 0.0f;
bool has_current_entry = false;
int current_entry = -1;
MenuEntry[] entries;
\end{lstlisting}

In the Initialize method we simply initialize the menu entries; notice how the callback for ``New Game" adds to $Game$, the manager of the main loop of the application, another component while removing itself from the list of active components:
\begin{lstlisting}
entries = new MenuEntry[]
{
  new MenuEntry()
  {
    Name = "New Game",
    OnSelect = () => {
      Game.Components.Add(new GameScreen(Game));
      Game.Components.Remove(this);
      this.Dispose();
    }
  },
  new MenuEntry() {
    Name = "Quit",
    OnSelect = () => Game.Exit()
  }
};
\end{lstlisting}

The update method takes as input from the main loop some timing information. This parameter, called $gameTime$, stores the amount of time since the launch of the game and the amount of time elapsed since the last call to update. This allows us to time our animations correctly: if the elapsed time is very short then we will have to perform little movement of our entities on screen, while if the elapsed time is longer then the movement of our entities on screen will be proportionately longer:
\begin{lstlisting}
public override void Update(GameTime gameTime)
\end{lstlisting}

The update function starts by incrementing the time counter:
\begin{lstlisting}
T += (float)gameTime.ElapsedGameTime.TotalSeconds;
\end{lstlisting}

If the animation is over then we can process the user input, either changing the index of the currently selected entry or invoking the callback of the current entry:
\begin{lstlisting}
if (input[Keys.Up])
{
  has_current_entry = true;
  current_entry = current_entry - 1;
}
else ...
\end{lstlisting}

When it's time to draw, then we iterate all the menu entries. We use a specific formula that allows us to interpolate the position of each menu entry between its starting position $start$ and its destination $end$ so that all the menu entries will enter the screen one at a time:
\begin{lstlisting}
var h = ScreenHeight / (entries.Length);
for (int i = 0; i < entries.Length; i++)
{
  var t = entries[i].Name;
  var c = has_current_entry && i == current_entry ? Color.LightYellow : Color.CornflowerBlue;
  var start = new Vector2(-200, i * h + h / 2);
  var end = new Vector2(400, i * h + h / 2);
  var pos = Vector2.SmoothStep(start, end, (T - EntryAnimationTime * i * 0.25f) / EntryAnimationTime);
  spriteBatch.DrawString(font, t, pos, c, 0.0f, font.MeasureString(t) * 0.5f,
    1.0f, 0, 0);
}
\end{lstlisting}

The $MenuScreen$ class is relatively straightforward, but we have to register a few aspects. First of all, we are using \textbf{a lot} of infrastructure in order to manage the macro-transitions of our game painlessly: we need a definition of the $DrawableGameComponent$ virtual class, and we also need a main loop that is smart enough to manage a list of instances of such class. Secondly, given the nature of the update and draw methods, the behavior of the class is harder to infer than it should be; in particular, the animation is computed analytically inside the draw method, rather than being described ``linearly" as we might wish to do with a straightforward loop such as (in pseudocode):
\begin{lstlisting}
for e in entries do
  e.start <- ...
  e.end <- ...
  while e.pos <> e.end do
    move e.pos towards e.end
\end{lstlisting}

The second $DrawableGameComponent$ that we will see is the one responsible for the actual gameplay. We will not see the $VictoryScreen$ and $DefeatScreen$ components since they resemble very closely the menu component.

The first definition that is needed by the main game is that of the $Entity$ structure. First of all this is a structure, not a class, so it will be compiled as a value type and not as a reference type. The main advantage is that we will not weigh too much on the garbage collector, since value types are not garbage collected:
\begin{lstlisting}
struct Entity
{
  public float life, damage;
  public Vector2 position, velocity;
  public float drag;
\end{lstlisting}

The $Entity$ struct has an update function to update its position with respect to its velocity and to apply ``drag" effect to its velocity:
\begin{lstlisting}
public void Update(float dt)
{
  position += velocity * dt;
  velocity *= (1.0f - dt * drag);
}
\end{lstlisting}

The $Entity$ struct also has some helper functions that allow us to determine whether the areas of two entities intersect; to determine intersection we also need to know the textures (the images) with which the entities will be drawn to screen:
\begin{lstlisting}
public static bool collision(Entity e1, Texture2D t1, Entity e2, Texture2D t2)
\end{lstlisting}

The actual $GameScreen$ class is declared as inheriting $DrawableGameComponent$:
\begin{lstlisting}
public class GameScreen : Microsoft.Xna.Framework.DrawableGameComponent
\end{lstlisting}

The most important fields of the $GameScreen$ class are those that describe the current state of the game; we have the description of the ship, of the asteroids, of the projectiles and the current score:
\begin{lstlisting}
Entity ship;
List<Entity> asteroids;
List<Entity> projectiles;
int missed_asteroids;
int destroyed_asteroids;
\end{lstlisting}

In the initialize method, we set up these fields so that they represent the ``initial" state of the game:
\begin{lstlisting}
missed_asteroids = 0;
destroyed_asteroids = 0;

ship = new Entity()
{
  damage = 50.0f,
  life = 100.0f,
  position = new Vector2(400, 440),
  velocity = Vector2.Zero,
  drag = 2.0f
};

asteroids = new List<Entity>();
projectiles = new List<Entity>();
\end{lstlisting}

The update methods begins by reading the current $\Delta t$; this represents the amount of time we will have to integrate for:
\begin{lstlisting}
var dt = (float)gameTime.ElapsedGameTime.TotalSeconds;
\end{lstlisting}

We update the ship position with respect to the user's input; we also force the ship to ``bounce" against the screen bounds:
\begin{lstlisting}
if (input.KeyboardState.IsKeyDown(Keys.Left))
  ship.velocity -= Vector2.UnitX * dt * SHIP_VELOCITY;
(** ... **)
\end{lstlisting}

If the user presses the \textit{space} key, then a new projectile is added exactly where the ship is. Notice that being $Entity$ a value type when we say $p\ =\ ship$ then we are actually copying the contents of $ship$ into $p$:
\begin{lstlisting}
if (input[Keys.Space])
{
  var p = ship;
  p.velocity = -Vector2.UnitY * PROJECTILE_VELOCITY;
  p.life = 1.0f;
  p.drag = 0.0f;
  p.damage = 5.0f;
  projectiles.Add(p);
}
\end{lstlisting}

We also generate new asteroids in random positions at the top of the screen with a certain (low) probabilty; this way if update is called $n$ times per second then we will generate on average $n \times p_{gen\ ast}$ asteroids per second:
\begin{lstlisting}
if ((float)random.NextDouble() <= ASTEROID_GENERATION_P)
{
  asteroids.Add(new Entity()
  {
    position = new Vector2((float)random.NextDouble() * ScreenWidth, 
                           -asteroid_t.Height),
    velocity = Vector2.UnitY * ASTEROID_VELOCITY * 
               ((float)random.NextDouble() * 1.0f + 0.5f),
    life = 10.0f,
    damage = 34.0f,
    drag = 0.0f
  });
}
\end{lstlisting}

At this point we need to update both asteroids and projectiles. We begin with projectiles:
\begin{lstlisting}
for (int i = 0; i < asteroids.Count; i++)
{
  var a = asteroids[i];
  a.Update(dt);
\end{lstlisting}

we store the value of each projectile into a temporary variable $a$. Until we perform $asteroids_i =\ a$ the modifications done on $a$ will not be reflected on the $asteroids$ list.

If there is a collision between the current asteroid and the ship, then we decrement the life of both:
\begin{lstlisting}
  if (Entity.collision(a, asteroid_t, ship, ship_t))
  {
    a.life -= ship.damage;
    ship.life -= a.damage;
  }
\end{lstlisting}

At this point we iterate all the projectiles. For each projectile that collides with an asteroid then we decrement the life of both; moreover, if a projectile is damaged to the point that its life is $\leq\ 0$, then we remove it from the $projectiles$ list:
\begin{lstlisting}
  for (int j = 0; j < projectiles.Count; j++)
  {
    var p = projectiles[j];

    if (Entity.collision(a, asteroid_t, p, plasma_t))
    {
      a.life -= p.damage;
      p.life -= a.damage;
    }

    if (p.life <= 0.0f)
    {
      projectiles.RemoveAt(j);
      j--;
      continue;
    }
    else
      projectiles[j] = p;
  }
\end{lstlisting}

The asteroids update loop then continues by checking if the current asteroid is destroyed or if it has moved beyond the bottom of the screen; if this is the case, then the asteroid is removed and the score is modified appropriately; otherwise the new value of the asteroid is copied back into the array:
\begin{lstlisting}
  if (a.life <= 0.0f || a.position.Y >= ScreenHeight)
  {
    if (a.life > 0.0f)
      missed_asteroids++;
    else
      destroyed_asteroids++;

    asteroids.RemoveAt(i);
    i--;
    continue;
  }
  else
    asteroids[i] = a;
}
\end{lstlisting}

The update for projectiles is very similar to the previous loop:
\begin{lstlisting}
for (int i = 0; i < projectiles.Count; i++)
{
  var p = projectiles[i];
  p.Update(dt);
  if (p.life <= 0.0f || p.position.Y <= -plasma_t.Height / 2)
    (** REMOVE PROJECTILE p **)
  else
    projectiles[i] = p;
}
\end{lstlisting}

Similarly to how we moved from the menu to the actual game, if the score counters have reached certain thresholds then the game has been won (or lost) and the $GameScreen$ must be removed while the appropriate screen must be added to the set of active components:
\begin{lstlisting}
if (destroyed_asteroids >= 10)
{
  Game.Components.Add(new VictoryScreen(Game));
  Game.Components.Remove(this);
  this.Dispose();
}

if (missed_asteroids >= 10 || ship.life <= 0.0f)
{
  Game.Components.Add(new DefeatScreen(Game));
  Game.Components.Remove(this);
  this.Dispose();
}\end{lstlisting}

The draw function simply iterates all the entities and draws them, together with the current score and the life of the ship:
\begin{lstlisting}
public override void Draw(GameTime gameTime)
{
  spriteBatch.Begin();
  asteroids.ForEach(a => draw(a, asteroid_t));
  projectiles.ForEach(p => draw(p, plasma_t));
  draw(ship, ship_t);

  spriteBatch.DrawString(font, "Life: " + ship.life.ToString("###"), 
                         new Vector2(5, GraphicsDevice.Viewport.Height - 30), 
                         Color.CornflowerBlue);
  spriteBatch.DrawString(font, "Destroyed: " + destroyed_asteroids, 
                         new Vector2(5, 5), Color.LawnGreen);
  var t = "Missed: " + missed_asteroids;
  var t_s = font.MeasureString(t);
  spriteBatch.DrawString(font, t, 
                         new Vector2(ScreenWidth - 5 - t_s.X, 5), Color.Orange);
  spriteBatch.End();

  base.Draw(gameTime);
}
\end{lstlisting}

The code above is very fast and wastes very little processing power. The collision detection is not optimized (as indeed it should and would be in an actual game) but nevertheless this very imperative style where collections are modified in-place has the advantage that we are wasting very little memory and very little CPU cycles. On the other hand parallelizing the above code is quite the challenge: splitting the update and draw loops in particular, which might be very desirable (especially given that virtually no PCs bought today have less than two cores). Sadly, this very style of programming means that whenever we update one collection (either $asteroids$ or $projectiles$) then if that collection is being drawn we will risk a memory leak (the .Net framework is actually very conservative about this: if a collection is modified during its enumeration an exception will be thrown; there is in fact no guarantee that the current element being enumerated is still part of the collection).
