%%%%%%%%%%%%%%%%%%%%%%%%%%%%%%%%%%%%%%%%%%%%%%%%%%%%%%%%%%
% compiling_scripts.tex
%%%%%%%%%%%%%%%%%%%%%%%%%%%%%%%%%%%%%%%%%%%%%%%%%%%%%%%%%%

Scripting is a very important part of game development \citep{BETTER_SCRIPTS_GAMES}. For this reason we have adapted to our system a scripting solution that is derived from game engines. Whereas many game engines either use Lua, Python or even C\# as scripting languages (with various advantages and disadvantages) \cite{SCRIPTING_LUA,SCRIPTING_PYTHON, UNITY_YIELD} we have used F\# which we believe offers a powerful mix of the best features of all these languages: coroutines, flexibility and a lightweight syntax make F\# scripts similar to LUA and Python while static typing and support for .Net libraries and IDEs put F\# on par with C\# in terms of broader support.

To give additional expressive power to our scene, we add support to external scripts; external scripts are all those scripts that cannot be expressed in terms of routes. External scripts are very general, that is they can perform complex data conversions when copying values across entities, and they may even create, remove and modify nodes in any way possible. To represent external scripts, rather than using arbitrary objects that can access the state we have chosen to use coroutines, a widely used mechanism for representing computations in interactive applications \cite{PYTHON_COROUTINES,GPU_GEMS_6}. Coroutines are subroutines that can be suspended and resumed at certain locations. With coroutines the code for a SM is written ``linearly'' one statement after another, but each action may suspend itself (an operation often called ``yield'') many times before completing. A coroutine stores a temporary, internal state transparently inside its continuation.

We build a monadic framework \cite{COMPR_MON,DECL_IMP,EFF_MON,MOGGI_MON} for coroutines that allows us greater customization flexibility. This way we can define our own system for combining scripts running them in parallel, concurrently, etc. For a detailed discussion of this monadic framework for scripts and coroutines see \cite{X3D_TR1}.

A script in our system is defined as a normal F\# program surrounded by \texttt{\{ \}} brackets. A script runs another script with the statements \texttt{let!} and \texttt{do!}, and scripts can be combined with a small set of operators.

The main operators to combine scripts are:
\begin{itemize}
\item \texttt{parallel} ($s_1 \wedge s_2$) executes two scripts in parallel and returns both results
\item \texttt{concurrent} ($s_1 \vee s_2$) executes two scripts concurrently and returns the result of the first to terminate
\item \texttt{guard} ($s_1 \Rightarrow s_2$) executes and returns the result of a script only when another script evaluates to \texttt{true}
\item \texttt{repeat} ($\uparrow s$) keeps executing a script over and over
\end{itemize}

A sample script that moves the box \texttt{myBox} when the user enters a certain region \texttt{myRegion} could be the following:

\begin{lstlisting}
let my_script (scene:Scene) =
  let rec animate =
    script {
      if scene.myBox.Position.Y < 100.0f then
        scene.myBox.Position.Y <- scene.myBox.Position.Y + 0.1f
        do! animate }
  script {
    do! guard
         script {
           return inside(scene.Camera.Position, myRegion) }
         animate }
\end{lstlisting}

Notice that our script has a parameter of type \texttt{Scene}. If this parameter is used incorrectly (for example the scene this script is applied to does not have a \texttt{Box} node with name \texttt{myBox}) we will get a compile-time error. This makes it easier to build larger, reusable script modules since a mistake in using a pre-made module is easier to spot and requires less testing. Using scripts which have been made for different scenes would require extensive testing to ensure at least that all node accesses are correct.

Our scripting system is expressive enough to represent many scripts running together, even if at a first glance it may appear that our system supports only a single script. By using the \texttt{parallel} operator we can combine together a large number of scripts. For example, let us say we have many scripts $s_1, ... , s_n$ that must all run together with our scene. Each scripts has a different duration, that is the not all scripts will end at the same time (indeed, a script may even run indefinitely). The main script would chain each of the various actual scripts in the following manner:

\begin{lstlisting}
let my_script (scene:Scene) =
  parallel $s_1$ (parallel $s_2$ ... $s_n$) ... )
\end{lstlisting}
