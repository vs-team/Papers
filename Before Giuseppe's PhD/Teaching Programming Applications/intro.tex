%%%%%%%%%%%%%%%%%%%%%%%%%%%%%%%%%%%%%%%%%%%%%%%%%%%%%%%%%%
% intro.tex
%%%%%%%%%%%%%%%%%%%%%%%%%%%%%%%%%%%%%%%%%%%%%%%%%%%%%%%%%%

Math is beauty, engineering is challenge, Computer Science is both...plus fun!
However, it's becoming more and more difficult to get high school students enrolled in CS university programmes. One reason is that this scientific discipline is widely confused with its technological applications. However, we believe that there is also another reason: the way Computer Science, and in particular Computer Programming is presented to beginners. One of the greatest features of Computer Science is stressed too little: the fact that even the most complex problems can be expressed and solved by engaging, visual applications, and that most theory in Computer Science arises from intuitive and fascinating problems.

In this paper we report the results of a project run at our university with the final aim of getting high school students to look at Computer Science as ``beautiful, challenging and fun''. We describe how we built a template around an original pedagogical process for creating short-courses that can be taught in one day; these short-courses offer high school students the opportunity to experiment with a complex application of Computer Science in a visual and engaging environment. Our short-courses do not aim at making Computer Science appear as just ``fun and games''. Rather, we wish to offer students noise-free environments in which to do practical experiments (so no complex APIs or too technical details) while still maintaining rigor and theoretical soundness. In this paper we focus on two short-courses about computer graphics and physical simulations; in these courses students learn some of the equations that model the appearance and the motion of physical objects, and they learn how to translate those equations into (functional) code that the computer understands. The code that students write is used for rendering a scene where the students can immediately receive feedback to see if they have implemented everything correctly or not.

In Section \ref{sec:short-courses_template} we describe the general template around which our short-courses are built. In Sections \ref{sec:computer_graphics} and \ref{sec:physics_simulation} we describe two of these short-courses: computer graphics and physics simulation; we will show a large portion of the actual code we used in the classroom. In Section \ref{sec:feedback_and_results} we report the feedback and the rating that we received from the first batch of students that experimented with these short-courses.

\paragraph{Related Work}

The study of how to better teach Computer Science is growing by the year. A very large part of these studies is centered on how to teach Computer Science by insisting heavily on visual, entertaining applications (\cite{GAMES_AND_TRADITIONAL_CS}, \cite{POWER_AND_PERIL_GAME_TEACHING}, \cite{GAMES_FRESHMEN} and many others) that more closely resemble the students' experience with modern applications \cite{TEACHING_CS_THROUGH_GAMES}. These programs are all built to match the expectations that young people have when they start studying programming: they want to build web sites, games and interesting applications. Like these authors, we wish to offer beginner students an environment where they can learn complex tasks (like programming complex algorithms or equations) with the same visual appeal that they have come to expect from computer programs.

Most authors (\cite{GAME_DESIGN_AND_CS}, \cite{GAMES_FIRST_APPROACH}, \cite{GAME_PROG_INTRO}, \cite{GAME_PROG_FACULTY}, \cite{GAME_THEMED_CS1_2}) have observed a large increase in the effort made by students to complete their game-themed assignments when compared with the lack of interest of traditional (shell-based) assignments; we have measured a similar outcome during our initiative.

In \cite{FUN_IO} and \cite{SICSC} the authors have experimented with building a curriculum that can be used from middle- to high-schools and even in a university introductory course. These studies, together with \cite{PURELY_FUN_FIRST_YEAR}, have all used functional languages like we have done in our approach, believing that the syntactic and semantic cleanliness of these languages is a large bonus when compared to more complex imperative computational runtimes, especially when put in the hands of programming novices.

\paragraph{Downloading and Testing}

We have made our implementations available for download at the following URLs:
\begin{itemize}
\item the computer graphics source can be downloaded at \texttt{www.dsi.unive.it/\~{}grafica/PLS}
\item the physics simulation source can be downloaded at \texttt{www.dsi.unive.it/\~{}sartoret/SistemaDinamico.zip}
\end{itemize}
