%%%%%%%%%%%%%%%%%%%%%%%%%%%%%%%%%%%%%%%%%%%%%%%%%%%%%%%%%%
% conclusions.tex
%%%%%%%%%%%%%%%%%%%%%%%%%%%%%%%%%%%%%%%%%%%%%%%%%%%%%%%%%%

In this paper we presented a very applicative, hands-on approach for teaching complex topics in computer science. We have designed a set of short-courses that can be completed by high school students in at most 8 hours, each short-course focusing on an advanced application of computer science In the paper we have focused on two short-courses in particular, even though other similar short-courses achieved comparable results.

In all short-courses the first step consists in explaining the problem tackled in simple, intuitive (yet rigorous) terms. Then the required mathematical background is given, always focusing on its relationship with the problem and not as important ``per se''. A brief introduction to the development tools and languages is given; we used languages with little syntactic and semantic complexity such as Python or F\#, in order to avoid explaining things like scope and brackets which can be confusing for beginners. The tools and frameworks offered to students must be tweaked to the point that there is virtually no distraction from the main task: the students are required to write only the minimum code necessary to solve the problem at hand and nothing more.

All of our students have solved all the exercises in the given time, even the most complex: the students' good feedback shows that even though there is room for improvement this kind of initiative leaves them happy from a pleasant learning experience; considering the fact that the topics are traditionally considered very hard and boring (algebra, geometry and differential equations) this is a very notable achievement.

This work is by no means complete. This paper has described the second step in a larger initiative that started last year and that we hope to continuously improve. Last year we created the first few short-courses, while this year we defined a more general template to create more short-courses all based on the same working formula. This initiative is resulting in stronger relationships with the high schools surrounding our university, and given the initial success we will try to expand our offering. On one hand we will create more and more short-courses on other topics such as game or mobile development, while on the other hand we will experiment with longer courses that take more than one day to complete.
