\section{Vectors}
We now implement a simple example that shows how we can define a system of mutable vectors.

We begin by defining a 2d vector ($Vector2$) with two methods and a 3d vector ($Vector3$) with two other methods and which inherits the 2d vector:
\begin{lstlisting}
type Vector2Def k = Field () 'Cons' Field Float 'Cons' Field Float 'Cons' Method k () () 'Cons' Method k () Float 'Nil'
data RecVector2 = RecVector2 (Vector2Def RecVector2)
type Vector2 = Vector2Def RecVector2
instance Recursive Vector2 where
  type Rec = RecVector2
  to = RecVector2
  from (RecVector2 v) = v
x :: Label Vector2 (Field Float)
x = labelAt Z
y :: Label Vector2 (Field Float)
y = labelAt (S Z)
norm2 :: Label Vector2 (Method RecVector2 () ())
norm2 = labelAt (S S Z)
len2 :: Label Vector2 (Method RecVector2 () Float)
len2 = labelAt (S S S Z)

mk_vector2 xv yv = Field () 'Cons' Field xv 'Cons' Field yv 'Cons' mk_method (\this->\()->do l <- (this <= len2) ()
                (this <= x) *= (/l)
                (this <= y) *= (/l)) 'Cons' mk_method (\this->\()->do xv <- this <= x
         yv <- this <= y
         return sqrt(xv * xv + yv * yv))

type Vector3Def k = Inherit Vector3 'Cons' Field Float 'Cons' Method k () () 'Cons' Method k () Float 'Nil'
data RecVector3 = RecVector3 (Vector3Def RecVector3)
type Vector3 = Vector3Def RecVector3
z :: Label Vector3 (Field Float)
z = labelAt (S Z)
norm3 :: Label Vector3 (Method RecVector3 () ())
norm3 = labelAt (S S Z)
len3 :: Label Vector3 (Method RecVector3 () Float)
len3 = labelAt (S S S Z)
instance Recursive Vector3 where
  type Rec = RecVector3
  to = RecVector3
  from (RecVector3 v) = v

mk_vector3 xv yv zv = Inherit (mk_vector2 xv yv) 'Cons' Field z 'Cons' mk_method (\this->\()->do l <- (this <= len3) ()
                (this <= x) *= (/l)
                (this <= y) *= (/l)
                (this <= z) *= (/l)) 'Cons' mk_method (\this->\()->do xv <- this <= x
         yv <- this <= y
         zv <- this <= z
         return sqrt(xv * xv + yv * yv + zv * zv))
\end{lstlisting}

Now we can use these vectors:
\begin{lstlisting}
main :: mem ~ (Vector3 'Cons' Nil) => ST Nil Nil Bool
main = do+ v <- new (mk_vector3 0.0 2.0 -1.0) :: Ref mem Vector3
           zv <- v <= x
           let v' = coerce v :: Ref mem Vector2
           zv' <- v' <= x
           (v <= norm2)()
           (v <= norm3)()
           delete
           delete
           return (zv = zv'))

res :: Bool
res = runST main Nil
\end{lstlisting}

where we expect that $res=True$. Notice how select labels that are defined for a 2d vector on an instance of a 3d vector, and how we access the same label on the value of $base$ obtained through coercion on the 3d vector.
